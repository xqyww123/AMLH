\section{研究背景与意义}

区块链是一种新兴的去中心化分布式技术,
建立在此技术上的智能合约平台允许分布式地运行一系列被叫做“智能合约”的程序
。
基于区块链技术与智能合约技术的应用已产生了广泛的影响。
其中以太坊  与 EOSIO  均是
目前商业市场上活跃的平台,本文撰写时
以太坊市场总值为1500亿人民币左右,EOSIO 市场总值为280亿人民币左右
。

智能合约往往应用于金融领域,智能合约的
%程序实现上的
缺陷将直接影响其业务,并可能造成严重的经济损失,
且已实际发生多起经济损失事件,
包括 DAO 事件 造成1.5亿美元的损失,
HackerGold 事件 造成 40万美元的损失,
Rubixi 事件 造成 2万美元的损失,
Governmental 事件 造成 1万美元的损失,
Parity Multisig 事件 造成3千万美元的损失。

由于区块链技术本身的特点,一旦智能合约被部署就难以被修改,
而任何部署后发现的缺陷都难以被修复。DAO 事件中,缺陷实际在数月前就被发现,
但因缺乏有效的补救措施而未能及时修复,事件发生后以太坊不得不通过
硬分叉来挽回攻击造成的损失,但仍造成其市值大幅下跌 。

因此对智能合约的安全分析非常重要,特别是在智能合约被实际部署应用之前。
已有大量文献利用形式化方法对智能合约进行安全研究,
Krupp 开发了程序 \textsc{teEther} 能自动发现以太坊智能合约漏洞,并进行自动渗透
;
Hildenbrandt 在 $\mathbb{K}$ Framework  上构建了
以太坊虚拟机 EVM (Ethereum Virtual Machine),
以允许使用 $\mathbb{K}$ Framework 对
以太坊智能合约进行包括形式化验证在内的安全分析 。

大量的工作对智能合约进行形式化验证已证明其对给定性质的正确性,
这些验证工作可以分为基于 Model check 的与基于演绎证明(Deductive Verification)的
。基于 Model check 的工作包括,
Luu 开发了符号执行工具 \textsc{Oyente} 寻找以太坊智能合约的潜在缺陷
,尽管这一工具被批评为既不可靠(sound)又不完备(complete)
;
Zhou 提供工具 \textsc{Sasc} 扩展了 \textsc{Oyente} 的功能,包括增加了额外的特征样式与
可视化 Solidity 代码的拓扑图中已探测到的风险 ;
\textsc{Maian} 也扩展了 \textsc{Oyente} 的功能,额外地考虑单个智能合约中的多次调用情形,
进而发现不正常的,贪婪或者自杀的智能合约 。
\textsc{Oyente} 相关工作的主要缺陷在可靠性,只能发现一些潜在的漏洞而无法证明
对任何性质的满足。
Kalra 开发了程序 \textsc{ZEUS} 通过抽象解释与符号模型检查分析去证明一个以太坊
智能合约的正确性并验证它的公平性,它将智能合约翻译到 LLVM 平台,并使用 LLVM 
上已有的形式化验证工具进行分析与证明 ;
Tsankov 类似地提供工具 \textsc{Securify} 分析并证明智能合约的性质
。
其他类似的基于 Model check 的验证或分析工具包括 MythX,Manticore
,solgraph,SmartCheck。

基于演绎证明的工作包括,Bhargavan 将以太坊智能合约翻译到具有依赖类型系统的
编程语言 F* 上,使用依赖类型系统对智能合约进行演绎证明 ;
Hirai 将以太坊虚拟机 EVM (Ethereum Virtual Machine)
形式地定义在 Isabelle/HOL 上并使用 Isabelle/HOL 对 EVM 上智能合约的字节码进行
证明 ;Amani 改进了 Hirai 的工作,使用来自原始编程语言
的基础构造块信息有效地优化了证明的速度 ;
Pettersson 设计并实现了依赖类型语言 Idris 到 EVM 的编译后端,以允许由被依赖类型
系统验证的 Idris 程序编译到 EVM 。
此外还有一些相关的工作,Sergey 设计了中间级别的编程语言 Scilla 用于
对智能合约的形式化证明,特别考虑了智能合约在链上的通讯 。

综合来看,
目前对智能合约的形式化验证工作还有很多欠缺。Model check 类的工作受到 Model check
本身能力的制约,只能证明一阶逻辑(First Order Logic)内的性质,进而
只能削除限定种类的智能合约缺陷,例如算术溢出、栈调用溢出等一些特定种类的缺陷。
对于智能合约中更抽象的逻辑错误往往是无能为力的,例如难以证明一段程序输出的值
始终是质数。
%工具 \textsc{ZEUS} 使用受限霍恩子句(constrained horn clauses)进行证明,
%而受限霍恩子句只能处理一阶逻辑的问题。
演绎验证类的工作则在可行性上有很大问题,多数工作都是难以实践的,即便是对
一些短小的智能合约,也需要消耗相当大的时间与人力。


