\chapter{绪论}
\section{研究背景与意义}

软件开发的主要挑战是在有限而可控的的时间与经费预算下开发高质量的软件
\cite{DinesSE1, brooks1995mythical}。
但却缺少一种简单而万能的软件工程方案,一如
F Brooks 所说“没有银弹”(“No Silver Bullet”)\cite{brooks1987no},
却又着实存在这样的需要,一种至少是有效的又也许尽可能简单的软件开发方法学。
这正是软件工程作为一个学科的意义所在,一如令软件工程伊始的
NATO(北大西洋公约组织)软件工程峰会与所谓的“软件危机”
\cite{nato1969software}。

大量的方法提出以提升软件工程的效率与产品的质量
\cite{DinesSE1, SoftwareQuality}。
形式化方法(formal method)也可有效地应用于此,
以提升软件质量,并辅助软件的开发以提升开发效率\cite{formal_method_view1} 。
实际上大量的形式化方法已经在实际地发挥作用\cite{pierce2002types, jackson2012software}。

其中重要的一点是程序实现的验证,即一实现是否正确实现了设计的全部功能,且
不具有违背设计的包括漏洞在内的任何缺陷。
传统的单元测试难以发掘程序的全部潜在缺陷,
而相比而言形式化验证可以更有效地验证程序实现而发现更多的缺陷
\cite{formal_method_view1}。
但问题在于,倘若某种可行的形式化验证仅仅是更有效而并非是彻底,
即被验证后的程序实现依旧可能存在种种缺陷,
那这种形式化验证只能称之为比传统测试更好,
但问题仍未被彻底解决,仍然不能说一个实现是对于设计正确的满足而无任何缺陷。

软件的正确性包括其设计的正确性与程序实现的正确性。
软件的设计可能有缺陷,也许难以预见而难以检验,
但是否有可能一个程序的实现是彻底正确而无缺陷地满足了其设计?
进而能否验证一个程序是否被正确实现?
再而能否构造正确实现的而无任何缺陷的程序,
即所谓被验证的程序(verified software)\cite{verified_soft_grand}?

这一点具有实际的意义。
特别是在安全严苛(Safety critical)\cite{DinesSE1}的场景上尤其紧要。
有太多的计算机硬件与软件系统直接与生命安全、重要的民生事务相关。

本工作聚焦的基于区块链技术的各种智能合约平台就是这样的安全严苛场景。
因为区块链技术与平台的特殊性,被叫做智能合约的程序一旦被部署往往就无法修改,
进而无法更新以修复任何部署前未被发现的缺陷 \cite{swan2015blockchain}。
而目前区块链技术与智能合约平台大量应用于金融领域,如 Bitcoin 等诸多的
加密货币\cite{narayanan2016bitcoin},与加密金融这一新领域\cite{came2019}。
短短的数年以来智能合约领域已有大量的缺陷案例\cite{atzei2016survey}。
这些缺陷往往会导致严重的经济损失 cite???。
这些案例中,软件的设计往往是正确的而因为程序错误的实现而造成了损失。

这凸显了验证程序实现的重要性。

我们不能因为过往与眼前经受的种种困难就放弃,而轻易地留下一句
“这始终是一门发展的学科而程序亦始终在发展而不存在完美的一刻”,
以对一切草草了事。
某一个程序是否可以宣称其实现是彻底正确的,而无任何缺陷与漏洞的,这一问题
事关计算机科学的基础,如果有一个回答,也必须是严谨的论证而不能仅凭经验
草草地回答,否则就是不负责任。

对正确而无缺陷的程序实现及这种开发方法的追求并非是痴人说梦。
1967年 Floyd 的论文就清晰地旨在寻找一种严格的对程序证明
包括正确性证明与停机性证明的方案\cite{floyd1967}。

此处涉及到的正是形式化方法的起源。
形式化方法并不只是一个软件工程学方法,软件工程在1969年NATO软件峰会上
才被定义,而其出现是为了解决当时的“软件危机”\cite{nato1969software}。
而形式化方法以及其中的形式化验证源自久远的计算机科学诞生伊始的1940年代中期,
最初 von Neumann, Goldstine 和 Turing 等人
试图基于流程图与断言等达成对程序的包括但不限于正确性的论证\cite{jones2003early},
而后是 Hoare 逻辑 \cite{hoare1969axiomatic} 与公理化语义(Axiomatic Semantic)。

软件工程要求程序作为产品的可靠性。
而计算机科学作为一门科学,自其诞生伊始并持续的近百年来,
一直在尝试寻找一种有效的通用手段,彻底而严谨地证明程序实现的正确性,
这一点既有软件工程角度生产上的意义,也有纯粹的科学意义。

以 Hoare 与 Milner 的话说,
这是计算科学界的“伟大挑战”(Grand challenge in computing research)
\cite{hoare2004grand, hoare2006ideal}。
众多的支持文章从不同角度涌现:“具有验证功能的编译器”(verifying compiler)
\cite{verifying_compiler},“被验证的软件”(verified software)
\cite{verified_soft_grand},“可依赖的系统演化”(Dependable Systems Evolution)
\cite{Dependable_Systems_Evolution_grand}。

本文研究的背景领域是形式化领域。但形式化并非仅是软件工程中的工具,
尽管在软件工程方面具有非常的现实经济意义。
对程序严谨的形式分析与证明,这直接体现的是计算机科学作为一门科学的意义。

形式化方法发展至今已经有非常丰富的流派依照各自的观念发展出非常丰富多样的成果。
在实际的软件工业生产中,形式化方法已经有许多成功又面临诸多困难。

各种类型系统均属于形式化方法的范畴,已成为现代编程语言的重中之重,而
被广泛应用于各种工业生产,并如约地有效削减了众多程序缺陷并极大地提升了
程序开发效率\cite{pierce2002types}。
甚至,一些高级类型系统特别是依赖类型系统(Dependent type system)
\cite{mcbride2000dependently}
在数学构建主义(Mathmatic Constuctism)的道路上成功地让“程序即是证明”
(Program is proof)进而达成了被验证的编程语言\cite{pierce2005advanced},如
Agda \cite{norell2008dependently}, Idris \cite{brady2013idris},
而一些限定的方面上非常接近“伟大挑战”所追求的。
形式化描述(Formal Specification)领域,也已有大量优秀的成果如 Z 
\cite{jacky1997Z}、B Method\cite{abrial1991B}。

然而相比数理逻辑的简洁,现实的需求是如此复杂,系统具有如此多的细节,
形式化模型难以描述这些,而对其进行分析就更加困难,
就意味着更加巨大的资源投入与消耗\cite{jackson2012software}。
J Daniel 说“形式化方法不是银弹”
(“formal methods were not the silver bullet”)\cite{jackson2012software}。

现状是,软件开发的商业与工业主流已牢牢地得益于各种形式化方法,
形式化方法已渗入软件开发的各个方面,在安全严苛场景已大量地应用,
但对软件普遍地形式化验证仍未到来。
形式化方法的确已经有效地帮助程序开发而形式化验证也已有效地发现并避免了诸多
软件缺陷,但缺陷与漏洞依旧存在。
在一些限定的范围,被验证的程序实现已经能够生产,却
在现实的工业生产的普遍范围内因为种种原因未被广泛应用。
现实软件工业的普遍现状是,缺陷与漏洞始终存在,随处可见。
现实工业生产尚未选择——因为尚未发现——一种更便宜的手段,生产完美无缺的程序实现;
而任何已有手段的实施成本本身,相比缺陷本身的潜在危害都更加昂贵。

无论是软件工程方面的现实经济意义,还是计算机科学角度的学术意义,
形式化方法都从未实现它的愿景,因此这“伟大挑战”也得以是一种挑战。

本文尝试寻找并实现这样的形式化方法,它代价不高,不消耗大量的开发成本,
只需要专业的数学知识与机器证明技能,而本身施行起来不复杂不困难,
进而能被普遍地应用在现实的普通工业生产中;
但却能有效而彻底地证明程序实现的正确性。

\subsection{智能合约}

区块链是一种新兴的去中心化分布式技术\cite{swan2015blockchain},
建立在此技术上智能合约平台允许分布式地运行一系列被叫做“智能合约”的程序
\cite{buterin2014next}。
基于区块链技术与智能合约技术的应用已产生了广泛的影响\cite{casino2019systematic}。
其中 Ethereum \cite{buterin2014next} 与 EOSIO \cite{eos.io} 均是
目前商业市场上活跃的平台,本文章撰写时
Ethereum 市场总值为1500亿人民币左右,EOSIO 市场总值为280亿人民币左右
\cite{coinmarket.cap}。

其上运行的智能合约直接与其商业业务关联,智能合约程序实现
上的缺陷将直接影响其业务,并可能造成严重的经济损失。
已有大量文献研究智能合约的安全性
\cite{dhawan2017analyzing, krupp2018teether, luu2016making, 
suiche2017porosity, kalra2018zeus, nikolic2018finding}。
已有诸多智能合约实现的缺陷造成了验证的财产损失\cite{vitalik2016think, atzei2016survey},
包括 DAO 事件\cite{DAOattack} 造成1.5亿美元的损失,
HackerGold 事件\cite{HackerGold} 造成 40万美元的损失,
Rubixi 事件\cite{vitalik2016think} 造成 2万美元的损失,
Governmental 事件\cite{vitalik2016think} 造成 1万美元的损失,
Parity Multisig 事件\cite{ParityMultisig} 造成 20亿美元的损失。

且由于区块链技术本身的特点,一旦智能合约被部署就难以被修改,
而任何部署后发现的缺陷都难以被更新。DAO 事件中,缺陷实际在数月前就被发现,
但因缺乏有效的补救措施而未能及时修复,事件发生后 Ethereum 不得不通过
硬分叉来挽回攻击造成的损失 \cite{EebhftrDf}。

对智能合约进行形式化分析以尽可能在发布前提早发现缺陷有重要意义。
已有大量文献利用形式化方法对智能合约进行安全研究
\cite{hildenbrandt2018kevm, bhargavan2016formal, hirai2017defining,
amani2018towards, pettersson2016safer, sergey2018scilla, grishchenko2018foundations}。

而测试与静态分析难以覆盖所有潜藏的缺陷。目前的智能合约都很短小,
对其进行形式化验证相对而言并不困难,而智能合约因其独特性其上一切缺陷的代价
都十分昂贵,因此值得对智能合约进行尽可能全面的形式化验证。
如上述,若能彻底形式化地证明智能合约实现的正确性,则可自信地宣称
此合约的实现已尽其所能的正确而不存在任何潜在的缺陷。

本工作着重以智能合约应用为场景,针对 EOSIO 平台设计。

\subsection{WASM}
