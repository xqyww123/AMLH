%\section{Noesis 系统}
\section{朴素 Noesis 系统}

\begin{defin}[朴素 Noesis 系统]
朴素 Noesis 系统是一个形式系统,基于无类型 λ 演算。

无穷的字符串集合 $\hat{\mathcal{P}},\ \hat{\mathcal{E}} \subseteq 
\Words$ 分别表示现象、本体的字母集。
其上的 λ 表达式 $\mathcal{P},\ \mathcal{E}$ 分别为现象集、本体集。
\begin{align*}
\mathcal{P} &= \bnf{\hat{\mathcal{P}}\mbar (\mathcal{P}\ \mathcal{P})
\mbar (\lambda\ \hat{\mathcal{P}}\ \mathcal{P})} &
\mathcal{E} &= \bnf{\hat{\mathcal{E}}\mbar (\mathcal{E}\ \mathcal{E})
\mbar (\lambda\ \hat{\mathcal{E}}\ \mathcal{E})}&
\end{align*}
纯粹的 Noesis 系统不支持对理解的演算,理解是原子的,理解集 
$\mathcal{I} \subseteq \Words$ 是单词集。

\noindent Noesis 对应集 $N_1$
\[ N_1 = \bnf{\mathcal{P} \widesim{\mathcal{I}} \mathcal{E}} \]
Noesis 同构集 $N^*$
\[ N^* = \bnf{\mathcal{P} \proctr{\mathcal{I}|\cdots|\mathcal{I}}{}
\mathcal{E}} \]
Noesis 表达式是集合 $N = N_1 \cup N^*$。
上下文集合 $\Gamma_\mathrm{N} = \powerset(N_1)$。
Noesis 形式系统的形式语言 $L_\mathrm{N}$ 由语法 
\[ \Gamma_\mathrm{N} \vdash N \]
确定,其上的规则 $\vdash$ 如下

\vspace{5mm}

\hspace{-8mm}%
\begin{minipage}[b]{0.25\linewidth} \begin{prooftree} \centering
    \AxiomC{\hfill}
    \RightLabel{(公理)}
    \UnaryInfC{$p \widesim{i} e \vdash p \widesim{i} e$}
\end{prooftree}\end{minipage}%
\begin{minipage}[b]{0.4\linewidth} \begin{prooftree}
    \AxiomC{$\Gamma \vdash p \widesim{i} e$}
    \RightLabel{(一阶同构引入)}
    \UnaryInfC{$\Gamma \vdash p \proctr{i}{} e$}
\end{prooftree}\end{minipage}%
\begin{minipage}[b]{0.3\linewidth} \begin{prooftree}
    \AxiomC{$\Gamma \vdash p \proctr{i}{} e$}
    \RightLabel{(一阶同构削除)}
    \UnaryInfC{$\Gamma \vdash p \widesim{i} e$}
\end{prooftree}\end{minipage}%}

\vspace{5mm}

\begin{minipage}[b]{0.45\linewidth} \begin{prooftree}
    \AxiomC{$\Gamma_1 \vdash p \widesim{i} e$}
    \AxiomC{$\Gamma_2 \vdash f \proctr{i|j|\cdots|k}{} \phi$}
    \RightLabel{(应用)}
    \BinaryInfC{$\Gamma_1 \cup \Gamma_2 \vdash
    f\ p \proctr{j|\cdots|k}{} \phi\ e$}
\end{prooftree}\end{minipage}
\begin{minipage}[b]{0.5\linewidth} \begin{prooftree}
    \AxiomC{$\Gamma,\ p \widesim{i} e \vdash 
    f \proctr{j|\cdots|k}{} \phi$}
    \RightLabel{(抽象)}
    \UnaryInfC{$\Gamma \vdash \lambda p.\ f \proctr{i|j|\cdots|k}{}
    \lambda e.\ \phi$}
\end{prooftree}\end{minipage}

\vspace{5mm}

\noindent 
形式系统朴素 Noesis 就是形式语言 $L_\mathrm{N}$ 和规则 $\vdash$ 
构成的形式系统。
\end{defin}

\section{HOL 逻辑上的 Noesis 系统 \noesishol}

形式系统 \noesishol 构建在 HOL 逻辑的基础上,使用 HOL 形式语言作为
\noesishol 的形式语言,而构建兼容 HOL 逻辑的 Noesis 形式系统的规则。
或者说,形式系统 \noesishol 是朴素 Noesis 形式系统在 HOL 逻辑上的实现
与适当扩广。这种实现通过在 HOL 逻辑上定义表示理解的类型与表示 Noesis
对应及 Noesis 同构的谓词完成。于是可以允许理解的运算,作为对朴素
Noesis 系统的主要扩广,例如那些产生理解的 HOL 函数,
诸如理解的并、序列的理解等。

接下来将逐步在 HOL 逻辑上定义现象与理解,而本体是
任意 HOL 逻辑上的数学对象,甚至包括定义在 HOL 逻辑上的现象与理解本身。

\begin{defin}[现象类型] \label{Def.phenomenon}
HOL 逻辑上的类型 phenomenon 表示 \noesishol 的现象,
根据不同的具体实现可以不同方式具体定义,不同的具体定义不影响本章理论。

唯独特别的,phenomenon 类型中需要包含一个特别的零元素$\mathrm{p}_0$,
表示计算客体中不需要被任何实在的现象表示而自然地于虚无中存在的,
例如0个二进制位或者C语言的void类型。
\end{defin}

\begin{defin}[到$\alpha$的理解] \label{Def.itp} 
类型 $\itp{\alpha}$ 是现象到类型为 $\alpha$ 的本体的理解,
包含可以构建现象与本体间联系的映射。
    \[ \begin{split}
        \itp{\alpha} \Coloneqq \mathbf{Noesis}\ &(\alpha \rightarrow \phenomenon)
        \ (\phenomenon \rightarrow \alpha)\\
        &(\alpha\ \mathrm{set})\ (\phenomenon\ \mathrm{set})
    \end{split} \]
    \begin{align*}
        \mathbf{NOE\_LIGHT}\ (\mathbf{Noesis}\ l\ tr\ s\ s_p) & = l&
        \mathbf{NOE\_TRANSCEND}\ (\mathbf{Noesis}\ l\ tr\ s\ s_p) & = tr&\\
        \mathbf{NOE\_SET}\ (\mathbf{Noesis}\ l\ tr\ s\ s_p) & = s&
        \mathbf{NOE\_PSET}\ (\mathbf{Noesis}\ l\ tr\ s\ s_p) & = s_p&
    \end{align*}
\end{defin}

$\itp{\alpha}$ 表示一个将现象对应为某种 $\alpha$ 类型的本体的理解。
例如现象到布尔值间的超越映射就要届由到 bool 类型的理解来完成。
$\mathbf{Noesis}\ l\ tr\ s\ s_p$ 构建了这样一个理解,$l$是投影函数,确定本体到
现象的映射;$tr$ 是超越函数,确定现象到本体的映射;$s$ 是本体集合,
有 $s \subseteq \mathrm{dom}\ l \land s \subseteq \mathrm{rng}\ tr$;
$s_p$ 是现象集合,有 $s_p \subseteq \mathrm{dom}\ tr \land s_p \subseteq \mathrm{rng}\ l$
,一个理解即是上述的四元组。


\begin{notation}[理解的相关属性的简写记号]
\begin{align*}
    \mathbf{Li}_i\ & \coloneqq \mathbf{NOE\_LIGHT}\ i
    & \mathbf{Tr}_i\ & \coloneqq \mathbf{NOE\_TRANSCEND}\ i \\
    \mathbf{Se}_i\ & \coloneqq \mathbf{NOE\_SET}\ i
    & \mathbf{Sp}_i\ & \coloneqq \mathbf{NOE\_PSET}\ i
\end{align*}
\end{notation}

\begin{defin}[到$\alpha$的有效理解] \label{Def.Vi}
\[ \begin{split}
    \mathbf{VALID\_NOE}\ i \coloneqq\ &
    (\forall e.\ e \in \mathbf{Se}_i \Rightarrow \mathbf{Li}_i\ e \in
    \mathbf{Sp}_i \land (\mathbf{Tr}_i(\mathbf{Li}_i\ e) = e)) \\
    & \land\ (\forall p.\ p \in \mathbf{Sp}_i \Rightarrow \mathbf{Tr}_i\ p \in 
    \mathbf{Se}_i)
\end{split} \]
\end{defin}

\begin{notation}[有效理解的简写]
    \[ \mathbf{V}_i \coloneqq \mathbf{VALID\_NOE}\ i \]
\end{notation}

即 $\mathbf{V}_i$ 意味着 $\mathbf{Li}_i$ 是 $\mathbf{Se}_i$ 到 $\mathbf{Sp}_i$ 的单射,
$\mathbf{Tr}_i$ 是 $\mathbf{Sp}_i$ 到 $\mathbf{Se}_i$ 的满射,
$\mathbf{Tr}_i$ 是 $\mathbf{Li}_i$ 的逆函数。

\begin{example}[到bool的有效理解的样例]
\[ \begin{split}
    \mathbb{B}_\mathrm{I} \coloneqq\ & \mathbf{Noesis}\ (\lambda e.\ \mathbf{if}\ e\ 
    \mathbf{then}\ 1\ \mathbf{else}\ 0) \\
& (\lambda p.\ \mathbf{if}\ p=0\ \mathbf{then}\ \mathrm{F}\ \mathbf{else}\ \mathrm{T})\quad
   \{\mathrm{T},\ \mathrm{F}\}\quad \mathbb{N}
\end{split} \]
\end{example}

\begin{defin}[ Noesis 对应关系] \label{Def.TR}
定义三元关系 $ p \widesim{i} e$ 表示现象与本体在某种理解下的的对应关系。
\[ p \widesim{i} e \coloneqq \mathbf{V}_i \land p \in \mathbf{Sp}_i \land
    (\mathbf{Tr}_i\ p = e)\]
\end{defin}

某个成立的 Noesis 对应关系可以用 HOL 逻辑证明,或者以形式系统的角度去说,
可以用HOL逻辑的一系列规则得到某个成立的Noesis 对应。

\begin{prooftree}
    \AxiomC{$\cdots$} 
    \RightLabel{对于某个成立的 $p \widesim{i} e$\quad\quad( Noesis 对应的证明律)} \doubleLine
    \UnaryInfC{$\vdash p \widesim{i} e$}
\end{prooftree}

由此构建了程序的语义感知(Semantic Reasoning)的基础,即在理解方式$i$下可以将
计算客体$p$解释成$e$,而$e$即$p$在理解方式$i$下的语义。

一个现象对于一个理解只有确定的一个本体,但一个现象可以同时具有多种理解,
而现象上的一次计算可以同时带有多个含义。例如现象1可以在布尔理解下对应真,
也可以在自然数理解下对应自然数1,也可以是某种有序集的理解下的第一个元素。

%\begin{theo}[Noesis 对应的一些性质]
%\begin{align*}
%  \forall p\ i\ e.\ p \widesim{i} e \Rightarrow
%    e \in \mathbf{Se}_i
%\end{align*}
%\end{theo}
%\begin{proof} 由定义 \ref{Def.Vi} 直接得到。
%\end{proof}

\begin{defin}[理解的合并] \label{Def.I*}
一个到 $\alpha$ 的理解$i_a$与到 $\beta$ 的理解$i_b$可以合并成
    到二元组$(\alpha,\beta)$的理解 $i_a \cdot i_b$
\[
    \forall l_a\ l_b\ a\ b.\ (l_a\ a = l_b\ b) \Rightarrow
    (\dot{\mathbf{l}}\ l_a\ l_b\ a\ b = l_a\ a )
\]
\begin{equation} \begin{split}
    (\mathbf{Noesis}\ & l_a\ tr_a\ s_a\ sp_a)\ \cdot\ 
    (\mathbf{Noesis}\ l_b\ tr_b\ s_b\ sp_b) \coloneqq \\
    & \mathbf{Noesis}\ (\dot{\mathbf{l}}\ l_a\ l_b)\ 
        (\lambda p.\ (tr_a\ p,\ tr_b\ p))\ \\
    & \quad\quad\quad (s_a \times s_b \cap (\lambda (a,\ b).\ l_a\ a = l_b\ b))\ 
    (sp_a \cap sp_b)
\end{split} \end{equation}
\end{defin}

这样合并的意义在于,多个基于不同理解的Noesis 对应,可以将这些不同的理解合并而表示为一个Noesis 对应。

\begin{theo}[理解合并的有效性] \label{T.V.IMerge}
\begin{equation}
    \mathbf{V}_a \land \mathbf{V}_b \Rightarrow \mathbf{V}_{a\cdot b}
\end{equation}
\end{theo}
\begin{proof}
\begin{align*}
    \mathbf{V}_{a\cdot b} =\ & (\forall e_a\ e_b.\ e_a \in \mathbf{Se}_a \ \ \land \ \ 
    e_b \in \mathbf{Se}_b \ \ \land \ \ 
    (\mathbf{Li}_a\ e_a = \mathbf{Li}_b\ e_b) \ \ \Rightarrow \\
    & \quad\quad\quad \mathbf{Li}_a\ e_a \in \mathbf{Sp}_a \cap \mathbf{Sp}_b \ \ \land
    \ \ (\mathbf{Tr}_a(\mathbf{Li}_a\ e_a) = e_a) \ \ \land \ \ 
    (\mathbf{Tr}_b(\mathbf{Li}_b\ e_b) = e_b)) \\
    & \land\ \ (\forall p.\ p \in \mathbf{Sp}_a \ \ \land \ \ p \in \mathbf{Sp}_b
    \Rightarrow \mathbf{Tr}_a\ p \in \mathbf{Se}_a \ \ \land \ \ \mathbf{Tr}_b\ p \in 
      \mathbf{Se}_b ) \\
    \mathbf{V}_a =\ & (\forall e_a.\ e_a \in \mathbf{Se}_a \Rightarrow
      \mathbf{Li}_a\ e_a \in \mathbf{Sp}_a \ \ \land \ \ 
      (\mathbf{Tr}_a(\mathbf{Li}_a\ e_a) = e_a)) \\
    & \land\ \ (\forall p.\ p \in \mathbf{Sp}_a \Rightarrow \mathbf{Tr}_a\ p \in 
      \mathbf{Se}_a) \\
\mathbf{V}_b =\ & (\forall e_b.\ e_b \in \mathbf{Se}_b \Rightarrow
      \mathbf{Li}_b\ e_b \in \mathbf{Sp}_b \ \ \land \ \ 
      (\mathbf{Tr}_b(\mathbf{Li}_b\ e_b) = e_b)) \\
    & \land\ \ (\forall p.\ p \in \mathbf{Sp}_b \Rightarrow \mathbf{Tr}_b\ p \in
      \mathbf{Se}_b) \\
\mathbf{V}_a,\ & \mathbf{V}_b,\ e_a \in \mathbf{Se}_a,\ e_b \in \mathbf{Se}_b,\ 
    \mathbf{Li}_a\ e_a = \mathbf{Li}_b\ e_b \ \vdash \ 
    \mathbf{Li}_a\ e_a \in \mathbf{Sp}_a \cap \mathbf{Sp}_b \\
    & \quad\quad\quad\quad \land (\mathbf{Tr}_a(\mathbf{Li}_a\ e_a) = e_a) \land
    (\mathbf{Tr}_b(\mathbf{Li}_b\ e_b) = e_b) \\
\mathbf{V}_a,\ & \mathbf{V}_b,\ p \in \mathbf{Sp}_a,\ p \in \mathbf{Sp}_b\ \vdash\ 
    \mathbf{Tr}_a\ p \in \mathbf{Se}_a\ \ \land \ \ \mathbf{Tr}_b\ p \in \mathbf{Se}_b \\
\mathbf{V}_a,\ &\mathbf{V}_b \vdash \mathbf{V}_{a \cdot b}
\end{align*}
\end{proof}

\begin{theo}[理解合并对于 Noesis 对应的同构]
\begin{equation}
    \forall i_a\ i_b\ p\ e_a\ e_b.\ p \widesim{i_a} e_a \land p \widesim{i_b} e_b = 
    p \widesim[2]{i_a \cdot i_b} (e_a,\ e_b) 
\end{equation}
\end{theo}

\begin{proof}
\[ p \widesim[2]{ia \cdot ib} (e_a,\ e_b) =\ \mathbf{V}_{ia\cdot ib} \land p \in 
    (\mathbf{Sp}_{ia} \cap \mathbf{Sp}_{ib}) \land 
    (\mathbf{Tr}_{ia}\ p = e_a) \land (\mathbf{Tr}_{ib}\ p = e_b) \]
结合定义\ref{Def.TR}以及理解合并的有效性定理\ref{T.V.IMerge}
\[p \widesim[2]{ia \cdot ib} (e_a,\ e_b) = p \widesim{i_a} e_a \land p \widesim{i_b} e_b \]
\end{proof}

由合并的理解的 Noesis 对应关系就可以提取出其中各个理解的 Noesis 对应。

\begin{gather*}
    p \widesim{i_1 \cdot i_2} (e_1,\ e_2) \Rightarrow p \widesim{i_1} e_1 \\
    p \widesim{i_1 \cdot i_2} (e_1,\ e_2) \Rightarrow p \widesim{i_2} e_2
\end{gather*}

\begin{minipage}[b]{0.45\linewidth}
\begin{prooftree}
\AxiomC{$\Gamma \vdash p \widesim{i_1 \cdot i_2} (e_1,\ e_2)$}
    \RightLabel{(合并分解律1)}
\UnaryInfC{$\Gamma \vdash p \widesim{i_1} e_1$}
\end{prooftree}
\end{minipage}%
\begin{minipage}[b]{0.4\linewidth}
\begin{prooftree}
\AxiomC{$\Gamma \vdash p \widesim{i_1 \cdot i_2} (e_1,\ e_2)$}
    \RightLabel{(合并分解律2)}
\UnaryInfC{$\Gamma \vdash p \widesim{i_2} e_2$}
\end{prooftree}
\end{minipage}

\begin{prooftree}
\AxiomC{$\Gamma_1 \vdash p \widesim{i_1} e_1$}
\AxiomC{$\Gamma_2 \vdash p \widesim{i_2} e_2$}
    \RightLabel{(合并律)}
\BinaryInfC{$\Gamma_1 \cup \Gamma_2 \vdash p \widesim{i_1 \cdot i_2} (e_1,\ e_2)$}
\end{prooftree}

\begin{theo}[理解合并在 Noesis 对应上的交换律同构与结合律同构]
\begin{gather*}
    \forall e_a\ e_b\ i_a\ i_b\ p.\ p \widesim[2]{i_a \cdot i_b} (e_a,\ e_b) =
        p \widesim[2]{i_b \cdot i_a} (e_b,\ e_a) \\
    \forall e_a\ e_b\ e_c\ i_a\ i_b\ i_c\ p.\ p \widesim[4]{(i_a \cdot i_b) \cdot i_c}
    ((e_a,\ e_b),\ e_c) = p \widesim[4]{i_a \cdot (i_b \cdot i_c)} (e_a,(e_b,\ e_c))
\end{gather*}
\end{theo}

\begin{proof} 显然,略
\end{proof}


\begin{defin}[过程的 Noesis 对应同构] \label{def.proctr}
定义4元符号 $f_p \proctr{i}{cond} f_e$ 表示有条件的值的 Noesis 对应。
定义5元符号 $f_p \proctr{i|j}{cond} f_e$ 表示1个参数的且以
$cond$为条件的过程的 Noesis 对应,$n+3$ 元符号
$f_p \proctr{i_1 | \cdots | i_n}{cond} f_e$ 表示 $n$ 个参数的过程的且条件为
$cond$的 Noesis 对应。

\begin{equation} \begin{split}
    p \proctr{i}{cond} e \coloneqq &\ (cond \Rightarrow p \widesim{i} e) \\
    (f_p:\mathrm{phenomenon}\rightarrow\mathrm{phenomenon})\ & \proctr{i|j}{cond}
    (f_e:\alpha \rightarrow \beta) \coloneqq \\
    (\forall a_p\ a_e.\quad & a_p \widesim{i} a_e \Rightarrow cond\ a_e \Rightarrow
    f_p\ a_p \widesim{j} f_e\ a_e) \\
    (f_p:\mathrm{phenomenon}\rightarrow \cdots \rightarrow \mathrm{phenomenon})\ & 
    \proctr{i | \cdots | l}{cond}
    (f_e:\alpha \rightarrow \cdots \rightarrow \gamma \rightarrow \delta) \coloneqq \\
    (\forall a_p\ a_e \cdots c_p\ c_e.\quad a_p \widesim{i} a_e \Rightarrow &
    \ b_p \widesim{j} b_e \Rightarrow \cdots \Rightarrow cond\ a_e\ \cdots\ c_e 
    \Rightarrow  \\
    & f_p\ a_p\ \cdots\ c_p \widesim{l} f_e\ a_e\ \cdots\ c_e)
\end{split} \end{equation}

    $cond$ 表示过程正确执行的条件。

    对于$\proctr{i | \cdots | k | l}{cond}$ 若理解
    $i, \cdots, k$ 的类型分别为 $\itp{\alpha},\ \cdots,\ \itp{\gamma}$,
    那么 $cond$ 的类型为 $\alpha \rightarrow \cdots \rightarrow \gamma \rightarrow 
    \mathrm{bool}$,即 $cond$ 是在本体层面上描述执行条件。

    过程的 Noesis 对应同构构成了一种契约,
    在输入参数可以被约定的理解$i,\cdots,k$解释,
    且解释后的意义也即本体$a_e,\cdots,c_e$满足给定的条件,
    那么此过程可以正确执行并返回以约定的理解$l$的解释下对应本体$f\ a_e\ \cdots\ c_e$ 的输出

\end{defin}

上述定义具有如下的递归形式。

\begin{equation} \label{f.proctr_rec} \begin{split}
    f_p \proctr{i|j|\cdots|l}{cond} f_e = (\forall a_p\ a_e.\quad
      a_p \widesim{i} a_e \Rightarrow f_p\ a_p \proctr{j|\cdots|l}{cond\ a_e} f_e\ a_e)
\end{split} \end{equation}

\begin{theo}[\noesishol 上的调用律、抽象律、一阶同构引入律、一阶同构削除律]
在 Noesis 对应的定义 \ref{Def.TR} 与 Noesis 同构的定义 \ref{def.proctr} 
下,Noesis 系统的调用律、抽象律、一阶同构引入律、一阶同构削除律,以
HOL 上的定理的形式出现。

\hspace{-16mm}
\begin{minipage}[b]{0.55\linewidth} \begin{prooftree}
    \AxiomC{$\Gamma_1 \vdash f_p \proctr{i|j|\cdots|l}{cond} f_e$}
    \AxiomC{$\Gamma_2 \vdash a_p \widesim{i} a_e$}
    \RightLabel{(调用)}
    \BinaryInfC{$\Gamma_1 \cup \Gamma_2 \vdash f_p\ a_p \proctr{j|\cdots|l}{cond\ a_e}
      f_e\ a_e$}
\end{prooftree}\end{minipage}%
\begin{minipage}[b]{0.45\linewidth} \begin{prooftree}
    \AxiomC{$\Gamma_1,\ a_p \widesim{i} a_e \vdash b_p \proctr{j|\cdots|l}{cond} b_e$}
    \RightLabel{(抽象)}
    \UnaryInfC{$\Gamma_1 \vdash (\lambda a_p.\ b_p)
      \proctr{i|j|\cdots|l}{\lambda a_e.\ cond\ a_e} (\lambda a_e.\ b_e)$}
\end{prooftree}\end{minipage}

\hspace{-10mm}
\begin{minipage}[b]{0.45\linewidth} \begin{prooftree}
    \AxiomC{$\Gamma, cond \vdash p \widesim{i} e$}
    \RightLabel{(一阶同构引入)}
    \UnaryInfC{$\Gamma \vdash p \proctr{i}{cond} e$}
\end{prooftree}\end{minipage}%
\begin{minipage}[b]{0.4\linewidth} \begin{prooftree}
    \AxiomC{$\Gamma_1 \vdash p \proctr{i}{cond} e$}
    \AxiomC{$\Gamma_2 \vdash cond$}
    \RightLabel{(一阶同构削除)}
    \BinaryInfC{$\Gamma_1 \cup \Gamma_2 \vdash p \widesim{i} e$}
\end{prooftree}\end{minipage}
\end{theo}

\begin{proof} 由定义 \ref{def.proctr},递归公式 \ref{f.proctr_rec},应用
    \ref{Sec.HOL} 节中介绍的HOL逻辑上的 MATCH\_MP 律、DISCH 律。
\end{proof}

作为一种定理,可以自然地被证明,或者被演绎得出。

%又也许理解可以尝试地作为这样一种类型,但问题在于,任何一个现象都可以被应用于大量的理解
%——只要现象在理解的现象集内就行,若尝试将理解作为类型,就意味着每一个现象都可以属于大量的类型,
%这显然是荒谬的。

%本文对所有将 Noesis 对应描述成某种类型系统的尝试都失败了。根本在于,类型是一种对值的分类,
%以良好地提取维护值的部分性质;
%但本文的思路, Noesis 对应根本不尝试对值分类,而是尝试显示地确定理解方式而找到值对应的以某种
%数理对象表示的意义,进而提取出值在该理解下的全部性质。
%这与与指称语义(Denotational semantics)非常相似。

%指称语义尝试建立起程序与数学函数的对应关系,跟 Noesis 对应非常相似。
%但$\amlh$的优势在于程序的抽象,基于$\lamst$使得程序本身亦是抽象的结构。
%这导致一些重要的区别,指称语义在递归定义的函数与数据结构上遇到的困难带领指称语义
%走向了域理论,而对于$\amlh$,不使用域理论而
%是在???节将看到递归与归纳(Induction)间的内在联系。
%指称语义与 Noesis 对应尝试实现类似的目标,但解决思路与方式完全不同。


\section{常量、基元指令、理解的引入方法} \label{Sec.op-const-noesis.def}

常量、基元指令、理解均有两种引入方法:第一种是给出具体的计算表达式来定义,
即基元指令作为一个值的函数,具体地给出值的计算过程;常量作为一个值,具体地给出值的数值;
理解作为定义 \ref{Def.itp} 中的4元组,给出各个映射与集合具体的表达。
例如32位加法 \texttt{i32.add}的定义
\[ \texttt{i32.add}\ a\ b = a + b \Mod 2^{32} \]
另一种定义方法分别是,常量定理,基元指令虚构,理解虚构。
三者均不需要给出具体的表达,只是由所期望的抽象语义,证明必定存在这样的常量、基元指令、
理解满足所期望的抽象语义,进而引入。

\begin{theo}[常量定理] \label{T.Vconst}
  常量的虚构可以通过理解 $i$ 的 $\mathbf{Li}_i$ 完成
\[ \forall i\ e.\ \mathbf{V}_i\ \land\ e \in \mathbf{Se}_i
\Rightarrow \mathbf{Li}_i\ e \widesim{i} e \]
给定一个合法的理解 $i$ 与理念中的抽象语义 $e$,$\mathbf{Li}_i\ e$ 是
Noesis 对应到抽象语义 $e$ 的{\phew}。
\vspace{-4mm}
\begin{prooftree}
\AxiomC{$\Gamma_1 \vdash \mathbf{V}_i$}
\AxiomC{$\Gamma_2 \vdash e \in \mathbf{Se}_i$}
\RightLabel{(常量律)} \BinaryInfC{$\Gamma_1 \cup \Gamma_2 \vdash
\mathbf{Li}_i\ e \widesim{i} e$}
\end{prooftree}

\begin{proof} 
Noesis 对应的定义 \ref{Def.TR}
\[ p \widesim{i} e = \mathbf{V}_i \land p \in \mathbf{Sp}_i \land (\mathbf{Tr}_i\ e = p) \]
有效理解的定义 \ref{Def.Vi},有
\[ \mathbf{V}_i \Rightarrow (\forall e.\ e \in \mathbf{Se}_i \Rightarrow \mathbf{Li}_i\ e \in
    \mathbf{Sp}_i \land (\mathbf{Tr}_i(\mathbf{Li}_i\ e) = e))\]
    于是
\[\mathbf{V}_i \Rightarrow (\forall e.\ e \in \mathbf{Se}_i \Rightarrow \mathbf{Li}_i\ e \widesim{i} e)\]
命题得证。

\end{proof}
\end{theo}

%基元指令的第一种基于{\phew}的定义方法 \ref{Def.phenomenon},
%具体地给出{\phew}上函数的定义,再由 Noesis 对应的定义 \ref{Def.TR} 证明此函数满足某个或某些
% Noesis 对应同构,
给出具体定义的{\phew}函数可以在不同的理解下具有多个 Noesis 对应同构。
%,于是由此可以定义具有多个抽象语义或者说是多个意义的函数。
例如在{\phew}上完整定义的加法函数,既可以 Noesis 同构到
自然数区间的加法,也可以是有限域的加法。

虚构基元指令方法则不需要明确定义,而是
%第二种是根据预先期望的 Noesis 对应同构,不给出明确的定义,而是
构造一个存在性命题,证明必定存在一个{\phew}函数满足此同构,
进而由此引入。%这一方式叫做 {\it 虚构引入}。
可以证明对于任意的 Noesis 对应同构,只要满足一些很宽松的条件,
就一定存在一个{\phew}函数满足此对应同构。
进而虚构的基元指令不需要具体地描述计算过程,可以根据意义直接引入,
%也根本不需要关注计算内容,
且只要得到了 Noesis 对应同构定理,就可以直接由组合律去应用。
%这点得益于基元指令 Noesis 同构的公理性,程序是不需要关注基元指令如何实现
%的,程序只要组合基元指令。

\begin{theo}[基元虚构定理]
给定所期望的抽象语义$f_e$,参数的理解$i\ \cdots\ k$,返回值的理解$l$,过程执行的条件$cond$,
只要 $l$ 有效,且 $cond$ 足以限定返回值的抽象语义 $f_e\ a_e\ \cdots\ c_e$到其理解$l$的抽象语义集
$\mathbf{Se}_l$内,则必定存在一个{\phew}上的过程 $f_p$,且满足期望的
Noesis 对应同构。
\[ \begin{split} \forall f_e\ i\ \cdots\ k\ l\ cond.\ 
 \mathbf{V}_l\ \land\ &\\
    (\forall a_e\ \cdots\ c_e.\ \mathbf{V}_i\ & \land a_e \in \mathbf{Se}_i \land \cdots
\land \mathbf{V}_k \land c_e \in \mathbf{Se}_k \land cond\ a_e\ \cdots\ c_e\ \Rightarrow\\
    & f\ a_e\ \cdots\ c_e \in \mathbf{Se}_l) \Rightarrow\\
    & \exists f_p.\ f_p \proctr{i|\cdots|k|l}{cond} f_e
\end{split} \]
\begin{proof} 此定理在没有参数时的特例,常量虚构定理
\[ \forall i\ e.\ \mathbf{V}_i \Rightarrow e \in \mathbf{Se}_i \Rightarrow
  \exists p.\ p \widesim{i} e \]
由常量定理 \ref{T.Vconst} 直接得到。

令 $x_e = f_e\ a_e\ \cdots\ c_e$,

$P_{cond} = (\forall a_e\ \cdots\ c_e.\ \mathbf{V}_i\ \land a_e \in 
\mathbf{Se}_i \land \cdots
\land \mathbf{V}_k \land c_e \in \mathbf{Se}_k \land cond\ a_e\ \cdots\ c_e\ \Rightarrow
    f\ a_e\ \cdots\ c_e \in \mathbf{Se}_l)$,

$\Gamma_\sim = \{a_p \widesim{i} a_e,\ \cdots\ ,\ c_p \widesim{k} c_e,\ cond\ x_e,\ 
\mathbf{V}_l,\ P_{cond} \}$

\begin{prooftree}
\AxiomC{$\Gamma_\sim \vdash P_{cond}$}
\AxiomC{$a_p \widesim{i} a_e \vdash \mathbf{V}_i \land a_e \in \mathbf{Se}_i$}
\AxiomC{$\cdots$}
\AxiomC{$c_p \widesim{i} c_e \vdash \mathbf{V}_i \land c_e \in \mathbf{Se}_i$}
\QuaternaryInfC{$\Gamma_\sim \vdash x_e \in \mathbf{Se}_l \land \mathbf{V}_l$}
\RightLabel{(常量虚构定理)}
\UnaryInfC{$\Gamma_\sim \vdash \exists p.\ p \widesim{i} x_e$}
\UnaryInfC{$\Gamma_\sim \vdash \exists f_p.\ f_p\ a_p\ \cdots\ c_p \widesim{i} f_e\ a_e\ \cdots c_e$}
\UnaryInfC{$\mathbf{V}_l,\ P_{cond} \vdash \exists f_p.\ 
    f_p \proctr{i|\cdots|k|l}{cond} f_e$}
\UnaryInfC{$\forall f_e\ i\ \cdots\ k\ l\ cond.\ 
\mathbf{V}_l\ \land\ P_{cond} \Rightarrow 
    \exists f_p.\ f_p \proctr{i|\cdots|k|l}{cond} f_e$}
\end{prooftree}
\end{proof}
\end{theo}

虚构基元的条件,是非常宽松的,下面的定理说明这一点。

\begin{theo}[Noesis 对应同构的条件的充分性]  \label{T.ptr.enough}
\[ \begin{split}
\forall f_p\ i\ &\cdots\ k\ l\ cond\ f_e.\ f_p \proctr{i|\cdots|k|l}{cond} f_e 
\Rightarrow\\ & \forall a_e\cdots c_e.\ \mathbf{V}_i \land a_e \in \mathbf{Se}_i
    \land \cdots \land \mathbf{V}_k \land c_e \in \mathbf{Se}_k \land
    cond\ a_e\ \cdots\ c_e \Rightarrow \\
    & \quad\quad\quad\quad \mathbf{V}_l \ \land\  f_e\ a_e\ \cdots\ c_e \in \mathbf{Se}_l
\end{split} \]
\end{theo}
\begin{proof}
\[\begin{split} f_p \proctr{i|\cdots|k|l}{cond} f_e = (\forall &a_e\ a_p\ \cdots c_e\ c_p.\
    a_p \widesim{i} a_e \ \Rightarrow \cdots \Rightarrow c_p \widesim{i} c_e \Rightarrow\\
    & cond\ a_e\cdots\ c_e \Rightarrow f_p\ a_p\ \cdots\ c_p \widesim{l} f_e\ a_e\ \cdots\ 
c_e) \end{split} \tag{1} \]
令 $a_p = \mathbf{Li}_i\ a_e,\ \cdots,\ c_p = \mathbf{Li}_k\ c_e$,
由定义\ref{Def.TR}在前提 $\mathbf{V}_i \land a_e \in \mathbf{Se}_i
    \land \cdots \land \mathbf{V}_k \land c_e \in \mathbf{Se}_k$ 下有
    \[ a_p \widesim{i} a_e \land \cdots \land c_p \widesim{k} c_e \]
带入(1)并结合前提$cond\ a_e\cdots\ c_e$ 得到
    \[ f_p\ a_p\ \cdots\ c_p \widesim{l} f_e\ a_e\ \cdots\ c_e \]
由定义\ref{Def.TR}
    \[ \mathbf{V}_l \ \land\ f_e\ a_e\ \cdots\ c_e \in \mathbf{Se}_l \]
\end{proof}

最后理解也可以被类似地虚构地引入,而完全不给出任何具体的细节。
这一点是通过同构实现的。

\begin{defin}[单射的定义] \label{Def.INJ}
\[
    \mathbf{INJ}\ f\ S\ T \coloneqq (\forall s.\ s \in S \Rightarrow f\ s \in T) \land
    (\forall s_1\ s_2.\ s_1 \in S \land s_2 \in S \land (f\ s_1 = f\ s_2) \Rightarrow
    s_1 = s_2)
\]
\end{defin}
\begin{lemma}[单射存在逆函数] \label{L.inji}
\[ \forall f\ S\ T.\ \mathbf{INJ}\ f\ S\ T \Rightarrow \exists g.\ (\forall t.\ t \in T
    \Rightarrow g\ t \in S) \ \land\ (\forall s.\ s \in S \Rightarrow
    g(f(s)) = s) \]
\end{lemma}
\begin{defin}[势的偏序关系的定义] \label{Def.Card}
    \[ \abs{A} \leq \abs{B} \coloneqq (\exists f.\ \mathbf{INJ}\ f\ A\ B) \]
\end{defin}

\begin{theo}[理解虚构定理]
\[ \forall S_e. \abs{S_e} \leq \abs{\mathrm{phenomenon}} \Rightarrow
    \exists i.\ \mathbf{V}_i\ \land\ (\mathbf{Se}_i = S_e) \]
\end{theo}
\begin{proof}
在假设 $\abs{S_e} \leq \abs{\mathrm{phenomenon}} $ 下,由定义 \ref{Def.Card} 得到
由$S_e$到{\phew}的单射$f$,再由引理$\ref{L.inji}$得到逆函数$g$,记单射$f$的值域为
$S_p = \{f(a)\mbar a \in S_e \}$,有:
\[ \forall e.\ e \in S_e \Rightarrow f\ e \in S_p \land g(f\ e) = e  \tag{1} \]
\[ \forall p.\ p \in S_p \Rightarrow g\ e \in S_e \tag{2} \]
令 $i = \mathbf{Interpretation}\ f\ g\ S_e\ S_p$ 现只需证明 $\mathbf{V}_i$ 即可。
\[ \mathbf{V}_i = (\forall e.\ e \in S_e \Rightarrow f\ e \in S_p \land
    f(g(e)) = e)\ \land\ (\forall p.\ p \in S_p \Rightarrow g\ p \in S_e) \]
由(1),(2) 这是显然的。
\end{proof}

一般而言$\amlh$的实现中{\phew}集同构于自然数集的一阶无穷,这就给$S_e$的选择带来很大的空间。
事实上任何计算机实际可触及的意义,或者说计算机上可表示的概念,一定是有限的,因为计算机的
内存是有限的,于是$S_e$是有限的;即便抛开内存的限制,需要去表示的概念也往往是一阶无穷。
这些一阶无穷集包括一阶无穷集上的列表、树结构、映射表等多种数据结构。

借助理解虚构定理,可以直接引入到期望本质集上的理解。尽管不给出理解的具体映射方式意味着
不可能直接证明某个{\phew}在该理解下的 Noesis 对应,但依旧可以通过虚构基元指令
与虚构常量的方式引入关于此虚构理解的基元指令。

至此,抽象机 \amlh 论述完成。
%它基于 \noesishol 系统,主要加入了
%基元指令与常量的引入方法。基元指令、常量、理解均可以用两种方式引入。
%
%第一种方法给出具体的包括完整的对{\phew}的计算的定义,根据定义证明其具有
% Noesis 同构、Noesis 对应、或者是合法的理解。一个{\phew}或{\phew}过程可能
% 在多种理解下有多种 Noesis 对应或同构,即有多个抽象语义对应。
%
%对于复杂的结构与过程第一种方法可能是困难的,
%第二种方法使用虚构基元指令、
%虚构常量、虚构理解,在给定的期望的 Noesis 同构、Noesis 对应、理解的
%抽象语义集下,证明及其简单与基本的几个条件,就可以虚构得到这些定理。
%虚构得到的基元指令、常量、理解只能拥有其虚构中给出的确定的一种
%Noesis 同构、Noesis 对应与抽象语义集,虽然不能有多重对应,但非常简单易用。
%
%虚构的基元指令、常量、理解不需要给出具体的实现,因为它们是作为公理式的。
%虚构的基元指令作为指令集中的指令,是由执行环境直接执行的,不需要实现的。
%

