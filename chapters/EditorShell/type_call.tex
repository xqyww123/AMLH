\subsection{类型的调用}

现在讨论诸如$\lamst$以及$\lambda2$中组合律的类型调用

\hfill

\begin{minipage}[b]{0.45\linewidth}
\begin{prooftree}
\AxiomC{$\Gamma \vdash M : \sigma \leftarrow \tau$}
\AxiomC{$\Gamma \vdash N : \sigma$} \RightLabel{(组合律)}
\BinaryInfC{$\Gamma \vdash M N : \tau$}
\end{prooftree}
\end{minipage}\begin{minipage}[b]{0.5\linewidth}
\begin{prooftree}
\AxiomC{$\Gamma_1 \vdash M : \forall \alpha\ \sigma$}
\AxiomC{$\Gamma_2 \vdash \tau : *$}
\RightLabel{(全称组合律)}
\BinaryInfC{$\Gamma_1,\ \Gamma_2 \vdash M\ \tau : \sigma$}
\end{prooftree}\end{minipage}

\hfill

自然地期望在 $\ES$ 中也应有类似的组合运算。
原本的 λ2 演算要求显示地实例化组合双方的类型,直到实例化成不具有全称量化的形式才可以进行
组合,而组合的结果也一定不具有全称量化的。
例如 $\forall x.\ x \rightarrow z \rightarrow x$ 与
$\forall y.\ y$ 进行组合的结果只能是 $z \rightarrow x$ 或 $z \rightarrow y$,
即进行了两次实例化,但其实只要一次实例化就可以的,更好的结果是 $\forall x.\ 
z \rightarrow x$

$\ES$ 对 λ2 主要的改进在于此,这也是几乎 $\ES$ 与 λ2 唯一的不同,$\ES$ 基于
上一节论述的{\it 全称量化的交}提出一种对组合双方的全称量化进行最小实例化以尽可能保持
全称量化而进行组合的运算,称作{\it 全称量化类型的函数调用}。
这种运算只进行必要的实例化而在结果中尽可能地保留了全称量化,在 $\forall x.\ 
x \rightarrow z \rightarrow x$ 与 $\forall y.\ y$ 的例子中,可以得到 
$\forall x.\ z \rightarrow x$。

接下来一步步地正式地数学定义{\it 全称量化类型的函数调用}
(function application of universal quantified type)。

\begin{defin}[函数类型]
$\rightarrow$ 是函数类型的类型构造器。
\begin{align*}
&\underline{\rightarrow} \in \hat{U} - \hat{U}_v&
&\mathrm{a}(\underline{\rightarrow}) = 2&
\end{align*}
诸如以下范式的类型被叫做从 $a$ 到 $b$ 的函数,简称函数。
\[ \underline{(\ \rightarrow\ a\ b\ )} \]
也可以使用记号$a \rightarrow b$表示 $\underline{(\ \rightarrow\ a\ b\ )}$

\noindent 所有的函数构成的集合为 $\PiE^\rightarrow$

\noindent 函数 $\DomF : \PiE^\rightarrow \rightarrow \PiE$
    \[ \DomF(a \rightarrow b) = a \]
函数 $\ImaF : \PiE^\rightarrow \rightarrow \PiE$
    \[ \ImaF(a \rightarrow b) = b \]
\end{defin}

\begin{defin}[普通类型集合的函数调用]
普通类型集合的函数调用是$\powerset(\PiE^\rightarrow) \times \powerset(\PiE)$到
$\powerset(\PiE)$的函数 $(\cdot)$
\[ A \cdot B = \{b \mbar (a \rightarrow b) \in \PiE^\rightarrow\ \land\ a \in \PiE\} \]
\end{defin}

\begin{algorithm}
\caption{全称量化类型的调用 QCall} \label{alg:QCall}
\begin{algorithmic}[1]
\Require 全称量化类型 $q_1,\ q_2 \in \PiAE$
\Ensure $q \in \PiAE$ 满足 $\QI q_1 \cdot \QI q_2 = \QI q$,或者 $\QI q_1 \cdot \QI q_2 = 
\emptyset$ 时输出 {\it 无解} 。
\State $\NC(q_1,\ q_2) \rightarrow (q_1',\ q_2')$ \Comment{调用算法 \ref{alg:NC} NC}
\State $(\QV q_1' \cup \QV q_2',\ \{\DomF \QB q_1' = \QB q_2'\}) \rightarrow M$ \label{l:qc1}
\Comment {构造类型匹配系统 $M \in \TMs$}
\State $\mathrm{SolveM}(M) \rightarrow M'$ \Comment{算法 \ref{alg:SolveM} SolveM}
\If {$M' \neq \MNS$}
\State $\mathrm{QIMI}(M',\ \ImaF \QB q_1') \rightarrow q$ \Comment{算法 \ref{alg:QIMI} QIMI} \label{l:qc2}
\State 输出 $q$
\Else
    \State 输出 {\it 无解}
\EndIf
\TimeComplexity $O(\ \ (\abs{\QV q_1} + \abs{\QV q_2} + 1)\ \ (\TS(\QB q_1) + \TS(
    \QB q_2))\ \ )$
\item[\textbf{说明}] 
这一算法非常类似算法 \ref{alg:QInter} QInter 仅在行 \ref{l:qc1}
,\ref{l:qc2} 有区别,注意 $\DomF$ 和 $\ImaF$ 的调用。
\end{algorithmic}
\end{algorithm}

\begin{lemma} \label{L.QC.I.I}
\[ q_1 \NC q_2 \vdash \QI q_1 \cdot \QI q_2 =\ 
\MI\ (\QV q_1 \cup \QV a_2,\ \{\DomF \QB q_1 = \QB q_2\})\ \ImaF \QB(q_1) \]
    \begin{proof} 效仿定理 \ref{T.TMS.I.I} 的证明。
    \end{proof}
\end{lemma}

\begin{theo}\label{T.QCall}
\[ \forall q_1,\ q_2 \in \PiAE.\ (\QI q_1 \cdot \QI q_2 = \emptyset)\ \lor\ 
(\exists q.\  \QI q_1 \cdot \QI q_2 = \QI q)\]
\begin{proof} 算法 \ref{alg:QCall} QCall 即是寻找这样的 $q$,且其有限停机。
算法 \ref{alg:QCall} QCall 的正确性与时间复杂度的证明
    效仿引理 \ref{L.inter.q},仅是将其中对定理 \ref{T.TMS.I.I} 的应用
    替换成引理 \ref{L.QC.I.I} 即可。
\end{proof}
\end{theo}

\begin{defin}[全称量化类型的调用] \label{D.QCall}
    由定理 \ref{T.QCall} 可以定义全称量化类型的调用,
    从 $\PiAE \times \PiAE$ 到 $\PiAE$ 的函数 $(\cdot)$
\[ \forall q_1,\ q_2 \in \PiAE.\ (\QI q_1 \cdot \QI q_2 \neq \emptyset)
\Rightarrow \QI (q_1 \cdot q_2) = \QI q_1 \cdot \QI q_2\]
    算法 \ref{alg:QCall} QCall 即是计算 $(\cdot)$ 并包含无解情况的算法。
\end{defin}
