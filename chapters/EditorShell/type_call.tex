\subsection{类型的调用}

现在讨论诸如$\lamst$以及$\lambda2$中组合律的类型调用

\hfill

\begin{minipage}[b]{0.45\linewidth}
\begin{prooftree}
\AxiomC{$\Gamma \vdash M : \sigma \leftarrow \tau$}
\AxiomC{$\Gamma \vdash N : \sigma$} \RightLabel{(组合律)}
\BinaryInfC{$\Gamma \vdash M N : \tau$}
\end{prooftree}
\end{minipage}\begin{minipage}[b]{0.5\linewidth}
\begin{prooftree}
\AxiomC{$\Gamma_1 \vdash M : \forall \alpha\ \sigma$}
\AxiomC{$\Gamma_2 \vdash \tau : *$}
\RightLabel{(全称组合律)}
\BinaryInfC{$\Gamma_1,\ \Gamma_2 \vdash M\ \tau : \sigma$}
\end{prooftree}\end{minipage}

\hfill

自然地期望在 $\ES$ 中也应有类似的组合运算。
原本的 λ2 演算要求显示地实例化组合双方的类型,直到实例化成不具有全称量化的形式才可以进行
组合,而组合的结果也一定不具有全称量化的。
例如 $\forall x.\ x \rightarrow z \rightarrow x$ 与
$\forall y.\ y$ 进行组合的结果只能是 $z \rightarrow x$ 或 $z \rightarrow y$,
即进行了两次实例化,但其实只要一次实例化就可以的,更好的结果是 $\forall x.\ 
z \rightarrow x$

$\ES$ 对 λ2 主要的改进在于此,这也是几乎 $\ES$ 与 λ2 唯一的不同,$\ES$ 基于
上一节论述的{\it 全称量化的交}提出一种对组合双方的全称量化进行最小实例化以尽可能保持
全称量化而进行组合的运算,称作{\it 全称量化类型的函数调用}。
这种运算只进行必要的实例化而在结果中尽可能地保留了全称量化,在 $\forall x.\ 
x \rightarrow z \rightarrow x$ 与 $\forall y.\ y$ 的例子中,可以得到 
$\forall x.\ z \rightarrow x$。

接下来一步步地正式地数学定义{\it 全称量化类型的函数调用}
(function application of universal quantified type)。

\begin{defin}[函数类型]
$\rightarrow$ 是函数类型的类型构造器。
\begin{align*}
&\underline{\rightarrow} \in \hat{U} - \hat{U}_v&
&\mathrm{a}(\underline{\rightarrow}) = 2&
\end{align*}
诸如以下范式的类型被叫做从 $a$ 到 $b$ 的函数,简称函数。
\[ \underline{(\ \rightarrow\ a\ b\ )} \]
也可以使用记号$a \rightarrow b$表示 $\underline{(\ \rightarrow\ a\ b\ )}$

\noindent 所有的函数构成的集合为 $\PiE^\rightarrow$
\end{defin}

\begin{defin}[普通类型的函数调用匹配]
普通类型的函数调用匹配是$\PiE^\rightarrow \times \PiE$上的二元关系 $\dotsim$
\[ \forall a\ b.\ (a \rightarrow b) \dotsim a \]
\end{defin}

\begin{defin}[普通类型集合的函数调用]
普通类型集合的函数调用是$\powerset(\PiE^\rightarrow) \times \powerset(\PiE)$到
$\powerset(\PiE)$的函数 $(\cdot)$
\[ A \cdot B = \{b \mbar (a \rightarrow b) \in \PiE^\rightarrow\ \land\ a \in \PiE\} \]
\end{defin}

\begin{theo}
\[ \forall q_1,\ q_2 \in \PiAE.\ (\QI q_1 \cdot \QI q_2 = \emptyset)\ \lor\ 
(\exists q.\  \QI q_1 \cdot \QI q_2 = \QI q)\]
\end{theo}

