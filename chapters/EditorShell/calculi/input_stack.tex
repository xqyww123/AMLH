
这一节介绍预定义的值,这些值构建起编辑环境并提供编辑定理以进行程序开发的
功能。这些讨论是在 \ESm{S} 上进行的。

简单地列出一些基础的算子于表 \ref{tab:ES.basic} 中。唯独需要说明的是
算子 $\I$,作为一个值被前置调用,而以任何参数调用它都返回参数本身,
于是结果是当状态值为 $\I$ 时,任何的输入只导致状态值变为此输入。
这是为何初始状态值为 $\I$,而 $\I$ 又被叫做空的原因。


\begin{table}
	\centering
	\caption{\ES 的基础算子} \label{tab:ES.basic} \begin{estable}
\xvaluen{$\I$}{\forall \alpha.\ \alpha \rightarrow \alpha}
{$\I x = x$}
\xvaluen{$\K$}{\forall \alpha\ \beta.\ \alpha \rightarrow \beta
\rightarrow \alpha} {$\K x\ y = x$}
    \end{estable}
	\centering
	\caption{操作栈算子} \label{tab:ES.stack} \begin{estable}
\modifiern{\texttt{;}}{\forall \alpha\ \beta.\ \alpha \rightarrow \beta
        \rightarrow \beta} {$\texttt{;} = \K \I$}
\modifier{\texttt{--}}{\forall \alpha.\ \alpha \rightarrow 
\mathrm{string} \rightarrow \alpha}{$\texttt{--}\ x\ name$ 将值 $x$ 
以名称 $name$ 加入到变量上下文中,并返回值 $x$。}
\xconstr{\textbf{InputStack}}{\forall \alpha\ \beta.\ 
        \alpha \rightarrow (\istack{(\beta\rightarrow\beta)}{
    \alpha})}{输入栈结构的构造函数,$\quad \textbf{InputStack}\ 
        x\ y$ 写作 $\istack{y}{x}$}
\modifier{\texttt{(}}{\forall \alpha\ \beta.\ 
        \alpha \rightarrow (\istack{(\beta\rightarrow\beta)}{
    \alpha})}{ \textbf{InputStack} 的别名,$\quad x\ \texttt{(}\ y
        = \istack{y}{x}$}
        \modifier{\texttt{)}}{\forall \alpha\ \beta.\ (\istack{
\alpha}{(\alpha\rightarrow\beta)})\rightarrow \beta}{
    弹出输入栈,将栈顶值弹出并
调用于新的栈顶值,$\quad (\istack{a}{f})\ \texttt{)} = f\ a$}
\modifier{\texttt{)}}{\forall \alpha\ \beta\ \gamma.\ 
(\istack{\alpha}{\istack{(\alpha\rightarrow\beta)}{\gamma}})\rightarrow 
    \istack{\beta}{\gamma}}{
弹出输入栈,将栈顶值弹出并调用于新的栈顶值,\newline
    $(\istack{a}{\istack{f}{b}})\ \texttt{)} = \istack{f(a)}{b}$}
\modifier{\text{⟭}}{\forall \alpha\ \beta.\ (\istack{\alpha}{\beta})
\rightarrow \beta}{弹出输入栈,将栈顶值抛弃,
$\quad(\istack{a}{b})\ \text{⟭} = b$}
\callingmethod{$\text{IS}_{\rightarrow}$}{\forall \alpha\ \beta\ \gamma\ \delta.\ 
        \CMT{\alpha}{\beta}{\gamma} \rightarrow \CMT{(\istack
{\alpha}{\delta})}{\beta}{(\istack{\gamma}{\delta})}}
        {$\mathrm{IS}_{\rightarrow}\ \CM(f) = \CM(\lambda\ (\istack{f}{b}),\ a.\ 
        \istack{f(a)}{b})$}
\callingmethod{$\text{IS}_{\leftarrow}$}{\forall \alpha\ \beta\ \gamma\ \delta.\ 
        \CMT{\alpha}{\beta}{\gamma} \rightarrow \CMT{\alpha}
{\CMT{\beta}{\delta}}{(\istack{\gamma}{\delta})}}
        {$\mathrm{IS}_{\leftarrow}\ \CM(f) = \CM(\lambda\ f,\ (\istack{a}{b}).\ 
        \istack{f(a)}{b})$}
\modifiern{\texttt{;}}{\forall \alpha\ \beta\ \gamma.\ 
        \istack{\alpha}{\gamma}\rightarrow (\istack{(
        \beta \rightarrow \beta)}{\gamma})}
        {输入栈版本的 \texttt{;}}
\modifiern{\texttt{--}}{\forall \alpha\ \gamma.\ (\istack{\alpha}
    {\gamma})\rightarrow \mathrm{string} \rightarrow (\istack{\alpha}
        {\gamma})}{输入栈版本的 \texttt{--}}
\end{estable}
\end{table}

接下来将首先介绍输入栈相关的算子。这些算子列于表 \ref{tab:ES.stack}
,提供的功能包括基础的编辑变量的存入;编辑栈的压入与弹出等。

分号 \texttt{;} 是 $\K$ 的别名,此分号与传统编程语言中的分号功能相同,
以从一行源代码切换到令一行源代码,在 \ES 中,分号算子 \text{;} 意味着
抛弃一个状态值而切换到空状态 $\I$,以接受新的输入。例如
\[ 1; 2; 3 \]
将状态值变为 $1$ 又抛弃并将状态值变为 $2$ 而又抛弃并最终状态值变为 $3$。

修饰符 \texttt{--} 用于绑定编辑变量,在实际的实现中会考虑局部变量,而
此处为理论描述的简洁仅考虑全局变量。
\[ 1\ \texttt{--}\ x; x \]
在输入 $1\ \texttt{--}$ 后,状态值的类型为 $\mathrm{string} \rightarrow
\mathrm{int}$,而之后输入的 $x$ 因为既不是 $\mathrm{string} \rightarrow
\mathrm{int}$ 上的修饰符又不是某个编辑变量的标识,于是被解析为字符串
字面量,即 \texttt{'x'},输入结果是将 $1$ 以标识 \texttt{x} 绑定入
全局变量空间中而状态值变回 $1$。接下来 \texttt{;} 算子切换状态值至 $\I$
然后输入标识 \texttt{x},此时 \texttt{x} 已经在全局变量空间中绑定了
值 $1$,于是效果是输入 $1$ 在 $\I$ 的作用下将状态值变为 $1$。

\textbf{InputStack} 作为一种构造函数将任意的 $x,\ y$ 构造成具有类型
$\istack{y}{x}$ 的值 $\mathbf{InputStack}\ x\ y$ 写作
$\istack{y}{x}$,表示输入中一系列括号构成的栈。
修饰符 \texttt{(} 表示压入一层输入栈,将当前的状态值压入栈内保存起来而
将栈顶替换成空以允许新的输入。
修饰符 \texttt{)} 表示出栈,将栈顶元素弹出作为参数调用于栈内保存的元素。
修饰符 ⟭ 更直接地丢弃栈顶元素。
两个调用算子 $\mathrm{IS}_{\rightarrow},\ \mathrm{IS}_{\leftarrow}$ 允许调用栈顶值就像调用
没有输入栈时的情况一样,即允许当状态值为 $\istack{f}{b}$ 时,
输入 $(\leftarrow,\ a)$ 的结果是理想的 $\istack{f(a)}{b}$,
或者输入 $(\rightarrow,\ g)$ 时也是理想的 $\istack{g(f)}{b}$。

一个示例脚本是
\[ 1 + (2 + (3) \eset x;\ x + x) \]
在输入完 $1\ +$ 后,状态值为 $+\ 1 : \mathrm{int} \rightarrow \mathrm
{int}$,接下来 $($ 作为一个修饰符,
调用结果为 
\[\mathbf{InputStack}\ (+\ 1) : \forall \alpha.\ \alpha
\rightarrow (\istack{\alpha}{\mathrm{int}})\]
然后 $2$ 令状态变为
\[\istack{2}{(+\ 1)} : \istack{\mathrm{int}}
{(\mathrm{int} \rightarrow \mathrm{int})}\]
输入 $+$,通过 $\mathrm{IS}_{\leftarrow}$ 算子
\[\istack{(+\ 2)}{(+\ 1)} : \istack{(\mathrm{int} \rightarrow
\mathrm{int})}{(\mathrm{int} \rightarrow \mathrm{int})}\]
再次 \texttt{(} 修饰符
\[\istack{\I}{\istack{(+\ 2)}{(+\ 1)}} : \forall \alpha.\ \istack{
    (\alpha \rightarrow \alpha)}{\istack{(\mathrm{int} \rightarrow
\mathrm{int})}{(\mathrm{int} \rightarrow \mathrm{int})}}\]
输入 $3$
\[\istack{3}{\istack{(+\ 2)}{(+\ 1)}} : \istack{\mathrm{int}}{
    \istack{(\mathrm{int} \rightarrow
\mathrm{int})}{(\mathrm{int} \rightarrow \mathrm{int})}}\]
\texttt{)} 修饰符弹出并调用 $+\ 2$
\[\istack{5}{(+\ 1)} : \istack{\mathrm{int}}{(\mathrm{int} 
\rightarrow \mathrm{int})}\]
输入栈版本的 $\eset$
\[\istack{(\eset\ 5)}{(+\ 1)} : \istack{\mathrm{string} \rightarrow 
\mathrm{int}}{(\mathrm{int} \rightarrow \mathrm{int})}\]
\texttt{x} 以字符串字面量输入,将 5 绑定到 \texttt{x}
\[\istack{5}{(+\ 1)} : \istack{\mathrm{int}}{(\mathrm{int} 
\rightarrow \mathrm{int})}\]
\texttt{;} 切换栈顶值到 $\I$
\[\istack{\I}{(+\ 1)} : \forall \alpha.\ \istack{(\alpha \rightarrow 
\alpha)}{(\mathrm{int} \rightarrow \mathrm{int})}\]
 $x + x$ 的结果是
\[\istack{10}{(+\ 1)} : \istack{\mathrm{int}}{(\mathrm{int} 
\rightarrow \mathrm{int})}\]
最后 \texttt{)} 将输入栈弹出
\[ 11 : \mathrm{int} \]

以上,这个示例完整地解释了输入栈算子的应用,揭示了 \ES 机器如何在
非常简洁的设定下达到强大而灵活的编辑能力,而完全不包含任何的关键字。
分号 \texttt{;} 括号 \texttt{()} 这些都不是关键字或者别的什么,而是
普普通通的函数,一种值,跟 $1,2,3$ 一样的值,体现了一种高度的优雅
与自恰。

接下来介绍段落算子,编辑壳层函数的定义方法。

\begin{table}
	\centering
	\caption{段落算子} \label{tab:ES.paragraph} \begin{estable}
        \xconstr{\textbf{Paragraph}}{\forall \alpha\ \beta\ \gamma.\ 
        (\alpha \rightarrow \beta) \rightarrow \gamma \rightarrow
        \combtyp{paragraph}{(\alpha,\ \beta,\ \gamma)}}{
        段落类型的构造函数}
\modifier{\pilcrow}{\forall \alpha\ \beta\ \gamma.\ 
       \gamma \rightarrow (\alpha \rightarrow \beta) \rightarrow 
        \combtyp{paragraph}{(\alpha,\ \beta,\ \gamma)}}
        {$x\ \pilcrow\ f = \textbf{Paragraph}(f,\ x)$}
\modifier{\texttt{(}}{\forall \alpha\ \beta\ \gamma.\ 
\combtyp{paragraph}{(\alpha,\ \beta,\ \gamma)} \rightarrow 
        (\istack{\gamma}{(\alpha \rightarrow \beta)})
        }{进入一个段落并加载段落中所有的临时变量。}
\modifier{\texttt{(}}{\forall \alpha\ \beta\ \gamma\ \delta.\ 
(\istack{(\combtyp{paragraph}{(\alpha,\ \beta,\ \gamma)})}{\delta})
    \rightarrow (\istack{\gamma}{\istack{
        (\alpha \rightarrow \beta)}{\delta}})
        }{进入一个段落并加载段落中所有的临时变量。}
\modifier{«}{\forall \alpha\ \beta\ \gamma\ \delta.\ 
    \combtyp{paragraph}{(\alpha,\ \beta,\ \gamma)} \rightarrow 
    \mathrm{string} \rightarrow \delta \rightarrow 
    \combtyp{paragraph}{(\alpha,\ \beta,\ \gamma)}}
    {$p\ «\ name\ x$ 为段落 $p$ 添加名为 $name$ 的临时变量}
    \end{estable}
	\caption{函数构建} \label{tab:ES.makef} \begin{estable}
\xvalue{\textbf{MakeFunc}}{\forall \alpha\ \beta.\ 
        \alpha \rightarrow \beta \rightarrow \alpha \rightarrow \beta}
{\textbf{MakeFunc} 是 \ES 内用于构建编辑壳层函数的编辑壳层功能,
$\textbf{MakeFunc}\ x\ b$,构建一个编辑壳层函数 $\lambda x.\ b$,
$x$ 必须是不包含任何内容的原子。}
        \xvalue{λE}{\forall \alpha\ \beta.\ \combtyp{type}{\alpha}
\rightarrow \combtyp{paragraph}{(\alpha,\ (\alpha \rightarrow \beta),\ 
\beta)}}{$\text{λE}\ \alpha$ 构造一个用于构建具有类型
        $\alpha \rightarrow \beta$ 的函数的段落}
\modifier{$\Leftarrow$}{\forall \alpha\ \beta\ \gamma\ \delta.\ 
\combtyp{paragraph}{(\alpha,\ \beta,\ \gamma)} \rightarrow
\mathrm{string} \rightarrow \combtyp{type}{\delta} \rightarrow
\combtyp{paragraph}{(\alpha,\ (\delta \rightarrow \beta),\ \gamma)}}
        {对于回调将返回$\beta$类型的段落 $p$,$p \Leftarrow
        name\ \delta$ 返回一个新段落,其拥有一个名为 $name$ 的新变量,
        并返回一个 $\delta \rightarrow \delta$ 的函数。
        可以用于构建多个参数的高阶函数。}
    \end{estable}
\end{table}

段落是具有类型 $\combtyp{paragraph}{(\alpha,\ \beta,\ \gamma)}$ 的值,
封装了 $\alpha \rightarrow \beta$ 类型的回调函数,$\gamma$ 类型的值与
一系列临时变量。段落是一个临时变量空间。
$\combtyp{paragraph}{(\alpha,\ \beta,\ \gamma)}$ 类型的段落意为,
给出 $\gamma$ 类型的一个值,即将此值放在输入栈顶上,并激活段落中的各个
临时变量,
而栈下放着 $\alpha \rightarrow \beta$ 类型的回调函数,让用户进行一系列
的操作,最后产生 $\alpha$ 类型的结果,而用 \texttt{)} 修饰符弹出栈顶
的 $\alpha$ 而调用于栈下层的回调 $\alpha \rightarrow \beta$。
例如
\[ 2\ \pilcrow\ (1 +)\ (+ 3) \]
$2\ \pilcrow\ (1 +)$ 构造了一个 $\combtyp{paragraph}{(\mathrm{int},\ 
\mathrm{int},\ \mathrm{int})}$ 类型的段落
\[ \mathbf{Paragraph}\ (+\ 1)\ 2\quad:\quad\combtyp{paragraph}{(\mathrm{int},\ 
\mathrm{int},\ \mathrm{int})}\]
而后 \texttt{(} 修饰符打开此段落,激活段落中的临时变量,状态值变成
\[ \istack{2}{(+\ 1)}\quad:\quad\istack{\mathrm{int}}{(\mathrm{int} \rightarrow
\mathrm{int})} \]
然后 $+\ 3$ 令状态值变为
\[ \istack{5}{(+\ 1)}\quad:\quad\istack{\mathrm{int}}{(\mathrm{int} \rightarrow
\mathrm{int})} \]
最后修饰符 \texttt{)} 弹出栈顶,得到
\[ 6\quad:\quad\mathrm{int} \]

值 λE 可以用于构建函数。$\combtyp{type}{\alpha}$ 是类型的类型,例如
$\mathrm{int}$ 在 \ES 中可以作为一个值而其类型为 $\combtyp{type}{\mathrm{int}}$
而 $\combtyp{type}{\mathrm{int}}$ 亦是一个值其类型为
$\combtyp{type}{(\combtyp{type}{\mathrm{int}})}$。
λE 作为一个值具有类型 
\[\forall \alpha\ \beta.\ \combtyp{type}{\alpha}
\rightarrow \combtyp{paragraph}{(\alpha,\ (\alpha \rightarrow \beta),\ 
\beta)}\]
其接受某个类型 $\alpha$,并返回一个段落。这段落给出一个具有该类型 $\alpha$ 的原子值 $x$,此值不包含任何内容无法参与任何编辑壳层功能,$x$
参与用户的任意计算后,用户任意编辑后得到某个 $\beta$ 类型的结果 $y$,
$y$ 中包含 $x$ 因为 $x$ 无法参与编辑壳层功能而不会被规约掉,
最后弹出给段落的回调而此回调构造编辑壳层函数 $\lambda x.\ y$

例如这段 \ES 输入
\[ \text{λE}\ \mathrm{int}\ (\ +\ 1) \]
$\text{λE}\ \mathrm{int}$ 后的状态值为
\[ \mathbf{Paragraph}\ (\mathbf{MakeFunc}\ x)\ x\quad:\quad
\forall \alpha.\ \combtyp{paragraph}{(\alpha,\ (\mathrm{int} 
\rightarrow \alpha),\ \mathrm{int})}\]
修饰符 \texttt{(} 打开段落
\[ \istack{x}{\mathbf{MakeFunc}\ x}\quad :\quad \forall \alpha.\ 
\istack{\mathrm{int}}{(\alpha \rightarrow 
\mathrm{int} \rightarrow \alpha)}\]
$+ 1$ 后
\[ \istack{x + 1}{\mathbf{MakeFunc}\ x}\quad :\quad \forall \alpha.\ 
\istack{\mathrm{int}}{(\alpha \rightarrow 
\mathrm{int} \rightarrow \alpha)}\]
修饰符 \texttt{)} 弹出
\[ \mathbf{MakeFunc}\ x\ (x + 1)\quad : \quad \mathrm{int} \rightarrow
\mathrm{int} \]
$\mathbf{MakeFunc}$ 是一个编辑壳层功能,执行后得到
\[ (\lambda x.\ x + 1)\quad : \quad \mathrm{int} \rightarrow
\mathrm{int}  \]
即是期望构建的函数。

至此,对 \ES 编辑环境的论述告一段落,已经能看到 \ES 有效的编辑能力。
但仍然有大量的功能与算子未被介绍,这些算子将在
未来的的用户手册中逐一论述。

