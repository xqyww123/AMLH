
这一节介绍预定义的值,这些值构建起编辑环境并提供编辑定理以进行程序开发的
功能。


根据一个值具有的期望被调用的方式,期望被前置调用的被直接叫做值,
而期望被后置调用的被叫做修饰符。
当每个值的调用方式被确定,按输入顺序依次写下这些值,就构成了对
\ES 机器的操作的文本,这种文本易于我们讨论与分析 \ES 机器上的操作。

例如文本 $1 + 2$,在确定 $+$ 是后置调用地作为修饰符后,这段文本就清晰
地指出对 \ES 机器的操作序列
\[ (\rightarrow,\ 1),\ (\leftarrow,\ +),\ (\rightarrow,\ 2) \]
初始态 \ES 机器状态中的值为 $\I$,第一个操作 $(\rightarrow,\ 1)$ 令
$\I$ 以参数 $1$ 调用,得到结果 $1$,第二个操作 $(\leftarrow,\ +)$ 令
$+$ 以参数 $1$ 调用,得到 $+\ 1$,第三个操作 $(\rightarrow,\ 2)$,
得到 $+\ 1\ 2$。调用就此结束,若 $+$ 与整数加法的编辑壳层功能关联,
那么此时的调用会触发此功能,而结果将成为 $3$。


\begin{table}
	\centering
	\caption{\ES 的基础算子} \label{tab:ES.basic} \begin{estable}
\xvaluen{$\I$}{\forall \alpha.\ \alpha \rightarrow \alpha}
{$\I x = x$}
\xvaluen{$\K$}{\forall \alpha\ \beta.\ \alpha \rightarrow \beta
\rightarrow \alpha} {$\K x\ y = x$}
    \end{estable}
\end{table}
\begin{table}
	\centering
	\caption{操作栈算子} \label{tab:ES.stack} \begin{estable}
\modifiern{\texttt{;}}{\forall \alpha\ \beta.\ \alpha \rightarrow \beta
\rightarrow \alpha} {$\K$ 的别名}
\modifier{\texttt{--}}{\forall \alpha.\ \alpha \rightarrow 
\mathrm{string} \rightarrow \alpha}{$\texttt{--}\ x\ name$ 将值 $x$ 
以名称 $name$ 加入到变量上下文中。}
\xconstr{\textbf{InputStack}}{\forall \alpha\ \beta.\ 
\alpha \rightarrow \combtyp{input\_stack}{(\beta\rightarrow\beta,\ 
\alpha)}}{输入栈结构的构造函数}
\modifiern{\texttt{(}}{\forall \alpha\ \beta.\ 
\alpha \rightarrow \combtyp{input\_stack}{(\beta\rightarrow\beta,\ 
\alpha)}}{ \textbf{InputStack} 的别名}
\modifier{\texttt{)}}{\forall \alpha\ \beta.\ \combtyp{input\_stack}
{(\alpha,\ \alpha\rightarrow\beta)}}{ 弹出输入栈,将栈顶值弹出并
调用于新的栈顶值}
    \end{estable}
\end{table}

简单地列出一些基础的算子于表 \ref{tab:ES.basic} 中,
接下来将首先介绍输入栈相关的算子。这些算子列于表 \ref{tab:ES.stack}
,提供的功能包括基础的编辑变量的存入;编辑栈的压入与弹出等。

一个示例脚本是
\[ 1 + (2 + 3) \eset x;\ x + x \]
在输入完 $1\ +$ 后,\ES 状态为 $+\ 1 : \mathrm{int} \rightarrow \mathrm
{int}$,接下来 $($ 作为一个修饰符,
调用结果为 
\[\mathbf{InputStack}\ (+\ 1) : \forall \alpha.\ \alpha
\rightarrow \combtyp{input\_stack}{(\alpha, \mathrm{int})}\]
然后 $2$ 令状态变为
\[\mathbf{InputStack}\ (+\ 1)\ 2 : \combtyp{input\_stack}{(\mathrm{int},
\mathrm{int})}\]


