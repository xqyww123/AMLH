\section{\ES 机器}

\subsection{\ES 机器的形式定义}

\begin{defin}[标识、编辑变量空间、编辑上下文]
索引集(index set)$I$ 表示所有的标识,这些标识用于命名编辑器变量。
一个编辑变量空间是一个 $I$ 到 $\LE^2$ 的有限映射 $I \mapsto \LE^2$,
在实现上可以用哈希表表示。
所有的编辑变量空间构成集合 $\vspac$。
另一个索引集 $\Is$ 中的元素标识编辑变量空间,$\Is$ 到 $\vspac$ 
的有限映射叫做 \ES 机器的上下文 $\{\vspa{i}\},\ \vspa{i} \in \vspac,\ 
i \in \Is$ 构成变量空间$\vspac$的族(families of set $\vspac$),
所有的上下文构成集合 $\CE$。
\end{defin}

对一个变量空间 $\Psi$,项 $\Psi(name),\ name \in I$ 是变量空间中名为
$name$的值。
一个上下文 $\{\vspa{i}\}$,项 $\vspaR,\ \mathbf{0} \in \Is$ 表示
标识为 $\mathbf{0}$ 的变量空间,继而 $\vspaR(name)$ 表示
标识为 $\mathbf{0}$ 的变量空间中名为 $name$ 的变量。

\ES 机器中非常重要的一点是预定义函数,这些函数是一些 \ES 形式语言上的值
,但与编辑器上预先实现的一些功能关联,当这些值进行 β 规约时,关联的
功能将被激活,编辑器上特定的程序过程将被执行,而这些值的参数将传入
这些过程作为参数。以此实现诸如编译、读取文件,或者在屏幕上打印一些
内容的功能。可以说,\ES 形式语言是为调用这些功能的脚本。

现在形式地说明这些预定义值与功能,但绝非说这些功能就是如此按数学定义来
实现的,它们在软件中的具体实现参见???章,而此处的形式定义仅仅是为了
方便论述它们的含义,以及精确论述 \ES 状态的抽象模型。
预定义值此处的定义仅仅是理论上的工具,暂且把具体实现搁置。

\begin{defin}[预定义值与编辑器功能]
编辑器功能的理论模型是 $\CE \times \{\LE\}_n$ 到 
$\CE \times \LE$ 的函数。对于输入上下文 $\gamma$,长度为$n$的输入参数
列表$\{a_i\}_n$,编辑器功能 $f$ 的执行结果 \[f(\gamma,\{a_i\}_n) = 
(\gamma',y)\] 表示新的上下文 $\gamma'$ 以及返回值 $y$。
所有的编辑器功能构成集合 $\FE$。
量化常量值 $L_c^2$ 中所有链接到编辑器功能的值构成集合 $\Lp^2$,
$\Lp^2 \subseteq L_c^2$,其中每个值$z_p$对应的编辑器功能为
$\fE(z_p)$,$\fE$ 是 $\Lp^2$ 到 $\FE$ 的函数。
每个编辑器功能$f$都有一个自然数对应$\omega(f)$ 表示参数数目。
因为 $\Lp$ 集中每个值$z_p$都对应了唯一的一个编辑器功能 $\fE(z_p)$,
就也能得到对应编辑器功能的参数数目 $\omega(\fE z_p)$,
不会引起混淆地同样写作$\omega(z_p)$
\[ \omega(z_p) = \omega(\fE\ z_p) \]
\end{defin}

\begin{defin}[\ES 状态与\ES 机器]
\ES 状态是上下文$\{\vspac{i}\}$ 与值 $z$ 的二元组 $(\{\vspac{i}\}, z)$,
所有的状态构成集合 $\mathbf{State} = \CE \times \LE^2$。
一个特殊的变量空间 $\vspaR$ 表示全局变量空间,初始状态 $s_0$
\[ s_0 = (\vspaR,\ \I) \]
定义操作集为 $\mathbf{Choice} = \{\rightarrow,\ \leftarrow\}$,
输入集为 $\mathbf{Input} = \mathbf{Choice} \times \LE^2$,
状态转移函数 $\delta : \mathbf{State}\times\mathbf{Input} \rightarrow
\mathbf{State}$
\begin{align*}
\delta((\Psi,\ z),\ \leftarrow, z') &= (\Psi,\     )
\end{align*}

\ES 机器是状态机。
\end{defin}


