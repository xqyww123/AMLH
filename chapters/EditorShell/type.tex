\subsection{类型与全称量化类型}

首先定义类型。

\begin{defin}[类型集$\PiE$]
无穷的单词集 $\hat{U}$ 为用于表示类型的字母表
$\hat{U} \subseteq \Words$,
函数$\xa : \hat{U} \mapsto \mathbb{N}$表示类型的元数(Arity),
$\hat{U}$ 中同样无穷的元数为0的子集$\hat{U}_\mathrm{v}$:
$\hat{U}_\mathrm{v} \subseteq \hat{U}\ \land\ \forall u.\ 
u \in \hat{U}_\mathrm{v} \Rightarrow \xa(u) = 0$
表示类型变量,主要用于全称量化。

\ES  的类型集 $\Pi_\mathrm{E}$ 是满足以下条件的$\String$的最小子集
\begin{equation} \label{Def.PiE}
\forall u.\ u \in U \Rightarrow \forall \bm{v}.\ \bm{v} 
\in \Pi_\mathrm{E}^{\xa(u)} \Rightarrow \underline{(\ u\ \bm{v}_1\ 
\cdots\ \bm{v}_{\xa(u)}\ )} \in \Pi_\mathrm{E}
\end{equation}
其中 $\Pi_\mathrm{E}^{\xa(u)}$ 表示 $\Pi_\mathrm{E}$ 的 $\xa(u)$ 维向量
空间,特别的 $\Pi_\mathrm{E}^0 = \{\mathbf{0}\}$
\begin{align*}
    \Uv &= \{\underline{(\ u\ )}\mbar u \in \hat{\mathrm{U}}_v\}&
    U &= \{\underline{(\ u\ )}\mbar u \in \hat{\mathrm{U}}\}&
\end{align*}
\end{defin}

\begin{example}[$\Pi_\mathrm{E}$] 以下命题成立:
\[ \forall u.\ u \in \hat{U}\ \land\ (\xa(u) = 0)\ \Rightarrow\ 
\underline{(\ u\ )} \in \Pi_\mathrm{E} \]
\[ \forall u\ v.\ u \in \hat{U}\ \land\ v \in \hat{U}_\mathrm{v}\ \land\ (\xa(u) = 1)
\ \Rightarrow\ \underline{(\ u\ (\ v\ )\ )} \in \Pi_\mathrm{E} \]
\[ \forall u\ v.\ u \in \hat{U}\ \land\ v \in U_\mathrm{v}\ \land\ (\xa(u) = 1)
\ \Rightarrow\ \underline{(\ u\ v\ )} \in \Pi_\mathrm{E} \]
\[ \forall u\ v.\ u \in \hat{U}\ \land\ v \in \Pi_\mathrm{E}\ \land\ (\xa(u) = 1)
\ \Rightarrow\ \underline{(\ u\ v\ )} \in \Pi_\mathrm{E} \]
\end{example}

\begin{theo}(类型的结构) \label{TS}
\[ \forall u'.\ u' \in \PiE \Rightarrow \exists! u\ \bm{v}.\ 
u \in \hat{U}\ \land\ \bm{v} \in \PiE^{\xa(u)}\ \land\ 
(u' = \underline{(\ u\ \bm{v}_1\ \cdots\ \bm{v}_{\xa(u')}\ )}) \]
\end{theo}
\begin{proof} 首先唯一性是显然的,由字符串理论就可以得到。
对存在性的证明使用反证法,假设
\begin{equation} \label{Hypo.TS}
\exists u'.\ u' \in \PiE \Rightarrow \forall u\ \bm{v}.\ 
u \in \hat{U}\ \land\ \bm{v} \in \PiE^{\xa(u)}\ \land\ 
(u' \neq \underline{(\ u\ \bm{v}_1\ \cdots\ \bm{v}_{\xa(u')}\ )})
\end{equation}
现证明 $\PiE - \{u'\}$ 满足条件 \ref{Def.PiE} 而
$\PiE - \{u'\} \subset \PiE$ 这样就构造了悖论,因为 $\PiE$ 不再是满足
条件 \ref{Def.PiE} 的最小子集,$\PiE - \{u'\}$ 比 $\PiE$更小。
即证明
\[ \forall u.\ u \in \hat{U} \Rightarrow \forall \bm{v}.\ 
\bm{v} \in \PiE^{\xa(u)} - \{u'\} \Rightarrow
\underline{(\ u\ \bm{v}_1\ \cdots\ \bm{v}_{\xa(u)})} \in \PiE - \{u'\} \]
因为有
\[ \forall u.\ u \in \hat{U} \Rightarrow \forall \bm{v}.\ 
\bm{v} \in \Pi_\mathrm{E}^{\xa(u)} \Rightarrow
\underline{(\ u\ \bm{v}_1\ \cdots\ \bm{v}_{\xa(u)})} \in \Pi_\mathrm{E} \]
所以只要证明
\[ \underline{(\ u\ \bm{v}_1\ \cdots\ \bm{v}_{\xa(u)}\ )} \neq u' \]
而由反证假设 \ref{Hypo.TS} 这是成立的,故而悖论被构造进而命题得证。
\end{proof}

定理 \ref{TS} 意味着一切类型$u$都具有且唯一地具有如下格式
\[ \underline{u_0\ \bm{v}_1\ \cdots\ \bm{v}_{\xa(u_0\ )}} \]
其中 $u_0 \in \hat{U}$,$\bm{v}_1,\ \cdots,\ \bm{v}_{\xa(u_0\ )} \in \PiE$

即每一个类型都构成一颗树,类型变量与0元类型构造器是叶子。

\begin{defin}[类型的构造器、元数、高度、变量集、重量] \label{Def.Th}
函数 $\mathrm{c}:\Pi_\mathrm{E} \mapsto \hat{U}$ 表示类型的构造器。
\[ \mathrm{c}\ \underline{(\ u\ v_1\ \cdots\ v_n\ )} = u\]
函数 $\xa:\Pi_\mathrm{E} \mapsto \mathbb{N}$ 表示类型的元数。
\[ \xa\ \underline{(\ u\ v_1\ \cdots\ v_n\ )} = n\]
$\xa:\Pi_\mathrm{E} \mapsto \mathbb{N}$ 不会跟上文定义的
$\xa:\hat{U} \mapsto \mathbb{N}$冲突,因为定义域不重合,且两者具有相同的意义,
不会造成歧义。

\noindent 函数 $\mathrm{h}:\Pi_\mathrm{E} \mapsto \mathbb{N}$ 表示类型的高度。
\[ \mathrm{h}\ \underline{(\ u\ v_1\ \cdots\ v_n\ )} = 1 + \max(\mathrm{h}\ v_1,\ \cdots,\ 
\mathrm{h}\ v_n) \]

\noindent 函数 $\mathrm{v}:\Pi_\mathrm{E} \mapsto \powerset{(\Uv)}$ 
表示类型中的所有变量。
\[ \begin{split}
&\mathrm{v}\ \underline{(\ c\ )}=\xif c \in \Uv \xthen \{
\underline{(\ c\ )}\} \xelse \emptyset \\
&\mathrm{v}\ \underline{(\ u\ v_1\ \cdots\ v_n\ )}=
\bigcup_{i=1\cdots n} \mathrm{v}(v_i)
\end{split} \]
函数 $\TS : \PiE \rightarrow \mathbb{N}$ 表示类型的重量
    \[ \TS \underline{(\ u\ v_1\ \cdots\ v_n\ )} = 1 + 
    \sum_{i=1,\cdots,n} \TS v_i \]
\end{defin}

有如下性质

\begin{lemma}[类型的元数、高度与重量的性质] \label{T.cah}
\[ \forall u.\ \mathrm{h}\ u  \geq 1 \quad\quad\text{(1)}\quad\quad
\quad\quad\quad\forall u.\ \xa(\mathrm{c}\ u) = \xa\ u \quad\quad\text{(2)} \quad\quad\quad
\forall u.\ (\mathrm{h}\ u = 1) \Rightarrow u \in U
\quad\quad\text{(3)} \]
\[ \begin{split}
\forall u.\ (\mathrm{h}\ u > 1) \Rightarrow \exists c\ &\bm{v}.\ c \in \hat{U}
\ \land\ \bm{v} \in \PiE^{\xa(u)}\ \land\ u = \underline{
(c\ \bm{v}_1\ \cdots\ \bm{v}_{\xa(u)}\ )} \ \land\ \\
&(\forall i.\ 1 \leq i \leq \xa(u) \Rightarrow \mathrm{h}\ \bm{v}_i < \mathrm{h}\ u )
\end{split} \tag{4} \]
    \[ \forall u.\ \TS u \geq 1 \quad\quad\text{(5)}\quad\quad\quad
    \forall u.\ (\TS u = 1) \Rightarrow u \in U\quad\quad\text{(6)}\]
\begin{proof} 由定义 \ref{Def.Th} 与定理 \ref{TS} 直接得到。
\end{proof}
\end{lemma}

这样就可以关于类型的高度进行归纳法。以后这些性质将暗含使用而不再另行引证。

\begin{defin}[全称量化类型 $\PiAE$]
集合 $\Pi_\mathrm{E}^\forall$ 表示全称量化类型,由所有满足如下语法的字符串构成。
\[ \Pi_\mathrm{E} \mbar \forall\ \Uv\ \Pi_\mathrm{E} \]
同样有记号 $\forall v_1\ \cdots\ v_n.\ b$ 表示 
$\underline{\forall\ v_1\ \cdots\  \forall\ v_n\ b}$
\end{defin}

\begin{defin}[全称量化类型的相关属性]
函数 $\mathrm{QV} : \Pi_\mathrm{E}^\forall \rightarrow \powerset(
U_\mathrm{v})$ 表示全称量化类型的绑定变量集。
\[ \mathrm{QV}(\forall v_1\ \cdots\ \v_n.\ b) = \{v_1,\ \cdots,\ v_n\} \]
函数 $\mathrm{QB} : \Pi_\mathrm{E}^\forall \rightarrow \Pi_\mathrm{E}$
表示全称量化的类型体。
\[ \mathrm{QB}(\forall v_1\ \cdots\ \v_n.\ b) = b \]
函数 $\Qv : \Pi_\mathrm{E}^\forall \rightarrow \powerset(
U_\mathrm{v})$ 表示全称量化类型所有的变量集。
\[ \Qv q = \QV q \cup \mathrm{v}(\QB q) \]
\end{defin}

\begin{defin}[实例化] \label{Def.inst}
函数 $\mathrm{inst} : (U_\mathrm{v} \rightarrow \Pi_\mathrm{E})
\rightarrow \Pi_\mathrm{E} \rightarrow \Pi_\mathrm{E}$ 对类型
进行变量实例化。
\begin{align*}
\mathrm{inst}\ f\ \underline{(\ v\ )} &= \xif \underline{(\ v\ )}
\in U_\mathrm{v} \xthen
 f\ \underline{(\ v\ )} \xelse \underline{(\ v\ )} \\
\mathrm{inst}\ f\ \underline{(\ u\ v_1\ \cdots\ v_{\xa(u)}\ )} &= 
\underline{(}\concat u \concat \inst f\ v_1\concat\cdots\concat
\inst f\ v_{\xa(u)}\concat\underline{)}
\end{align*}

以及部分实例化 $\inst_V$ 
\[ \inst_V f = \inst\ (\lambda v.\ \xif v \in V \xthen f\ v \xelse v) \]

函数 $\mathrm{inst}_\forall : (U_\mathrm{v} \rightarrow \Pi_\mathrm{E})
\rightarrow \Pi_\mathrm{E}^\forall \rightarrow \Pi_\mathrm{E}$ 
实例化全称量化类型。
\begin{gather*}
\mathrm{inst}_\forall\ f\ q = \inst_{\ \mathrm{QV}(q)} f\ \mathrm{QB}(q)
\end{gather*}
即$\mathrm{inst}_\forall$只会实例化全称量化的类型变量。

\noindent 给定类型变量集 $Q \subseteq \Uv$ 可以将 $\inst_\forall$ 实例化
的结果重新全称量化,函数 $\inst_\forall^Q$
\begin{align*} 
\QB(\inst_\forall^Q\ f\ q) &= \inst_\forall\ f\ q&
    \QV(\inst_\forall^Q\ f\ q) &= Q&
\end{align*}
\end{defin}

\begin{algorithm}
\caption{实例化函数 $\inst$} \label{alg:Iv}
\begin{algorithmic}[1]
\Require 集合 $V \in \powerset(\Uv)$ 表示实例化的范围
\Require 实例化函数 $f : \Uv \rightarrow \PiE$ 表示变量到值的对应
\Require $u \in \PiE,\ u = \underline{(\ c\ v_1\ \cdots\ v_o\ )}$
表示要实例化的目标
\Ensure $\inst_V f\ u \in \PiE$
\If {$o = 0$}
\If {\quad $u \in V$ \quad} \quad 输出 $f(u)$
\Else {\quad 输出 $u$}
\EndIf
\Else
\State $\underline{(\ c} \rightarrow s$
\For{$i=1,\ \dots,\ o$}
\State $s \concat \inst(V,f,v_i) \rightarrow s$
\EndFor
\State 输出 $s \concat \underline{)}$
\EndIf
\TimeComplexity 若 $f$ 的时间复杂度为 $t$ 则算法 inst 的时间复杂度为 $O(t\TS(u))$
\end{algorithmic}
\end{algorithm}
\begin{algorithm}
\caption{全称量化的实例化函数 $\inst_\forall$} \label{alg:IvQ}
\begin{algorithmic}[1]
\Require 实例化函数 $f : \Uv \rightarrow \PiE$ 表示变量到值的对应
\Require $q \in \PiAE,\ q = \underline{\forall\ v_1\ \cdots \forall\ v_p
\ u}$
表示要实例化的目标
\Ensure $\inst_\forall f\ q \in \PiE$
\State $\{\} \rightarrow s$
\For{$i=1,\ \dots,\ p$}
\State 集合 $s$ 加入 $q_i$
\EndFor
\State 调用算法 \ref{alg:Iv}:$\inst(s,f,u)$ 将结果输出。
\TimeComplexity 若 $f$ 的时间复杂度为 $t$ 则算法 inst 的时间复杂度为 $O(t\TS(\QB(u)))$
\end{algorithmic}
\end{algorithm}
\begin{algorithm}
\caption{构造全称量化类型 $\mathrm{MakeQT}$} \label{alg:MakeQT}
\begin{algorithmic}[1]
\Require 集合 $V \in \powerset(\Uv)$
\Require 类型 $u \in \PiE$
\Ensure $q \in \PiAE$ 满足 $(\QV q = V) \ \land\ (\QB q = u)$
\For{$v \in V$}
\State $\underline{\forall} \concat v \concat u \rightarrow u$
\EndFor
\State 输出 $u$
\TimeComplexity $O(\abs{V})$
\end{algorithmic}
\end{algorithm}

\begin{lemma} \label{Lem.Iv.V}
\[ \inst_V f\ u = \inst_{\ V \cap \mathrm{v}(u)} f\ u \]
\begin{proof} 对 $u$ 进行类型高度的归纳法即可。
\end{proof}
\end{lemma}
\begin{lemma} \label{L.Iv.h}
\[ \h(\inst_V f\ u) \geq \h u \]
\begin{proof} 对 $u$ 进行类型高度的归纳法。
\end{proof}
\end{lemma}

\begin{defin}[实例化类型集] \label{Def.QI}
全称量化类型$q$的实例化类型集$\mathrm{QI}\ q$为
\[ \mathrm{QI}\ q = \{\mathrm{inst}_\forall\ f\ q\mbar f \in 
(U_\mathrm{v} \rightarrow \Pi_\mathrm{E})\} \]
\end{defin}

\begin{defin}[α等价] \label{Def.aE}
二元关系$\sim_\alpha$ 定义为
    \[ (q_1 \sim_\alpha q_2) = (\QI q_1 = \QI q_2) \]
显然是一种等价关系,被叫做α等价。
\end{defin}

α等价类$\Pi_\mathrm{E}^\forall/[\sim_\alpha]$即是本质不同的全称量化类型。

\begin{lemma}[实例化类型集非空]
\[\forall q.\ \mathrm{QI}\ q \neq \emptyset\]
\end{lemma}
\begin{proof}
因为 $\quad\forall q.\ \mathrm{inst}_\forall\ \I\ q = \mathrm{QB}\ q 
\quad$ 所以有 $\quad\forall q.\ \mathrm{QB}\ q \in \mathrm{QI}\ q$
\end{proof}

接下来尝试证明一个重要命题 
\[ \forall q_1,\ q_2 \in \Pi_\mathrm{E}^\forall \Rightarrow
(\mathrm{QI}\ q_1 \cap \mathrm{QI}\ q_2 = \emptyset)\ \lor\ 
(\exists q \in \Pi_\mathrm{E}^\forall.\ \mathrm{QI}\ q_1
\cap \mathrm{QI}\ q_2 = \mathrm{QI}\ q) \]
并找到一个算法用于求解上述的 $q$,为此要引入诸多工具。
