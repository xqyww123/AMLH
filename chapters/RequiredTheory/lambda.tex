\chapter{λ 演算介绍} \label{Ch.lambda}

Sørensen 在其讲义中对$\lambda$演算的形式定义非常简洁,
本节直接从 $\lamst$ 开始介绍,朴素 $\lambda$ 演算请参见其讲义第一章,
 $\lamst$参考自讲义第三章,$\lambda2$参考自讲义第十三章
\cite{sorensen2006lecturesC1}。

\section{λ→}

λ→ 演算有两种样式,Curry 式和 Church 式的,本文延续 Sørensen 的提法,
简单地说 Curry 式的λ→演算为λ→演算,而说Church式的为Church式λ→演算。
这两种样式的λ→ 演算本质是相同的,Sørensen 在其讲义中对此有论述,
本文不再赘述。

\begin{defin}[$\lamst$演算] \label{Def.slam}
\begin{enumerate}
\item $V$为无穷的字母表,表示符号。

\begin{equation}
V = \{v_0,v_1,\cdots\}
\end{equation}

\item 字符串集合 $L$ 是无类型 $\lambda$ 演算上的表达式($\lambda$ term),由如下语法定义。
\begin{equation}
L = \bnf{V\ |\ (L\ L)\ |\ (\lambda V \ L)}
\end{equation}

简写 $\underline{(\ \lambda\ x_1\ (\ \lambda\ x_2\ \cdots\ (\ 
    \lambda\ x_n\ y\ )\ \cdots\ )\ )}$ 为
    $\lambda x_1\ x_2\ \cdots\ x_n.\ y$

\item 集合$U$是另一个无穷的字母表,表示类型变量集(type variables)。
\begin{equation}
U = \{\alpha,\beta,\cdots\} 
\end{equation}

\item 字符串集合 $\Pi$ 表示简单类型。
\begin{equation} \label{Pi}
    \Pi = \bnf{U\ |\ (\Pi \rightarrow \Pi)}
\end{equation}

\item 集合 $C$ 表示上下文,是由语法$V : \Pi$定义的字符串集合的幂集。
\begin{equation}
    C = \powerset \bnf{V:\Pi}
\end{equation}
即$C$ 是所有具有如下形式的集合。
        \[ \{x_1:\tau_1, \cdots, x_n : \tau_n\} \]
其中 $x_1,\cdots,x_n \in V$,$\tau_1,\cdots,\tau_n \in \Pi$。

\item 定义上下文$\Gamma = \{x_1:\tau_1,\cdots,x_n:\tau_n\}$ 的符号域 $\mathrm{dom}$
\[ \mathrm{dom}(\Gamma) = \{x_1,\cdots,x_n\}\]
将$\Gamma_1 \cup \Gamma_2$ 写作 $\Gamma_1, \Gamma_2$ 当
        $\mathrm{dom}(\Gamma_1) \cap \mathrm{dom}(\Gamma_2)$ 时。


%\item 定义上下文$\Gamma = \{x_1:\tau_1,\cdots,x_n:\tau_n\}$ 的类型域 $\rvert \Gamma \lvert$
%
%\[ \lvert \Gamma \rvert = \{\tau_1,\cdots,\tau_n\}\]
%
\item $\mathcal{L} = \bnf{L : \Pi}$ 是λ→表达式,
由如下规则定义$C \times \mathcal{L}$ 上的二元关系 $\vdash$

\hfill

\begin{minipage}[b]{0.5\linewidth}
\begin{prooftree}
\AxiomC{$\ $} \RightLabel{(公理)}
\UnaryInfC{$\Gamma, x : \tau \vdash x : \tau$}
\end{prooftree}
\end{minipage}%
\begin{minipage}[b]{0.4\linewidth}
\begin{prooftree}
\AxiomC{$\Gamma, x : \sigma \vdash M : \tau$} \RightLabel{(抽象律)}
\UnaryInfC{$\Gamma \vdash \lambda x. M : \sigma \rightarrow \tau$}
\end{prooftree}
\end{minipage}

\hfill

\begin{minipage}[b]{0.5\linewidth}
\begin{prooftree}
\AxiomC{$\Gamma \vdash M : \sigma \rightarrow \tau$}
\AxiomC{$\Gamma \vdash N : \sigma$} \RightLabel{(组合律)}
\BinaryInfC{$\Gamma \vdash M N : \tau$}
\end{prooftree}
\end{minipage}\begin{minipage}[b]{0.5\linewidth}
\begin{prooftree}
\AxiomC{$\Gamma \vdash (\lambda v\ M)\ x : \tau$}
\RightLabel{($\beta$规约)}
\UnaryInfC{$\Gamma \vdash M[v/x] : \tau$}
\end{prooftree}
\end{minipage}

\hfill

\end{enumerate}

简单类型$\lambda$演算$\lamst$就是三元组 $(L, \Pi, \vdash)$
\end{defin}


\section{Church式λ→}

Church式λ→ 与 Curry 式 λ→ 的主要区别是,Church 式的抽象中的变量需要
显示地标记类型。即Curry式的抽象写作
\[ \lambda x.\ x : \sigma \rightarrow \sigma \]
而 Church 式的写作
\[ \lambda x{:}\sigma.\ x : \sigma \rightarrow \sigma \]

\begin{defin}[Church式λ→]
\begin{gather}
V_C = \bnf{V:\Pi} \\
L_C = \bnf{V \mbar (L\ L) \mbar (\lambda V_C\ L)} \\
\mathcal{L}_C = \bnf{L_C : \Pi}
\end{gather}
定义 $C \times \mathcal{L}_C$ 上的关系 $\vdash$

\begin{minipage}[b]{0.5\linewidth}
\begin{prooftree}
\AxiomC{$\ $} \RightLabel{(公理)}
\UnaryInfC{$\Gamma, x : \tau \vdash x : \tau$}
\end{prooftree}
\end{minipage}%
\begin{minipage}[b]{0.4\linewidth}
\begin{prooftree}
\AxiomC{$\Gamma, x : \sigma \vdash M : \tau$} \RightLabel{(抽象律)}
\UnaryInfC{$\Gamma \vdash \lambda x{:}\sigma. M : \sigma \rightarrow \tau$}
\end{prooftree}
\end{minipage}

\begin{prooftree}
\AxiomC{$\Gamma \vdash M : \sigma \rightarrow \tau$}
\AxiomC{$\Gamma \vdash N : \sigma$} \RightLabel{(组合律)}
\BinaryInfC{$\Gamma \vdash M N : \tau$}
\end{prooftree}

Church 式$\lamst$就是三元组 $(L_C, \Pi, \vdash)$
\end{defin}

\section{λ2}

现在介绍λ2 演算,它有很多名字,System F,二阶λ演算,Girard–Reynolds
多态λ演算。
λ2 演算基于Church式λ→演算论述。

\begin{defin}[λ2]
\[ L_* = \bnf{L \mbar \Lambda\ U\ L_*} \]
并类似地记号 $\Lambda t_1\ t_2\ \cdots\ t_n.\ b$ 表示
    $\underline{\Lambda\ t_1\ \Lambda\ t_2\ \cdots\ \Lambda\ t_n\ b}$,
\[ \Pi_* = \bnf{\Pi \mbar \forall\ U\ \Pi_*} \]
记号 $\forall \tau_1\ \tau_2\ \cdots\ \tau_n.\ b$ 表示
$\underline{\forall\ \tau_1\ \forall\ \tau_2\ \cdots\ \forall\
    \tau_n\ b}$
\[ C_* = \powerset \bnf{V:\Pi \mbar U:*} \]
$\mathcal{L}_* = \bnf{L_* : \Pi_*}$ 是λ2表达式,
定义 $C_* \times \mathcal{L}_*$上的关系 $\vdash$

\hfill

\begin{minipage}[b]{0.5\linewidth}
\begin{prooftree}
\AxiomC{$\ $} \RightLabel{(公理)}
\UnaryInfC{$\Gamma, x : \tau \vdash x : \tau$}
\end{prooftree}
\end{minipage}%
\begin{minipage}[b]{0.4\linewidth}
\begin{prooftree}
\AxiomC{$\Gamma, x : \sigma \vdash M : \tau$} \RightLabel{(抽象律)}
\UnaryInfC{$\Gamma \vdash \lambda x{:}\sigma. M : \sigma \rightarrow \tau$}
\end{prooftree}
\end{minipage}

\begin{prooftree}
\AxiomC{$\Gamma \vdash M : \sigma \rightarrow \tau$}
\AxiomC{$\Gamma \vdash N : \sigma$} \RightLabel{(组合律)}
\BinaryInfC{$\Gamma \vdash M N : \tau$}
\end{prooftree}

\begin{minipage}[b]{0.5\linewidth}
\begin{prooftree}
\AxiomC{$\Gamma,\ \alpha:* \vdash M : \tau$}
\RightLabel{(全称抽象律)}
\UnaryInfC{$\Gamma \vdash \Lambda \alpha\ M : \forall \alpha\ \tau$}
\end{prooftree}
\end{minipage}\begin{minipage}[b]{0.5\linewidth}
\begin{prooftree}
\AxiomC{$\Gamma_1 \vdash M : \forall \alpha\ \sigma$}
\AxiomC{$\Gamma_2 \vdash \tau : *$}
\RightLabel{(全称组合律)}
\BinaryInfC{$\Gamma_1,\ \Gamma_2 \vdash M\ \tau : \sigma$}
\end{prooftree}\end{minipage}

\hfill

符号 $*$ 可以看作类型的类型。

多态λ演算λ2就是三元组 $(L_*, \Pi_*, \vdash)$
\end{defin}

\section{β规约}

β 规约是一种 λ表达式上的偏序关系,
上述的 λ 演算均具有 β 规约。

\begin{notation}[表达式替换]
记号 $t[x/a]$ 表示将表达式 $t$ 中的变量 $x$ 替换为 $a$,
并额外的 $t[x/x']$ 表示将表达式 $t$ 中的变量替换为一个未曾在 $t$ 中出现
的变量 $x'$
\end{notation}

\begin{defin}[β规约] \label{D.breduce}
β规约$\breduce$是集合$L$即 λ 表达式上满足
    \[ (\lambda x\ b)\ a \breduce b[x/a] \]
且在下述规则下闭合的最小关系
\[ \begin{array}{lcrcl}
    P \breduce P' &\Rightarrow&\forall x \in V&:&\lambda x.P 
    \breduce \lambda x. P'\\
    P \breduce P' &\Rightarrow&\forall Z \in L&:&P\ Z
    \breduce P'\ Z\\
    P \breduce P' &\Rightarrow&\forall Z \in L&:& Z\ P
    \breduce Z\ P'
\end{array} \]
多步 β 规约 $\bbreduce$ 是 $\breduce$ 的传递性自反性闭包,即
$\bbreduce$ 是最小的在下述规则下闭合的关系
\[ \begin{array}{lcl}
    P \breduce P' & \Rightarrow & P \bbreduce P'\\
    P \bbreduce P'\ \land\ P' \bbreduce P'' & \Rightarrow & P \bbreduce
    P''\\ & \Rightarrow & P \bbreduce P
\end{array} \]
\end{defin}

\begin{theo}[多步β规约的唯一性] \label{T.bbreduce.11}
    \[ \forall P\ P'\ P''.\ P \bbreduce P'\ \land\ P \bbreduce P''
    \Rightarrow (P' = P'')\]
\begin{proof} 不是本文的重点,参见 Sørensen 的讲义
    \cite{sorensen2006lecturesC1}。
\end{proof}
\end{theo}

