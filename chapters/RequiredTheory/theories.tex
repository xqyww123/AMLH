\section{一些需要的理论}

本文是计算机领域的论文,故仅将一些需要的理论列在这里而并不去深究更多。

\begin{defin}[函数] \label{Def.Func}
此定义来自 Halmos 的朴素集合论 \cite{halmos2017naive}。
所有从 $U$ 到 $V$ 的函数构成的集合 $U \rightarrow V$ 是
$\powerset(U \times V)$ 的子集
\[ \begin{split}
U \rightarrow&\ V = \{ s \mbar s \in \powerset(U \times V) \ \land\\
&(\Dom s = U)\ \land\ 
(\forall u\ v_1\ v_2.\ (u,v_1) \in s \land (u,v_2) \in s
\Rightarrow (v_1 = v_2))\\
\} \quad &
\end{split} \]
\end{defin}

接下来将简单地定义字符串与字符串代数,
本文后续讨论中的形式系统大多基于此,以此为
形式系统中形式语言(Formal Language)的编纂系统(Collation System)。

使用下划线记号 $\underline{abcd}$ 表示一个单词。
一个单词可以是任何的不被空白分割开的文字、符号,所有的单词构成
无穷集 $\Words$,作为形式语言的字母表。
本文不更深入地探讨 $\Words$ 的本质,这涉及到更多的形式系统的理论而
超出了本文的范畴。
只是简单地举几个例子

\begin{example}[单词的例子]
\begin{align*}
\underline{abc} &\in \Words&\underline{\text{单词}} &\in \Words&
\underline{(} &\in \Words&\underline{)} &\in \Words&\\
\underline{\forall} &\in \Words&\underline{:} &\in \Words&
\underline{\quad} &\notin \Words&\underline{\text{这里有\ \ 空格}} &\notin \Words&
\end{align*}
\end{example}

\begin{defin}[字符串代数] \label{D.StringAlgebra}
字符串代数(String Algebra)是满足如下条件的
集合 $\String$ 与其上的拼接函数 $\concat : \String \rightarrow 
\String \rightarrow \String$

\begin{itemize}
\item $\Words \subseteq \String$
\item $\forall a,\ b,\ c \in \String.
(a \concat (b \concat c) = (a \concat b) \concat c)$ 即
拼接具有结合律。
\item $\forall a \in \Words.\ \nexists b\ c.\ a = b \concat c$
即单词是原子的。
\item $\forall s \in \String.\ \exists b_1,\cdots,b_n
\in \Words.\ b_1 \concat b_2 \concat \cdots \concat b_n = a$ 这也意味着
$\String$ 中任意的字符串都由有限个单词组成。
\item $\forall s,\ x,\ y_1,\ y_2 \in \String.\ (a = x\concat y_1)\ 
\land\ (a = x\concat y_2) \Rightarrow (y_1 = y_2)$
即所谓{\it 唯一解读性}({\it unique readability})
\end{itemize}

\begin{notation}[字符串的记号]
使用包含空格的下划线文本表示多个单词拼接成的字符串。
\[ \underline{a\ b}\quad\text{表示}\quad \underline{a} \concat
\underline{b}\]
额外的,斜体表示字符串变量。
\begin{example}[一些下划线记号表示的字符串]
\begin{align*}
&\underline{a:b}\quad\text{表示}\quad a \concat\underline{:}\concat b&
&\underline{(\ u\ v_1\ v_2\ )}\quad\text{表示}\quad
\underline{(}\concat u \concat v_1 \concat v_2 \concat\underline{)}&
\end{align*}
\end{example}
\end{notation}
\end{defin}

\begin{notation}[语法限定的字符串集合]
使用大尖括号 $\bnf{Grammar}$ 表示所有满足语法 $Grammar$ 的字符串构成的集合。
\begin{example}[一个语法限定的字符串集合]
\[ \bnf{A:B}\quad\text{表示}\quad\{\underline{a:b}
\mbar a \in A \land b \in B\}\]
\end{example}
\end{notation}



