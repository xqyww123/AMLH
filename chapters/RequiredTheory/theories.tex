\section{一些需要的理论}

本文是计算机领域的论文,故仅将一些需要的理论列在这里而并不去深究更多。

\begin{defin}[函数] \label{Def.Func}
此定义来自 Halmos 的朴素集合论 \cite{halmos2017naive}。
所有从 $X$ 到 $Y$ 的函数构成的集合 $X \rightarrow Y$ 是
$\powerset(X \times Y)$ 的子集。
\[ \begin{split}
X \rightarrow&\ Y = \{ s \mbar s \in \powerset(X \times Y) \ \land\\
&(\Dom s = X)\ \land\ 
(\forall x\ y_1\ y_2.\ (x,y_1) \in s \land (x,y_2) \in s
\Rightarrow (y_1 = y_2))\\
\} \quad &
\end{split} \]
函数$f : X \rightarrow Y$的定义域为 $\Dom f = X$,值域为 $\Ima f = Y$。
\end{defin}

\begin{defin}[有限映射] 一个函数 $f : X \rightarrow Y$ 若其定义域 $X$。
    有限,这种函数也被叫做有限映射,记作 $f : X \mapsto Y$。
    数学上有限映射就是函数是不需要特殊处理的,此处之所以单独地提出是
    因为计算机科学的范畴内,有限映射可以使用哈希表这一数据结构表示,
    可以容易地对有限映射进行修改。
    这种修改表示为运算 $(\fupdate) : (X \times Y) \times (X \mapsto
    Y) \rightarrow (X \mapsto Y)$
    \[ (x,y) \fupdate f = \lambda v.\ \xif v = x \xthen y \xelse f\ v \]
    以及删除运算 $(\fdelete) : (X \mapsto Y) \times X \rightarrow
    (X \mapsto Y)$
    \[ \Dom (f \fdelete x) = \Dom f - \{x\}\quad\quad\land\quad\quad
    (f \fdelete x)\ v = f\ v\]
    合并运算 $\multiset : (X \mapsto Y) \times (X \mapsto Y) \rightarrow
    (X \mapsto Y)$
    \[ \Dom f \multiset g = \Dom f \cup \Dom g \quad\quad\land\quad\quad
    (f \multiset g)\ x = \xif x \in \Dom f \xthen f\ x \xelse g\ x\]
    此处 $X, Y$ 指代某个集合,在计算机科学中这被叫做泛型(Generic),
    在数学中 $\fupdate,\ \fdelete,\ \multiset$ 
    属于泛函而不是函数,因为其定义域并非确定。
\end{defin}

作为计算机科学领域的文章,泛函与函数的区别不影响本文的严谨性,而细致
的纠纷只会干扰论述而影响读者的理解,本文选择性地忽视泛函与函数间
的区别而在以下统称为函数。

\begin{defin}[其他一些基础的函数(泛函)]
\begin{align*}
    &\K x\ y = x&&\I x = x&
\end{align*}
\end{defin}

接下来将简单地定义字符串与字符串代数,
本文后续讨论中的形式系统大多基于此,以此为
形式系统中形式语言(Formal Language)的编纂系统(Collation System)。

使用下划线记号 $\underline{abcd}$ 表示一个单词。
一个单词可以是任何的不被空白分割开的文字、符号,所有的单词构成
无穷集 $\Words$,作为形式语言的字母表。
本文不更深入地探讨 $\Words$ 的本质,这涉及到更多的形式系统的理论而
超出了本文的范畴,
只是简单地举几个例子。

\begin{example}[单词的例子]
\begin{align*}
\underline{abc} &\in \Words&\underline{\text{单词}} &\in \Words&
\underline{(} &\in \Words&\underline{)} &\in \Words&\\
\underline{\forall} &\in \Words&\underline{:} &\in \Words&
\underline{\quad} &\notin \Words&\underline{\text{这里有\ \ 空格}} &\notin \Words&
\end{align*}
\end{example}

\begin{defin}[字符串代数] \label{D.StringAlgebra}
字符串代数(String Algebra)是满足如下条件的
集合 $\String$ 与其上的拼接函数 $\concat : \String \rightarrow 
\String \rightarrow \String$。

\begin{itemize}
\item $\Words \subseteq \String$。
\item $\forall a,\ b,\ c \in \String.
(a \concat (b \concat c) = (a \concat b) \concat c)$ 即
拼接具有结合律。
\item $\forall a \in \Words.\ \nexists b\ c.\ a = b \concat c$
即单词是原子的。
\item $\forall s \in \String.\ \exists b_1,\cdots,b_n
\in \Words.\ b_1 \concat b_2 \concat \cdots \concat b_n = a$ 这也意味着
$\String$ 中任意的字符串都由有限个单词组成。
\item $\forall s,\ x,\ y_1,\ y_2 \in \String.\ (a = x\concat y_1)\ 
\land\ (a = x\concat y_2) \Rightarrow (y_1 = y_2)$
即所谓{\it 唯一解读性}({\it unique readability})。
\end{itemize}

\begin{notation}[字符串的记号]
使用包含空格的下划线文本表示多个单词拼接成的字符串。
\[ \underline{a\ b}\quad\text{表示}\quad \underline{a} \concat
\underline{b}\]
额外的,斜体表示字符串变量。
\begin{example}[一些下划线记号表示的字符串]
\begin{align*}
&\underline{a:b}\quad\text{表示}\quad a \concat\underline{:}\concat b&
&\underline{(\ u\ v_1\ v_2\ )}\quad\text{表示}\quad
\underline{(}\concat u \concat v_1 \concat v_2 \concat\underline{)}&
\end{align*}
\end{example}
\end{notation}
\end{defin}

\begin{notation}[语法限定的字符串集合]
使用大尖括号 $\bnf{Grammar}$ 表示所有满足语法 $Grammar$ 的字符串构成的集合。
\begin{example}[一个语法限定的字符串集合]
\[ \bnf{A:B}\quad\text{表示}\quad\{\underline{a:b}
\mbar a \in A \land b \in B\}\]
\end{example}
\end{notation}

最后本文还需要 λ 演算相关理论,λ 演算因为过于基础,本文将其介绍放于
附录 \ref{Ch.lambda},有需要的读者可以自行参阅。

