\begin{abstract}
\pagenumbering{roman}

软件的正确性由其设计的正确性与其程序实现的正确性构成。
软件设计的正确与否也许是难以评判的。
而对程序实现正确性验证的探索,围绕着形式化方法这种技术,
自计算机科学伊始之时的1940年代就开始了。
近百年过去了,大量的成果涌现。
程序实现的正确性是可以被彻底而严谨地形式证明的。
且已有大量的形式化方法验证程序实现的正确性,并同样有很多方法构造
被证明的正确实现的程序。
但怪相是,近百年过去了,这些方法仅被应用在部分安全严苛领域,
而在剩余的安全严苛领域与更广泛的普通软件开发领域,
实现上的缺陷始终存在,且一直在造成各种严重的损失。

本文提出一种新的形式方法,核心思想是,
    \begin{center} \it 演绎定理以构建程序, \end{center}
于是 1. 可以超越类型系统而重新寻找更适合程序开发的形式系统,本文提出了
一种名为 Noesis 的形式系统。
2. 可以在一个成熟的证明工具上演绎与证明定理,以程序开发,
    并由证明工具提供编程辅助与程序的形式化验证。
本文在一个证明系统内构建了一台抽象机并实现了一套在此抽象机上
开发程序的软件工程方案,包括最终编译到现实计算机的功能。
以力图严谨而彻底证明实现正确性的同时兼顾证明复杂度、编程复杂度、
可执行程序的时空性能,进而第一易于工程实践,第二可生产具有实际工业价值
的程序,第三产品的实现正确性被严谨而彻底地证明。
最后此方法实验性地实现在区块链智能合约平台这种新兴的安全严苛场景上,
作为其可行性的验证。

\keywords{形式化方法,机器证明,类型系统,编程语言,智能合约}
\end{abstract}
