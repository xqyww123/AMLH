\begin{abstract}
\pagenumbering{roman}

%软件的正确性由其设计的正确性与其程序实现的正确性组成。
%软件设计的正确与否也许是难以评判的。
%而对程序实现正确性验证的探索,围绕着形式化方法特别是其中的形式化验证,
%自计算机科学伊始的1940年代就开始了。
%近百年过去了,大量的成果涌现,已经有很多方法能够证明程序实现的正确性,
%并构造这些正确性被证明的程序实现。
%然而,因为实践上的困难,特别是对于复杂软件过于高昂的证明成本,
%这些方法并未普遍地用于普通软件工业领域。
%程序实现上的缺陷始终存在,且一直在造成各种严重的损失。
%本文提出一种新的形式方法,试图完整地证明程序实现的正确性,并保持合理的成本。
%但这些方案中大多数都只是可靠(soundness)却并不是性质完备(completeness of 
%property)的,
%只能证明智能合约的部分性质进而只能削除部分缺陷;而那些性质完备的方案,
%均难以实践,即便是对一些非常短小的智能合约。
%本文认为对智能合约的性质完备验证的主要困难之一是

智能合约因其特性一旦被部署就难以更改,任何的缺陷在部署后都难以
修补而会造成巨大的损失,特别是智能合约往往被应用在事关重大
财产安全的金融领域,因此对智能合约的形式化验证非常有意义。
但智能合约的形式化验证与其他领域的形式化验证一样,面临现实程序的复杂性。
具体的代码形式的程序表达缺失了很多开发者的逻辑信息,
而从具体代码中还原开发者的意图是困难的。
若能表达并得到开发者原始的逻辑与意图,则能有效辅助形式化验证。

本文首先提出一种新的公理化程序构造逻辑,{\it Noesis 逻辑},以及 Noesis 逻辑中的考察
对象,{\it 抽象语义}。
Noesis 逻辑中,一个值或一段程序的抽象语义是一个易于分析的抽象数学对象,
用以完整地表达开发者的逻辑信息与意图。
基于 Noesis 逻辑本文设计了一种智能合约构建方法,开发者通过演绎 Noesis 定理
构建具有显式抽象语义的程序,这种方法要求用户在构建程序的同时
也显式地将逻辑信息与意图完整地表达在抽象语义中。
因此,抽象语义完整地承载了值或程序的所有开发者所关切的性质,
对值或程序的验证就可以简化为对抽象语义的验证。
本文提供工具 \Eamlh 实现了此构建方法。\Eamlh 构建在 HOL 交互式定理证明器上,
将 Noesis 逻辑实现在 HOL 逻辑上,抽象语义表达为 HOL 逻辑中任意的数学对象。
HOL 逻辑类似高阶逻辑能够表达几乎所有的逻辑信息,
于是 \Eamlh 实现的抽象语义有能力完整地表达用户任意的,可由程序实现的,
逻辑信息与意图。
且抽象语义是 HOL 逻辑上易于分析的数学对象,
\Eamlh 允许用户后续使用 HOL 定理证明器证明抽象语义的任何性质,
以证明智能合约具有此性质,进而证明智能合约的任何性质。因此
对具象的复杂程序的验证转化为对易于分析的抽象语义的验证,就被有效简化。
最后 \Eamlh 能够将程序编译成区块链平台 EOS.IO 上可用的智能合约,
且编译过程亦被验证。
论文在 EOS.IO 平台实际测试,实现并验证了一段代币合约,测试结果表明 \Eamlh 初步有能力构建
较高执行性能并易于被形式化验证的智能合约。
%因此 \Eamlh 给出了一种可靠的(sound)、性质完备的(completed )、
%半自动的形式化验证方法,且在抽象语义的辅助下,比现有的任何完备验证方法更加便捷。



%
%
%
%
%
%
%\Eamlh 提供扎实的程序到抽象语义的证明,而后续的抽象语义的
%具体性质的证明交由专业的专用证明工具。
%
%
%
%
%
%
%
%
%
%
%
%
%
%
%
%
%
%  本文提出一种新的智能合约开发方法与工具 \Eamlh,
%允许构建具有明确的抽象语义的智能合约,且程序与抽象语义间的关联
%通过本文提出的公理系统 {\it Noesis 逻辑} 保障,。
%用户通过演绎 Noesis 定理来构建程序,能始终容易地得到
%程序的抽象语义。
%
%
%
%
%
%  本文构造了一种公理系统,{\it Noesis 逻辑},其上的 {\it Noesis 定理}
%描述并保障抽象语义与程序间的关联,于是抽象语义的性质即是程序的性质,
%对抽象意义的完备验证延伸为对程序的完备验证。
%
%然后 \Eamlh 使用 HOL 定理证明器实现 Noesis 逻辑,
%抽象语义表达为 HOL 逻辑中任意的数学对象,
%而 HOL 逻辑类似高阶逻辑能够表达几乎所有性质,
%于是用户可以完整地保留所有逻辑信息与意图到抽象语义中。
%且抽象语义是 HOL 逻辑上易于分析的数学对象,
%用户后续可以使用 HOL 定理证明器轻易地证明抽象语义的任何性质,
%以证明智能合约具有此性质,进而证明智能合约的任何性质。
%因此 \Eamlh 给出了一种可靠的(sound)、完备的(complete)、
%半自动的形式化验证方法。
%\Eamlh 提供扎实的程序到抽象语义的证明,而后续的抽象语义的
%具体性质的证明交由专业的专用证明工具。
%
%
%
%
%
%
%
%
%用户通过演绎 Noesis 定理来构建程序,可以同时构造抽象语义与
%具体的工程实现。
%
%
%
%
%
%
%
%
%
%
%
%
%
%
%
%  \Eamlh 的一切
%形式化证明均由 HOL 定理证明器验证,是高度可信的。
%
%
%
%
%
%
%
%
%  事实上在
%本文实现的工具 \Eamlh 中,
%
%且 \Eamlh 使用 HOL 定理证明器作为证明引擎,\Eamlh 的一切
%形式化证明均由 HOL 定理证明器验证,是高度可信的。
%
%
%
%
%
%
%
%
%
%
%  Noesis
%系统被实现在 HOL 逻辑上,
%
%
%
%
%
%
%
%
%
%
%
%
%
%
%
%
%故而目前的形式化验证多数只能证明程序一些具象的性质,例如不会
%算术溢出或者不会抛出异常,但却难以证明更抽象的属性,例如
%证明一段程序在正确地判断素数。
%即便开发者可以将程序的设计以某种形式语言精准地形式地描述,
%目前的形式化验证仍只能证明部分性质,而无法证明程序精确地等价于
%此描述。这些形式化验证多数只是可靠的(sound),而不是完备的
%(complete)。
%
%
%
%
%
%
%
%
%
%本文首先提出一种新的公理系统,Noesis 逻辑,描述程序与其抽象语义(Abstract
% Semantic)的关联,以允许证明抽象语义的性质来证明程序的性质,
%于是对程序的形式化验证转化为对抽象语义的验证,形式化验证就被简化。
%然后本文围绕 Noesis 逻辑提出一种技术,通过演绎 Noesis 定理构造具有明确抽象语义的
%程序。如此构造的程序具有良好的执行性能,并且对其的分析转变为更容易的,
%对其抽象语义的分析,进而易于形式化验证,并可以利用已有的交互式定理证明工具最终
%完成对其的形式化验证。
%最终本文设计并实现了工具 \Eamlh 以实现该技术,并完成了编译到智能合约平台 EOS.IO 的编译后端,
%可以生产高执行效率的并且易于形式化分析的智能合约。
%
%指令集、常量集跟它们的抽象语义是 Noesis 逻辑的公理;
%Noesis 逻辑上的定理表达,由这些
%指令与常量组合而成的某个程序在此公理下所具有的抽象语义。
%当作为公理的指令与常量的抽象语义正确地映射到某个执行环境时,
%Noesis 逻辑上的定理即正确反映此执行环境上的程序的抽象语义。
%若指令与常量存在到某个执行环境的机器代码的映射,则 Noesis 逻辑上的程序
%可以编译到此执行环境。
%
%Noesis 逻辑可以实现在经典逻辑上或者说作为经典逻辑的子集,
%在本文中它被实现在 HOL 定理证明器的 HOL 逻辑上,
%程序的抽象语义被自然地表达为 HOL 证明器上的数学对象,
%可以使用 HOL 证明器分析与证明抽象语义的性质。
%于是对程序的形式化验证变为对其抽象语义的形式化验证,而抽象语义是易于
%分析的,于是形式化验证就被有效简化。
%

\keywords{公理系统,形式化验证,类型系统,智能合约,程序语义}
\end{abstract}
