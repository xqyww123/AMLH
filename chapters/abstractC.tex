\begin{abstract}
\pagenumbering{roman}

%软件的正确性由其设计的正确性与其程序实现的正确性组成。
%软件设计的正确与否也许是难以评判的。
%而对程序实现正确性验证的探索,围绕着形式化方法特别是其中的形式化验证,
%自计算机科学伊始的1940年代就开始了。
%近百年过去了,大量的成果涌现,已经有很多方法能够证明程序实现的正确性,
%并构造这些正确性被证明的程序实现。
%然而,因为实践上的困难,特别是对于复杂软件过于高昂的证明成本,
%这些方法并未普遍地用于普通软件工业领域。
%程序实现上的缺陷始终存在,且一直在造成各种严重的损失。
%本文提出一种新的形式方法,试图完整地证明程序实现的正确性,并保持合理的成本。

本文首先提出一种新的公理系统,Noesis 系统,描述程序与其抽象语义(Abstract
 Semantic)的关联,以允许证明抽象语义的性质来证明程序的性质,
于是对程序的形式化验证转化为对抽象语义的验证,形式化验证就被简化。
然后本文围绕 Noesis 系统提出一种技术,通过演绎 Noesis 定理构造具有明确抽象语义的
程序。如此构造的程序具有良好的执行性能,并且对其的分析转变为更容易的,
对其抽象语义的分析,进而易于形式化验证,并可以利用已有的交互式定理证明工具最终
完成对其的形式化验证。
最终本文设计并实现了工具 \Eamlh 以实现该技术,并完成了编译到智能合约平台 EOS.IO 的编译后端,
可以生产高执行效率的并且易于形式化分析的智能合约。

指令集、常量集跟它们的抽象语义是 Noesis 系统的公理;
Noesis 系统上的定理表达,由这些
指令与常量组合而成的某个程序在此公理下所具有的抽象语义。
当作为公理的指令与常量的抽象语义正确地映射到某个执行环境时,
Noesis 系统上的定理即正确反映此执行环境上的程序的抽象语义。
若指令与常量存在到某个执行环境的机器代码的映射,则 Noesis 系统上的程序
可以编译到此执行环境。

Noesis 系统可以实现在经典逻辑上或者说作为经典逻辑的子集,
在本文中它被实现在 HOL 定理证明器的 HOL 逻辑上,
程序的抽象语义被自然地表达为 HOL 证明器上的数学对象,
可以使用 HOL 证明器分析与证明抽象语义的性质。
于是对程序的形式化验证变为对其抽象语义的形式化验证,而抽象语义是易于
分析的,于是形式化验证就被有效简化。


\keywords{公理系统,形式化验证,类型系统,智能合约,程序语义}
\end{abstract}
