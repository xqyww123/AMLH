\begin{abstract}
\pagenumbering{roman}

软件的正确性由其设计的正确性与其程序实现的正确性构成。
软件设计的正确与否也许是难以评判的。
而对程序实现正确性验证的探索,围绕着形式化方法这种技术,
自计算机科学伊始之时的1940年代就开始了。
近百年过去了,大量的成果涌现。
程序实现的正确性是可以被彻底而严谨地形式证明的。
且已有大量的形式化方法验证程序实现的正确性,并同样有很多方法构造
被证明的正确实现的程序。
但怪相是,近百年过去了,这些方法仅被应用在部分安全严苛领域,
而在剩余的安全严苛领域与更广泛的普通软件开发领域,
实现上的缺陷始终存在,且一直在造成各种严重的损失。
本文尝试总结阻碍这些形式化方法被普遍软件工业界接受的因素,
并基于本文提出的逻辑虚拟机构造一种新的形式化方法,它第一
易于工业实践,第二能生产正确性被彻底证明的程序实现。
最后此方法实验性地实现在区块链智能合约平台这种新兴的安全严苛场景上,
作为其可行性的验证。

\keywords{形式化方法}
\end{abstract}
