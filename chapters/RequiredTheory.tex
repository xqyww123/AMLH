\chapter{必要的理论基础与工程工具介绍}

\amlh 是 \HOL 上的 \lamst 演算,
正式论述前有必要事先引入 \lamst 演算的正式定义与 \HOL 交互式定理证明器的证明系统。

\section{\lamst 演算引论}

Sørensen 在其讲义中对$\lambda$演算的形式定义非常简洁,本节主要基于此讲义第一章与第三章
\cite{sorensen2006lecturesC1}。

\begin{defin}[$\lamst$演算] \label{Def.slam}
\begin{enumerate}
\item $V$为无穷的字母表,表示符号。

\begin{equation}
V = \{v_0,v_1,\cdots\}
\end{equation}

\item 字符串集合 $\Lambda$ 是 $\lambda$ 演算上的表达式,由如下语法定义。

\begin{equation}
\Lambda \Coloneqq V\ |\ (\Lambda\ \Lambda)\ |\ (\lambda V \ \Lambda)
\end{equation}

简写 $\lambda x_1\ \lambda x_2\ \cdots\ \lambda x_n\ y$ 为
    $\lambda x_1\ x_2\ \cdots\ x_n.\ y$

\item 集合$U$是另一个无穷的字母表,表示类型变量集(type variables)。

\begin{equation}
U = \{\alpha,\beta,\cdots\} 
\end{equation}

\item 字符串集合 $\Pi$ 表示简单类型。

\begin{equation} \label{Pi}
 \Pi \Coloneqq U\ |\ (\Pi \rightarrow \Pi)
\end{equation}

\item 集合$L$ 表示 $\lamst$ 上的语言 。

    \[ L ::= \Lambda : \Pi \]

\item 集合 $C$ 表示上下文,是由语法$V : \Pi$定义的字符串集合的幂集。

\begin{equation}
    C = \powerset \{x:\tau\ |\ x \in V, \tau \in \Pi\}
\end{equation}

即$C$ 是所有具有如下形式的集合。

        \[ \{x_1:\tau_1, \cdots, x_n : \tau_n\} \]

其中 $x_1,\cdots,x_n \in V$,$\tau_1,\cdots,\tau_n \in \Pi$。

\item 定义上下文$\Gamma = \{x_1:\tau_1,\cdots,x_n:\tau_n\}$ 的符号域 $\mathrm{dom}$

\[ \mathrm{dom}(\Gamma) = \{x_1,\cdots,x_n\}\]

将$\Gamma_1 \cup \Gamma_2$ 写作 $\Gamma_1, \Gamma_2$ 当
        $\mathrm{dom}(\Gamma_1) \cap \mathrm{dom}(\Gamma_2)$ 时。

\item 定义上下文$\Gamma = \{x_1:\tau_1,\cdots,x_n:\tau_n\}$ 的类型域 $\rvert \Gamma \lvert$

\[ \lvert \Gamma \rvert = \{\tau_1,\cdots,\tau_n\}\]

\item 由如下规则定义$C \times L$ 上的二元关系 $\vdash$

\hfill

\begin{minipage}[b]{0.2\linewidth}
\begin{prooftree}
\AxiomC{$\ $}
\UnaryInfC{$\Gamma, x : \tau \vdash x : \tau$}
\end{prooftree}
\end{minipage}%
\begin{minipage}[b]{0.3\linewidth}
\begin{prooftree}
\AxiomC{$\Gamma, x : \sigma \vdash M : \tau$}
\UnaryInfC{$\Gamma \vdash \lambda x. M : \sigma \leftarrow \tau$}
\end{prooftree}
\end{minipage}
\begin{minipage}[b]{0.3\linewidth}
\begin{prooftree}
\AxiomC{$\Gamma \vdash M : \sigma \leftarrow \tau$}
\AxiomC{$\Gamma \vdash N : \sigma$}
\BinaryInfC{$\Gamma \vdash M N : \tau$}
\end{prooftree}
\end{minipage}

\hfill

\item 简单类型$\lambda$演算$\lamst$就是三元组 $(\Lambda, \Pi, \vdash)$

\end{enumerate}
\end{defin}

\section{定理证明工具}

定理证明工具(Theorem Prover)是一个发展已久的领域,有众多优秀的成果。
著名的工具包括 HOL 系列,Coq\cite{coq.itp},PVS\cite{pvs.itp},Twelf\cite{twelf.itp},
ACL2\cite{acl2.itp},Isabelle/HOL\cite{isabelle.itp}。
其中 HOL 系列起源于 1972年 Robin Milner 的 LCF 程序\cite{milner1972logic},
经过近半个世纪的发展已经演化出诸多版本与分支,HOL Light\cite{hol.light.itp},
HOL4\cite{hol4.itp} 等是其中的佼佼者。

本工作使用 HOL4 作为定理证明器。

HOL4 是一种交互式定理证明工具,允许用户交互式地决定证明策略而后基于策略进行自动证明,
故而是一种半自动工具。而数学上已知不存在一个确定的算法解决任意的问题,故而不可能有
完全自动的证明工具,故而必须通过半自动的方式届由用户决定的策略完成证明。


