\chapter{必要的理论基础与工程工具介绍}


\amlh 是 \HOL 上的 \lamst 演算,
正式论述前有必要事先引入 \lamst 演算的正式定义与 \HOL 交互式定理证明器的证明系统。

\section{记号与注意事项}

本工作在形式系统的记号上继承 Greg 在 \textit{An introduction to substructural logics} 
第二章前几节形式定义的表达方式\cite{restall2002introduction},
本文不再赘述,如有需要可自行查阅。

\section{一些需要的理论}

本文是计算机领域的论文,故仅将一些需要的理论列在这里而并不去深究更多。

\begin{defin}[函数] \label{Def.Func}
此定义来自 Halmos 的朴素集合论 \cite{halmos2017naive}。
所有从 $X$ 到 $Y$ 的函数构成的集合 $X \rightarrow Y$ 是
$\powerset(X \times Y)$ 的子集。
\[ \begin{split}
X \rightarrow&\ Y = \{ s \mbar s \in \powerset(X \times Y) \ \land\\
&(\Dom s = X)\ \land\ 
(\forall x\ y_1\ y_2.\ (x,y_1) \in s \land (x,y_2) \in s
\Rightarrow (y_1 = y_2))\\
\} \quad &
\end{split} \]
函数$f : X \rightarrow Y$的定义域为 $\Dom f = X$,值域为 $\Ima f = Y$。
\end{defin}

\begin{defin}[有限映射] 一个函数 $f : X \rightarrow Y$ 若其定义域 $X$。
    有限,这种函数也被叫做有限映射,记作 $f : X \mapsto Y$。
    数学上有限映射就是函数是不需要特殊处理的,此处之所以单独地提出是
    因为计算机科学的范畴内,有限映射可以使用哈希表这一数据结构表示,
    可以容易地对有限映射进行修改。
    这种修改表示为运算 $(\fupdate) : (X \times Y) \times (X \mapsto
    Y) \rightarrow (X \mapsto Y)$
    \[ (x,y) \fupdate f = \lambda v.\ \xif v = x \xthen y \xelse f\ v \]
    以及删除运算 $(\fdelete) : (X \mapsto Y) \times X \rightarrow
    (X \mapsto Y)$
    \[ \Dom (f \fdelete x) = \Dom f - \{x\}\quad\quad\land\quad\quad
    (f \fdelete x)\ v = f\ v\]
    合并运算 $\multiset : (X \mapsto Y) \times (X \mapsto Y) \rightarrow
    (X \mapsto Y)$
    \[ \Dom f \multiset g = \Dom f \cup \Dom g \quad\quad\land\quad\quad
    (f \multiset g)\ x = \xif x \in \Dom f \xthen f\ x \xelse g\ x\]
    此处 $X, Y$ 指代某个集合,在计算机科学中这被叫做泛型(Generic),
    在数学中 $\fupdate,\ \fdelete,\ \multiset$ 
    属于泛函而不是函数,因为其定义域并非确定。
\end{defin}

作为计算机科学领域的文章,泛函与函数的区别不影响本文的严谨性,而细致
的纠纷只会干扰论述而影响读者的理解,本文选择性地忽视泛函与函数间
的区别而在以下统称为函数。

\begin{defin}[其他一些基础的函数(泛函)]
\begin{align*}
    &\K x\ y = x&&\I x = x&
\end{align*}
\end{defin}

接下来将简单地定义字符串与字符串代数,
本文后续讨论中的形式系统大多基于此,以此为
形式系统中形式语言(Formal Language)的编纂系统(Collation System)。

使用下划线记号 $\underline{abcd}$ 表示一个单词。
一个单词可以是任何的不被空白分割开的文字、符号,所有的单词构成
无穷集 $\Words$,作为形式语言的字母表。
本文不更深入地探讨 $\Words$ 的本质,这涉及到更多的形式系统的理论而
超出了本文的范畴,
只是简单地举几个例子。

\begin{example}[单词的例子]
\begin{align*}
\underline{abc} &\in \Words&\underline{\text{单词}} &\in \Words&
\underline{(} &\in \Words&\underline{)} &\in \Words&\\
\underline{\forall} &\in \Words&\underline{:} &\in \Words&
\underline{\quad} &\notin \Words&\underline{\text{这里有\ \ 空格}} &\notin \Words&
\end{align*}
\end{example}

\begin{defin}[字符串代数] \label{D.StringAlgebra}
字符串代数(String Algebra)是满足如下条件的
集合 $\String$ 与其上的拼接函数 $\concat : \String \rightarrow 
\String \rightarrow \String$。

\begin{itemize}
\item $\Words \subseteq \String$。
\item $\forall a,\ b,\ c \in \String.
(a \concat (b \concat c) = (a \concat b) \concat c)$ 即
拼接具有结合律。
\item $\forall a \in \Words.\ \nexists b\ c.\ a = b \concat c$
即单词是原子的。
\item $\forall s \in \String.\ \exists b_1,\cdots,b_n
\in \Words.\ b_1 \concat b_2 \concat \cdots \concat b_n = a$ 这也意味着
$\String$ 中任意的字符串都由有限个单词组成。
\item $\forall s,\ x,\ y_1,\ y_2 \in \String.\ (a = x\concat y_1)\ 
\land\ (a = x\concat y_2) \Rightarrow (y_1 = y_2)$
即所谓{\it 唯一解读性}({\it unique readability})。
\end{itemize}

\begin{notation}[字符串的记号]
使用包含空格的下划线文本表示多个单词拼接成的字符串。
\[ \underline{a\ b}\quad\text{表示}\quad \underline{a} \concat
\underline{b}\]
额外的,斜体表示字符串变量。
\begin{example}[一些下划线记号表示的字符串]
\begin{align*}
&\underline{a:b}\quad\text{表示}\quad a \concat\underline{:}\concat b&
&\underline{(\ u\ v_1\ v_2\ )}\quad\text{表示}\quad
\underline{(}\concat u \concat v_1 \concat v_2 \concat\underline{)}&
\end{align*}
\end{example}
\end{notation}
\end{defin}

\begin{notation}[语法限定的字符串集合]
使用大尖括号 $\bnf{Grammar}$ 表示所有满足语法 $Grammar$ 的字符串构成的集合。
\begin{example}[一个语法限定的字符串集合]
\[ \bnf{A:B}\quad\text{表示}\quad\{\underline{a:b}
\mbar a \in A \land b \in B\}\]
\end{example}
\end{notation}

最后本文还需要 λ 演算相关理论,λ 演算因为过于基础,本文将其介绍放于
附录 \ref{Ch.lambda},有需要的读者可以自行参阅。


\chapter{λ 演算介绍} \label{Ch.lambda}

Sørensen 在其讲义中对$\lambda$演算的形式定义非常简洁,
本节直接从 $\lamst$ 开始介绍,朴素 $\lambda$ 演算请参见其讲义第一章,
 $\lamst$参考自讲义第三章,$\lambda2$参考自讲义第十三章
\cite{sorensen2006lecturesC1}。

\section{λ→}

λ→ 演算有两种样式,Curry 式和 Church 式的,本文延续 Sørensen 的提法,
简单地说 Curry 式的λ→演算为λ→演算,而说Church式的为Church式λ→演算。
这两种样式的λ→ 演算本质是相同的,Sørensen 在其讲义中对此有论述,
本文不再赘述。

\begin{defin}[$\lamst$演算] \label{Def.slam}
\begin{enumerate}
\item $V$为无穷的字母表,表示符号。

\begin{equation}
V = \{v_0,v_1,\cdots\}
\end{equation}

\item 字符串集合 $L$ 是无类型 $\lambda$ 演算上的表达式($\lambda$ term),由如下语法定义。
\begin{equation}
L = \bnf{V\ |\ (L\ L)\ |\ (\lambda V \ L)}
\end{equation}

简写 $\underline{(\ \lambda\ x_1\ (\ \lambda\ x_2\ \cdots\ (\ 
    \lambda\ x_n\ y\ )\ \cdots\ )\ )}$ 为
    $\lambda x_1\ x_2\ \cdots\ x_n.\ y$

\item 集合$U$是另一个无穷的字母表,表示类型变量集(type variables)。
\begin{equation}
U = \{\alpha,\beta,\cdots\} 
\end{equation}

\item 字符串集合 $\Pi$ 表示简单类型。
\begin{equation} \label{Pi}
    \Pi = \bnf{U\ |\ (\Pi \rightarrow \Pi)}
\end{equation}

\item 集合 $C$ 表示上下文,是由语法$V : \Pi$定义的字符串集合的幂集。
\begin{equation}
    C = \powerset \bnf{V:\Pi}
\end{equation}
即$C$ 是所有具有如下形式的集合。
        \[ \{x_1:\tau_1, \cdots, x_n : \tau_n\} \]
其中 $x_1,\cdots,x_n \in V$,$\tau_1,\cdots,\tau_n \in \Pi$。

\item 定义上下文$\Gamma = \{x_1:\tau_1,\cdots,x_n:\tau_n\}$ 的符号域 $\mathrm{dom}$
\[ \mathrm{dom}(\Gamma) = \{x_1,\cdots,x_n\}\]
将$\Gamma_1 \cup \Gamma_2$ 写作 $\Gamma_1, \Gamma_2$ 当
        $\mathrm{dom}(\Gamma_1) \cap \mathrm{dom}(\Gamma_2)$ 时。


%\item 定义上下文$\Gamma = \{x_1:\tau_1,\cdots,x_n:\tau_n\}$ 的类型域 $\rvert \Gamma \lvert$
%
%\[ \lvert \Gamma \rvert = \{\tau_1,\cdots,\tau_n\}\]
%
\item $\mathcal{L} = \bnf{L : \Pi}$ 是λ→表达式,
由如下规则定义$C \times \mathcal{L}$ 上的二元关系 $\vdash$

\hfill

\begin{minipage}[b]{0.5\linewidth}
\begin{prooftree}
\AxiomC{$\ $} \RightLabel{(公理)}
\UnaryInfC{$\Gamma, x : \tau \vdash x : \tau$}
\end{prooftree}
\end{minipage}%
\begin{minipage}[b]{0.4\linewidth}
\begin{prooftree}
\AxiomC{$\Gamma, x : \sigma \vdash M : \tau$} \RightLabel{(抽象律)}
\UnaryInfC{$\Gamma \vdash \lambda x. M : \sigma \rightarrow \tau$}
\end{prooftree}
\end{minipage}

\hfill

\begin{minipage}[b]{0.5\linewidth}
\begin{prooftree}
\AxiomC{$\Gamma \vdash M : \sigma \rightarrow \tau$}
\AxiomC{$\Gamma \vdash N : \sigma$} \RightLabel{(组合律)}
\BinaryInfC{$\Gamma \vdash M N : \tau$}
\end{prooftree}
\end{minipage}\begin{minipage}[b]{0.5\linewidth}
\begin{prooftree}
\AxiomC{$\Gamma \vdash (\lambda v\ M)\ x : \tau$}
\RightLabel{($\beta$规约)}
\UnaryInfC{$\Gamma \vdash M[v/x] : \tau$}
\end{prooftree}
\end{minipage}

\hfill

\end{enumerate}

简单类型$\lambda$演算$\lamst$就是三元组 $(L, \Pi, \vdash)$
\end{defin}


\section{Church式λ→}

Church式λ→ 与 Curry 式 λ→ 的主要区别是,Church 式的抽象中的变量需要
显示地标记类型。即Curry式的抽象写作
\[ \lambda x.\ x : \sigma \rightarrow \sigma \]
而 Church 式的写作
\[ \lambda x{:}\sigma.\ x : \sigma \rightarrow \sigma \]

\begin{defin}[Church式λ→]
\begin{gather}
V_C = \bnf{V:\Pi} \\
L_C = \bnf{V \mbar (L\ L) \mbar (\lambda V_C\ L)} \\
\mathcal{L}_C = \bnf{L_C : \Pi}
\end{gather}
定义 $C \times \mathcal{L}_C$ 上的关系 $\vdash$

\begin{minipage}[b]{0.5\linewidth}
\begin{prooftree}
\AxiomC{$\ $} \RightLabel{(公理)}
\UnaryInfC{$\Gamma, x : \tau \vdash x : \tau$}
\end{prooftree}
\end{minipage}%
\begin{minipage}[b]{0.4\linewidth}
\begin{prooftree}
\AxiomC{$\Gamma, x : \sigma \vdash M : \tau$} \RightLabel{(抽象律)}
\UnaryInfC{$\Gamma \vdash \lambda x{:}\sigma. M : \sigma \rightarrow \tau$}
\end{prooftree}
\end{minipage}

\begin{prooftree}
\AxiomC{$\Gamma \vdash M : \sigma \rightarrow \tau$}
\AxiomC{$\Gamma \vdash N : \sigma$} \RightLabel{(组合律)}
\BinaryInfC{$\Gamma \vdash M N : \tau$}
\end{prooftree}

Church 式$\lamst$就是三元组 $(L_C, \Pi, \vdash)$
\end{defin}

\section{λ2}

现在介绍λ2 演算,它有很多名字,System F,二阶λ演算,Girard–Reynolds
多态λ演算。
λ2 演算基于Church式λ→演算论述。

\begin{defin}[λ2]
\[ L_* = \bnf{L \mbar \Lambda\ U\ L_*} \]
并类似地记号 $\Lambda t_1\ t_2\ \cdots\ t_n.\ b$ 表示
    $\underline{\Lambda\ t_1\ \Lambda\ t_2\ \cdots\ \Lambda\ t_n\ b}$,
\[ \Pi_* = \bnf{\Pi \mbar \forall\ U\ \Pi_*} \]
记号 $\forall \tau_1\ \tau_2\ \cdots\ \tau_n.\ b$ 表示
$\underline{\forall\ \tau_1\ \forall\ \tau_2\ \cdots\ \forall\
    \tau_n\ b}$
\[ C_* = \powerset \bnf{V:\Pi \mbar U:*} \]
$\mathcal{L}_* = \bnf{L_* : \Pi_*}$ 是λ2表达式,
定义 $C_* \times \mathcal{L}_*$上的关系 $\vdash$

\hfill

\begin{minipage}[b]{0.5\linewidth}
\begin{prooftree}
\AxiomC{$\ $} \RightLabel{(公理)}
\UnaryInfC{$\Gamma, x : \tau \vdash x : \tau$}
\end{prooftree}
\end{minipage}%
\begin{minipage}[b]{0.4\linewidth}
\begin{prooftree}
\AxiomC{$\Gamma, x : \sigma \vdash M : \tau$} \RightLabel{(抽象律)}
\UnaryInfC{$\Gamma \vdash \lambda x{:}\sigma. M : \sigma \rightarrow \tau$}
\end{prooftree}
\end{minipage}

\begin{prooftree}
\AxiomC{$\Gamma \vdash M : \sigma \rightarrow \tau$}
\AxiomC{$\Gamma \vdash N : \sigma$} \RightLabel{(组合律)}
\BinaryInfC{$\Gamma \vdash M N : \tau$}
\end{prooftree}

\begin{minipage}[b]{0.5\linewidth}
\begin{prooftree}
\AxiomC{$\Gamma,\ \alpha:* \vdash M : \tau$}
\RightLabel{(全称抽象律)}
\UnaryInfC{$\Gamma \vdash \Lambda \alpha\ M : \forall \alpha\ \tau$}
\end{prooftree}
\end{minipage}\begin{minipage}[b]{0.5\linewidth}
\begin{prooftree}
\AxiomC{$\Gamma_1 \vdash M : \forall \alpha\ \sigma$}
\AxiomC{$\Gamma_2 \vdash \tau : *$}
\RightLabel{(全称组合律)}
\BinaryInfC{$\Gamma_1,\ \Gamma_2 \vdash M\ \tau : \sigma$}
\end{prooftree}\end{minipage}

\hfill

符号 $*$ 可以看作类型的类型。

多态λ演算λ2就是三元组 $(L_*, \Pi_*, \vdash)$
\end{defin}

\section{β规约}

β 规约是一种 λ表达式上的偏序关系,
上述的 λ 演算均具有 β 规约。

\begin{notation}[表达式替换]
记号 $t[x/a]$ 表示将表达式 $t$ 中的变量 $x$ 替换为 $a$,
并额外的 $t[x/x']$ 表示将表达式 $t$ 中的变量替换为一个未曾在 $t$ 中出现
的变量 $x'$
\end{notation}

\begin{defin}[β规约] \label{D.breduce}
β规约$\breduce$是集合$L$即 λ 表达式上满足
    \[ (\lambda x\ b)\ a \breduce b[x/a] \]
且在下述规则下闭合的最小关系
\[ \begin{array}{lcrcl}
    P \breduce P' &\Rightarrow&\forall x \in V&:&\lambda x.P 
    \breduce \lambda x. P'\\
    P \breduce P' &\Rightarrow&\forall Z \in L&:&P\ Z
    \breduce P'\ Z\\
    P \breduce P' &\Rightarrow&\forall Z \in L&:& Z\ P
    \breduce Z\ P'
\end{array} \]
多步 β 规约 $\bbreduce$ 是 $\breduce$ 的传递性自反性闭包,即
$\bbreduce$ 是最小的在下述规则下闭合的关系
\[ \begin{array}{lcl}
    P \breduce P' & \Rightarrow & P \bbreduce P'\\
    P \bbreduce P'\ \land\ P' \bbreduce P'' & \Rightarrow & P \bbreduce
    P''\\ & \Rightarrow & P \bbreduce P
\end{array} \]
\end{defin}

\begin{theo}[多步β规约的唯一性] \label{T.bbreduce.11}
    \[ \forall P\ P'\ P''.\ P \bbreduce P'\ \land\ P \bbreduce P''
    \Rightarrow (P' = P'')\]
\begin{proof} 不是本文的重点,参见 Sørensen 的讲义
    \cite{sorensen2006lecturesC1}。
\end{proof}
\end{theo}



\section{定理证明工具}

定理证明工具(Theorem Prover)是一个发展已久的领域,有众多优秀的成果
\cite{nawaz2019survey}。
著名的工具包括 HOL 系列,Coq\cite{coq.itp},PVS\cite{pvs.itp},Twelf\cite{twelf.itp},
ACL2\cite{acl2.itp},Isabelle/HOL\cite{isabelle.itp}。
其中 HOL 系列起源于 1972年 Robin Milner 的 LCF 程序\cite{milner1972logic},
经过近半个世纪的发展已经演化出诸多版本与分支,HOL Light\cite{hol.light.itp},
HOL4\cite{hol4.itp} 等是其中的佼佼者。

本工作使用 HOL4 作为定理证明器。

HOL4 是一种交互式定理证明工具,允许用户交互式地决定证明策略而后基于策略进行自动证明,
故而是一种半自动工具。而数学上已知不存在一个确定的算法解决任意的问题,故而不可能有
完全自动的证明工具,故而必须通过半自动的方式届由用户决定的策略完成证明。
个定理证明器得出的结论是正确的”。


\subsection{de Burijn 标准}

机器证明界一个重要的问题是,“如何相信一个定理证明器得出的结论是正确的”,机器证明的正确性。
即所谓 \textit{de Burijn criterion} (de Burijn 标准),若一个定理证明器的结果,
可以被一个独立的简单的方式验证,例如使用一个小的程序或者人工地手算地验证,
那么此定理证明器满足 de Burijn 标准 \cite{barendregt2002autarkic,nawaz2019survey}。

通常的观点认为,HOL系列以及其中的 HOL4 是满足 de Burijn 标准的\cite{nawaz2019survey}。
HOL 系列内部使用一个微小的“核”(kernel)程序,决定了基元推理过程,
一如证明系统中的基元规则,而一切证明与推理过程都是基元推理过程的复合,
一如证明系统中对基元规则的复合。于是所有的推理与证明都经由这个微小的核,
复制核程序或者人工模拟核的运算就可以独立验证任何定理证明器得出的结论。

\subsection{HOL 逻辑概述} \label{Sec.HOL}

HOL 逻辑是 HOL 定理证明器上的证明系统,而 $\amlh$ 构建在 HOL 逻辑上,故有必要介绍 HOL逻辑。

HOL 逻辑的详细定义非常复杂,参见 HOL 系统描述\cite{norrish2019hol},这里仅简单概要
为描述本文工作必须的基础。

\begin{defin}[HOL 逻辑上的表达式与类型]
    集合 $\mathrm{V}_\mathrm{H}$ 表示 HOL 逻辑上的表达式(term),每一个表达式
    $t \in \mathrm{V}_\mathrm{H}$ 拥有确定的类型 $\mathcal{T}(t)$,所有的类型构成集合
    $\mathrm{T}_\mathrm{H}$
\end{defin}

一般使用英文字母$t_1,t_2,\cdots$表示变量,$x_1,x_2,\cdots$表示量词内的变量,
而以小写希腊字母$\alpha,\beta,\cdots$
表示类型变量。如非必要,表示时一般省略表达式的类型。

HOL 逻辑上的推理由一系列规则构成,以下是部分本文需要使用的规则。

\begin{defin}[HOL 逻辑上的部分推理规则]

\hfill

\begin{minipage}[b]{0.5\linewidth}
\begin{prooftree}
    \AxiomC{$ \Gamma_1 \vdash t_1 \Rightarrow t_2 $}
    \AxiomC{$\Gamma_2 \vdash t_1 $}
    \RightLabel{(MP)}
    \BinaryInfC{$\Gamma_1 \cup \Gamma_2 \vdash t_2$}
\end{prooftree}
\end{minipage}%
\begin{minipage}[b]{0.3\linewidth}
\begin{prooftree}
    \AxiomC{$ \Gamma_1 \vdash t_1 = t_2 $}
    \AxiomC{$\Gamma_2 \vdash t_1 $}
    \RightLabel{(EQ\_MP)}
    \BinaryInfC{$\Gamma_1 \cup \Gamma_2 \vdash t_2$}
\end{prooftree} \end{minipage}

\hfill

\begin{minipage}[b]{0.6\linewidth} \begin{prooftree}
    \AxiomC{$ \Gamma_1 \vdash \forall x_1\cdots x_n.\ t_1 \Rightarrow t_2 $}
    \AxiomC{$\Gamma_2 \vdash t_1 $}
    \RightLabel{(MATCH\_MP)}
    \BinaryInfC{$\Gamma_1 \cup \Gamma_2 \vdash \forall x_a\cdots x_k.\ t_2$}
\end{prooftree} \end{minipage}%
\begin{minipage}[b]{0.3\linewidth} \begin{prooftree}
    \AxiomC{$\ $}
    \RightLabel{(ASSUME)}
    \UnaryInfC{$ t \vdash t$}
\end{prooftree} \end{minipage}%

\hfill

\begin{minipage}[b]{0.4\linewidth} \begin{prooftree}
    \AxiomC{$ \Gamma \vdash t $}
    \RightLabel{(DISCH $u$)}
    \UnaryInfC{$\Gamma - \{u\} \vdash u \Rightarrow t$}
\end{prooftree} \end{minipage}%
\begin{minipage}[b]{0.3\linewidth} \begin{prooftree}
    \AxiomC{$ \Gamma \vdash t $}
    \RightLabel{(GEN $x$)}
    \UnaryInfC{$\Gamma \vdash \forall x.\ t$}
\end{prooftree} \end{minipage}


\end{defin}

HOL 定理证明器是基于 SML 语言(标准元语言,Standard Meta Language)实现的,
包括定义、推理、证明在内的操作都需要通过 SML 程序完成。
以下介绍本文构造HOL逻辑上定义的记号,以及这些记号到 SML 代码的对应。

\begin{defin}[用于定义HOL逻辑上新的类型的记号] 使用如下文法表示 HOL 逻辑上的新的类型定义的构造。
    \[ \begin{split} 
        type\_name\ ::= \ & Constructor1\ arg1_1\ \cdots\ arg1_i \mbar \cdots \\
                        \mbar & ConstructorN\ argN_1\ \cdots\ argN_k
    \end{split} \]
\end{defin}

$Constructor1 \cdots ConstructorN$ 表示构造函数的名称,$arg1_1 \cdots arg1_i \cdots argN_k$
均为类型,表示对应构造函数的参数的类型,即各构造函数具有如下类型。

\begin{gather*}
    Constructor1 :\ arg1_1 \rightarrow arg1_2 \cdots \rightarrow arg1_i \rightarrow type\_name \\
    \cdots \\
    ConstructorN :\ argN_1 \rightarrow argN_2 \cdots \rightarrow argN_k \rightarrow type\_name
\end{gather*}

这一记号对应的 SML 代码是:

\begin{lstlisting}[language=ML]
val _ = Datatype `type_name = Constructor1 arg11 ... arg1i | ...
                   | ConstructorN argN1 ... argNk`;
\end{lstlisting}

然后是值与函数的定义。

\begin{defin}[用于定义HOL逻辑上值与函数的记号]
    \[ \begin{split}
        & value \coloneqq expression \\
        & function\ x_1\ \cdots\ x_n \coloneqq expression
    \end{split}   \]

其中 $x_1\cdots x_n$ 为参数符号。可以显示地指定参数类型:
    \[ 
        function\ (x_1:\tau_1) \cdots\ (x_n:\tau_n) \coloneqq expression
    \]
\end{defin}

不造成混淆的情况下可以省略类型指示。

以上对应的 SML 代码是:

\begin{lstlisting}[language=ML]
val value_def = Define `value = expression`;
val func_def = Define `function x1 ... xn = expression`;
\end{lstlisting}

%\section{其他一些需要的理论与工具}
%
%这些逻辑上的理论与工具通用于经典逻辑与大多数非经典逻辑,下文的论述默认在 HOL 逻辑的范畴下。
%
%\begin{defin}[常用逻辑函数]
%\begin{align*}
%    & (\mathrm{I}:\alpha \rightarrow \alpha)\ x\ \coloneqq x \\
%    & (\mathrm{K}:\alpha \rightarrow \beta \rightarrow \alpha)\ x\ y\ \coloneqq x
%\end{align*}
%\end{defin}
%
%\begin{defin}[自然数类型] 遵循于皮亚诺定义方式。
%    \[ \mathrm{number} \Coloneqq 0 \mbar \Suc \mathrm{number} \]
%$\Suc$ 表示后继函数,即一切自然数都是0或者另一个自然数的后继。
%例如 1 是 $\Suc 1$,2 是 $\Suc(\Suc 1)$
%\end{defin}
%
%\begin{defin}[有限映射] \label{Def.FM}
%有限映射是定义域与值域均为有限集合的映射关系,
%从类型$\alpha$到类型$\beta$的有限映射类型写作
%    \[ \alpha \mapsto \beta \]
%有限映射的定义域函数为 $\mathbf{FDom} : (\alpha \mapsto \beta)
%    \rightarrow (\alpha\ \mathrm{set})$,值域函数为
%$\mathbf{FRng} : (\alpha \mapsto \beta)
%    \rightarrow (\beta\ \mathrm{set})$
%\[ \mathbf{FRng}\ f \coloneqq \{f\ x\mbar x \in \mathbf{FDom}\ f\} \]
%有限映射的空集为 $\mathbf{FEmpty}$
%    \[ \mathbf{FDom}\ \mathbf{FEmpty} = \emptyset\]
%有限映射的更新函数 $\fupdate$
%\[ \begin{split}
%\mathbf{FDom}\ ((x, y)\ \fupdate\ f) &= x \INSERT \mathbf{FDom}\ f \\
%    ((x, y)\ \fupdate\ f)\ a &= \xif x = a \xthen y \xelse f\ x
%\end{split} \]
%一切有限映射都是从$\mathbf{FEmpty}$ 开始,经有限次 $\fupdate$ 更新得到。
%\end{defin}
%
