\chapter{必要的理论基础}

\amlh 是 \HOL 上的 \lamst 演算,
正式论述前有必要事先引入 \lamst 演算的正式定义与 \HOL 交互式定理证明器的证明系统。

\section{\lamst 演算引论}

Sørensen 在其讲义中对$\lambda$演算的形式定义非常简洁,本节主要基于此讲义第一章与第三章
\cite{sorensen2006lecturesC1}。

\begin{defin}[$\lamst$演算] \label{Def.slam}
 $\lamst$ 是一个形式系统。
\begin{enumerate}
\item $V$为无穷的字母表,表示符号。

\begin{equation}
V = \{v_0,v_1,\cdots\}
\end{equation}

\item 字符串集合 $\Lambda$ 是 $\lambda$ 演算上的表达式,由如下语法定义。

\begin{equation}
\Lambda \Coloneqq V\ |\ (\Lambda\ \Lambda)\ |\ (\lambda V \ \Lambda)
\end{equation}

\item 集合$U$是另一个无穷的字母表,表示类型变量集(type variables)。

\begin{equation}
U = \{\alpha,\beta,\cdots\} 
\end{equation}

\item 字符串集合 $\Pi$ 表示简单类型。

\begin{equation} \label{Pi}
 \Pi \Coloneqq U\ |\ (\Pi \rightarrow \Pi)
\end{equation}

\item 集合 $C$ 表示上下文,
对应形式系统中推论(consecution)\cite{restall2002introductionP35}中
的前提(antecedent),定义为如下形式。

\begin{equation}
    C ::= <empty> \mbar V : \Pi \mbar C ; C
\end{equation}

\item 形式系统中的主要对象推论(consecution)$S$,定义为如下形式。

\begin{equation}
    S ::= C \vdash \Lambda : \Pi
\end{equation}

\item 接下来定义形式系统的规则。

首先引入三个结构化规则,是变量上下文的交换律与结合律。

\begin{minipage}[b]{0.3\linewidth}
\begin{prooftree}
    \AxiomC{$X;(Y;Z) \vdash v:\tau$}
    \RightLabel{(B)}
    \UnaryInfC{$(X;Y);Z \vdash v:\tau$}
\end{prooftree}
\end{minipage}%
\begin{minipage}[b]{0.3\linewidth}
\begin{prooftree}
    \AxiomC{$(X;Y);Z \vdash v:\tau$}
    \RightLabel{(C)}
    \UnaryInfC{$(X;Z);Y \vdash v:\tau$}
\end{prooftree}
\end{minipage}
\begin{minipage}[b]{0.3\linewidth}
\begin{prooftree}
    \AxiomC{$X;Y \vdash v:\tau$}
    \RightLabel{(CI)}
    \UnaryInfC{$Y;X \vdash v:\tau$}
\end{prooftree}
\end{minipage}


\begin{minipage}[b]{0.2\linewidth}
\begin{prooftree}
\AxiomC{$\ $}
\UnaryInfC{$\Gamma, x : \tau \vdash x : \tau$}
\end{prooftree}
\end{minipage}%
\begin{minipage}[b]{0.3\linewidth}
\begin{prooftree}
\AxiomC{$\Gamma, x : \sigma \vdash M : \tau$}
\UnaryInfC{$\Gamma \vdash \lambda x. M : \sigma \leftarrow \tau$}
\end{prooftree}
\end{minipage}
\begin{minipage}[b]{0.3\linewidth}
\begin{prooftree}
\AxiomC{$\Gamma \vdash M : \sigma \leftarrow \tau$}
\AxiomC{$\Gamma \vdash N : \sigma$}
\BinaryInfC{$\Gamma \vdash M N : \tau$}
\end{prooftree}
\end{minipage}

\hfill

其中第一条与第三条规则中 $x \notin \mathrm{dom}(\Gamma)$ 

\item 简单类型$\lambda$演算$\lamst$就是三元组 $(\Lambda, \Pi, \vdash)$

\end{enumerate}
\end{defin}

\subsection{细粒度抽象机}

\begin{defin}[细粒度抽象机] 基于定义\ref{Def.slam},细粒度抽象机是简单类型$\lambda$演算
的应用。
\begin{enumerate}
  \item 保留 $V$, $\Lambda$ 不变
  \begin{align}
   & V_\mathrm{f} = V & \Lambda_\mathrm{f} = \Lambda
  \end{align}
  \item 令
  \[ U_{\mathrm{f}} = \{ \mathrm{phenomenon} \} \]
  即细粒度抽象机的类型变量集只有一个类型变量 $\mathrm{phenomenon}$,
  根据\ref{Pi}求得对应的 $\Pi_\mathrm{f}$
\end{enumerate}
\end{defin}


