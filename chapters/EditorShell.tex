\chapter{\ES 编辑壳层}

$\amlh$ 已经在上一章论述清楚。但$\amlh$只是一个抽象机理论,尽管被
定义在 HOL 交互式证明工具的 HOL 逻辑上,借助定理证明工具似乎具象了一点,
但显然不能让用户直接操作数学命题与定理。
需要有一个壳层包裹起$\amlh$理论并向外提供给用户编辑程序的功能,
这是本章将论述的编辑壳层\ES 的意义。

\section{\ES 概述}

实际实现中,上一章的 $\amlh$ 运作在 HOL 交互式定理证明器上,是其中的
理论。要操作 $\amlh$ 以在其上开发程序必须届由 HOL 交互式定理证明器完成。
HOL 定理证明器是一个构建在 SML 语言上的工具,或是不准确但更易于理解地说
,一个 SML 语言的库(library)。
HOL 证明器始终运行在 SML 解释器环境下,其执行程序就是加载了 HOL 工具
并被轻微改造过的 SML 解释器。HOL 证明器暴露了一系列 SML 函数并装入
SML 解释环境中,需要通过编写调用这些函数的 SML 语句并交由 SML 解释器
执行以操作 HOL 证明器工作。

这样对于期望使用 $\amlh$ 的用户,这个体系意味着至少要经过两层,
先编写 SML 语句操作 HOL 证明器,进而操作其上定义的 $\amlh$ 理论,
最后用 $\amlh$ 理论中定义的定理与证明策略构造程序。
这显然难以接受。

\ES  就为此而生,构造一个壳层包裹起一切的中间环节而提供给用户
一个简单的接口,允许轻松地在 $\amlh$ 上进行程序开发。

\ES  壳层首先构造了一种形式语言,名为\ES 形式语言。它同样具有 SML 
所拥有的 System F 式的类型系统,却对 System F 进行了改进而拥有更强
的功能,但保持对 SML 的对应,即 \ES  语言可以翻译到 SML 语言。
进而允许 \ES  壳层包裹在 HOL 证明器定制的 SML 解释器外,
\ES  壳层内部使用 \ES  语言而享受到改进的功能,在与 HOL 证明器对接时
将 \ES  语言翻译到 SML 语言再输入 SML 解释器进而控制 HOL 证明器操作
$\amlh$ 理论进行程序开发。

但 \ES  壳层并不强迫用户使用某种文本上书写的语言进行程序开发。
\ES  形式语言是抽象的理论工具,并不是用于开发程序的编程语言。

\ES  壳层设计了一种状态机并实现在 $\Eamlh$ 软件体系中,不妨就叫做
\ES  机器。\ES  机器既是一个抽象理论也是一个被真实实现的
软件工具,是 $\Eamlh$ 软件体系的后端,
内部执行 \ES  语言,外部暴露确定的两种类型的操作。
\ES  机器启动时加载预先编写的
 \ES  语言环境,包括一系列用于 $\amlh$ 上程序开发的
的预定义值与函数。
通过\ES 机器暴露的两种操作控制\ES 机器执行 \ES  语言,进而使用
提供的函数与值进行 $\amlh$ 上的程序开发。

用户并不是使用 \ES  形式语言进行 $\amlh$ 上的程序开发,而是通过
反复调用 \ES  机器暴露的那两种操作改变 \ES  机器的状态进行的。

故严格来说,这种程序开发方式并不强迫用户使用某种编程语言。而同样严格地说
,$\Eamlh$ 方式开发的程序并没有文本形式的源代码,这种开发方式的源代码
即是 \ES  机器的状态。\ES  机器将提供保存 \ES  机器状态到本地文件以及
从文件中重新加载状态的功能,即是源代码的保存与读取,这种功能也是作为
\ES  语言上的函数的形式被提供与被调用。

即 \ES  机器实质是一种集成开发环境(IDE)。

而尚未提及的前端,可以多种多样,它的功能是对接到后端亦即 \ES 机器并将
用户各种形式对前端的输入转换成 \ES 机器开放的那两种操作。前端的输入
形式是不需要确定的,可以是图形图像界面,可以是命令行,甚至是通过语音输入
或者是云端场景下的某个 Web 服务器。本文的工作实现了相当简易的命令行式
的一个前端。

这么做是因为,既然 $\amlh$ 上的程序开发是在数理抽象的氛围中使用形式定理
与数学对象构造程序,在程序的编辑方面,就不应该吝啬地保持在纸上。
既然一切跟逻辑是如此相关,是如此抽象,程序的编写与编辑就该更抽象一点。

任何时刻开发者思维的注意力集中在一个对象上。
而一切操作都是一种编辑壳层上的函数,每一步可以用意味某种操作的函数调用
此对象,或者如果这对象也是个函数,用它去调用别的什么。
然后将操作的结果,即函数执行的结果,作为一个新的对象,放在开发者的
注意力前。

而程序开发,就是不断地重复这一步骤。
这是 $\Eamlh$ 开发方式对程序开发的建模。
而这显然就是一种状态机模型,就是通过 \ES 机器完成的。

泛泛而谈,\ES 机器的模型是一个状态机,类似函数式语言其上的函数
也是一种值,任意时刻都有一个唯一确定的值$x$
作为状态,且可以进行两种操作进行状态转移而步入下一时刻:
\begin{enumerate}
\item 给一个值 $a$,以 $a$ 调用 $f$,将计算后的结果作为新的状态。
\item 给一个值 $f$,以 $x$ 调用 $f$,将计算后的结果作为新的状态。
\end{enumerate}

一切操作都通过编辑壳层的函数完成,不妨将这函数命名为编辑函数(Editing 
Function)。这些作为函数的操作包括,将值装入一个编辑环境变量,
这是传统语言中等号'\texttt{=}'或者其他赋值符号的含义;
定义一个新的编辑函数;将当前对象压入输入栈并开始一个新的输入空间,
这是传统语言中左括号'\texttt{(}'或者左大括号'\texttt{\{}'的含义;
关闭一个输入空间并弹出输入栈,这是传统语言中右括号'\texttt{)}'
或者右大括号'\texttt{\}}'的含义。于是 \ES 壳层上没有关键字,实际上
文本都没有,一切都是编辑函数。


难以论证这种方式比传统在纸上或者文本编辑器里的方式有何优势,
这是一个全新的方式,仅在本文涉及的非常有限的工作中被尝试,
尚没有任何有说服力的统计实验或者数据以支撑。
但它的确比传统的编辑方式抽象的多的多,一切编辑行为作为一种函数,
允许被包含在别的新被定义的函数中,而那些新的函数也是新的编辑操作,
这看起来似乎富有魅力而能提供强大的编辑功能,也许
古老的 Vim 编辑器的编辑指令能作为一个类似案例。
而 $\amlh$ 上的程序其实是些定理证明器上的定理,
根本没有文本形式的源代码。既然程序始终是高度抽象的,
已经是定理证明器证明完成的数学对象,再尝试用文本编辑工具
去修改编辑,不是很寒碜么,不是弄巧成拙么,
不如直接地交互式地输入编辑指令,交互式地执行各种指令以变换作为程序的
定理,以构造出新的定理也是新的程序,就是 \ES 机器的方式。

接下来将分别论述\ES 形式语言、\ES 机器、\ES 环境。

\section{\ES  形式语言}
\ES  形式语言基本是一种具有 System F 式类型系统的函数式程序语言,
且具有到 SML 语言的映射,可以翻译到 SML 。
\ES 的类型系统跟标准 System F 的差异在于,标准 System F 中全称量化必须被
显示地特化成非量化类型后才可被调用,而 \ES 形式语言设计了一种方式直接
调用两个全称量化类型。

尽管标准的 System F 中全称量化与其他要素相同都只是普通的不需要有任何
其他意义的符号,\ES 形式语言这样看待全称量化:
全称量化类型对应于所有可以由其特化而出的非全称量化类型,这些类型构成一
集合叫做特化集,全称量化类型对应于这特化集。即全称量化表示了一系列
具有同一结构的类型。全称量化类型的调用应是这种结构上的调用。
更具体地,两个特化集的各个元素依次调用,所有可行的调用的结果构成了另一个
集合,而可以证明这集合恰好就是某个全称量化类型的特化集。
这一定理意味着全称量化所表示的那些相似结构间确实可以调用,
而调用的结果是调用双方进行最小特化后能满足调用要求的最大泛化的结构。

例如全称量化类型 $\forall \alpha.\ (\alpha,\ \mathrm{int})\ 
\mathrm{pair}\rightarrow (\mathrm{int},\ \alpha)\ \mathrm{pair}$ 在
$\forall \beta.\ (\mathrm{string},\ \beta)\ \mathrm{pair}$ 上的调用,
双方进行所有可能的特化后只有类型 $(\mathrm{string},\ 
\mathrm{int})\ \mathrm{pair} \rightarrow (\mathrm{int},\ 
\mathrm{string})\ \mathrm{pair}$ 与 $(\mathrm{string},\ \mathrm{int})
\ \mathrm{pair}$ 
可以被调用,于是调用结果自然是 $(\mathrm{int},\ \mathrm{string})
\ \mathrm{pair}$。
但若函数的全称量化类型被适当扩大,
$\forall \alpha\ \gamma.\ (\alpha,\ \gamma)\ 
\mathrm{pair}\rightarrow (\gamma,\ \alpha)\ \mathrm{pair}$
那么此时所有诸如 $(\mathrm{int},\ *)\ 
\mathrm{pair}\rightarrow (*,\ \mathrm{int})\ \mathrm{pair}$
与 $(*,\ \mathrm{int})\ \mathrm{pair}$ 的结构的类型都可被调用,
所有的调用结果具有 $(\mathrm{int},\ *)\ \mathrm{pair}$ 结构,
是全称量化类型 $\forall \delta.\ (\mathrm{int},\ \delta)\ 
\mathrm{pair}$ 的特化集。故可以说全称量化类型
$\forall \delta.\ (\mathrm{int},\ \delta)\ \mathrm{pair}$
是全称量化类型 
$\forall \alpha\ \gamma.\ (\alpha,\ \gamma)\ 
\mathrm{pair}\rightarrow (\gamma,\ \alpha)\ \mathrm{pair}$
调用在同为全称量化类型的
$\forall \beta.\ (\mathrm{string},\ \beta)\ \mathrm{pair}$
的结果。
此即全称量化类型的调用。

相应的,全称量化的值也有如此的调用。

现在 $\I : \forall \alpha.\ \alpha \rightarrow
\alpha $ 调用 $x : \forall \alpha.\ \alpha$ 的结果就是
$x : \forall \alpha.\ \alpha$ 其本身。

SML 的类型系统是函数参数的类型是隐式标注的, Curry 式的,
然而 Curry 式的 System F 是不定且无法进行类型推导的。
SML 采用一种 Hindley-Milner
算法下被限制的 System F,而 Hindley-Milner 算法给出了在
近似 System F 类型系统下的自动类型推导算法。XEE!!! 引用!
在 \ES 的情况,类型系统是 Church 式的,用户需要显示地给出
函数参数的类型。这是与 \ES 的函数的构建相关的。

\ES 上,没有文本形式的源代码,用户无法写下一段文本以构建一个新的函数。
\ES 上一切函数的定义都是通过执行一个用于定义函数的函数完成,
即用户调用用于定义函数的函数以定义新的函数。
\ES 是抽象的,函数对应编辑操作,而新建一个编辑器操作自然也是一种操作,
自然也应调用某个函数以进行。

这些都很天马行空,将在接下来几节详细阐述。


%但这并不意味着用户必须显示地标注类型,
%类型是不需要被自动推导的,因为根本没有文本,没有
%文本形式的源代码,一切用户操作都只是 \ES 机器上定义的那两种。
%严格来说,用户没法用什么文本去产生一个
%全新的值,一切值都已经是存在的,而新的值只能通过操纵 \ES 机器
%执行已有的函数的计算而产生,这样实际不需要自动类型推导。甚至函数变量
%也不是由用户创造的,而是计算产生的,\ES 上的一切函数的定义都是
%通过执行一个用于定义函数的函数完成,即用户调用用于定义函数的函数以
%定义新的函数。因为函数对应编辑操作,而新建一个编辑器操作也是一种操作,
%自然也应当调用某个函数以进行。
%最后 \ES 的类型系统完整地包含了 System F。
%
%这意味着 \ES 的类型系统相当的灵活。



\subsection{类型与全称量化类型}

首先定义类型。

\begin{defin}[类型集$\PiE$]
无穷的单词集 $\hat{U}$ 为用于表示类型的字母表
$\hat{U} \subseteq \Words$,
函数$\xa : \hat{U} \mapsto \mathbb{N}$表示类型的元数(Arity),
$\hat{U}$ 中同样无穷的元数为0的子集$\hat{U}_\mathrm{v}$:
$\hat{U}_\mathrm{v} \subseteq \hat{U}\ \land\ \forall u.\ 
u \in \hat{U}_\mathrm{v} \Rightarrow \xa(u) = 0$
表示类型变量,主要用于全称量化。

\ES  的类型集 $\Pi_\mathrm{E}$ 是满足以下条件的$\String$的最小子集
\begin{equation} \label{Def.PiE}
\forall u.\ u \in U \Rightarrow \forall \bm{v}.\ \bm{v} 
\in \Pi_\mathrm{E}^{\xa(u)} \Rightarrow \underline{(\ u\ \bm{v}_1\ 
\cdots\ \bm{v}_{\xa(u)}\ )} \in \Pi_\mathrm{E}
\end{equation}
其中 $\Pi_\mathrm{E}^{\xa(u)}$ 表示 $\Pi_\mathrm{E}$ 的 $\xa(u)$ 维向量
空间,特别的 $\Pi_\mathrm{E}^0 = \{\mathbf{0}\}$
\begin{align*}
    \Uv &= \{\underline{(\ u\ )}\mbar u \in \hat{\mathrm{U}}_v\}&
    U &= \{\underline{(\ u\ )}\mbar u \in \hat{\mathrm{U}}\}&
\end{align*}
\end{defin}

\begin{example}[$\Pi_\mathrm{E}$] 以下命题成立:
\[ \forall u.\ u \in \hat{U}\ \land\ (\xa(u) = 0)\ \Rightarrow\ 
\underline{(\ u\ )} \in \Pi_\mathrm{E} \]
\[ \forall u\ v.\ u \in \hat{U}\ \land\ v \in \hat{U}_\mathrm{v}\ \land\ (\xa(u) = 1)
\ \Rightarrow\ \underline{(\ u\ (\ v\ )\ )} \in \Pi_\mathrm{E} \]
\[ \forall u\ v.\ u \in \hat{U}\ \land\ v \in U_\mathrm{v}\ \land\ (\xa(u) = 1)
\ \Rightarrow\ \underline{(\ u\ v\ )} \in \Pi_\mathrm{E} \]
\[ \forall u\ v.\ u \in \hat{U}\ \land\ v \in \Pi_\mathrm{E}\ \land\ (\xa(u) = 1)
\ \Rightarrow\ \underline{(\ u\ v\ )} \in \Pi_\mathrm{E} \]
\end{example}

\begin{theo}(类型的结构) \label{TS}
\[ \forall u'.\ u' \in \PiE \Rightarrow \exists! u\ \bm{v}.\ 
u \in \hat{U}\ \land\ \bm{v} \in \PiE^{\xa(u)}\ \land\ 
(u' = \underline{(\ u\ \bm{v}_1\ \cdots\ \bm{v}_{\xa(u')}\ )}) \]
\end{theo}
\begin{proof} 首先唯一性是显然的,由字符串理论就可以得到。
对存在性的证明使用反证法,假设
\begin{equation} \label{Hypo.TS}
\exists u'.\ u' \in \PiE \Rightarrow \forall u\ \bm{v}.\ 
u \in \hat{U}\ \land\ \bm{v} \in \PiE^{\xa(u)}\ \land\ 
(u' \neq \underline{(\ u\ \bm{v}_1\ \cdots\ \bm{v}_{\xa(u')}\ )})
\end{equation}
现证明 $\PiE - \{u'\}$ 满足条件 \ref{Def.PiE} 而
$\PiE - \{u'\} \subset \PiE$ 这样就构造了悖论,因为 $\PiE$ 不再是满足
条件 \ref{Def.PiE} 的最小子集,$\PiE - \{u'\}$ 比 $\PiE$更小。
即证明
\[ \forall u.\ u \in \hat{U} \Rightarrow \forall \bm{v}.\ 
\bm{v} \in \PiE^{\xa(u)} - \{u'\} \Rightarrow
\underline{(\ u\ \bm{v}_1\ \cdots\ \bm{v}_{\xa(u)})} \in \PiE - \{u'\} \]
因为有
\[ \forall u.\ u \in \hat{U} \Rightarrow \forall \bm{v}.\ 
\bm{v} \in \Pi_\mathrm{E}^{\xa(u)} \Rightarrow
\underline{(\ u\ \bm{v}_1\ \cdots\ \bm{v}_{\xa(u)})} \in \Pi_\mathrm{E} \]
所以只要证明
\[ \underline{(\ u\ \bm{v}_1\ \cdots\ \bm{v}_{\xa(u)}\ )} \neq u' \]
而由反证假设 \ref{Hypo.TS} 这是成立的,故而悖论被构造进而命题得证。
\end{proof}

定理 \ref{TS} 意味着一切类型$u$都具有且唯一地具有如下格式
\[ \underline{u_0\ \bm{v}_1\ \cdots\ \bm{v}_{\xa(u_0\ )}} \]
其中 $u_0 \in \hat{U}$,$\bm{v}_1,\ \cdots,\ \bm{v}_{\xa(u_0\ )} \in \PiE$

即每一个类型都构成一颗树,类型变量与0元类型构造器是叶子。

\begin{defin}[类型的构造器、元数、高度、变量集、重量] \label{Def.Th}
函数 $\mathrm{c}:\Pi_\mathrm{E} \mapsto \hat{U}$ 表示类型的构造器。
\[ \mathrm{c}\ \underline{(\ u\ v_1\ \cdots\ v_n\ )} = u\]
函数 $\xa:\Pi_\mathrm{E} \mapsto \mathbb{N}$ 表示类型的元数。
\[ \xa\ \underline{(\ u\ v_1\ \cdots\ v_n\ )} = n\]
$\xa:\Pi_\mathrm{E} \mapsto \mathbb{N}$ 不会跟上文定义的
$\xa:\hat{U} \mapsto \mathbb{N}$冲突,因为定义域不重合,且两者具有相同的意义,
不会造成歧义。

\noindent 函数 $\mathrm{h}:\Pi_\mathrm{E} \mapsto \mathbb{N}$ 表示类型的高度。
\[ \mathrm{h}\ \underline{(\ u\ v_1\ \cdots\ v_n\ )} = 1 + \max(\mathrm{h}\ v_1,\ \cdots,\ 
\mathrm{h}\ v_n) \]

\noindent 函数 $\mathrm{v}:\Pi_\mathrm{E} \mapsto \powerset{(\Uv)}$ 
表示类型中的所有变量。
\[ \begin{split}
&\mathrm{v}\ \underline{(\ c\ )}=\xif c \in \Uv \xthen \{
\underline{(\ c\ )}\} \xelse \emptyset \\
&\mathrm{v}\ \underline{(\ u\ v_1\ \cdots\ v_n\ )}=
\bigcup_{i=1\cdots n} \mathrm{v}(v_i)
\end{split} \]
函数 $\TS : \PiE \rightarrow \mathbb{N}$ 表示类型的重量
    \[ \TS \underline{(\ u\ v_1\ \cdots\ v_n\ )} = 1 + 
    \sum_{i=1,\cdots,n} \TS v_i \]
\end{defin}

有如下性质

\begin{lemma}[类型的元数、高度与重量的性质] \label{T.cah}
\[ \forall u.\ \mathrm{h}\ u  \geq 1 \quad\quad\text{(1)}\quad\quad
\quad\quad\quad\forall u.\ \xa(\mathrm{c}\ u) = \xa\ u \quad\quad\text{(2)} \quad\quad\quad
\forall u.\ (\mathrm{h}\ u = 1) \Rightarrow u \in U
\quad\quad\text{(3)} \]
\[ \begin{split}
\forall u.\ (\mathrm{h}\ u > 1) \Rightarrow \exists c\ &\bm{v}.\ c \in \hat{U}
\ \land\ \bm{v} \in \PiE^{\xa(u)}\ \land\ u = \underline{
(c\ \bm{v}_1\ \cdots\ \bm{v}_{\xa(u)}\ )} \ \land\ \\
&(\forall i.\ 1 \leq i \leq \xa(u) \Rightarrow \mathrm{h}\ \bm{v}_i < \mathrm{h}\ u )
\end{split} \tag{4} \]
    \[ \forall u.\ \TS u \geq 1 \quad\quad\text{(5)}\quad\quad\quad
    \forall u.\ (\TS u = 1) \Rightarrow u \in U\quad\quad\text{(6)}\]
\begin{proof} 由定义 \ref{Def.Th} 与定理 \ref{TS} 直接得到。
\end{proof}
\end{lemma}

这样就可以关于类型的高度进行归纳法。以后这些性质将暗含使用而不再另行引证。

\begin{defin}[全称量化类型 $\PiAE$]
集合 $\Pi_\mathrm{E}^\forall$ 表示全称量化类型,由所有满足如下语法的字符串构成。
\[ \Pi_\mathrm{E} \mbar \forall\ \Uv\ \Pi_\mathrm{E} \]
同样有记号 $\forall v_1\ \cdots\ v_n.\ b$ 表示 
$\underline{\forall\ v_1\ \cdots\  \forall\ v_n\ b}$
\end{defin}

\begin{defin}[全称量化类型的相关属性]
函数 $\mathrm{QV} : \Pi_\mathrm{E}^\forall \rightarrow \powerset(
U_\mathrm{v})$ 表示全称量化类型的绑定变量集。
\[ \mathrm{QV}(\forall v_1\ \cdots\ \v_n.\ b) = \{v_1,\ \cdots,\ v_n\} \]
函数 $\mathrm{QB} : \Pi_\mathrm{E}^\forall \rightarrow \Pi_\mathrm{E}$
表示全称量化的类型体。
\[ \mathrm{QB}(\forall v_1\ \cdots\ \v_n.\ b) = b \]
函数 $\Qv : \Pi_\mathrm{E}^\forall \rightarrow \powerset(
U_\mathrm{v})$ 表示全称量化类型所有的变量集。
\[ \Qv q = \QV q \cup \mathrm{v}(\QB q) \]
\end{defin}

\begin{defin}[实例化] \label{Def.inst}
函数 $\mathrm{inst} : (U_\mathrm{v} \rightarrow \Pi_\mathrm{E})
\rightarrow \Pi_\mathrm{E} \rightarrow \Pi_\mathrm{E}$ 对类型
进行变量实例化。
\begin{align*}
\mathrm{inst}\ f\ \underline{(\ v\ )} &= \xif \underline{(\ v\ )}
\in U_\mathrm{v} \xthen
 f\ \underline{(\ v\ )} \xelse \underline{(\ v\ )} \\
\mathrm{inst}\ f\ \underline{(\ u\ v_1\ \cdots\ v_{\xa(u)}\ )} &= 
\underline{(}\concat u \concat \inst f\ v_1\concat\cdots\concat
\inst f\ v_{\xa(u)}\concat\underline{)}
\end{align*}

以及部分实例化 $\inst_V$ 
\[ \inst_V f = \inst\ (\lambda v.\ \xif v \in V \xthen f\ v \xelse v) \]

函数 $\mathrm{inst}_\forall : (U_\mathrm{v} \rightarrow \Pi_\mathrm{E})
\rightarrow \Pi_\mathrm{E}^\forall \rightarrow \Pi_\mathrm{E}$ 
实例化全称量化类型。
\begin{gather*}
\mathrm{inst}_\forall\ f\ q = \inst_{\ \mathrm{QV}(q)} f\ \mathrm{QB}(q)
\end{gather*}
即$\mathrm{inst}_\forall$只会实例化全称量化的类型变量。

\noindent 给定类型变量集 $Q \subseteq \Uv$ 可以将 $\inst_\forall$ 实例化
的结果重新全称量化,函数 $\inst_\forall^Q$
\begin{align*} 
\QB(\inst_\forall^Q\ f\ q) &= \inst_\forall\ f\ q&
    \QV(\inst_\forall^Q\ f\ q) &= Q&
\end{align*}
\end{defin}

\begin{algorithm}
\caption{实例化函数 $\inst$} \label{alg:Iv}
\begin{algorithmic}[1]
\Require 集合 $V \in \powerset(\Uv)$ 表示实例化的范围
\Require 实例化函数 $f : \Uv \rightarrow \PiE$ 表示变量到值的对应
\Require $u \in \PiE,\ u = \underline{(\ c\ v_1\ \cdots\ v_o\ )}$
表示要实例化的目标
\Ensure $\inst_V f\ u \in \PiE$
\If {$o = 0$}
\If {\quad $u \in V$ \quad} \quad 输出 $f(u)$
\Else {\quad 输出 $u$}
\EndIf
\Else
\State $\underline{(\ c} \rightarrow s$
\For{$i=1,\ \dots,\ o$}
\State $s \concat \inst(V,f,v_i) \rightarrow s$
\EndFor
\State 输出 $s \concat \underline{)}$
\EndIf
\TimeComplexity 若 $f$ 的时间复杂度为 $t$ 则算法 inst 的时间复杂度为 $O(t\TS(u))$
\end{algorithmic}
\end{algorithm}
\begin{algorithm}
\caption{全称量化的实例化函数 $\inst_\forall$} \label{alg:IvQ}
\begin{algorithmic}[1]
\Require 实例化函数 $f : \Uv \rightarrow \PiE$ 表示变量到值的对应
\Require $q \in \PiAE,\ q = \underline{\forall\ v_1\ \cdots \forall\ v_p
\ u}$
表示要实例化的目标
\Ensure $\inst_\forall f\ q \in \PiE$
\State $\{\} \rightarrow s$
\For{$i=1,\ \dots,\ p$}
\State 集合 $s$ 加入 $q_i$
\EndFor
\State 调用算法 \ref{alg:Iv}:$\inst(s,f,u)$ 将结果输出。
\TimeComplexity 若 $f$ 的时间复杂度为 $t$ 则算法 inst 的时间复杂度为 $O(t\TS(\QB(u)))$
\end{algorithmic}
\end{algorithm}
\begin{algorithm}
\caption{构造全称量化类型 $\mathrm{MakeQT}$} \label{alg:MakeQT}
\begin{algorithmic}[1]
\Require 集合 $V \in \powerset(\Uv)$
\Require 类型 $u \in \PiE$
\Ensure $q \in \PiAE$ 满足 $(\QV q = V) \ \land\ (\QB q = u)$
\For{$v \in V$}
\State $\underline{\forall} \concat v \concat u \rightarrow u$
\EndFor
\State 输出 $u$
\TimeComplexity $O(\abs{V})$
\end{algorithmic}
\end{algorithm}

\begin{lemma} \label{Lem.Iv.V}
\[ \inst_V f\ u = \inst_{\ V \cap \mathrm{v}(u)} f\ u \]
\begin{proof} 对 $u$ 进行类型高度的归纳法即可。
\end{proof}
\end{lemma}
\begin{lemma} \label{L.Iv.h}
\[ \h(\inst_V f\ u) \geq \h u \]
\begin{proof} 对 $u$ 进行类型高度的归纳法。
\end{proof}
\end{lemma}

\begin{defin}[实例化类型集] \label{Def.QI}
全称量化类型$q$的实例化类型集$\mathrm{QI}\ q$为
\[ \mathrm{QI}\ q = \{\mathrm{inst}_\forall\ f\ q\mbar f \in 
(U_\mathrm{v} \rightarrow \Pi_\mathrm{E})\} \]
\end{defin}

\begin{defin}[α等价] \label{Def.aE}
二元关系$\sim_\alpha$ 定义为
    \[ (q_1 \sim_\alpha q_2) = (\QI q_1 = \QI q_2) \]
显然是一种等价关系,被叫做α等价。
\end{defin}

α等价类$\Pi_\mathrm{E}^\forall/[\sim_\alpha]$即是本质不同的全称量化类型。

\begin{lemma}[实例化类型集非空]
\[\forall q.\ \mathrm{QI}\ q \neq \emptyset\]
\end{lemma}
\begin{proof}
因为 $\quad\forall q.\ \mathrm{inst}_\forall\ \I\ q = \mathrm{QB}\ q 
\quad$ 所以有 $\quad\forall q.\ \mathrm{QB}\ q \in \mathrm{QI}\ q$
\end{proof}

接下来尝试证明一个重要命题 
\[ \forall q_1,\ q_2 \in \Pi_\mathrm{E}^\forall \Rightarrow
(\mathrm{QI}\ q_1 \cap \mathrm{QI}\ q_2 = \emptyset)\ \lor\ 
(\exists q \in \Pi_\mathrm{E}^\forall.\ \mathrm{QI}\ q_1
\cap \mathrm{QI}\ q_2 = \mathrm{QI}\ q) \]
并找到一个算法用于求解上述的 $q$,为此要引入诸多工具。

\subsection{类型匹配系统}

\begin{defin}[类型匹配系统(Type Match System)] 类型匹配系统集 $\TMs$ 是集合
\[ \TMs = \powerset(\Uv) \times \powerset(\PiE \times \PiE) \]
类型匹配系统是所有集合 $\TMs$ 中的元素。
\[ (V,\ \{ (x_1, y_1),\ (x_2, y_2),\ (x_3, y_3),\ \cdots \}) \]
其中 $\{ (x_1, y_1),\ (x_2, y_2),\ (x_3, y_3),\ \cdots \}$ 叫做
匹配系统中的方程组,$x_1 = y_1,\ \cdots$ 是方程组中的方程,
$V \subseteq \Uv$ 是类型匹配系统的变量集。
\end{defin}
\begin{notation}[类型匹配系统的记号] 
记号
\[ \begin{Bmatrix}
x_1 &=& y_1 \\
x_2 &=& y_2 \\
x_3 &=& y_3 \\
&\cdots& 
\end{Bmatrix}_V \]
表示类型匹配系统
\[ (V,\ \{ (x_1, y_1),\ (x_2, y_2),\ (x_3, y_3),\ \cdots \}) \]
\end{notation}
\begin{defin}[类型方程组的重量]
类型方程组的重量函数 $\EQW : (\PiE \times \PiE) \rightarrow \mathbb{N}$
    \[ \EQW X = \sum_{(x,y) \in X} \TS(x) + \TS(y) \]
$\abs{\UV X}$ 表示集合 $\UV X$ 的大小,$\EQW$ 仅在 $X$ 有限时被定义。
\end{defin}
\begin{defin}[类型匹配系统的重量]
类型匹配系统的重量函数 $\TMW : \TMs \rightarrow
    \mathbb{N} \times \mathbb{N}$
    \[ \TMW (V,X) = (\abs{V},\EQW X)\]
同样 $\TMW$ 仅在 $X$ 有限时被定义。
\end{defin}
\begin{defin}[类型匹配系统的解与解集] \label{Def.MF}
一个类型匹配系统$(V,X)$的解$f$是一个$(\Uv \rightarrow \PiE)$函数,
满足
\[ \forall u_1\ u_2.\ (u_1,u_2) \in X \Rightarrow 
\Inst f\ u_1 = \Inst f\ u_2 \]
所有这样的解构成的集合叫解集。

函数 $\MF : \TMs \rightarrow \powerset(\Uv \rightarrow \PiE)$
将一个类型匹配系统映射到其解集
\[ \MF (V,X) = \{ f \mbar \forall u_1\ u_2.\ (u_1,u_2) \in X
\Rightarrow \Inst_V f\ u_1 = \Inst_V f\ u_2 \} \]
\end{defin}

\begin{defin}[类型匹配系统的M等价] \label{Def.Meq}
二元等价关系 $\Meq$
\[ \forall A\ B.\ (A \Meq B) = (\MF A = \MF B) \]
\end{defin}
\begin{lemma} \label{L.Meq.refl}
\[ \forall X\ V\ v.\ (V,X) \Meq (V, X \cup \{(v,v)\}) \]
\begin{proof} 由定义 \ref{Def.Meq} 定义 \ref{Def.MF} 直接得到
\end{proof}
\end{lemma}

\begin{lemma} \label{L.MF.XUX}
\[ \MF (V,X_1 \cup X_2) = \MF(V,X_1) \cap \MF(V,X_2) \]
\begin{proof} 将$\MF$ 的定义展开,命题等价于
\begin{multline*}
(\forall u_1\ u_2.\ (u_1,u_2) \in (X_1 \cup X_2)
\Rightarrow \Inst_V f\ u_1 = \Inst_V f\ u_2) = \\
(\forall u_1\ u_2.\ 
(u_1,u_2) \in X_1
\Rightarrow \Inst_V f\ u_1 = \Inst_V f\ u_2)\ \land\\
(\forall u_1\ u_2.\ (u_1,u_2) \in X_2
\Rightarrow \Inst_V f\ u_1 = \Inst_V f\ u_2)
\end{multline*}
这时显然的。
\end{proof}
\end{lemma}

\begin{defin}[有意义类型方程组] \label{Def.SF}
有意义(Senseful Form)类型方程组函数
$\SF : \powerset(\PiE \times \PiE) \rightarrow \powerset(\PiE \times \PiE)$
\[ \SF X = \{(x,y) \mbar (x,y) \in X \ \land\ x \neq y\} \]
即是削除了恒等式 $(x,x)$ 后的方程。
\end{defin}


\begin{algorithm}
\caption{计算有意义的类型方程组 $\SF X$} \label{alg:SF}
\begin{algorithmic}[1]
\Require 类型方程组 $X$
\Ensure 类型方程组的有意义形式 $\SF X$
\State $\{\} \rightarrow X'$
\For {$(x,y) \in X$}
\If {$x \neq y$} \State 将 $(x,y)$ 加入 $X'$ \EndIf
\EndFor
\State 输出 $X'$
\TimeComplexity $O(\abs{X})$
\end{algorithmic}
\end{algorithm}

\begin{defin}[类型匹配系统的并] \label{Def.M.U}
类型匹配系统 $(V,X)$ 与 $(V',X')$ 的并 $(V,X)\cup(V',X')$ 定义为
\[ (V,X)\cup(V',X') = (V \cup V',\ X \cup X') \]
\end{defin}

\begin{defin}[已解的类型匹配系统] \label{Def.Solved}
类型匹配系统在的{\it 已解}形式下的未知量(Unknown Variable)函数
$\UV : (\PiE \times \PiE) \rightarrow \powerset(\Uv)$
\[ \UV X = \{x \mbar (x,y) \in \SF X\} \]
一个类型匹配系统 $(V,X)$ 若满足 $\Solved (V,X)$ 则被叫做{\it 已解}的。
\[ \begin{split}
\Solved (V,X) = (\forall x\ &y.\ (x,y) \in \SF X \Rightarrow
  x \in V \ \land\ \BV y \cap \UV X = \emptyset)\ \land \\
  & (\forall x\ y_1\ y_2.\ (x,y_1) \in \SF X \ \land\ (x,y_2) \in \SF X
  \Rightarrow (y_1 = y_2))
\end{split} \]
\end{defin}

有一些显然的性质,
\begin{equation} \Solved(V,X) \vdash \UV X \subseteq V \end{equation}
\begin{equation} \Solved(V,X) \vdash \forall x\ y.\ (x,y) \in X
\Rightarrow x \in \UV X \label{UVX} \end{equation}
\begin{equation} \label{Solved.sub.refl}
\Solved(V,X) \vdash \Solved(V,X - {u,u}) \end{equation}

\begin{lemma} \label{Lem.SFXF}
\[ \Solved(V,X) \vdash \SF X \in (\UV X \rightarrow \PiE) \]
即在已解形式下,有意义的方程组 $\SF X$ 就是一个$V$到$\PiE$的函数。
\end{lemma}
\begin{proof}
由定义 \ref{Def.Func},命题等价于
\[ \Solved(V,X) \vdash \forall x\ y_1\ y_2.\ (x,y_1) \in \SF X \ \land\ 
(x,y_2) \in \SF X \Rightarrow (y_1 = y_2) \ \land\ x \in \UV X \ \land\ 
y_1 \in \PiE \]
由定义 \ref{Def.Solved} 这是显然的。
\end{proof}
\begin{lemma}[引理\ref{Lem.SFXF}的推论] \label{Lem.SFXapplied}
结合引理\ref{Lem.SFXF}与公式 \ref{UVX}
\[ \Solved(V,X),\ (x,y) \in X \vdash \SF X\ x = y \]
\end{lemma}

\begin{defin}[部分函数的I扩展]
$\EI\ f$ 将一个部分函数 $f$ 扩展成完全函数
\[ \EI\ f\ x = \xif x \in \Dom f \xthen f\ x \xelse x \]
\end{defin}

\begin{lemma} \label{L.vEi}
\[ x \in V \vdash \mathrm{v}(\Inst_V\ (\EI \{(x,y)\})\ u) =
\mathrm{v}(u) - \{x\} \]
\begin{proof} 对 $u$ 进行类型高度的归纳法,只需要证明叶子类型的情形
命题就能得证。
\[ x \in V,\ u\in \Uv \vdash \mathrm{v}(\Inst_V\ (\EI \{(x,y)\})\ u) =
\mathrm{v}(u) - \{x\} \]
由 $\Inst_V$ 的定义 \ref{Def.Inst}
\[ x \in V,\ u\in \Uv \vdash \mathrm{v}(\EI \{(x,y)\}\ u) =
\mathrm{v}(u) - \{x\} \]
分类讨论 $u = x$ 的情况,结合$\mathrm{v}$的定义 \ref{Def.Th} 命题得证。
\end{proof}
\end{lemma}

\begin{algorithm}
\caption{二元组集的函数扩张 $\mathrm{AsFunc}$} \label{alg:AsFunc}
\begin{algorithmic}[1]
\Require 二元组集 $f \in \powerset(X \times Y)$ 满足
$\forall x\ y_1\ y_2.\ (x,y_1) \in f \ \land\ (x,y_2) \in f \Rightarrow
(y_1 = y_2)$
\Statex 以哈希表形式表达
\Ensure $\EI f \in (X \rightarrow Y)$
\Function{F\ }{$v$}
\If {$\exists y.\ (v,y) \in f$} \Comment {$v$ 是否作为键存在于哈希表 $f$ 中}
\State 返回 $y$ \Else \State 返回 $v$
\EndIf
\EndFunction
\State 输出 F
\TimeComplexity AsFunc 的时间复杂度 $O(1)$,输出 F 的时间复杂度 $O(1)$,因为使用了
    哈希表
\end{algorithmic}
\end{algorithm}

\begin{theo}[已解方程组的解]
\[ \Solved (V,X) \vdash \MF(V,X) = \{ \Inst_V g
\circ \EI (\SF X) \mbar g \} \]
\end{theo}
\begin{proof} 由定义 \ref{Def.MF} 命题等价于
\[ \begin{split}
\Solved&(V,X) \vdash (\forall u_1\ u_2.\ (u_1,u_2) \in X
\Rightarrow \Inst_V f\ u_1 = \Inst_V f\ u_2) = \\
&(\exists g.\ \forall x.\ x \in \Uv \Rightarrow 
(f(x) = \Inst_V g(\EI (\SF X)\ x)))
\end{split} \]
当 $X = \emptyset$ 时等式左边即是 $T$,右边也是 $T$ 因为
$\Dom (\SF X) = \emptyset$ 故而 $\EI(\SF X) = I$ 这样右式就等于
\[ \exists g.\ \forall x.\ x \in \Uv \Rightarrow
(f\ x = \Inst_V\ g\ x)\]
而这样的 $g$ 是始终存在的,令 $g = f$ 则有
\[ \forall x.\ x \in \Uv \Rightarrow (f\ x = \Inst_V\ f\ x)\]
而由 $\Inst_V$ 的定义,当 $x \in \Uv$ 时,$\Inst_V f\ x = f\ x$,
故而右式就是恒真式。

\hfill

\noindent 接下来证明 $X \neq \emptyset$ 的情况,先
证明 $\Rightarrow$,即命题
\[ \begin{split}
X \neq \emptyset,\ 
\Solved(V,X),\ (\forall x\ y.\ &(x, y) \in \SF X \Rightarrow
(\Inst_V f\ x = \Inst_V f\ y)) \vdash\\
&\exists g.\ \forall u.\ u \in \Uv \Rightarrow
(f(u) = \Inst_V g(\EI (\SF X)\ u)) \end{split}\]
存在这样的 $g = f$,命题变为
\[ \begin{split}
X \neq \emptyset,\ 
\Solved(V,X),\ (&\forall x\ y.\ (x, y) \in \SF X \Rightarrow
(\Inst_V f\ x = \Inst_V f\ y)) \vdash\\
&\forall u.\ u \in \Uv \Rightarrow
(f\ u = \Inst_V f(\EI (\SF X)\ u))
\end{split} \]
由引理 \ref{Lem.SFXF},$\Dom(\SF X) = \UV X$,再由$\EI$的定义,
当 $u \notin \UV X$ 时,$\EI(\SF X) u = u$,
而由 $\Inst_V$ 的定义,在$u \in \Uv$时$\Inst_V u = u$,
故而最后只要下述证明命题
\[ \begin{split}
X \neq \emptyset,\ 
Solved(V,X),\ &(\forall x\ y.\ (x, y) \in \SF X \Rightarrow
(\Inst_V f\ x = \Inst_V f\ y)) \vdash\\
&\forall u.\ u \in \UV\ X
\Rightarrow (f\ u = \Inst_V f\ (\SF X\ u)) \end{split} \]
再由公式 \ref{UVX} 且 $X \neq \emptyset$
\[X \neq \emptyset,\ 
\Solved(V,X),\ (x, y) \in \SF X,\ 
\Inst_V f\ x = \Inst_V f\ y \vdash f\ x = \Inst_V f\ (\SF X\ x)\]
引理 \ref{Lem.SFXapplied}
\[X \neq \emptyset,\ 
\Solved(V,X),\ (x, y) \in \SF X,\ 
\Inst_V f\ x = \Inst_V f\ y \vdash f\ x = \Inst_V f\ y\]
而 $\Solved(V,X)\ \land\ (x, y) \in \SF X$ 所以 $x \in \Uv$ 所以
$\Inst_V\ f\ x = x$ 于是命题得证。

\hfill

\noindent 接下来证明 $\Leftarrow$,即命题
\[ \begin{split}
X \neq \emptyset,\ 
\Solved(V,X),\ (\forall u.\ &u \in \Uv \Rightarrow
(f(u) = \Inst_V g(\EI (\SF X)\ u))) \vdash \\
&\forall x\ y.\ (x, y) \in \SF X \Rightarrow
(\Inst_V f\ x = \Inst_V f\ y)
\end{split}\]
等价于
\[ \begin{split}
X \neq \emptyset,\ 
\Solved(V,X),\ (\forall u.\ &u \in \Uv \Rightarrow
(f(u) = \Inst_V g\ (\EI (\SF X)\ u))),\ (x, y) \in \SF X
\vdash \\
& \Inst_V f\ x = \Inst_V f\ y
\end{split}\]
$(x,y) \in \SF X$ 故而 $x \in \UV X$
故而 $x \in \Uv$ 这样 $\Inst_V f\ x = f\ x$ 且
$f\ x = \Inst_V g\ (\EI (\SF X)\ x)$

\noindent 且因为 $x \in \UV X$ 所以 $\EI (\SF X)\ x = \SF X\ x = y$
最后命题等价于
\[ \begin{split}
X \neq \emptyset,\ 
\Solved(V,X),\ (\forall u.\ &u \in \Uv \Rightarrow
(f(u) = \Inst_V g\ (\EI (\SF X)\ u))),\ (x, y) \in \SF X
\vdash \\
& \Inst_V g\ y = \Inst_V f\ y
\end{split}\]
对 $y$ 进行类型高度的递归法,只要证明下式命题就能得证。
\[ \begin{split}
\Solved(V,X),\ (\forall u.\ &u \in \Uv \Rightarrow
(f(u) = \Inst_V g\ (\EI (\SF X)\ u))),\ y \in \Uv,\ (x, y) \in \SF X
\vdash \\
& \Inst_V g\ y = \Inst_V f\ y
\end{split}\]
进而
\[ \begin{split}
\Solved(V,X),\ (\forall u.\ &u \in \Uv \Rightarrow
(f(u) = \Inst_V g\ (\EI (\SF X)\ u))),\ y \in \Uv,\ (x, y) \in \SF X
\vdash \\ & g\ y = f\ y
\end{split}\]
由前提中的 $(\forall u.\ u \in \Uv \Rightarrow
(f(u) = \Inst_V g\ (\EI (\SF X)\ u)))$
\[ \begin{split}
\Solved(V,X),\ (\forall u.\ &u \in \Uv \Rightarrow
(f(u) = \Inst_V g\ (\EI (\SF X)\ u))),\ y \in \Uv ,\ (x, y) \in \SF X
\vdash \\& g\ y = \Inst_V g\ (\EI (\SF X)\ y)
\end{split}\]
注意至今为止一直没用到的 $\Solved$ 的一个条件
\[ \Solved(V,X),\ (x,y) \in \SF X \vdash \BV y \cap \UV X = \emptyset \]
而 $\Dom (\SF X) = \emptyset$,这意味着  $\EI (\SF X)\ y = y$
于是命题变成
\[ \begin{split}
\Solved(V,X),\ (\forall u.\ &u \in \Uv \Rightarrow
(f(u) = \Inst_V g\ (\EI (\SF X)\ u))),\ y \in \Uv ,\ (x, y) \in \SF X
\vdash \\& g\ y = \Inst_V g\ y
\end{split}\]
因为 $y \in \Uv$ 所以 $\Inst_V g\ y = g\ y$ 命题得证。
\end{proof}

\begin{defin}[类型匹配系统的实例集] \label{Def.MI}
类型匹配系统 $(V,X) \in \TMs$ 在 $u \in \PiE$ 上的实例集 
$\MI\ (V,X)\ u \in \powerset(\PiE)$
\[\MI\ (V,X)\ u = \{ \Inst_V f\ u \mbar f \in \MF(V,X) \}\]
\end{defin}
\begin{lemma}[类型匹配系统的实例集的M等价] \label{Lem.MI.Meq}
\[ \forall M_1\ M_2\ u.\ u \in \PiE\ \land\ (M_1 \Meq M_2) 
\Rightarrow (\MI M_1\ u \Meq \MI M_2\ u) \]
\end{lemma}
\begin{proof} 由$\MI$的定义 \ref{Def.MI},M等价的定义 \ref{Def.Meq}
直接得到
\end{proof}

\begin{lemma} \label{Lem.RecIv}
\[ \Solved(V,X) \vdash \Inst_V (\Inst_V g \circ
\EI\ (\SF X)) = \Inst_V g \circ \Inst_V (\EI\ (\SF X))\]
\end{lemma}
\begin{proof} 等价于
\[ \Solved(V,X) \vdash \Inst_V (\Inst_V g \circ
\EI\ (\SF X))\ u = \Inst_V g\ (\Inst_V (\EI\ (\SF X))\ u)\]
对 $u$ 进行高度的归纳法,当 $\h u = 1$ 时,$u = \underline{(\ c\ )}$,
$c \in U$ 由 $\Inst$ 的定义 \ref{Def.Inst} 命题成立。

\noindent 当 $\h u = \Suc n$ 时,$u = \underline{(\ c\ v_1\ \cdots\ v_o\ )}
,\ \h v_1 \leq n,\ \cdots,\ \h v_o \leq n$,且有
\[ \Solved(V,X),\ \h u \leq n \vdash \Inst_V (\Inst_V g \circ
\EI\ (\SF X))\ u = \Inst_V g\ (\Inst_V (\EI\ (\SF X))\ u)
\tag{归纳假设} \]
由 $\Inst$ 的定义 \ref{Def.Inst} 命题等价于
\begin{align*}
\Solved(V,X) \vdash \Inst_V (\Inst_V g \circ
\EI\ (\SF X))\ v_1\ &= \Inst_V g\ (\Inst_V (\EI\ (\SF X))\ v_1)\\
\cdots&\\
\Solved(V,X) \vdash \Inst_V (\Inst_V g \circ
\EI\ (\SF X))\ v_o\ &= \Inst_V g\ (\Inst_V (\EI\ (\SF X))\ v_o)
\end{align*}
逐个应用归纳假设,命题得证。
\end{proof}

\begin{lemma} \label{L.QIMI}
\[ \Solved(V,X),\ \QV q = V,\ \QB q = 
\Inst_V\ (\EI (\SF X))\ u \vdash
\QI q = \MI\ (V,X)\ u \]
\end{lemma}
\begin{proof}
等价于
\[ \begin{split}
\Solved(V,&X),\ \QV q = V,\ \QB q = 
\Inst_V\ (\EI (\SF X))\ u\vdash\\
&\{\Inst_{\ \QV q} f\ \QB(q)\mbar f\} = \{\Inst_V (\Inst_V g \circ
\EI\ (\SF X))\ u \mbar g\} \end{split} \]
由引理 \ref{Lem.RecIv}
\[ \begin{split}
\Solved(V,&X),\ \QV q = V,\ \QB q =  
\Inst_V\ (\EI (\SF X))\ u\vdash\\
&\{\Inst_{\ \QV q} f\ \QB(q)\mbar f\} = \{\Inst_V g\ (
\Inst_V (\EI\ (\SF X))\ u) \mbar g\} \end{split} \]
这是显然成立的。
\end{proof}

引理 \ref{L.QIMI} 意味着很多,首先对于任意的类型匹配系统 $M \in \TMs$
如果其有可解形式 $M'$ 即 $\Solved M'\ \land\ M \Meq M'$,
那么必定存在存在一个 $q \in \PiAE$ 使得 $\QI q = \MI M$,
其次,算法 \ref{alg:QIMI} QIMI 可以由 $\Solved M'$ 求解 $q$,
其正确性直接由引理 \ref{L.QIMI} 得到。

\begin{theo} \label{T.QIMI.S}
\[ \forall M\ u.\ 
\Solved M \Rightarrow \exists q.\ \QI q = \MI M\ u \]
\end{theo}
\begin{proof} 由引理 \ref{L.QIMI} 直接得到。
\end{proof}


\begin{algorithm}
\caption{已解类型匹配系统的实例化 MI} \label{alg:MI}
\begin{algorithmic}[1]
\Require 已解的类型匹配系统$(V,X)$ 满足 $\Solved(V,X)$
\Require 类型 $u \in \PiE$
\Ensure $\Inst_V\ (\EI(\SF X))\ u$
\State $\mathrm{AsFunc}(\SF X) \rightarrow f$
\Comment {调用算法 \ref{alg:SF} 和算法 \ref{alg:AsFunc},
由引理 \ref{Lem.SFXF} $\SF X$ 满足
算法 \ref{alg:AsFunc} 的条件 }
\State 输出 $\Inst(V,f,u)$  \Comment {调用算法 \ref{alg:Iv}}
    \TimeComplexity $O(\abs{X} + \TS u)$
\end{algorithmic}
\end{algorithm}
\begin{algorithm}
\caption{已解类型匹配系统的全称量化类型 QIMI} \label{alg:QIMI}
\begin{algorithmic}[1]
\Require 已解的类型匹配系统$(V,X)$ 满足 $\Solved(V,X)$
\Require 类型 $u \in \PiE$
\Ensure $q \in \PiAE$ 满足 $\QI q = \MI (V,X)\ u$
\State $\MI((V,X),u) \rightarrow b$  \Comment {调用算法 \ref{alg:MI}}
\State 输出 $\mathrm{MakeQT}(V,b)$
\Comment {调用算法 \ref{alg:MakeQT}}
    \TimeComplexity $O(\abs{X} + \abs{V} + \TS u)$
\end{algorithmic}
\end{algorithm}
\begin{algorithm}
\caption{无冲突化量化类型对 $\NC$} \label{alg:NC}
\begin{algorithmic}[1]
\Require $q_1,\ q_2 \in \PiAE,\quad q_1 = \underline{\forall a_1\ \cdots\ 
    \forall a_n\ u},\quad q_2 = \underline{\forall b_1\ \cdots\ 
    \forall b_m\ w}$
\Ensure $q_1', q_2' \in \PiAE$ 满足 $q_1 \sim_\alpha q_1'\ \land\ q_2 \sim_\alpha q_2'
    \ \land\ q_1' \NC q_2'$
\State $\{\} \rightarrow f_1\quad\quad \{\} \rightarrow f_2$
    \Comment {构造 $f_1,\ f_2$ 为 $\Uv$ 到 $\Uv$ 的哈希表}
\For {$i=1,\cdots,n$} \State $\{(a_i,v_i)\} \cup f_1 \rightarrow f_1$ 
\Comment {取 $v_i \in \Uv\quad\quad i=1,\cdots,n+m$}
\EndFor
\For {$i=1,\cdots,m$} \State $\{(b_i,v_{n+i})\} \cup f_2 \rightarrow f_2$ 
\EndFor
\State $\Inst(\Dom f_1,\ f_1,\ u) \rightarrow q_1'\quad\quad \Inst(\Dom
    f_2,\ f_2,\ w) \rightarrow q_2'$ \Comment{算法 \ref{alg:Iv} $\Inst$}
\For {$i=1,\cdots,n$} \State $\underline{\forall}\concat v_i \concat q_1'
    \rightarrow q_1'$ \EndFor
\For {$i=1,\cdots,m$} \State $\underline{\forall}\concat v_{n+i} \concat q_2'
\rightarrow q_2'$ \EndFor
    \State 输出 $(q_1',q_2')$
    \TimeComplexity $O(\abs{\QV q_1}) + \abs{\QV q_2} + \TS(\QB q_1) + \TS(\QB q_2)$
\end{algorithmic}
\end{algorithm}

\begin{defin}[无冲突全称量化类型]
定义$\PiAE \times \PiAE$上的二元关系 $\NC$ 表示两个全称量化类型的全称量化变量不相互冲突。
\[ q_1 \NC q_2 = (\QV q_1 \cap \Qv q_2 = \emptyset) \ \land\ 
(\QV q_2 \cap \Qv q_1 = \emptyset) \]
显然有
\[ q_1 \NC q_2 \vdash \QV q_1 \cap \QV q_2 = \emptyset \]
\end{defin}

\begin{theo}[$\QI$ 等价的无冲量化类型对] \label{T.NCize}
    \[ \forall q_1,\ q_2 \in \PiAE\ \ \exists q_1',\ q_2' \in \PiAE\ \ .\ 
    (\QI q_1 = \QI q_1')\ \land\ (\QI q_2 = \QI q_2')\ \land\ (q_1' \NC q_2')\]
\begin{proof} 算法 \ref{alg:NC} NC 就是寻找这样的 $q_1',\ q_2'$ 且其显然有限停机。
\end{proof}
\end{theo}




\begin{theo}[类型匹配系统与实例化类型集交集的关系] \label{T.TMS.I.I}
\[ q_1 \NC q_2 \vdash \QI q_1 \cap \QI q_2 =\ 
\MI\ (\QV q_1 \cup \QV a_2,\ \{\QB q_1 = \QB q_2\})\ \QB(q_1)\]
\end{theo}
\begin{proof} 简洁起见,令
\begin{align*}
&u_1 = \QB q_1&&u_2 = \QB q_2&\\
&V_1 = \QV q_1&&V_2 = \QV q_2&
\end{align*}
将QI的定义与MI的定义展开,原命题等价于
\[ \begin{split} q_1\ &\NC q_2 \vdash \\
&\{\Inst_{V_1} f\ u_1 \mbar \Inst_{V_1} f\ u_1 = \Inst_{V_2} g\ u_2 \}
= \{\Inst_{V_1 \cup V_2} f\ u_1 \mbar \Inst_{V_1 \cup V_2} f\ u_1
= \Inst_{V_1 \cup V_2} f\ u_2 \} \end{split} \]
由 $q_1 \NC q_2$ 可得
$(V_1 \cup V_2) \cap V_1 = V_1$,
$(V_1 \cup V_2) \cap V_2 = V_2$,
用引理 \ref{Lem.Iv.V},得到
\begin{align*}
&\Inst_{V_1 \cup V_2} f\ u_1 = \Inst_{V_1} f\ u_1&
&\Inst_{V_1 \cup V_2} f\ u_2 = \Inst_{V_2} f\ u_2&
\end{align*}
带入命题,命题变为
\[ q_1\ \NC q_2 \vdash
\{\Inst_{V_1} f\ u_1 \mbar \Inst_{V_1} f\ u_1 = \Inst_{V_2} g\ u_2 \}
= \{\Inst_{V_1} f\ u_1 \mbar \Inst_{V_1} f\ u_1= \Inst_{V_1} f\ u_2 \}\]
显然有
\[ \{\Inst_{V_1} f\ u_1 \mbar \Inst_{V_1} f\ u_1 = \Inst_{V_2} g\ u_2 \}
\supseteq
\{\Inst_{V_1} f\ u_1 \mbar \Inst_{V_1} f\ u_1= \Inst_{V_1} f\ u_2 \}\]
只要证明
\[ q_1\ \NC q_2 \vdash
 \{\Inst_{V_1} f\ u_1 \mbar \Inst_{V_1} f\ u_1 = \Inst_{V_2} g\ u_2 \}
\subseteq
\{\Inst_{V_1} f\ u_1 \mbar \Inst_{V_1} f\ u_1= \Inst_{V_1} f\ u_2 \}\]
即证明命题
\[ q_1\ \NC q_2 ,\ 
\Inst_{V_1} f\ u_1 = \Inst_{V_2} g\ u_2 \vdash \exists h.\ 
\Inst_{V_1} h\ u_1 = \Inst_{V_2} h\ u_2 \]
$h = (\lambda v.\ \xif v \in V_1 \xthen f\ v \xelse g\ v)$ 
是满足上述条件的实例,因为 $V_1 \cap V_2 = \emptyset$

\noindent 这样命题得证。
\end{proof}

\begin{defin}[无解的类型匹配系统] \label{Def.NS}
谓词 $\NoSolution$ 表示一个类型匹配系统是无解的。
\[ \NoSolution M = (\MF M = \emptyset) \]
相反的 $\Solvable$ 表示有解
\[ \Solvable M = (\MF M \neq \emptyset) \]
$\MNS$ 是某个无解的类型匹配系统
\[ \NoSolution \MNS \]
$\MNS$ 具体的表达是无关紧要的,例如下面的类型匹配系统即可
\[ (\emptyset,\ \{u_1 = u_2\})\quad\quad u_1 \neq u_2 \]
\end{defin}

\begin{theo} \label{T.TMS.NS}
    \[ \NoSolution M \vdash \MI M\ u = \emptyset \]
\begin{proof} 由 $\MI$ 的定义 \ref{Def.MI} 与 $\NoSolution$ 的定义 \ref{Def.NS}
    直接得到。
\end{proof}
\end{theo}

\begin{lemma} \label{L.Rec.NoSolution}
    \[ (x,y) \in X,\ x \in \BV y,\ x \neq y \vdash \NoSolution(V,X) \]
\begin{proof}
    弱于以下命题
    \[ x \in \BV y,\ x \neq y \vdash \nexists f.\ \Inst_V f\ x = \Inst_V f\ y \]
证明等价的逆否命题
\[ x \in \BV y,\ \Inst_V f\ x = \Inst_V f\ y \vdash x = y\]
因为 $x \in \Uv$ 命题等价于
\[ x \in \BV y,\ x = \Inst_V f\ y \vdash x = y\]
$\h x = 1$ 于是 $\h(\Inst_V f\ y) = 1$,由引理 \ref{L.Iv.h},
$\h y \leq \h(\Inst_V f\ y)$ 而 $\forall u.\ \h u \geq 1$
于是 $\h y = 1$ 于是 $y \in U$ 于是 $\Inst_V f\ y = y$ 于是命题得证。
\end{proof}
\end{lemma}

\begin{defin}[类型方程组中全部的类型变量]
对类型方程组 $X \in \powerset(\PiE \times \PiE)$
\[ \MV\ X = \bigcup_{(x,y) \in X} \mathrm{v}(x) \cup \mathrm{v}(y) \]
表示 $X$ 中每一个类型方程的所有类型变量的合并。
\end{defin}

\begin{defin}[类型匹配系统的实例化] \label{Def.IM}
$\IM : (\Uv \rightarrow \PiE) \rightarrow \TMs \rightarrow \TMs$ 
将一个类型匹配系统的所有方程的等式两边都实例化
\[ \IM f\ (V,X) = \{(\Inst_V f\ x,\ \Inst_V f\ y) \mbar (x,y) \in X\}\]
\end{defin}

\begin{algorithm}
\caption{类型匹配系统的实例化 $\IM$} \label{alg:IM}
\begin{algorithmic}[1]
\Require 类型匹配系统 $(V,X) \in \TMs$,
实例化函数 $f : \Uv \rightarrow \PiE$ 表示变量到值的对应
\Ensure $\IM\ f\ (V,X) \in \TMs$
\State $\{\} \rightarrow X'$
\For {$(x,y) \in X$}
\State $\Inst(V,f,x) \rightarrow x'\quad\quad
    \Inst(V,f,y) \rightarrow y' \quad\quad
    \{(x',y')\} \cup X' \rightarrow X'$
\Comment {算法 \ref{alg:Iv} $\Inst$}
\EndFor
\State 输出 $X'$
    \TimeComplexity $O(t\EQW(X))$,$t$ 是 $f$ 的时间复杂度
\end{algorithmic}
\end{algorithm}


\begin{lemma} \label{L.IIuv}
\[ (u,v) \in X,\ u \in V,\ \Inst_V g\ u = \Inst_V g\ v,\ x \in \PiE
\vdash \Inst_V g\ (\Inst_V\ \{(u,v)\}\ x) = \Inst_V\ g\ x \]
\begin{proof} 对 $x$ 进行类型高度的归纳法,只需要证明叶子类型时的情况
\[ (u,v) \in X,\ u \in V,\ \Inst_V g\ u = \Inst_V g\ v,\ x \in \Uv
\vdash \Inst_V g\ (\Inst_V\ \{(u,v)\}\ x) = \Inst_V\ g\ x \]
由 $\Inst_V$ 的定义,此时 $\Inst_V\ \{(u,v)\}\ x = \{(u,v)\}\ x$

\noindent 当 $x \neq v$,$\Inst_V\ \{(u,v)\}\ x = x$,命题显然成立

\noindent 当 $x = v$,$\Inst_V\ \{(u,v)\}\ x = v$
\[ (u,v) \in X,\ u \in V,\ \Inst_V g\ u = \Inst_V g\ v,\ x \in \Uv
\vdash \Inst_V g\ v = \Inst_V\ g\ x \]
于前提 $\Inst_V g\ u = \Inst_V g\ v$ 命题得证。
\end{proof}
\end{lemma}

\begin{lemma} \label{L.EM.IMEiX}
\[ (u,v) \in X,\ u \in V \vdash (V,X) 
\Meq (V,(\IM\ \EI\{(u,v)\}\ \ (X - \{(u,v)\})) \cap \{(u,v)\})\]
\begin{proof} 命题等价于
\[ \begin{split} (u,v) \in X,\ &u \in V\ \vdash
\{g \mbar (x,y) \in X\ \land\  \Inst_V g\ x = \Inst_V g\ y \} = \\
\{g \mbar &(x,y) \in (X-\{(u,v)\})\ \land\  \Inst_V g\ (\Inst_V\ \{(u,v)\}\ x) = 
\Inst_V g\ (\Inst_V\ \{(u,v)\}\ y)\ \land\\
&\quad\quad\Inst_V g\ u = \Inst_V g\ v \} \end{split} \]
分步证明等式两端
\begin{align}
\begin{split} (u,v) \in X,\ u \in V,&\ 
\forall x\ y.\ (x,y) \in X\Rightarrow (\Inst_V g\ x = \Inst_V g\ y)
\vdash \\ \forall x\ y.\ (x,y) \in\ (X-\{(u,v)\})&\Rightarrow
(\Inst_V g\ (\Inst_V\ \{(u,v)\}\ x) = 
\Inst_V g\ (\Inst_V\ \{(u,v)\}\ y))\\
&\land\ (\Inst_V g\ u = \Inst_V g\ v)
\end{split} \tag{1}\\ \notag \\
\begin{split} (u,v) \in X,\ u \in V,&\ \Inst_V g\ u = \Inst_V g\ v,\ \\
\forall x\ y.\ 
(x,y) \in (X-\{(u,v)\})&\Rightarrow
(\Inst_V g\ (\Inst_V\{(u,v)\}\ x) = 
\Inst_V g\ (\Inst_V\{(u,v)\}\ y)) \vdash \\
\forall x\ y.\ (x,&y) \in X\Rightarrow  \Inst_V g\ x = \Inst_V g\ y
\end{split} \tag{2}
\end{align}
对于 (1) 有
\[ \begin{split} (u,v) \in X,\ u \in V,\ 
\forall x\ y.\ (x,y)\ &\in X\Rightarrow (\Inst_V g\ x = \Inst_V g\ y)
\vdash \\
\Inst_V g\ u = \Inst_V g\ v
\end{split} \]
对于 (2) ,$\Inst_V g\ u = \Inst_V g\ v$ 就在前提列表中

\noindent 在这两种情况都可以应用引理 \ref{L.IIuv},这样命题就瞬间被简化成非常简单的形式
\begin{align}
\begin{split} (u,v) \in X,\ u \in V,&\ 
\forall x\ y.\ (x,y) \in X\Rightarrow(\Inst_V g\ x = \Inst_V g\ y)
\vdash \\ (\forall x\ y.\ 
(x,y) \in &(X-\{(u,v)\})\Rightarrow \Inst_V g\ x = \Inst_V g\ y)\ 
\land\ (\Inst_V g\ u = \Inst_V g\ v)
\end{split} \tag{1'}\\
\begin{split} (u,v) \in X,\ u \in V,&\ \Inst_V g\ u = \Inst_V g\ v,\\
\forall x\ y.\ &(x,y)
\in (X-\{(u,v)\})\Rightarrow \Inst_V g\ x = \Inst_V g\ y \vdash \\
\forall x\ y.\ (x,y) \in &X\Rightarrow (\Inst_V g\ x = \Inst_V g\ y)
\end{split} \tag{2'}
\end{align}
这些显然是成立的。
\end{proof}
\end{lemma}

\begin{lemma} \label{L.MVIM}
\[ (u,v) \in X,\ u \in V \vdash \MV(\IM\ \EI\{(u,v)\}\ (X-\{(u,v)\}))
= \MV(X) - \{u\} \]
\begin{proof} 由引理 \ref{L.vEi} 以及 $\IM$ 的定义 \ref{Def.IM} 直接得到。
\end{proof}
\end{lemma}

\begin{defin} \label{Def.SBP}
$\SBP\ (V_x,X_x)\ (V_o,X_o)$ 描述类型匹配系统 
$(V_x,X_x),\ (V_o,X_o) \in \TMs$ 具有如下性质
\[ \SBP\ (V_x,X_x)\ (V_o,X_o) = \Solved(V_o,X_o) \ \land\ 
(V_o = \UV X_o)\ \land\ (\MV X_x \cap V_o = \emptyset) \]
\end{defin}
\begin{lemma} \label{L.SBP.1}
\[ \begin{split} \SBP\ &(V_x,X_x)\ (V_o,X_o),\ (x,x) \in X_x \vdash
\SBP\ (V_x,X_x - \{(x,x)\})\ (V_o,X_o)\ \land\\
&((V_x,X_x - \{(x,x)\})\cup(V_o,X_o) \Meq (V_x,X_x)\cup(V_o,X_o)) \end{split}\]
\begin{proof}
$((V_x,X_x - \{(x,x)\})\cup(V_o,X_o) \Meq (V_x,X_x)\cup(V_o,X_o))$
是显然的,由引理 \ref{L.Meq.refl} 直接得到。
现证明
\[ \SBP\ (V_x,X_x)\ (V_o,X_o),\ (x,x) \in X_x \vdash
\SBP\ (V_x,X_x - \{(x,x)\})\ (V_o,X_o)\]
由$\SBP$ 的定义,命题变为
\[ \SBP\ (V_x,X_x)\ (V_o,X_o),\ (x,x) \in X_x \vdash 
(\MV (X_x - \{(x,x)\}) \cap V_o = \emptyset) \]
而这是显然的。
\end{proof}
\end{lemma}

\begin{lemma} \label{L.SBP.2}
\[ \begin{split} \SBP\ &(V_x,X_x)\ (V_o,X_o),\ (x,y) \in X_x,\ 
x = \underline{(\ c\ v_1\ \cdots\ v_n \ )},\ 
y = \underline{(\ c\ w_1\ \cdots\ w_n \ )} \vdash\\
&\SBP\ (V_x,X_x')\ (V_o,X_o)\ \land\ 
((V_x,X_x')\cup(V_o,X_o) \Meq (V_x,X_x)\cup(V_o,X_o)) \end{split}\]
\[ \text{其中\quad} X_x' = X_x - \{(x,y)\} \cup \bigcup_{i=1\cdots n} \{(v_i,w_i)\} \]
\begin{proof}
首先证明 $((V_x,X_x')\cup(V_o,X_o) \Meq (V_x,X_x)\cup(V_o,X_o))$ 部分,

\noindent 这是成立的因为
$x = \underline{(\ c\ v_1\ \cdots
\ v_n\ )},\ y = \underline{(\ c\ w_1\ \cdots\ w_n\ )}$ 下
\[ \{f \mbar \Inst_V f\ x = \Inst_V f\ y \} = 
\{ f \mbar \bigwedge_{i=1\cdots n} \Inst_V f\ v_i = \Inst_V f\ w_i \} \]
然后证明 $\SBP\ (V_x,X_x')\ (V_o,X_o)$ ,只要证明如下命题即可
\[ \begin{split} \SBP\ &(V_x,X_x)\ (V_o,X_o),\ (x,y) \in X_x,\ 
x = \underline{(\ c\ v_1\ \cdots\ v_n \ )},\ 
y = \underline{(\ c\ w_1\ \cdots\ w_n \ )} \vdash\\
&\MV(X_x - \{(x,y)\} \cup \bigcup_{i=1\cdots n} \{(v_i,w_i)\}) = 
\MV(X_x)
\end{split} \]
这是成立的因为
\[ x = \underline{(\ c\ v_1\ \cdots\ v_n \ )},\ 
y = \underline{(\ c\ w_1\ \cdots\ w_n \ )} \vdash
\mathrm{v}(x)\cup\mathrm{v}(y) = \bigcup_{i=1\cdots n} \mathrm{v}(v_i)
\cup \mathrm{v}(w_i) \]
\end{proof}
\end{lemma}

\begin{lemma} \label{L.SBP.subst}
\[ \begin{split} \SBP\ &(V_x,X_x)\ (V_o,X_o),\ (x,y) \in X_x,\ 
x \in V_x,\ x \notin \BV y \vdash\\
&\SBP\ (V_x',X_x')\ (V_o',X_o')\ \land\ 
((V_x',X_x')\cup(V_o',X_o') \Meq (V_x,X_x)\cup(V_o,X_o)) \end{split}\]
其中
\begin{align*}
&X_x' = \IM\ (\EI \{(x,y)\})\ (X_x - \{(x,y)\})&
&V_x' = V_x - \{x\}&\\
&X_o' = X_o \cup \{(x,y)\}&
&V_o' = V_o \cup \{x\}&
\end{align*}
\begin{proof}
令$V = V_x \cup V_o$
\[ V_x' \cup V_o' = (V_x - \{x\}) \cup (V_o \cup \{x\}) = V \]
首先证明 $(V_x',X_x')\cup(V_o',X_o') \Meq (V_x,X_x)\cup(V_o,X_o)$ 部分,即
\[ \begin{split} \SBP\ &(V_x,X_x)\ (V_o,X_o),\ (x,y) \in X_x,\ 
x \in V_x,\ x \notin \BV y \vdash\\
&(\MF(V,X_x' \cup X_o') = \MF(V,X_x \cup X_o)) \end{split}\]
\[ X_x' \cup X_o' = \IM\ (\EI\{(x,y)\})\ (X_x - \{(x,y)\})\cup
\{(x,y)\}\cup X_o\]
由引理 \ref{L.MF.XUX} 命题等价于
\[ \begin{split} \SBP\ &(V_x,X_x)\ (V_o,X_o),\ (x,y) \in X_x,\ 
x \in V_x,\ x \notin \BV y \vdash\\
(\MF(V,\IM\ (\EI\{(x,y)\})\ (X_x - &\{(x,y)\})\cup
\{(x,y)\}) \cap \MF(V,X_o) = \MF(V,X_x) \cap \MF(V,X_o))
\end{split}\]
弱于
\[ \begin{split} \SBP\ &(V_x,X_x)\ (V_o,X_o),\ (x,y) \in X_x,\ 
x \in V_x,\ x \notin \BV y \vdash\\
&(\MF(V,\IM\ (\EI\{(x,y)\})\ (X_x - \{(x,y)\})\cup
\{(x,y)\}) = \MF(V,X_x)
\end{split}\]
而这就是引理 \ref{L.EM.IMEiX} 所证明的。

\hfill

\noindent 接下来证明 $\SBP\ (V_x',X_x')\ (V_o',X_o')$ 部分,命题
\[ \begin{split} \SBP\ &(V_x,X_x)\ (V_o,X_o),\ (x,y) \in X_x,\ 
x \in V_x,\ x \notin \BV y \vdash\\
&\SBP\ (V_x',X_x')\ (V_o',X_o')
\end{split}\]
将$\SBP$ 的定义 \ref{Def.SBP} 展开
\[ \begin{split} 
\Solved&(V_o,X_o)\ \land\ (V_x\cap V_o = \emptyset)\ \land\ 
(\MV X_x \cap V_o = \emptyset),\ (x,y) \in X_x,\ 
x \in V_x,\ x \notin \BV y \vdash\\
&\Solved(V_o',X_o')\ \land\ (V_o' = \UV X_o')\ \land\ 
(\MV X_x' \cap V_o' = \emptyset)
\end{split}\]
$(V_o' = \UV X_o')$ 是显然的。

\noindent $\MV X_x' \cap V_o' = \emptyset$ 由引理 \ref{L.MVIM} 对
$\MV(X_x') = \MV(X_x) - \{x\}$ 的证明得到

\noindent 现证明 $\Solved(V_o',X_o')$,现将 $\Solved$ 的定义 \ref{Def.Solved}
展开
\[ \begin{split}
\Solved (V_o',X_o') = (\forall x\ &y.\ (x,y) \in \SF X_o' \Rightarrow
  x \in V_o' \ \land\ \BV y \cap \UV X_o' = \emptyset)\ \land \\
  & (\forall x\ y_1\ y_2.\ (x,y_1) \in \SF X_o' \ \land\ (x,y_2) \in \SF X_o'
  \Rightarrow (y_1 = y_2))
\end{split} \]
\[ \begin{split}
\Solved (V_o,X_o) = (\forall x\ &y.\ (x,y) \in \SF X_o \Rightarrow
  x \in V_o \ \land\ \BV y \cap \UV X_o = \emptyset)\ \land \\
  & (\forall x\ y_1\ y_2.\ (x,y_1) \in \SF X_o \ \land\ (x,y_2) \in \SF X_o
  \Rightarrow (y_1 = y_2))
\end{split} \]
\begin{align*}
&V_o' = V_o\cup \{x\}&&X_o' = X_o\cup\{(x,y)\}& \end{align*}

因为 $\BV y \subseteq \MV X_x$ 而由命题假设有 $\MV X_x \cap V_o = 
\emptyset$ 于是 $\BV y \cap V_o = \emptyset$
另外 $x \notin \BV y$ 于是
$\BV y \cap (V_o \cup \{x\}) = \emptyset$
这样 $\Solved(V_o',X_o')$ 中的第一部分
\[\forall x\ y.\ (x,y) \in \SF X_o' \Rightarrow
  x \in V_o' \ \land\ \BV y \cap \UV X_o' = \emptyset\]
得证。

又因为 $V_o = \UV X_o$ 于是 $\MV X_x \cap \UV X_o = \emptyset$
而 $x \in \MV X_x$ 于是
\[ \nexists y.\ (x,y) \in X_x \]
这样 $\Solved(V_o',X_o')$ 中的第二部分
\[ (\forall x\ y_1\ y_2.\ (x,y_1) \in \SF X_o \ \land\ (x,y_2)
\in \SF X_o \Rightarrow (y_1 = y_2)) \]
得证。

于是所有子命题得证,引理 \ref{L.SBP.subst} 得证。
\end{proof}
\end{lemma}

\begin{algorithm}
\caption{ $\mathrm{SolveM_r}$ } \label{alg:SolveMr}
\begin{algorithmic}[1]
\Require 任意的类型匹配系统 $(V_x,X_x)$,
已解的类型匹配系统 $\Solved(V_o,X_o)$
\Statex 满足 $X_x \neq \emptyset \ \land\ \SBP\ (V_x,X_x)\ (V_o,X_o)$
\Ensure 类型匹配系统 $(V_x',X_x')$,已解的类型匹配系统 $\Solved (V_o',X_o')$
\Statex 满足 $\SBP\ (V_x,X_x)\ (V_o,X_o)
\ \land\ ((V_x,X_x) \cup (V_o,X_o) \Meq (V_x',X_x') \cup 
(V_o',X_o')) $
\State 从 $X_x$ 中随意取出元素 $(x,y) \in X_x$,
$x = \underline{(\ c\ v_1\ \cdots\ v_n \ )},\ 
y = \underline{(\ d\ w_1\ \cdots\ w_m \ )}$
\State $X_x - (x,y) \rightarrow X_x$
\Comment {接下来将削除方程组中的等式 $(x,y)$ 而保持方程组的解不变}
\If {$x = y$} \label{l:mrif1}
\State 输出 $((V_x,X_x),(V_o,X_o))$ \label{l:mro1}
\Comment{恒等式的情况可以直接削除} \Else
\If {$n \neq 0 \ \land\ m \neq 0$} \label{l:mrif2}
\Comment {若等式两边均不是类型变量或单个类型}
\If {$c = d$}
\Comment {$(c = d) \Rightarrow (n = m)$}
\For {$i = 1\cdots n$}
\Comment {等式$(x,y)$就可以分解为$n$个子等式 $(v_i,w_i)$}
\State $\{(v_i,w_i)\} \cup X_x \rightarrow X_x$ \label{l:mrot1}
\Comment{将这些等式分别加入方程组}
\EndFor
\State 输出 $((V_x,X_x),(V_o,X_o))$ \label{l:mro2}
\Comment{就完成了$(x,y)$ 的削除}
\Else \State 输出 $(\MNS,\MNS)$ \Comment{$c \neq d$ 时,方程组一定无解}
\label{l:mro3} \EndIf
\Else \Comment{等式两边某一方是类型变量或单个类型}
\If {$c \in \Uv\ \lor\ d \in \Uv$} 
\Comment{若等式两边某一方是类型变量}
\State 令 $c,\ d$ 中属于 $\Uv$ 的为 $u$,另一方为 $v$
\State 若 $u \in \BV v$ 则直接输出 $(\MNS,\MNS)$ \label{l:mro6}
\State $\{(u,v)\} \cup X_o \rightarrow X_o\quad,\quad
\{u\} \cup V_o \rightarrow V_o\quad,\quad V_x - \{u\} \rightarrow V_x$
\State $\IM(X_x,\mathrm{AsFunc}(\{(u,v)\})) \rightarrow X_x$
\label{l:mrot2} \Comment {算法 \ref{alg:IM} $\IM$}
\State 输出 $((V_x,X_x'),(V_o,X_o))$ \label{l:mro4}
\Else \Comment{若等式两边都是单个类型而不是类型变量}
\State 输出 $(\MNS,\MNS)$ \Comment{因为$x \neq y$ 这时也是不可解的}
\label{l:mro5} \EndIf
 \EndIf \EndIf
\end{algorithmic}
\end{algorithm}

\begin{algorithm}
\caption{求解类型匹配系统 SolveM } \label{alg:SolveM}
\begin{algorithmic}[1]
\Require 任意的类型匹配系统 $(V,X)$
\Ensure $M' \in \TMs$ 满足 $M'$ 是已解的即 $\Solved M'$ 或者等于 $\MNS$ 
,且 $M \Meq M'$
\State $\{\} \rightarrow V' \quad,\quad \{\} \rightarrow X'$
\While {$(V,X) \neq \MNS \ \land\ X \neq \emptyset$}
\State $\mathrm{SolveM_r}((V,X),(V',X')) \rightarrow ((V,X),(V',X'))$
\Comment {$\mathrm{SolveM_r}$ 是算法 \ref{alg:SolveMr}}
\EndWhile
\State 输出 $(V',X')$
\TimeComplexity $O((\abs{V} + 1)(\EQW(X) + 1))$
\end{algorithmic}
\end{algorithm}


算法 SolveM 用于求解任意给定的类型匹配系统,主要是对子算法 \SolveMr
的迭代。接下来将分析 \SolveMr 的性质以证明 SolveM 的正确性。

\begin{lemma}[\SolveMr 的正确性] \label{L.SMr.Correct}
\begin{multline*}
\SBP\ (V_x,X_x)\ (V_o,X_o),\ X_x \neq \emptyset,\\
(\mathrm{SolveM_r}((V_x,X_x),(V_o,X_o)) = ((V_x',X_x'),(V_o',X_o')))
\vdash\\ \SBP\ (V_x',X_x')\ (V_o',X_o')\ \land\ 
((V_x,X_x) \cup (V_o,X_o) \Meq (V_x',X_x') \cup (V_o',X_o'))
\end{multline*}
\begin{proof} 
行 \ref{l:mro1} 的输出的正确性是引理 \ref{L.SBP.1}。
行 \ref{l:mro2} 的输出的正确性是引理 \ref{L.SBP.2}。
行 \ref{l:mro3} 的情况下,$M_x \cup M_o$ 是无解的因为
$(x,y) \in X_x, x = \underline{(\ c\ v_1\ \cdots
\ v_n\ )},\ y = \underline{(\ d\ w_1\ \cdots\ w_p\ )}$ 时
\[ c \neq d \Rightarrow \nexists f.\ \Inst_V f\ x = \Inst_V f\ y \]
行 \ref{l:mro5} 的无解情况也是一样的。
行 \ref{l:mro6} 的无解情况由引理 \ref{L.Rec.NoSolution} 证明。
行 \ref{l:mro4} 的输出的正确性则是引理 \ref{L.SBP.subst}。

\noindent 所有输出的正确性就被证明。
\end{proof}
\end{lemma}

接下来证明 $\mathrm{SolveM}$ 的停机性。

\begin{defin}[$\mathbb{N} \times \mathbb{N}$ 上的良序 $<$]
$\mathbb{N} \times \mathbb{N}$ 上的良序关系 $<$
    \[ (n_1,w_1) < (n_2,w_2) \Leftrightarrow (n_1 < n_2)\ \lor\ 
    (n_1 = n_2\ \land\ w_1 < w_2)\]
\end{defin}

\begin{lemma}[\SolveMr 重量递减] \label{L.SMr.W}
\[ \begin{split}
\SBP\ (V_x,X_x)\ (V_o,X_o),\ &X_x \neq \emptyset,\ \FINITE\ X_x\\
(\mathrm{SolveM_r}((V_x,X_x),(V_o,&X_o)) = ((V_x',X_x'),(V_o',X_o')))
    \vdash\\ (\EQW X_x' < \EQW\ &X_x)\ \lor\ ((V_x',X_x') = \MNS)
\end{split} \]
\begin{proof}
行 \ref{l:mro1},\ref{l:mro3},\ref{l:mro5},\ref{l:mro6} 是显然的,
行 \ref{l:mro2} 用简单的类型高度的归纳法即可证明,
行 \ref{l:mro4} 因为 $\abs{\UV X_x'} < \abs{\UV X_x}$ 所以也是显然的。
    如此所有的输出语句上的引理 \ref{L.SMr.W} 就被证明。
\end{proof}
\end{lemma}

\begin{theo}[算法 SolveM 的正确性与停机性] \label{T.SM}
    \[\text{当输入的类型方程组有限时,算法 SolveM 是正确且可停机的}\]
\begin{proof} 由字符串代数的定义 \ref{D.StringAlgebra} 任何字符串都是
    有限的,故而类型亦如此。故当输入的类型方程组有限,其重量就有限。
    由引理 \ref{L.SMr.W} 算法 SolveM 就是有限停机的。
    再而由引理 \ref{L.SMr.Correct} 算法 SolveM 是正确的。
\end{proof}
\end{theo}

于是直接地推论:

\begin{theo}[类型匹配系统的求解] \label{T.TMS.S}
\[ \forall M.\ \exists M'.\ (M \Meq M') \ \land\ (\Solved M' \lor 
\NoSolution M') \]
另一个等价的表达是
\[ \forall M.\ \Solvable M \Rightarrow \exists M'.\ (M \Meq M')\ \land\ 
\Solved M'\]
    \begin{proof} 由定理 \ref{T.SM} 算法 SolveM 就是寻找这样的 $M'$ 
        的算法,且其有限停机。
\end{proof}
\end{theo}

\begin{theo}[SolveM 的时间复杂度] \label{T.TMS.S.T}
    \[ \text{SolveM 的时间复杂度为}\quad O((\abs{V} + 1)(\EQW(X) + 1)) \]
\begin{proof} 算法 SolveM 对算法 \SolveMr 的不断迭代中,行 \ref{l:mrot1} 
    花费时间 $O(1)$ 最多执行 $O(\EQW(X))$ 次,行 \ref{l:mrot2} 单次花费时间 $O(\EQW(X))$
    最多执行 $O(\abs{V})$ 次,其余均是常量时间,故算法 SolveM 总的时间复杂度
    $O((\abs{V} + 1)(\EQW(X) + 1))$
\end{proof}
\end{theo}

现在,终于是时候把定理 \ref{T.QIMI.S} 定理 \ref{T.TMS.I.I} 定理 \ref{T.TMS.NS} 
定理 \ref{T.TMS.S} 联系到一起。

\begin{lemma} \label{L.inter.q}
    \[ q_1 \NC q_2 \vdash (\QI q_1 \cap \QI q_2 = \emptyset)\ \lor\ 
    (\exists q.\ \QI q_1 \cap \QI q_2 = \QI q)\]
\begin{proof} 首先定理 \ref{T.TMS.I.I}
\[ q_1 \NC q_2 \vdash \QI q_1 \cap \QI q_2 =\ 
\MI\ (\QV q_1 \cup \QV a_2,\ \{\QB q_1 = \QB q_2\})\ \QB(q_1) \tag{定理 \ref{T.TMS.I.I}}\]
令 $M = (\QV q_1 \cup \QV a_2,\ \{\QB q_1 = \QB q_2\})$ 
\[ q_1 \NC q_2 \vdash \QI q_1 \cap \QI q_2 = \MI\ M\ \QB(q_1) \]
使用 SolveM 算法求解得到
$M' = \mathrm{SolveM}\ M$ 且由定理 \ref{T.TMS.S}
    \[ M \Meq M'\ \land\ (\Solved(M')\ \lor\ \NoSolution(M'))\]
再由引理 \ref{Lem.MI.Meq}
\[ q_1 \NC q_2 \vdash \QI q_1 \cap \QI q_2 = \MI\ M'\ \QB(q_1) \]
对 $\Solved(M')$ 的情况,待证命题就是定理 \ref{T.QIMI.S},对 $\NoSolution(M')$,
待证命题就是定理 \ref{T.TMS.NS} 于是证明完毕。
\end{proof}
\end{lemma}

\begin{algorithm}
\caption{全称量化类型的交 QInter} \label{alg:QInter}
\begin{algorithmic}[1]
\Require 全称量化类型 $q_1,\ q_2 \in \PiAE$
\Ensure $q \in \PiAE$ 满足 $\QI q_1 \cap \QI q_2 = \QI q$,或者 $\QI q_1 \cap \QI q_2 = 
\emptyset$ 时输出 {\it 无解} 。
\State $\NC(q_1,\ q_2) \rightarrow (q_1',\ q_2')$ \Comment{调用算法 \ref{alg:NC} NC}
\State $(\QV q_1' \cup \QV q_2',\ \{\QB q_1' = \QB q_2'\}) \rightarrow M$
\Comment {构造类型匹配系统 $M \in \TMs$}
\State $\mathrm{SolveM}(M) \rightarrow M'$ \Comment{算法 \ref{alg:SolveM} SolveM}
\If {$M' \neq \MNS$}
\State $\mathrm{QIMI}(M',\ \QB q_1') \rightarrow q$ \Comment{算法 \ref{alg:QIMI} QIMI}
\State 输出 $q$
\Else
    \State 输出 {\it 无解}
\EndIf
\TimeComplexity $O(\ \ (\abs{\QV q_1} + \abs{\QV q_2} + 1)\ \ (\TS(\QB q_1) + \TS(
    \QB q_2))\ \ )$
\end{algorithmic}
\end{algorithm}

\begin{theo}[全称量化类型的交] \label{T.inter.q}
    \[\forall q_1,\ q_2 \in \PiAE.\  (\QI q_1 \cap \QI q_2 = \emptyset)\ \lor\ 
    (\exists q.\ \QI q_1 \cap \QI q_2 = \QI q)\]
    \begin{proof} 在引理 \ref{L.inter.q} 的基础上继续应用定理 \ref{T.NCize} 直接得到。
\end{proof}
\end{theo}


算法 QInter 求解这样的 $q$,这样的 $q$ 是 $q_1,\ q_2$ 经最少的实例化操作能得到的
最广的全称量化类型,在下一节类型调用中将发挥重要的作用。



\subsection{类型的调用}

现在讨论诸如$\lamst$以及$\lambda2$中组合律的类型调用

\hfill

\begin{minipage}[b]{0.45\linewidth}
\begin{prooftree}
\AxiomC{$\Gamma \vdash M : \sigma \leftarrow \tau$}
\AxiomC{$\Gamma \vdash N : \sigma$} \RightLabel{(组合律)}
\BinaryInfC{$\Gamma \vdash M N : \tau$}
\end{prooftree}
\end{minipage}\begin{minipage}[b]{0.5\linewidth}
\begin{prooftree}
\AxiomC{$\Gamma_1 \vdash M : \forall \alpha\ \sigma$}
\AxiomC{$\Gamma_2 \vdash \tau : *$}
\RightLabel{(全称组合律)}
\BinaryInfC{$\Gamma_1,\ \Gamma_2 \vdash M\ \tau : \sigma$}
\end{prooftree}\end{minipage}

\hfill

自然地期望在 $\ES$ 中也应有类似的组合运算。
原本的 λ2 演算要求显示地实例化组合双方的类型,直到实例化成不具有全称量化的形式才可以进行
组合,而组合的结果也一定不具有全称量化的。
例如 $\forall x.\ x \rightarrow z \rightarrow x$ 与
$\forall y.\ y$ 进行组合的结果只能是 $z \rightarrow x$ 或 $z \rightarrow y$,
即进行了两次实例化,但其实只要一次实例化就可以的,更好的结果是 $\forall x.\ 
z \rightarrow x$

$\ES$ 对 λ2 主要的改进在于此,这也是几乎 $\ES$ 与 λ2 唯一的不同,$\ES$ 基于
上一节论述的{\it 全称量化的交}提出一种对组合双方的全称量化进行最小实例化以尽可能保持
全称量化而进行组合的运算,称作{\it 全称量化类型的函数调用}。
这种运算只进行必要的实例化而在结果中尽可能地保留了全称量化,在 $\forall x.\ 
x \rightarrow z \rightarrow x$ 与 $\forall y.\ y$ 的例子中,可以得到 
$\forall x.\ z \rightarrow x$。

接下来一步步地正式地数学定义{\it 全称量化类型的函数调用}
(function application of universal quantified type)。

\begin{defin}[函数类型]
$\rightarrow$ 是函数类型的类型构造器。
\begin{align*}
&\underline{\rightarrow} \in \hat{U} - \hat{U}_v&
&\mathrm{a}(\underline{\rightarrow}) = 2&
\end{align*}
诸如以下范式的类型被叫做从 $a$ 到 $b$ 的函数,简称函数。
\[ \underline{(\ \rightarrow\ a\ b\ )} \]
也可以使用记号$a \rightarrow b$表示 $\underline{(\ \rightarrow\ a\ b\ )}$

\noindent 所有的函数构成的集合为 $\PiE^\rightarrow$

\noindent 函数 $\DomF : \PiE^\rightarrow \rightarrow \PiE$
    \[ \DomF(a \rightarrow b) = a \]
函数 $\ImaF : \PiE^\rightarrow \rightarrow \PiE$
    \[ \ImaF(a \rightarrow b) = b \]
\end{defin}

\begin{defin}[普通类型集合的函数调用]
普通类型集合的函数调用是$\powerset(\PiE^\rightarrow) \times \powerset(\PiE)$到
$\powerset(\PiE)$的函数 $(\cdot)$
\[ A \cdot B = \{b \mbar (a \rightarrow b) \in \PiE^\rightarrow\ \land\ a \in \PiE\} \]
\end{defin}

\begin{algorithm}
\caption{全称量化类型的调用 QCall} \label{alg:QCall}
\begin{algorithmic}[1]
\Require 全称量化类型 $q_1,\ q_2 \in \PiAE$
\Ensure $q \in \PiAE$ 满足 $\QI q_1 \cdot \QI q_2 = \QI q$,或者 $\QI q_1 \cdot \QI q_2 = 
\emptyset$ 时输出 {\it 无解} 。
\State $\NC(q_1,\ q_2) \rightarrow (q_1',\ q_2')$ \Comment{调用算法 \ref{alg:NC} NC}
\State $(\QV q_1' \cup \QV q_2',\ \{\DomF \QB q_1' = \QB q_2'\}) \rightarrow M$ \label{l:qc1}
\Comment {构造类型匹配系统 $M \in \TMs$}
\State $\mathrm{SolveM}(M) \rightarrow M'$ \Comment{算法 \ref{alg:SolveM} SolveM}
\If {$M' \neq \MNS$}
\State $\mathrm{QIMI}(M',\ \ImaF \QB q_1') \rightarrow q$ \Comment{算法 \ref{alg:QIMI} QIMI} \label{l:qc2}
\State 输出 $q$
\Else
    \State 输出 {\it 无解}
\EndIf
\TimeComplexity $O(\ \ (\abs{\QV q_1} + \abs{\QV q_2} + 1)\ \ (\TS(\QB q_1) + \TS(
    \QB q_2))\ \ )$
\item[\textbf{说明}] 
这一算法非常类似算法 \ref{alg:QInter} QInter 仅在行 \ref{l:qc1}
,\ref{l:qc2} 有区别,注意 $\DomF$ 和 $\ImaF$ 的调用。
\end{algorithmic}
\end{algorithm}

\begin{lemma} \label{L.QC.I.I}
\[ q_1 \NC q_2 \vdash \QI q_1 \cdot \QI q_2 =\ 
\MI\ (\QV q_1 \cup \QV a_2,\ \{\DomF \QB q_1 = \QB q_2\})\ \ImaF \QB(q_1) \]
    \begin{proof} 效仿定理 \ref{T.TMS.I.I} 的证明。
    \end{proof}
\end{lemma}

\begin{theo}\label{T.QCall}
\[ \forall q_1,\ q_2 \in \PiAE.\ (\QI q_1 \cdot \QI q_2 = \emptyset)\ \lor\ 
(\exists q.\  \QI q_1 \cdot \QI q_2 = \QI q)\]
\begin{proof} 算法 \ref{alg:QCall} QCall 即是寻找这样的 $q$,且其有限停机。
算法 \ref{alg:QCall} QCall 的正确性与时间复杂度的证明
    效仿引理 \ref{L.inter.q},仅是将其中对定理 \ref{T.TMS.I.I} 的应用
    替换成引理 \ref{L.QC.I.I} 即可。
\end{proof}
\end{theo}

\begin{defin}[全称量化类型的调用] \label{D.QCall}
    由定理 \ref{T.QCall} 可以定义全称量化类型的调用,
    从 $\PiAE \times \PiAE$ 到 $\PiAE$ 的函数 $(\cdot)$
\[ \forall q_1,\ q_2 \in \PiAE.\ (\QI q_1 \cdot \QI q_2 \neq \emptyset)
\Rightarrow \QI (q_1 \cdot q_2) = \QI q_1 \cdot \QI q_2\]
    算法 \ref{alg:QCall} QCall 即是计算 $(\cdot)$ 并包含无解情况的算法。
\end{defin}

\subsection{值}

现在定义值(Value)。

\begin{defin}[值的单词集] 
无穷单词集 $\hLv$ 为表示值变量的字母表 $\hLv \subseteq
\Words$,无穷单词集 $\hLc$ 为用于表示预定义常量的字母表 $\hLc \subseteq
\Words$,两集合不相交 $\hLc \cap \hLv = \emptyset$, 且 $\hLc$ 包含所有的数字单词、所有的小数单词、所有的字符串字面量单词。
\[ \underline{0},\ \underline{1},\ \underline{2},\cdots,\ 
\underline{0.1},\ \underline{0.01},\cdots,\ \underline{\squote
a \squote},\ \underline{\squote ab \squote},\ 
\cdots \in \hLc \]
\end{defin}

\begin{defin}[值常量、值变量、值] \label{D.Value}
所有的值常量构成集合 $L_c$ ,所有值变量构成集合 $L_v$ 
$L_c^2$
\begin{align*} 
L_c &= \bnf{\hLc : \PiAE}&L_v &= \bnf{\hLv : \PiE}&\end{align*}
所有的值构成的集合 $\LE$ 是所有满足如下规则的最小集合
\[ \begin{array}{rcl}
x \in L_c & \Rightarrow & x \in \LE\\
x \in L_v & \Rightarrow & x \in \LE \\
\underline{x\ :\ (\ \rightarrow\ \sigma\ \tau\ )} \in \LE\ \land\ 
\underline{y\ :\ \sigma} \in \LE & \Rightarrow & 
\underline{(\ x\ :\ (\ \rightarrow\ \sigma\ \tau\ )\ y\ :\ \sigma\ )}
\in \LE\\
v \in L_v\ \land\ x \in \LE & \Rightarrow & \underline{(\ λ\ v\ x\ )}
\in \LE
\end{array} \]
上述的第三条规则类似 λ 演算中的组合律,也可以将 $\LE$ 集理解为
满足 λ 类型规则(主要是组合律)的满足语法
\[ L_c \mbar L_v \mbar (\ \LE\ \LE\ ) \mbar (\ \lambda\ L_v\ \LE \ )\]
的集合。

然后是量化值集 $\LE^2$ 量化值常量集 $L_c^2$
\begin{align*}
\LE^2 &= \bnf{\LE \mbar \Lambda \Uv\ \LE^2}& 
L_c^2 &= \bnf{L_c \mbar \Lambda \Uv\ L_c^2} &
\end{align*}
$\underline{(\ \lambda\ x_1\ (\ \lambda\ x_2\ 
\cdots\ (\ \lambda\ x_n\ u\ )\ \cdots\ )\ )}$ 同样可以被写作
$\lambda x_1\ x_2\ \cdots\ x_n.\ u$ \quad,

\noindent
 $\underline{(\ \Lambda\ \tau_1\ (\ \Lambda\ \tau_2\ \cdots\ (\ \Lambda
\ \tau_m u\ )\ \cdots\ )\ )}$ 可以被写作
$\Lambda \tau_1\ \tau_2\ \cdots\ \tau_m.\ u$ \quad,

\noindent 抽象变量是有类型标记的,类似Church式。另外显然有
\begin{align*} L_c &\subseteq \LE& L_v &\subseteq \LE&
L_c \cap L_v &=\emptyset& L_c^2 &\subseteq \LE^2
\end{align*}
量化值的量化类型变量 $\LV : \LE^2 \rightarrow \powerset(\Uv)$ 被递归定义
\begin{align*} \LV x &= \{\}&&\text{当}\quad x \in \LE&\\
\LV \underline{\Lambda \tau\ x} &=\{\tau\} \cup \LV x&&\text{当}\quad
\underline{\Lambda \tau\ x}\in \LE^2&
\end{align*}
自然地同样递归定义量化值的值 $\LB : \LE^2 \rightarrow \LE$
\begin{align*} \LB x &= x&&\text{当}\quad x \in \LE&\\
\LB \underline{\Lambda \tau\ x} &=\LB x&&\text{当}\quad
\underline{\Lambda \tau\ x}\in \LE^2&
\end{align*}
递归定义值的值重量(value only weight)
    $\vw : \LE \rightarrow \mathbb{N}$
\[ \begin{array}{lclcl}
\vw \underline{(\ u:\tau\ )} & = & 1 & \when & \underline{(\ u:\tau\ )}
\in \LE\\
\vw \underline{(\ f\ a\ )} & = & \vw(f) + \vw(a) & \when & 
    \underline{(\ f\ a\ )} \in \LE\\
\vw \underline{(\ \Lambda\ v\ x\ )} & = & 1 + \vw(x) & \when & 
    \underline{(\ \Lambda\ v\ x\ )} \in \LE\\
\end{array} \]
注意不要混淆值重量与类型重量,它们同为符号 w,但定义域不同且不相交,可
根据定义域区分。
递归定义值的类型重量(type only weight)$\vwt : \LE \rightarrow \mathbb{N}$
\[ \begin{array}{lclcl}
\vwt \underline{(\ u:\tau\ )} & = & \TS \tau & \when & \underline{(\ u:\tau\ )}
    \in \LE\quad\text{注意}\ \TS \tau\ \text{是类型重量}\\
\vwt \underline{(\ f\ a\ )} & = & \vwt(f) + \vwt(a) & \when & 
    \underline{(\ f\ a\ )} \in \LE\\
\vwt \underline{(\ \Lambda\ v\ x\ )} & = & 1 + \vwt(x) & \when & 
    \underline{(\ \Lambda\ v\ x\ )} \in \LE\\
\end{array} \]
最后是值的总重量 $\vW : \LE \rightarrow \mathbb{N}$
\[ \vW x = \vw x + \vwt x \]
以及相应的量化值的值重量、类型重量、总重量
\begin{align*} 
    \Vw z &= \abs{\LV z} + \vw(\LB z) \\
    \Vwt z &= \abs{\LV z} + \vwt(\LB z) \\
    \VW z &= \Vw z + \Vwt z
\end{align*}
同样可以通过定义域区分这些量化值的重量与值的重量,额外的,因为任何值
也都是一种量化值 $\LE \subseteq \LE^2$,任何值被视作量化值时的量化值
的各种重量跟作为普通值时值的各种重量是相同的,故不会引起冲突。
\end{defin}

\begin{defin}[$\LE$ 与 $\LE^2$ 的类型] \label{D.type}
递归函数 $\type : \LE \rightarrow \PiE$ 计算值的类型
\[ \begin{array}{lcl}
\type \underline{(\ x\ :\ \tau\ )}  =  \tau &\when&
\underline{(\ x\ :\ \tau\ )} \in L_c \cup L_v\\
\type \underline{(\ x\ :\ (\ \rightarrow\ \sigma\ \tau\ )\ y\ :\ 
\sigma\ )}  =  \tau &\when&
\underline{x\ :\ (\ \rightarrow\ \sigma\ \tau\ )} \in \LE\ \land\ 
\underline{y\ :\ \sigma} \in \LE\\
\type \underline{(\ \lambda\ v\ x\ )}  =  \type(v) \rightarrow 
\type(x) & \when & x \in \LE \ \land\ v \in \LE
\end{array} \]
由字符串代数的定义 \ref{D.StringAlgebra} 中字符串的有限性以及定义
\ref{D.Value} 这样的递归定义是可行的。

定义量化值的类型 $\type^2 : \LE^2 \rightarrow \PiAE$
\begin{align*} \QB \type^2(x) &= \type \LB(x)&
\QV \type^2(x) &= \LV(x) & \end{align*}
\end{defin}

这一定义也给出了计算值的类型的递归算法 \ref{alg:type} type,与
计算量化值了类型的算法 \ref{alg:type2} $\type^2$。

\begin{algorithm}
\caption{值的类型 type} \label{alg:type} \algcase
\begin{multicols}{2}
\begin{algorithmic}[1]
\Require 值 $x \in \LE$
\Ensure $\type x$
    \Case {x} \When {$\underline{(\ v:\tau\ )}$} \State 输出 $\tau$
    \When {$\underline{(\ f\ a\ )}$} \State $\type(f) \rightarrow
    \underline{(\ \rightarrow\ \sigma\ \tau\ )}$
    \State 输出 $\sigma$ \When {$\underline{(\ λ\ v\ x\ )}$} \State
    输出 $\type(v) \rightarrow \type(x)$
    \EndCase
\TimeComplexity $O(\vw x)$
\end{algorithmic}
\end{multicols}
\end{algorithm}
\begin{algorithm}
\caption{量化值的类型 $\type^2$} \label{alg:type2} \algcase
\begin{multicols}{2}
\begin{algorithmic}[1]
\Require 值 $z \in \LE^2$
\Ensure $\type^2 z$
\Case {$z$} \When $\underline{(\ \Lambda\ v\ x\ )}$
\State $\type^2 x \rightarrow \tau$
\State $\type v \rightarrow \sigma$
\State $\underline{(\ \forall\ \sigma\ \tau\ )}$
\ElseCase \Comment{那么 $z \in \LE$}
    \State 输出 $\type z$ \EndCase
\TimeComplexity $O(\Vw x)$
\end{algorithmic}
\end{multicols}
\end{algorithm}

值集 $\LE$ 上也具有 β 规约,与定义 \ref{D.breduce} 是一样的。

\begin{lemma} \label{L.Le.expand}
\[ \begin{array}{lcc}
\underline{(\ x\ y\ )} \in \LE&\vdash& x \in \LE\ \land\ y \in \LE\\
\underline{(\ \lambda\ v\ x\ )} \in \LE&\vdash& x \in \LE
\ \land\ v \in L_v\\
\end{array} \]
\begin{proof} 由$\LE$的定义 \ref{D.Value} 这是显然的。
\end{proof}
\begin{lemma} \label{L.breduce.in1}
\[ \underline{(\ (\ \lambda\ v\ x\ )\ y\ )} \in \LE \vdash
x \in \LE \]
\begin{proof} 由引理 \ref{L.Le.expand} 直接推论。 \end{proof}
\end{lemma}
\begin{lemma} \label{L.breduce.in2}
\[ a \in \LE,\ a \breduce b \vdash b \in \LE \]
\begin{proof} 由引理 \ref{L.breduce.in1} 以及归纳法证得。 \end{proof}
\end{lemma}
\begin{lemma} \label{L.breduce.in3}
\[ a \in \LE,\ a \bbreduce b \vdash b \in \LE \]
\begin{proof} 由引理 \ref{L.breduce.in2} 以及 $\bbreduce$ 的定义
\ref{D.breduce} 证得。 \end{proof}
\end{lemma}

\begin{defin}[\LE 上的 β 规约函数]
由β规约的唯一性定理 \ref{T.bbreduce.11} 和显然的存在性定理,
以及引理 \ref{L.breduce.in3} 证明的闭包性,允许
定义进行 β 规约的函数 $\B : \LE \rightarrow \LE$
\[ \forall x \in \LE.\ x \bbreduce \B x \]
\end{defin}

\end{lemma}
\begin{lemma} \label{L.Le.Comp.Texpand}
\[ \underline{(\ x\ y\ )} \in \LE,\ \tau \in \PiE,\ 
(\type \underline{(\ x\ y\ )} = \tau) \vdash \exists \sigma \in \PiE
.\ (\type x = \sigma \rightarrow \tau)\ \land\ (\type y = \sigma) \]
\begin{proof} 由引理 \ref{L.Le.expand} 与定义 \ref{D.type} 
定义 \ref{D.Value} 证得。
\end{proof}
\end{lemma}
\begin{lemma} \label{L.Le.type.lxyv}
\[ \underline{(\ (\ \lambda\ x\ y\ )\ v\ )} \in \LE \vdash
\type \underline{(\ (\ \lambda\ x\ y\ )\ v\ )} = \type x \]
\begin{proof} 由引理 \ref{L.Le.Comp.Texpand} 与定义 \ref{D.type} 
定义 \ref{D.Value} 证得。
\end{proof}
\end{lemma}
\begin{lemma}[$\LE$上β规约类型不变] \label{L.breduce.type.consist}
\[ x,y \in \LE,\ x \breduce y \vdash \type x = \type y \]
\begin{proof} 由引理 \ref{L.Le.type.lxyv} 以及对 β 规约归纳法证得。
\end{proof}
\end{lemma}
\begin{lemma}[$\LE$上多步β规约类型不变] \label{L.bbreduce.type.consist}
\[x,y \in \LE,\ x \bbreduce y \vdash \type x = \type y \]
\begin{proof} 由引理 \ref{L.breduce.type.consist}
以及多步 β 规约的定义 \ref{D.breduce} 证得。
\end{proof}
\end{lemma}
\begin{theo}[B规约函数类型不变]
\[x \in \LE \vdash \type \B(x) = \type x\]
\end{theo}

\begin{defin}[值的调用与可调用性] \label{D.V.call}
若值 $f$ 与值 $a$ 满足条件 $f \vcallable a$则被叫做 $f$ 可被 $a$ 调用。
\[ f \vcallable a = \exists \sigma,\ \tau \in \PiE.\ 
(\type f = \sigma \rightarrow \tau)\ \land\ (\type a = \sigma) \]
显然有
\[ f \vcallable a \Rightarrow \underline{(\ f\ a\ )} \in \LE \]
进而有值的调用函数 $(\vcall) : \LE \times \LE \rightarrow \LE$
\[ f \vcallable a \Rightarrow (f \vcall a = \B\underline{(\ f\ a\ )}) \]
同样此处的符号 $(\vcall)$ 不应被混淆因为定义域不同。
\end{defin}

\begin{algorithm}
\caption{值的可调用测试 suit} \label{alg:suit}
\begin{algorithmic}[1]
\Require 值 $f,\ x \in \LE$
\Ensure Bool 值 $f \vcallable x$
    \State 输出 $(\type f\ \text{是函数类型}\quad\land\quad \Dom(\type f) =  \type x)$
    \TimeComplexity $O(1)$
\end{algorithmic}
\end{algorithm}
\begin{algorithm}
\caption{值的调用 vCall} \label{alg:vCall}
\begin{algorithmic}[1]
\Require 值 $f,\ x \in \LE\quad$满足 $f \vcallable x$
\Ensure $f \vcall x$
\State 对 $\underline{(\ f\ x\ )}$ 进行 β 规约,然后输出。
    \TimeComplexity $O(\vw(f) + \vw(x))$
\end{algorithmic}
\end{algorithm}
\begin{algorithm}\algcase
\caption{值的特化 Inst} \label{alg:Inst}
\begin{multicols}{2}
\begin{algorithmic}[1]
\Require 特化范围 $V \in \powerset(\Uv)$,类型特化函数 $f : \Uv \rightarrow \PiE$,值 $x \in \LE$
\Ensure $\Inst_V\ f\ x$
    \Case {$x$}
    \When {$\underline{(\ v:\tau\ )}$}
    \State $\inst(V,f,\tau) \rightarrow \tau'$
    \State 输出 $\underline{(\ v:\tau'\ )}$
    \When {$\underline{(\ u\ v\ )}$}
    \State $\Inst(V,f,u) \rightarrow u'$
    \State $\Inst(V,f,v) \rightarrow v'$
    \State 输出 $\underline{(\ u'\ v'\ )}$
    \When {$\underline{(\ \lambda\ v\ u\ )}$}
    \State $\Inst(V,f,u) \rightarrow u'$
    \State $\Inst(V,f,v) \rightarrow v'$
    \State 输出 $\underline{(\ \lambda\ v'\ u'\ )}$
    \EndCase
\TimeComplexity $O(\vW x)$
\end{algorithmic} \end{multicols}
\end{algorithm}
\begin{algorithm}
\caption{值量化 MakeQV} \label{alg:MakeQV}
\begin{algorithmic}[1]
\Require 类型变量集 $V \in \powerset(\Uv)$,量化目标 $x \in \LE$
\Ensure $z \in \LE^2\quad$ 满足 $\LV z = V\quad\land\quad\LB z = x$
\For {$v \in V$} \State $\underline{\Lambda}\concat v \concat
    x \rightarrow x$ \EndFor
\State 输出 $x$ \TimeComplexity $O(\abs{V})$
\end{algorithmic} 
\end{algorithm}

至此定义完了普通值的调用运算,可以由此执行大量的函数调用。接下来将定义
量化值的调用运算。首先从量化值的特化开始。

\begin{defin}[值的特化] \label{D.Inst}
递归定义值的特化函数 $\Inst : (\Uv \rightarrow \PiE) \rightarrow \LE
\rightarrow \LE$,但其值域 $\LE$ 是需要证明,这将在引理 \ref{D.Inst.LE}
中完成,追求严谨的读者可以先认定其值域为 $\String$
\[ \begin{array}{llcr}
\Inst f\ \underline{(\ x : \tau\ )} = \underline{(\ x : \tau' \ )}
,& \tau' = \inst f\ \tau & \when & \underline{(\ x : \tau\ )}
\in \LE\\
\Inst f\ \underline{(\ g\ a\ )} = \underline{(\ g'\ a'\ )},&
g' = \Inst f\ g,\quad a' = \Inst f\ a& \when & \underline{(\ g\ a\ )}
\in \LE\\
\Inst f\ \underline{(\ \lambda\ v\ x\ )} = \underline{(\ \lambda\ v'
\ x'\ )},&v' = \Inst f\ v,\quad x' = \Inst f\ x&\when&
\underline{(\ \lambda\ v\ x\ )} \in \LE
\end{array} \]
同样的部分特化 $\Inst_V$
\[ \begin{array}{llcr}
\Inst_V f\ \underline{(\ x : \tau\ )} = \underline{(\ x : \tau' \ )}
,& \tau' = \inst_V f\ \tau & \when & \underline{(\ x : \tau\ )}
\in \LE\\
\Inst_V f\ \underline{(\ g\ a\ )} = \underline{(\ g'\ a'\ )},&
g' = \Inst_V f\ g,\quad a' = \Inst_V f\ a& \when & \underline{(\ g\ a\ )}
\in \LE\\
\Inst_V f\ \underline{(\ \lambda\ v\ x\ )} = \underline{(\ \lambda\ v'
\ x'\ )},&v' = \Inst_V f\ v,\quad x' = \Inst_V f\ x&\when&
\underline{(\ \lambda\ v\ x\ )} \in \LE
\end{array} \]
最后是量化值的特化 $\Inst_\Lambda : (\Uv \rightarrow \PiE) \rightarrow 
    \LE^2 \rightarrow \LE$
    \[ \Inst_\Lambda f\ z = \Inst_{\ \LV z}\ f\ \LB(z) \]
\end{defin}

\begin{lemma} \label{D.Inst.LE}
\begin{align*}
    &x \in \LE \vdash \Inst\ f\ x \in \LE&
    &x \in \LE \vdash \Inst_V\ f\ x \in \LE&
\end{align*}
\begin{proof} 由$\LE$集的定义 \ref{D.Value} 特化的定义 \ref{D.Inst}
    容易得到。
\end{proof}
\end{lemma}


值的特化的递归定义 \ref{D.Inst} 同时也给出了值特化的算法,形式地写作
算法 \ref{alg:Inst} Inst。



\begin{defin}[量化值的特化集]
量化值的特化集函数 $\VI : \LE^2 \rightarrow \powerset(\LE)$
    \[ \VI z = \{ \Inst_\Lambda\ f\ z \mbar f \} \]
即所有量化值 $z$ 可以特化成的值构成的集合。
\end{defin}

\begin{defin}[量化值的 α 等价]
$\LE^2 \times \LE^2$ 上的等价关系 $\sim_\alpha$
    \[ z_1 \sim_\alpha z_2 = (\VI z_1 = \VI z_2) \]
即可以理解为本质不同的量化值。
\end{defin}

接下来构建量化值的调用,方法跟 \ref{S.T.Call} 节一样,先定义特化集
的调用而后证明存在一个量化值其特化集等于此调用集,于是此量化值
就能代表特化集调用的结果。

\begin{defin}[值的集的调用]
值的集的调用是函数 $(\vcalls) : \powerset(\LE) \times \powerset(\LE)
    \rightarrow \powerset(\LE)$
    \[ Z_1 \vcalls Z_2 = \{ z_1 \vcall z_2 \mbar z_1 \in Z_1\ \land\ 
    z_2 \in Z_2\ \land\ z_1 \vcallable z_2 \} \]
同样此 $(\vcalls)$ 不应跟任何别的运算诸如类型的调用 $(\cdot)$ 混淆,
因为定义域不同,可以通过区分定义域而区分这些运算。
\end{defin}

\begin{algorithm}
\caption{量化值调用 VCall} \label{alg:VCall}
\begin{multicols}{2}
\begin{algorithmic}[1]
\Require 量化值 $z_1, z_2 \in \LE^2$
\Ensure $z \in \LE^2\quad$ 满足 $\quad \VI z = \VI z_1 \vcalls \VI z_2
    \quad$或者输出 {\it 无解}
\State $\type^2 z_1 \rightarrow \zeta_1$ 
\State $\type^2 z_2 \rightarrow \zeta_2$ 
\If {$\mathrm{QCall}(\zeta_1,\ \zeta_2) =\ \text{\it 无解}$}
    \State 输出 {\it 无解}
\Else \State $\mathrm{QCall}(\zeta_1,\ \zeta_2) \rightarrow (\zeta,\ f)$
\EndIf
    \State $\Inst(f,\ \LV z_1) \rightarrow x_1$
    \State $\Inst(f,\ \LV z_2) \rightarrow x_2$
    \If {$\mathrm{vCall}(x_1,\ x_2) =\ \text{无解}$}
    \State 输出 {\it 无解}
    \Else
\State $\mathrm{vCall}(x_1,\ x_2) \rightarrow x$
    \EndIf
    \State 输出 $\mathrm{MakeQV}(\QV \zeta,\ x)$
    \TimeComplexity $O((\abs{\LV z_1} + \abs{\LV z_2} + 1)(
    \TS(\type \LB(z_1)) + \TS(\type \LB(z_2))) + \vW(\LB z_1) + 
    \vW(\LB z_2))$
\end{algorithmic}
\end{multicols}
\end{algorithm}

\begin{theo} \label{T.VCall}
    \[ \VI z_1 \vcalls \VI z_2 \neq \emptyset \vdash \exists z \in \LE^2
    .\ \VI z_1 \vcalls \VI z_2 = \VI z\]
\begin{proof} 算法 \ref{alg:VCall} VCall 就是寻找这样的 $z$,且根据
    算法 \ref{alg:QCall} QCall 的正确性定理与停机性定理 \ref{T.QCall}
    算法 \ref{alg:VCall} VCall 也是有限停机的。
\end{proof}
\end{theo}

算法 \ref{alg:VCall} VCall 的时间复杂度比较复杂,但是多项式的。
由算法 \ref{alg:VCall} VCall 以及定理 \ref{T.VCall} 可以定义量化值的
调用。

\begin{defin}[量化值的调用]
量化值的可调用性测定函数 $\vcallable : \LE^2 \times \LE^2 \rightarrow
    \Bool$
\[ z_1 \vcallable z_2 = (\VI z_1 \vcalls \VI z_2 \neq \emptyset) \]
算法 \ref{alg:VCall} VCall 可计算 $\vcallable$,当 $\mathrm{VCall}(
    z_1,\ z_2)$ 有解时 $z_1 \vcallable z_2 = \T$ 无解时 $z_1 \vcallable
    z_2 = \F$。接下来是
量化值的调用函数 $(\vcall)$
    \[ z_1 \vcallable z_2 \vdash \VI (z_1 \vcall z_2) = \VI z_1 \vcalls
    \VI z_2 \]
在 $z_1 \vcallable z_2$ 时算法 \ref{alg:VCall} VCall 有解且其结果即是
对 $z_1 \vcall z_2$ 的计算。
\end{defin}

量化调用的灵感在于,可以将量化值对应到一个集合,集合由此量化值所有
可以特化出的值组成,即特化集合,
$z_1 \vcallable z_2$ 的情况下 $z_1 \vcall z_2$ 是也一个量化值而此
量化值对应的特化集合即是将$z_1,\ z_2$对应的特化集的所有值一一尝试调用
得到的结果构成的集合。

一个值 $z$ 是全称量化的,意味它可以特化到很多非量化类型,
其本身代表了一种泛性,例如 $\I$ 函数就是全称量化的,它的类型是
$\forall \tau.\ \tau \rightarrow \tau$


\section{\ES 机器}

\subsection{\ES 机器的形式定义}

\begin{defin}[标识、编辑变量空间、编辑上下文]
索引集(index set)$I$ 表示所有的标识,这些标识用于命名编辑器变量。
一个编辑变量空间是一个 $I$ 到 $\LE^2$ 的有限映射 $I \mapsto \LE^2$,
在实现上可以用哈希表表示。
所有的编辑变量空间构成集合 $\vspac$。
另一个索引集 $\Is$ 中的元素标识编辑变量空间,$\Is$ 到 $\vspac$ 
的有限映射叫做 \ES 机器的上下文 $\{\vspa{i}\},\ \vspa{i} \in \vspac,\ 
i \in \Is$ 构成变量空间$\vspac$的族(families of set $\vspac$),
所有的上下文构成集合 $\CE$。
\end{defin}

对一个变量空间 $\Psi$,项 $\Psi(name),\ name \in I$ 是变量空间中名为
$name$的值。
一个上下文 $\{\vspa{i}\}$,项 $\vspaR,\ \mathbf{0} \in \Is$ 表示
标识为 $\mathbf{0}$ 的变量空间,继而 $\vspaR(name)$ 表示
标识为 $\mathbf{0}$ 的变量空间中名为 $name$ 的变量。

\ES 机器中非常重要的一点是预定义函数,这些函数是一些 \ES 形式语言上的值
,但与编辑器上预先实现的一些功能关联,当这些值进行 β 规约时,关联的
功能将被激活,编辑器上特定的程序过程将被执行,而这些值的参数将传入
这些过程作为参数。以此实现诸如编译、读取文件,或者在屏幕上打印一些
内容的功能。可以说,\ES 形式语言是为调用这些功能的脚本。

现在形式地说明这些预定义值与功能,但绝非说这些功能就是如此按数学定义来
实现的,它们在软件中的具体实现参见???章,而此处的形式定义仅仅是为了
方便论述它们的含义,以及精确论述 \ES 状态的抽象模型。
预定义值此处的定义仅仅是理论上的工具,暂且把具体实现搁置。

\begin{defin}[预定义值与编辑器功能]
编辑器功能的理论模型是 $\CE \times \{\LE\}_n$ 到 
$\CE \times \LE$ 的函数。对于输入上下文 $\gamma$,长度为$n$的输入参数
列表$\{a_i\}_n$,编辑器功能 $f$ 的执行结果 \[f(\gamma,\{a_i\}_n) = 
(\gamma',y)\] 表示新的上下文 $\gamma'$ 以及返回值 $y$。
所有的编辑器功能构成集合 $\FE$。
量化常量值 $L_c^2$ 中所有链接到编辑器功能的值构成集合 $\Lp^2$,
$\Lp^2 \subseteq L_c^2$,其中每个值$z_p$对应的编辑器功能为
$\fE(z_p)$,$\fE$ 是 $\Lp^2$ 到 $\FE$ 的函数。
每个编辑器功能$f$都有一个自然数对应$\omega(f)$ 表示参数数目。
因为 $\Lp$ 集中每个值$z_p$都对应了唯一的一个编辑器功能 $\fE(z_p)$,
就也能得到对应编辑器功能的参数数目 $\omega(\fE z_p)$,
不会引起混淆地同样写作$\omega(z_p)$
\[ \omega(z_p) = \omega(\fE\ z_p) \]
\end{defin}

\begin{defin}[\ES 状态与\ES 机器]
\ES 状态是上下文$\{\vspac{i}\}$ 与值 $z$ 的二元组 $(\{\vspac{i}\}, z)$,
所有的状态构成集合 $\mathbf{State} = \CE \times \LE^2$。
一个特殊的变量空间 $\vspaR$ 表示全局变量空间,初始状态 $s_0$
\[ s_0 = (\vspaR,\ \I) \]
定义操作集为 $\mathbf{Choice} = \{\rightarrow,\ \leftarrow\}$,
输入集为 $\mathbf{Input} = \mathbf{Choice} \times \LE^2$,
状态转移函数 $\delta : \mathbf{State}\times\mathbf{Input} \rightarrow
\mathbf{State}$
\begin{align*}
\delta((\Psi,\ z),\ \leftarrow, z') &= (\Psi,\     )
\end{align*}

\ES 机器是状态机。
\end{defin}





\ES 

类似λ2演算,令字符串集合 $U_v$ 表示\ES 的类型系统上所有的类型变量,
字符串集合$U_0$表示所有预定义的类型,集合$ \UE = U_v \cup U_0 $
\[ \PiE = \bnf{\UE \mbar \UE \rightarrow \UE} \]
表示所有的非全称量化类型或简单地叫做普通类型,
\[ \PiAE = \bnf{\PiE \mbar \forall U_v\ \PiAE } \]
表示全称量化类型或简单地叫做量化类型。


