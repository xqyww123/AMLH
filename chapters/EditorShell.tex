\chapter{编辑壳层$\ES$}

$\amlh$ 已经在上一章论述清楚。但$\amlh$只是一个抽象机理论,尽管被
定义在 HOL 交互式证明工具的 HOL 逻辑上,借助定理证明工具似乎具象了一点,
但显然不能让用户直接操作数学命题与定理。
需要有一个壳层包裹起$\amlh$理论并向外提供给用户编辑程序的功能,
这是本章将论述的编辑壳层$\ES$的意义。

\section{$\ES$概述}

$\ES$ 是一个形式系统,基于此形式系统构建起同名为$\ES$的下推自动机
然后是同样同名的交互式工具进而允许用户由此有效地在$\amlh$上开发程序。

$\ES$ 形式系统 $\rightarrow$ $\ES$ 状态机 $\rightarrow$ $\ES$ 交互工具

本节概述这三者以给读者一个直观的轮廓而将精细的形式定义留在之后。
$\ES$ 形式系统,简单来说是一个改造过的 λ2 演算,继承了 λ2 表达式、
多态类型系统、α 规约、β 规约

\section{$\ES$ 形式系统}

$\ES$ 形式系统是一种改造过的 λ2 演算,且非常接近SML语言


但抛开与 λ2 的联系而
直接论述$\ES$ 形式系统本身更易与表述。先暂且不管与 λ2 的联系而开始直接
论述。

\subsection{类型系统}

$\ES$

类似λ2演算,令字符串集合 $U_v$ 表示$\ES$的类型系统上所有的类型变量,
字符串集合$U_0$表示所有预定义的类型,集合$ \UE = U_v \cup U_0 $
\[ \PiE = \bnf{\UE \mbar \UE \rightarrow \UE} \]
表示所有的非全称量化类型或简单地叫做普通类型,
\[ \PiAE = \bnf{\PiE \mbar \forall U_v\ \PiAE } \]
表示全称量化类型或简单地叫做量化类型。


