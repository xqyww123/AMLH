\chapter{$\ES$编辑壳层}

$\amlh$ 已经在上一章论述清楚。但$\amlh$只是一个抽象机理论,尽管被
定义在 HOL 交互式证明工具的 HOL 逻辑上,借助定理证明工具似乎具象了一点,
但显然不能让用户直接操作数学命题与定理。
需要有一个壳层包裹起$\amlh$理论并向外提供给用户编辑程序的功能,
这是本章将论述的编辑壳层$\ES$的意义。

\section{$\ES$概述}

$\ES$ 是一个形式语言,且具有同名为$\ES$的
基于 System F 的一阶多态类型系统。围绕此构建起同名为$\ES$的状态机,
最后是名为$\ES$编辑壳层的软件工具。

接下来将分别论述$\ES$形式语言、$\ES$类型系统、$\ES$状态机。

\section{$\ES$ 形式语言}

\subsection{类型与全称量化类型}

首先定义类型。

\begin{defin}[类型集$\PiE$]
字符串集合 $U$ 为用于表示类型的字母表,
函数$\xa : U \mapsto \mathbb{N}$表示类型的元数(Arity),
从$U$元数为0的子集$U_\mathrm{v}$:
$U_\mathrm{v} \subseteq U\ \land\ \forall u.\ u \in U_\mathrm{v}
\Rightarrow \xa(u) = 0$ 表示类型变量,主要用于全称量化。

所有由字母表 $U$ 组成的字符串记为集合$\mathrm{String}(U)$,
$\ES$ 的类型集 $\Pi_\mathrm{E}$ 是满足以下条件的$\mathrm{String}(U)$最小子集
\begin{equation} \label{Def.PiE}
\forall u.\ u \in U \Rightarrow \forall \bm{v}.\ \bm{v} 
\in \Pi_\mathrm{E}^{\xa(u\ )} \Rightarrow \underline{(\ u\ \bm{v}_1\ 
\cdots\ \bm{v}_{\xa(u\ )}\ )} \in \Pi_\mathrm{E}
\end{equation}
其中 $\Pi_\mathrm{E}^{\xa(u\ )}$ 表示 $\Pi_\mathrm{E}$ 的 $\xa(u)$ 维向量
空间,特别的 $\Pi_\mathrm{E}^0 = \{\mathbf{0}\}$
\end{defin}

\begin{example}[$\Pi_\mathrm{E}$] 以下命题成立:
\[ \forall u.\ u \in U\ \land\ (\xa(u) = 0)\ \Rightarrow\ 
\underline{(\ u\ )} \in \Pi_\mathrm{E} \]
\[ \forall u\ v.\ u \in U\ \land\ v \in U_\mathrm{v}\ \land\ (\xa(u) = 1)
\ \Rightarrow\ \underline{(\ u\ v\ )} \in \Pi_\mathrm{E} \]
\[ \forall u\ v.\ u \in U\ \land\ v \in \Pi_\mathrm{E}\ \land\ (\xa(u) = 1)
\ \Rightarrow\ \underline{(\ u\ v\ )} \in \Pi_\mathrm{E} \]
\end{example}

\begin{theo}(类型的结构) \label{TS}
\[ \forall u'.\ u' \in \PiE \Rightarrow \exists! u\ \bm{v}.\ 
u \in U_\mathrm{\bm{v}}\ \land\ \bm{v} \in \PiE^{\xa(u\ )}\ \land\ 
(u' = \underline{(\ u\ \bm{v}_1\ \cdots\ \bm{v}_{\xa(u'\ )}\ )}) \]
\end{theo}
\begin{proof} 首先唯一性是显然的,由字符串理论就可以得到。
对存在性的证明使用反证法,假设
\begin{equation} \label{Hypo.TS}
\exists u'.\ u' \in \PiE \Rightarrow \forall u\ \bm{v}.\ 
u \in U_\mathrm{\bm{v}}\ \land\ \bm{v} \in \PiE^{\xa(u\ )}\ \land\ 
(u' \neq \underline{(\ u\ \bm{v}_1\ \cdots\ \bm{v}_{\xa(u'\ )}\ )})
\end{equation}
现证明 $\PiE - \{u'\}$ 满足条件 \ref{Def.PiE} 而
$\PiE - \{u'\} \subset \PiE$ 这样就构造了悖论,因为 $\PiE$ 不再是满足
条件 \ref{Def.PiE} 的最小子集,$\PiE - \{u'\}$ 比 $\PiE$更小。
即证明
\[ \forall u.\ u \in U \Rightarrow \forall \bm{v}.\ 
\bm{v} \in \PiE^{\xa(u\ )} - \{u'\} \Rightarrow
\underline{(\ u\ \bm{v}_1\ \cdots\ \bm{v}_{\xa(u\ )}\ )} \in \PiE - \{u'\} \]
因为有
\[ \forall u.\ u \in U \Rightarrow \forall \bm{v}.\ 
\bm{v} \in \Pi_\mathrm{E}^{\xa(u\ )} \Rightarrow
\underline{(\ u\ \bm{v}_1\ \cdots\ \bm{v}_{\xa(u\ )}\ )} \in \Pi_\mathrm{E} \]
所以只要证明
\[ \underline{(\ u\ \bm{v}_1\ \cdots\ \bm{v}_{\xa(u\ )}\ )} \neq u' \]
而由反证假设 \ref{Hypo.TS} 这是成立的,故而悖论被构造进而命题得证。
\end{proof}

定理 \ref{TS} 意味着一切类型$u$都具有且唯一地具有如下格式
\[ \underline{u_0\ \bm{v}_1\ \cdots\ \bm{v}_{\xa(u_0\ )}} \]
其中 $u_0 \in \mathrm{U}$,$\bm{v}_1,\ \cdots,\ \bm{v}_{\xa(u_0\ )} \in \PiE$

即每一个类型都构成一颗树,类型变量与0元类型构造器是叶子。

\begin{defin}[类型的构造器、元数、高度] \label{Def.Th}
函数 $\mathrm{c}:\Pi_\mathrm{E} \mapsto U$ 表示类型的构造器。
\[ \mathrm{c}\ \underline{(\ u\ v_1\ \cdots\ v_n\ )} = u\]
函数 $\xa:\Pi_\mathrm{E} \mapsto \mathbb{N}$ 表示类型的元数。
\[ \xa\ \underline{(\ u\ v_1\ \cdots\ v_n\ )} = n\]
$\xa:\Pi_\mathrm{E} \mapsto \mathbb{N}$ 不会跟上文定义的
$\xa:U \mapsto \mathbb{N}$冲突,因为定义域不重合,且两者具有相同的意义,
不会造成歧义。

\noindent 函数 $\mathrm{h}:\Pi_\mathrm{E} \mapsto \mathbb{N}$ 表示类型的高度。
\[ \mathrm{h}\ \underline{(\ u\ v_1\ \cdots\ v_n\ )} = 1 + \max(\mathrm{h}\ v_1,\ \cdots,\ 
\mathrm{h}\ v_n) \]

\noindent 函数 $\mathrm{v}:\Pi_\mathrm{E} \mapsto \powerset{(\Uv)}$ 
表示类型中的所有变量。
\[ \begin{split}
&\mathrm{v}\ \underline{(\ c\ )}=\xif c \in \Uv \xthen \{c\} \xelse \emptyset \\
&\mathrm{v}\ \underline{(\ u\ v_1\ \cdots\ v_n\ )}=
\bigcup_{i=1\cdots n} \mathrm{v}(v_i)
\end{split} \]
\end{defin}

有如下性质

\begin{theo}[类型的元数与高度的性质] \label{T.cah}
\[ \forall u.\ \mathrm{h}\ u  \geq 1 \quad\quad\text{(1)}\quad\quad
\quad\quad\quad\forall u.\ \xa(\mathrm{c}\ u) = \xa\ u \quad\quad\text{(2)} \]
\[ \forall u.\ (\mathrm{h}\ u = 1) \Rightarrow \exists \mathrm{c}.\ \mathrm{c} \in U
\ \land\ u = \underline{(\ \mathrm{c}\ )} \tag{3} \]
\[ \begin{split}
\forall u.\ (\mathrm{h}\ u > 1) \Rightarrow \exists \mathrm{c}\ &\bm{v}.\ \mathrm{c} \in U
\ \land\ \bm{v} \in \PiE^{\xa(u\ )}\ \land\ u = \underline{
(\mathrm{c}\ \bm{v}_1\ \cdots\ \bm{v}_{\xa(u\ )}\ )} \ \land\ \\
&(\forall i.\ 1 \leq i \leq \xa(u) \Rightarrow \mathrm{h}\ \bm{v}_i < \mathrm{h}\ u )
\end{split} \tag{4} \]
\end{theo}
\begin{proof} 由定义 \ref{Def.Th} 与定理 \ref{TS} 直接得到。
\end{proof}

这样就可以关于类型的高度进行归纳法。

\begin{defin}[全称量化类型 $\PiAE$]
集合 $\Pi_\mathrm{E}^\forall$ 表示全称量化类型,由所有满足如下语法的字符串构成。
\[ \Pi_\mathrm{E} \mbar \forall U_\mathrm{v}\ \Pi_\mathrm{E} \]
同样有记号 $\forall v_1\ \cdots\ v_n.\ b$ 表示 $\underline{\forall v_1
\ \cdots\  \forall v_n\ b}$
\end{defin}

\begin{defin}[全称量化类型的相关属性]
函数 $\mathrm{QV} : \Pi_\mathrm{E}^\forall \rightarrow \powerset(
U_\mathrm{v})$ 表示全称量化类型的绑定变量集。
\[ \mathrm{QV}(\forall v_1\ \cdots\ \v_n.\ b) = \{v_1,\ \cdots,\ v_n\} \]
函数 $\mathrm{QB} : \Pi_\mathrm{E}^\forall \rightarrow \Pi_\mathrm{E}$
表示全称量化的类型体。
\[ \mathrm{QB}(\forall v_1\ \cdots\ \v_n.\ b) = b \]
函数 $\Qv : \Pi_\mathrm{E}^\forall \rightarrow \powerset(
U_\mathrm{v})$ 表示全称量化类型所有的变量集。
\[ \Qv q = \QV q \cup \mathrm{v}(\QB q) \]
\end{defin}

\begin{defin}[实例化] \label{Def.Inst}
函数 $\mathrm{Inst} : (U_\mathrm{v} \rightarrow \Pi_\mathrm{E})
\rightarrow \Pi_\mathrm{E} \rightarrow \Pi_\mathrm{E}$ 对类型
进行变量实例化。
\begin{align*}
\mathrm{Inst}\ f\ \underline{(\ v\ )} &= \xif v \in U_\mathrm{v} \xthen
 f\ v \xelse \underline{(\ v\ )} \\
\mathrm{Inst}\ f\ \underline{(\ u\ v_1\ \cdots\ v_{\xa(u\ )}\ )} &= 
\underline{(\ u\ f(v_1)\ \cdots\ f(v_{\xa(u\ )})\ )}
\end{align*}

以及部分实例化 $\Inst_V$ 
\[ \Inst_V f = \Inst (\lambda v.\ \xif v \in V \xthen f\ v \xelse v) \]

函数 $\mathrm{Inst}_\forall : (U_\mathrm{v} \rightarrow \Pi_\mathrm{E})
\rightarrow \Pi_\mathrm{E}^\forall \rightarrow \Pi_\mathrm{E}$ 
实例化全称量化类型。
\begin{gather*}
\mathrm{Inst}_\forall\ f\ q = \Inst_{\ \mathrm{QV}(q\ )} f\ \mathrm{QB}(q)
\end{gather*}
即$\mathrm{Inst}_\forall$只会实例化全称量化的类型变量。
\end{defin}

\begin{algorithm}
\caption{实例化函数 $\Inst$} \label{alg:Iv}
\begin{algorithmic}[1]
\Require 集合 $V \in \powerset(\Uv)$ 表示实例化的范围
\Require 实例化函数 $f : \Uv \rightarrow \PiE$ 表示变量到值的对应
\Require $u \in \PiE,\ u = \underline{(\ c\ v_1\ \cdots\ v_o\ )}$
表示要实例化的目标
\Ensure $\Inst_V f\ u \in \PiE$
\If {$o = 0$}
\If {\quad $c \in V$ \quad} \quad 输出 $f(c)$
\Else {\quad 输出 $\underline{(\ c\ )}$}
\EndIf
\Else
\State $\underline{(\ c} \rightarrow s$
\For{$i=1,\ \dots,\ o$}
\State $s \concat \Inst(V,f,v_i) \rightarrow s$
\EndFor
\State 输出 $s \concat \underline{)}$
\EndIf
\end{algorithmic}
\end{algorithm}
\begin{algorithm}
\caption{全称量化的实例化函数 $\Inst_\forall$} \label{alg:IvQ}
\begin{algorithmic}[1]
\Require 实例化函数 $f : \Uv \rightarrow \PiE$ 表示变量到值的对应
\Require $q \in \PiAE,\ q = \underline{\forall\ v_1\ \cdots \forall\ v_p
\ u}$
表示要实例化的目标
\Ensure $\Inst_\forall f\ q \in \PiE$
\State $\{\} \rightarrow s$
\For{$i=1,\ \dots,\ p$}
\State 集合 $s$ 加入 $q_i$
\EndFor
\State 调用算法 \ref{alg:Iv}:$\Inst(s,f,u)$ 将结果输出。
\end{algorithmic}
\end{algorithm}
\begin{algorithm}
\caption{构造全称量化类型 $\mathrm{MakeQT}$} \label{alg:MakeQT}
\begin{algorithmic}[1]
\Require 集合 $V \in \powerset(\Uv)$
\Require 类型 $u \in \PiE$
\Ensure $q \in \PiAE$ 满足 $(\QV q = V) \ \land\ (\QB q = u)$
\For{$v \in V$}
\State $\underline{\forall} \concat v \concat u \rightarrow u$
\EndFor
\State 输出 $u$
\end{algorithmic}
\end{algorithm}

\begin{lemma} \label{Lem.Iv.V}
\[ \Inst_V f\ u = \Inst_{\ V \cap \mathrm{v}(u)} f\ u \]
\begin{proof} 对 $u$ 进行类型高度的归纳法即可。
\end{proof}
\end{lemma}

\begin{defin}[α等价] \label{Def.aE}
二元关系$\sim_\alpha$ 定义为
\[ (q_1 \sim_\alpha q_2) = (\forall f.\ \mathrm{Inst}_\forall\ f\ q_1
= \mathrm{Inst}_\forall\ f\ q_2) \]
显然是一种等价关系,被叫做α等价。
\end{defin}

α等价类$\Pi_\mathrm{E}^\forall/[\sim_\alpha]$即是本质不同的全称量化类型。

\begin{defin}[实例化类型集] \label{Def.QI}
全称量化类型$q$的实例化类型集$\mathrm{QI}\ q$为
\[ \mathrm{QI}\ q = \{\mathrm{Inst}_\forall\ f\ q\mbar f \in 
(U_\mathrm{v} \rightarrow \Pi_\mathrm{E})\} \]
\end{defin}
\begin{theo}[实例化类型集的相等即是$\sim_\alpha$等价]
\[(q_1 \sim_\alpha q_2) = (\mathrm{QI}\ q_1 = \mathrm{QI}\ q_2)\]
\end{theo}
\begin{proof} 由定义 \ref{Def.aE} 与定义 \ref{Def.QI} 直接得到。
\end{proof}
\begin{theo}[实例化类型集非空]
\[\forall q.\ \mathrm{QI}\ q \neq \emptyset\]
\end{theo}
\begin{proof}
因为 $\quad\forall q.\ \mathrm{Inst}_\forall\ \I\ q = \mathrm{QB}\ q 
\quad$ 所以有 $\quad\forall q.\ \mathrm{QB}\ q \in \mathrm{QI}\ q$
\end{proof}

接下来尝试证明一个重要命题 
\[ \forall q_1\ q_2.\ q_1,\ q_2 \in \Pi_\mathrm{E}^\forall 
\ \land\ (\mathrm{QI}\ q_1 \cap \mathrm{QI}\ q_2 \neq \emptyset)
\Rightarrow
\exists q.\ q \in \Pi_\mathrm{E}^\forall\ \land\ ((\mathrm{QI}\ q_1)
\cap (\mathrm{QI}\ q_2) = \mathrm{QI}\ q) \]
并找到一个算法用于求解上述的 $q$

首先要引入诸多工具的定义。

\subsection{类型匹配系统}

\begin{defin}[类型匹配系统(Type Match System)] 类型匹配系统集 $\EQs$ 是集合
\[ \EQs = \powerset(\Uv) \times \powerset(\PiE \times \PiE) \]
类型匹配系统是所有集合 $\EQs$ 中的元素。
\[ (V,\ \{ (x_1, y_1),\ (x_2, y_2),\ (x_3, y_3),\ \cdots \}) \]
其中 $\{ (x_1, y_1),\ (x_2, y_2),\ (x_3, y_3),\ \cdots \}$ 叫做
匹配系统中的方程组,$x_1 = y_1,\ \cdots$ 是方程组中的方程,
$V \subseteq \Uv$ 是类型匹配系统的变量集。
\end{defin}
\begin{notation}[类型匹配系统的记号] 
记号
\[ \begin{Bmatrix}
x_1 &=& y_1 \\
x_2 &=& y_2 \\
x_3 &=& y_3 \\
&\cdots& 
\end{Bmatrix}_V \]
表示类型匹配系统
\[ (V,\ \{ (x_1, y_1),\ (x_2, y_2),\ (x_3, y_3),\ \cdots \}) \]
\end{notation}
\begin{defin}[类型匹配系统的解与解集] \label{Def.MF}
一个类型匹配系统$(V,X)$的解$f$是一个$(\Uv \rightarrow \PiE)$函数,
满足
\[ \forall u_1\ u_2.\ (u_1,u_2) \in X \Rightarrow 
\Inst f\ u_1 = \Inst f\ u_2 \]
所有这样的解构成的集合叫解集。

函数 $\MF : \EQs \rightarrow \powerset(\Uv \rightarrow \PiE)$
将一个类型匹配系统映射到其解集
\[ \MF (V,X) = \{ f \mbar \forall u_1\ u_2.\ (u_1,u_2) \in X
\Rightarrow \Inst_V f\ u_1 = \Inst_V f\ u_2 \} \]
\end{defin}

\begin{defin}[类型匹配系统的M等价] \label{Def.Meq}
二元等价关系 $\Meq$
\[ \forall A\ B.\ (A \Meq B) = (\MF A = \MF B) \]
\end{defin}
\begin{lemma} \label{L.Meq.refl}
\[ \forall X\ V\ v.\ (V,X) \Meq (V, X \cup \{(v,v)\}) \]
\begin{proof} 由定义 \ref{Def.Meq} 定义 \ref{Def.MF} 直接得到
\end{proof}
\end{lemma}

\begin{lemma} \label{L.MF.XUX}
\[ \MF (V,X_1 \cup X_2) = \MF(V,X_1) \cap \MF(V,X_2) \]
\begin{proof} 将$\MF$ 的定义展开,命题等价于
\begin{multline*}
(\forall u_1\ u_2.\ (u_1,u_2) \in (X_1 \cup X_2)
\Rightarrow \Inst_V f\ u_1 = \Inst_V f\ u_2) = \\
(\forall u_1\ u_2.\ 
(u_1,u_2) \in X_1
\Rightarrow \Inst_V f\ u_1 = \Inst_V f\ u_2)\ \land\\
(\forall u_1\ u_2.\ (u_1,u_2) \in X_2
\Rightarrow \Inst_V f\ u_1 = \Inst_V f\ u_2)
\end{multline*}
这时显然的。
\end{proof}
\end{lemma}

\begin{defin}[有意义类型方程组] \label{Def.SF}
有意义(Senseful Form)类型方程组函数
$\SF : \powerset(\PiE \times \PiE) \rightarrow \powerset(\PiE \times \PiE)$
\[ \SF X = \{(x,y) \mbar (x,y) \in X \ \land\ x \neq y\} \]
即是削除了恒等式 $(x,x)$ 后的方程。
\end{defin}


\begin{algorithm}
\caption{计算有意义的类型方程组 $\SF X$} \label{alg:SF}
\begin{algorithmic}[1]
\Require 类型方程组 $X$
\Ensure 类型方程组的有意义形式 $\SF X$
\State $\{\} \rightarrow X'$
\For {$(x,y) \in X$}
\If {$x \neq y$} \State 将 $(x,y)$ 加入 $X'$ \EndIf
\EndFor
\State 输出 $X'$
\end{algorithmic}
\end{algorithm}

\begin{defin}[类型匹配系统的并] \label{Def.M.U}
类型匹配系统 $(V,X)$ 与 $(V',X')$ 的并 $(V,X)\cup(V',X')$ 定义为
\[ (V,X)\cup(V',X') = (V \cup V',\ X \cup X') \]
\end{defin}

\begin{defin}[已解的类型匹配系统] \label{Def.Solved}
类型匹配系统在的{\it 已解}形式下的未知量(Unknown Variable)函数
$\UV : (\PiE \times \PiE) \rightarrow \powerset(\Uv)$
\[ \UV X = \{x \mbar (x,y) \in \SF X\} \]
一个类型匹配系统 $(V,X)$ 若满足 $\Solved (V,X)$ 则被叫做{\it 已解}的。
\[ \begin{split}
\Solved (V,X) = (\forall x\ &y.\ (x,y) \in \SF X \Rightarrow
  x \in V \ \land\ \BV y \cap \UV X = \emptyset)\ \land \\
  & (\forall x\ y_1\ y_2.\ (x,y_1) \in \SF X \ \land\ (x,y_2) \in \SF X
  \Rightarrow (y_1 = y_2))
\end{split} \]
\end{defin}

有一些显然的性质,
\begin{equation} \Solved(V,X) \vdash \UV X \subseteq V \end{equation}
\begin{equation} \Solved(V,X) \vdash \forall x\ y.\ (x,y) \in X
\Rightarrow x \in \UV X \label{UVX} \end{equation}
\begin{equation} \label{Solved.sub.refl}
\Solved(V,X) \vdash \Solved(V,X - {u,u}) \end{equation}

\begin{lemma} \label{Lem.SFXF}
\[ \Solved(V,X) \vdash \SF X \in (\UV X \rightarrow \PiE) \]
即在已解形式下,有意义的方程组 $\SF X$ 就是一个$V$到$\PiE$的函数。
\end{lemma}
\begin{proof}
由定义 \ref{Def.Func},命题等价于
\[ \Solved(V,X) \vdash \forall x\ y_1\ y_2.\ (x,y_1) \in \SF X \ \land\ 
(x,y_2) \in \SF X \Rightarrow (y_1 = y_2) \ \land\ x \in \UV X \ \land\ 
y_1 \in \PiE \]
由定义 \ref{Def.Solved} 这是显然的。
\end{proof}
\begin{lemma}[引理\ref{Lem.SFXF}的推论] \label{Lem.SFXapplied}
结合引理\ref{Lem.SFXF}与公式 \ref{UVX}
\[ \Solved(V,X),\ (x,y) \in X \vdash \SF X\ x = y \]
\end{lemma}

\begin{defin}[部分函数的I扩展]
$\EI\ f$ 将一个部分函数 $f$ 扩展成完全函数
\[ \EI\ f\ x = \xif x \in \Dom f \xthen f\ x \xelse x \]
\end{defin}

\begin{algorithm}
\caption{二元组集的函数扩张 $\mathrm{AsFunc}$} \label{alg:AsFunc}
\begin{algorithmic}[1]
\Require 二元组集 $f \in \powerset(X \times Y)$ 满足
$\forall x\ y_1\ y_2.\ (x,y_1) \in f \ \land\ (x,y_2) \in f \Rightarrow
(y_1 = y_2)$
\Ensure $\EI f \in (X \rightarrow Y)$
\Function{F\ }{$v$}
\For {$(x,y) \in f$}
\If {$x = v$} \State 返回 $y$ \EndIf
\EndFor
\State 返回 $v$
\EndFunction
\State 输出 F
\end{algorithmic}
\end{algorithm}

\begin{theo}[已解方程组的解]
\[ \Solved (V,X) \vdash \MF(V,X) = \{ \Inst_V g
\circ \EI (\SF X) \mbar g \} \]
\end{theo}
\begin{proof} 由定义 \ref{Def.MF} 命题等价于
\[ \begin{split}
\Solved&(V,X) \vdash (\forall u_1\ u_2.\ (u_1,u_2) \in X
\Rightarrow \Inst_V f\ u_1 = \Inst_V f\ u_2) = \\
&(\exists g.\ \forall x.\ x \in \Uv \Rightarrow 
(f(x) = \Inst_V g(\EI (\SF X)\ x)))
\end{split} \]
当 $X = \emptyset$ 时等式左边即是 $T$,右边也是 $T$ 因为
$\Dom (\SF X) = \emptyset$ 故而 $\EI(\SF X) = I$ 这样右式就等于
\[ \exists g.\ \forall x.\ x \in \Uv \Rightarrow
(f\ x = \Inst_V\ g\ x)\]
而这样的 $g$ 是始终存在的,令 $g = f$ 则有
\[ \forall x.\ x \in \Uv \Rightarrow (f\ x = \Inst_V\ f\ x)\]
而由 $\Inst_V$ 的定义,当 $x \in \Uv$ 时,$\Inst_V f\ x = f\ x$,
故而右式就是恒真式。

\hfill

\noindent 接下来证明 $X \neq \emptyset$ 的情况,先
证明 $\Rightarrow$,即命题
\[ \begin{split}
X \neq \emptyset,\ 
\Solved(V,X),\ (\forall x\ y.\ &(x, y) \in \SF X \Rightarrow
(\Inst_V f\ x = \Inst_V f\ y)) \vdash\\
&\exists g.\ \forall u.\ u \in \Uv \Rightarrow
(f(u) = \Inst_V g(\EI (\SF X)\ u)) \end{split}\]
存在这样的 $g = f$,命题变为
\[ \begin{split}
X \neq \emptyset,\ 
\Solved(V,X),\ (&\forall x\ y.\ (x, y) \in \SF X \Rightarrow
(\Inst_V f\ x = \Inst_V f\ y)) \vdash\\
&\forall u.\ u \in \Uv \Rightarrow
(f\ u = \Inst_V f(\EI (\SF X)\ u))
\end{split} \]
由引理 \ref{Lem.SFXF},$\Dom(\SF X) = \UV X$,再由$\EI$的定义,
当 $u \notin \UV X$ 时,$\EI(\SF X) u = u$,
而由 $\Inst_V$ 的定义,在$u \in \Uv$时$\Inst_V u = u$,
故而最后只要下述证明命题
\[ \begin{split}
X \neq \emptyset,\ 
Solved(V,X),\ &(\forall x\ y.\ (x, y) \in \SF X \Rightarrow
(\Inst_V f\ x = \Inst_V f\ y)) \vdash\\
&\forall u.\ u \in \UV\ X
\Rightarrow (f\ u = \Inst_V f\ (\SF X\ u)) \end{split} \]
再由公式 \ref{UVX} 且 $X \neq \emptyset$
\[X \neq \emptyset,\ 
\Solved(V,X),\ (x, y) \in \SF X,\ 
\Inst_V f\ x = \Inst_V f\ y \vdash f\ x = \Inst_V f\ (\SF X\ x)\]
引理 \ref{Lem.SFXapplied}
\[X \neq \emptyset,\ 
\Solved(V,X),\ (x, y) \in \SF X,\ 
\Inst_V f\ x = \Inst_V f\ y \vdash f\ x = \Inst_V f\ y\]
而 $\Solved(V,X)\ \land\ (x, y) \in \SF X$ 所以 $x \in \Uv$ 所以
$\Inst_V\ f\ x = x$ 于是命题得证。

\hfill

\noindent 接下来证明 $\Leftarrow$,即命题
\[ \begin{split}
X \neq \emptyset,\ 
\Solved(V,X),\ (\forall u.\ &u \in \Uv \Rightarrow
(f(u) = \Inst_V g(\EI (\SF X)\ u))) \vdash \\
&\forall x\ y.\ (x, y) \in \SF X \Rightarrow
(\Inst_V f\ x = \Inst_V f\ y)
\end{split}\]
等价于
\[ \begin{split}
X \neq \emptyset,\ 
\Solved(V,X),\ (\forall u.\ &u \in \Uv \Rightarrow
(f(u) = \Inst_V g\ (\EI (\SF X)\ u))),\ (x, y) \in \SF X
\vdash \\
& \Inst_V f\ x = \Inst_V f\ y
\end{split}\]
$(x,y) \in \SF X$ 故而 $x \in \UV X$
故而 $x \in \Uv$ 这样 $\Inst_V f\ x = f\ x$ 且
$f\ x = \Inst_V g\ (\EI (\SF X)\ x)$

\noindent 且因为 $x \in \UV X$ 所以 $\EI (\SF X)\ x = \SF X\ x = y$
最后命题等价于
\[ \begin{split}
X \neq \emptyset,\ 
\Solved(V,X),\ (\forall u.\ &u \in \Uv \Rightarrow
(f(u) = \Inst_V g\ (\EI (\SF X)\ u))),\ (x, y) \in \SF X
\vdash \\
& \Inst_V g\ y = \Inst_V f\ y
\end{split}\]
对 $y$ 进行类型高度的递归法,只要证明下式命题就能得证。
\[ \begin{split}
\Solved(V,X),\ (\forall u.\ &u \in \Uv \Rightarrow
(f(u) = \Inst_V g\ (\EI (\SF X)\ u))),\ y \in \Uv,\ (x, y) \in \SF X
\vdash \\
& \Inst_V g\ y = \Inst_V f\ y
\end{split}\]
进而
\[ \begin{split}
\Solved(V,X),\ (\forall u.\ &u \in \Uv \Rightarrow
(f(u) = \Inst_V g\ (\EI (\SF X)\ u))),\ y \in \Uv,\ (x, y) \in \SF X
\vdash \\ & g\ y = f\ y
\end{split}\]
由前提中的 $(\forall u.\ u \in \Uv \Rightarrow
(f(u) = \Inst_V g\ (\EI (\SF X)\ u)))$
\[ \begin{split}
\Solved(V,X),\ (\forall u.\ &u \in \Uv \Rightarrow
(f(u) = \Inst_V g\ (\EI (\SF X)\ u))),\ y \in \Uv ,\ (x, y) \in \SF X
\vdash \\& g\ y = \Inst_V g\ (\EI (\SF X)\ y)
\end{split}\]
注意至今为止一直没用到的 $\Solved$ 的一个条件
\[ \Solved(V,X),\ (x,y) \in \SF X \vdash \BV y \cap \UV X = \emptyset \]
而 $\Dom (\SF X) = \emptyset$,这意味着  $\EI (\SF X)\ y = y$
于是命题变成
\[ \begin{split}
\Solved(V,X),\ (\forall u.\ &u \in \Uv \Rightarrow
(f(u) = \Inst_V g\ (\EI (\SF X)\ u))),\ y \in \Uv ,\ (x, y) \in \SF X
\vdash \\& g\ y = \Inst_V g\ y
\end{split}\]
因为 $y \in \Uv$ 所以 $\Inst_V g\ y = g\ y$ 命题得证。
\end{proof}

\begin{defin}[类型匹配系统的实例集] \label{Def.MI}
类型匹配系统 $(V,X) \in \EQs$ 在 $u \in \PiE$ 上的实例集 
$\MI\ (V,X)\ u \in \powerset(\PiE)$
\[\MI\ (V,X)\ u = \{ \Inst_V f\ u \mbar f \in \MF(V,X) \}\]
\end{defin}
\begin{lemma}[类型匹配系统的实例集的M等价] \label{Lem.MI.Meq}
\[ \forall M_1\ M_2\ u.\ u \in \PiE\ \land\ (M_1 \Meq M_2) 
\Rightarrow (\MI M_1\ u \Meq \MI M_2\ u) \]
\end{lemma}
\begin{proof} 由$\MI$的定义 \ref{Def.MI},M等价的定义 \ref{Def.Meq}
直接得到
\end{proof}

\begin{lemma} \label{Lem.RecIv}
\[ \Solved(V,X) \vdash \Inst_V (\Inst_V g \circ
\EI\ (\SF X)) = \Inst_V g \circ \Inst_V (\EI\ (\SF X))\]
\end{lemma}
\begin{proof} 等价于
\[ \Solved(V,X) \vdash \Inst_V (\Inst_V g \circ
\EI\ (\SF X))\ u = \Inst_V g\ (\Inst_V (\EI\ (\SF X))\ u)\]
对 $u$ 进行高度的归纳法,当 $\h u = 1$ 时,$u = \underline{(\ c\ )}$,
$c \in U$ 由 $\Inst$ 的定义 \ref{Def.Inst} 命题成立。

\noindent 当 $\h u = \Suc n$ 时,$u = \underline{(\ c\ v_1\ \cdots\ v_o\ )}
,\ \h v_1 \leq n,\ \cdots,\ \h v_o \leq n$,且有
\[ \Solved(V,X),\ \h u \leq n \vdash \Inst_V (\Inst_V g \circ
\EI\ (\SF X))\ u = \Inst_V g\ (\Inst_V (\EI\ (\SF X))\ u)
\tag{归纳假设} \]
由 $\Inst$ 的定义 \ref{Def.Inst} 命题等价于
\begin{align*}
\Solved(V,X) \vdash \Inst_V (\Inst_V g \circ
\EI\ (\SF X))\ v_1\ &= \Inst_V g\ (\Inst_V (\EI\ (\SF X))\ v_1)\\
\cdots&\\
\Solved(V,X) \vdash \Inst_V (\Inst_V g \circ
\EI\ (\SF X))\ v_o\ &= \Inst_V g\ (\Inst_V (\EI\ (\SF X))\ v_o)
\end{align*}
逐个应用归纳假设,命题得证。
\end{proof}

\begin{lemma} \label{L.QIMI}
\[ \Solved(V,X),\ \QV q = V,\ \QB q = 
\Inst_V\ (\EI (\SF X))\ u \vdash
\QI q = \MI\ (V,X)\ u \]
\end{lemma}
\begin{proof}
等价于
\[ \begin{split}
\Solved(V,&X),\ \QV q = V,\ \QB q = 
\Inst_V\ (\EI (\SF X))\ u\vdash\\
&\{\Inst_{\ \QV q} f\ \QB(q)\mbar f\} = \{\Inst_V (\Inst_V g \circ
\EI\ (\SF X))\ u \mbar g\} \end{split} \]
由引理 \ref{Lem.RecIv}
\[ \begin{split}
\Solved(V,&X),\ \QV q = V,\ \QB q =  
\Inst_V\ (\EI (\SF X))\ u\vdash\\
&\{\Inst_{\ \QV q} f\ \QB(q)\mbar f\} = \{\Inst_V g\ (
\Inst_V (\EI\ (\SF X))\ u) \mbar g\} \end{split} \]
这是显然成立的。
\end{proof}

引理 \ref{L.QIMI} 意味着很多,首先对于任意的类型匹配系统 $M \in \EQs$
如果其有可解形式 $M'$ 即 $\Solved M'\ \land\ M \Meq M'$,
那么必定存在存在一个 $q \in \PiAE$ 使得 $\QI q = \MI M$,
其次,算法 \ref{alg:QIMI} 可以由 $\Solved M'$ 求解 $q$,
其正确性直接由引理 \ref{L.QIMI} 得到。

\begin{theo} \label{T.QIMI.S}
\[ \forall M\ u.\ 
\Solved M \Rightarrow \exists q.\ \QI q = \MI M\ u \]
\end{theo}
\begin{proof} 由引理 \ref{L.QIMI} 直接得到。
\end{proof}


\begin{algorithm}
\caption{已解类型匹配系统的实例化 MI} \label{alg:MI}
\begin{algorithmic}[1]
\Require 已解的类型匹配系统$(V,X)$ 满足 $\Solved(V,X)$
\Require 类型 $u \in \PiE$
\Ensure $\Inst_V\ (\EI(\SF X))\ u$
\State $\mathrm{AsFunc}(\SF X) \rightarrow f$
\Comment {调用算法 \ref{alg:SF} 和算法 \ref{alg:AsFunc},
由引理 \ref{Lem.SFXF} $\SF X$ 满足
算法 \ref{alg:AsFunc} 的条件 }
\State 输出 $\Inst(V,f,u)$  \Comment {调用算法 \ref{alg:Iv}}
\end{algorithmic}
\end{algorithm}
\begin{algorithm}
\caption{已解类型匹配系统的全称量化类型 QIMI} \label{alg:QIMI}
\begin{algorithmic}[1]
\Require 已解的类型匹配系统$(V,X)$ 满足 $\Solved(V,X)$
\Require 类型 $u \in \PiE$
\Ensure $q \in \PiAE$ 满足 $\QI q = \MI (V,X)\ u$
\State $\MI((V,X),u) \rightarrow b$  \Comment {调用算法 \ref{alg:MI}}
\State 输出 $\mathrm{MakeQT}(V,b)$
\Comment {调用算法 \ref{alg:MakeQT}}
\end{algorithmic}
\end{algorithm}

\begin{defin}[无冲突全称量化类型]
定义$\PiAE \times \PiAE$上的二元关系 $\NC$ 表示两个全称量化类型的全称量化变量不相互冲突。
\[ q_1 \NC q_2 = (\QV q_1 \cap \Qv q_2 = \emptyset) \ \land\ 
(\QV q_2 \cap \Qv q_1 = \emptyset) \]
显然有
\begin{equation}
q_1 \NC q_2 \vdash \QV q_1 \cap \QV q_2 = \emptyset
\end{equation}
\end{defin}

\begin{theo}[类型匹配系统与实例化类型集交集的关系]
\[ q_1 \NC q_2 \vdash \QI q_1 \cap \QI q_2 =\ 
\MI\ (\QV q_1 \cup \QV a_2,\ \{\QB q_1 = \QB q_2\})\ \QB(q_1)\]
\end{theo}
\begin{proof} 简洁起见,令
\begin{align*}
&u_1 = \QB q_1&&u_2 = \QB q_2&\\
&V_1 = \QV q_1&&V_2 = \QV q_2&
\end{align*}
将QI的定义与MI的定义展开,原命题等价于
\[ \begin{split} q_1\ &\NC q_2 \vdash \\
&\{\Inst_{V_1} f\ u_1 \mbar \Inst_{V_1} f\ u_1 = \Inst_{V_2} g\ u_2 \}
= \{\Inst_{V_1 \cup V_2} f\ u_1 \mbar \Inst_{V_1 \cup V_2} f\ u_1
= \Inst_{V_1 \cup V_2} f\ u_2 \} \end{split} \]
由 $q_1 \NC q_2$ 可得
$(V_1 \cup V_2) \cap V_1 = V_1$,
$(V_1 \cup V_2) \cap V_2 = V_2$,
用引理 \ref{Lem.Iv.V},得到
\begin{align*}
&\Inst_{V_1 \cup V_2} f\ u_1 = \Inst_{V_1} f\ u_1&
&\Inst_{V_1 \cup V_2} f\ u_2 = \Inst_{V_2} f\ u_2&
\end{align*}
带入命题,命题变为
\[ q_1\ \NC q_2 \vdash
\{\Inst_{V_1} f\ u_1 \mbar \Inst_{V_1} f\ u_1 = \Inst_{V_2} g\ u_2 \}
= \{\Inst_{V_1} f\ u_1 \mbar \Inst_{V_1} f\ u_1= \Inst_{V_1} f\ u_2 \}\]
显然有
\[ \{\Inst_{V_1} f\ u_1 \mbar \Inst_{V_1} f\ u_1 = \Inst_{V_2} g\ u_2 \}
\supseteq
\{\Inst_{V_1} f\ u_1 \mbar \Inst_{V_1} f\ u_1= \Inst_{V_1} f\ u_2 \}\]
只要证明
\[ q_1\ \NC q_2 \vdash
 \{\Inst_{V_1} f\ u_1 \mbar \Inst_{V_1} f\ u_1 = \Inst_{V_2} g\ u_2 \}
\subseteq
\{\Inst_{V_1} f\ u_1 \mbar \Inst_{V_1} f\ u_1= \Inst_{V_1} f\ u_2 \}\]
即证明命题
\[ q_1\ \NC q_2 ,\ 
\Inst_{V_1} f\ u_1 = \Inst_{V_2} g\ u_2 \vdash \exists h.\ 
\Inst_{V_1} h\ u_1 = \Inst_{V_2} h\ u_2 \]
$h = (\lambda v.\ \xif v \in V_1 \xthen f\ v \xelse g\ v)$ 
是满足上述条件的实例,因为 $V_1 \cap V_2 = \emptyset$

\noindent 这样命题得证。
\end{proof}

\begin{defin}[无解的类型匹配系统]
谓词 $\NoSolution$ 表示一个类型匹配系统是无解的。
\[ \NoSolution M = (\MF M = \emptyset) \]
相反的 $\Solvable$ 表示有解
\[ \Solvable M = (\MF M \neq \emptyset) \]
$\MNS$ 是某个无解的类型匹配系统
\[ \NoSolution \MNS \]
$\MNS$ 具体的表达是无关紧要的,例如下面的类型匹配系统即可
\[ (\emptyset,\ \{u_1 = u_2\})\quad\quad u_1 \neq u_2 \]
\end{defin}

\begin{defin}[类型方程组中全部的类型变量]
对类型方程组 $X \in \powerset(\PiE \times \PiE)$
\[ \MV\ X = \bigcup_{(x,y) \in X} \mathrm{v}(x) \cup \mathrm{v}(y) \]
表示 $X$ 中每一个类型方程的所有类型变量的合并。
\end{defin}

\begin{defin}[类型匹配系统的实例化]
$\IM : (\Uv \rightarrow \PiE) \rightarrow \EQs \rightarrow \EQs$ 
将一个类型匹配系统的所有方程的等式两边都实例化
\[ \IM f\ (V,X) = \{(\Inst_V f\ x,\ \Inst_V f\ y) \mbar (x,y) \in X\}\] 
\end{defin}
\begin{lemma} \label{L.IIuv}
\[ (u,v) \in X,\ u \in V,\ \Inst_V g\ u = \Inst_V g\ v,\ x \in \PiE
\vdash \Inst_V g\ (\Inst_V\ \{(u,v)\}\ x) = \Inst_V\ g\ x \]
\begin{proof} 对 $x$ 进行类型高度的归纳法,只需要证明叶子类型时的情况
\[ (u,v) \in X,\ u \in V,\ \Inst_V g\ u = \Inst_V g\ v,\ x \in \Uv
\vdash \Inst_V g\ (\Inst_V\ \{(u,v)\}\ x) = \Inst_V\ g\ x \]
由 $\Inst_V$ 的定义,此时 $\Inst_V\ \{(u,v)\}\ x = \{(u,v)\}\ x$

\noindent 当 $x \neq v$,$\Inst_V\ \{(u,v)\}\ x = x$,命题显然成立

\noindent 当 $x = v$,$\Inst_V\ \{(u,v)\}\ x = v$
\[ (u,v) \in X,\ u \in V,\ \Inst_V g\ u = \Inst_V g\ v,\ x \in \Uv
\vdash \Inst_V g\ v = \Inst_V\ g\ x \]
于前提 $\Inst_V g\ u = \Inst_V g\ v$ 命题得证。
\end{proof}
\end{lemma}

\begin{lemma}
\[ (u,v) \in X,\ u \in V \vdash (V,X) 
\Meq (V,(\IM\ \EI\{(u,v)\}\ \ (X - \{(u,v)\})) \cap \{(u,v)\})\]
\begin{proof} 命题等价于
\[ \begin{split} (u,v) \in X,\ &u \in V\ \vdash
\{g \mbar (x,y) \in X\ \land\  \Inst_V g\ x = \Inst_V g\ y \} = \\
\{g \mbar &(x,y) \in (X-\{(u,v)\})\ \land\  \Inst_V g\ (\Inst_V\ \{(u,v)\}\ x) = 
\Inst_V g\ (\Inst_V\ \{(u,v)\}\ y)\ \land\\
&\quad\quad\Inst_V g\ u = \Inst_V g\ v \} \end{split} \]
分步证明等式两端
\begin{align}
\begin{split} (u,v) \in X,\ u \in V,&\ 
\forall x\ y.\ (x,y) \in X\Rightarrow (\Inst_V g\ x = \Inst_V g\ y)
\vdash \\ \forall x\ y.\ (x,y) \in\ (X-\{(u,v)\})&\Rightarrow
(\Inst_V g\ (\Inst_V\ \{(u,v)\}\ x) = 
\Inst_V g\ (\Inst_V\ \{(u,v)\}\ y))\\
&\land\ (\Inst_V g\ u = \Inst_V g\ v)
\end{split} \tag{1}\\ \notag \\
\begin{split} (u,v) \in X,\ u \in V,&\ \Inst_V g\ u = \Inst_V g\ v,\ \\
\forall x\ y.\ 
(x,y) \in (X-\{(u,v)\})&\Rightarrow
(\Inst_V g\ (\Inst_V\{(u,v)\}\ x) = 
\Inst_V g\ (\Inst_V\{(u,v)\}\ y)) \vdash \\
\forall x\ y.\ (x,&y) \in X\Rightarrow  \Inst_V g\ x = \Inst_V g\ y
\end{split} \tag{2}
\end{align}
对于 (1) 有
\[ \begin{split} (u,v) \in X,\ u \in V,\ 
\forall x\ y.\ (x,y)\ &\in X\Rightarrow (\Inst_V g\ x = \Inst_V g\ y)
\vdash \\
\Inst_V g\ u = \Inst_V g\ v
\end{split} \]
对于 (2) ,$\Inst_V g\ u = \Inst_V g\ v$ 就在前提列表中

\noindent 在这两种情况都可以应用引理 \ref{L.IIuv},这样命题就瞬间被简化成非常简单的形式
\begin{align}
\begin{split} (u,v) \in X,\ u \in V,&\ 
\forall x\ y.\ (x,y) \in X\Rightarrow(\Inst_V g\ x = \Inst_V g\ y)
\vdash \\ (\forall x\ y.\ 
(x,y) \in &(X-\{(u,v)\})\Rightarrow \Inst_V g\ x = \Inst_V g\ y)\ 
\land\ (\Inst_V g\ u = \Inst_V g\ v)
\end{split} \tag{1'}\\
\begin{split} (u,v) \in X,\ u \in V,&\ \Inst_V g\ u = \Inst_V g\ v,\\
\forall x\ y.\ &(x,y)
\in (X-\{(u,v)\})\Rightarrow \Inst_V g\ x = \Inst_V g\ y \vdash \\
\forall x\ y.\ (x,y) \in &X\Rightarrow (\Inst_V g\ x = \Inst_V g\ y)
\end{split} \tag{2'}
\end{align}
这些显然是成立的。
\end{proof}
\end{lemma}

\begin{defin}
$\SBP\ (V_x,X_x)\ (V_o,X_o)$ 描述类型匹配系统 
$(V_x,X_x),\ (V_o,X_o) \in \EQs$ 具有如下性质
\[ \SBP\ (V_x,X_x)\ (V_o,X_o) = \Solved(V_o,X_o) \ \land\ 
(V_x \cap V_o = \emptyset)\ \land\ (\MV X_x \cap V_o = \emptyset) \]
\end{defin}
\begin{lemma}
\[ \begin{split} \SBP\ &(V_x,X_x)\ (V_o,X_o),\ (x,x) \in X_x \vdash
\SBP\ (V_x,X_x - \{(x,x)\})\ (V_o,X_o)\ \land\\
&((V_x,X_x - \{(x,x)\})\cup(V_o,X_o) \Meq (V_x,X_x)\cup(V_o,X_o)) \end{split}\]
\begin{proof}
$((V_x,X_x - \{(x,x)\})\cup(V_o,X_o) \Meq (V_x,X_x)\cup(V_o,X_o))$
是显然的,由引理 \ref{L.Meq.refl} 直接得到。
现证明
\[ \SBP\ (V_x,X_x)\ (V_o,X_o),\ (x,x) \in X_x \vdash
\SBP\ (V_x,X_x - \{(x,x)\})\ (V_o,X_o)\]
由$\SBP$ 的定义
\[ \SBP\ (V_x,X_x)\ (V_o,X_o),\ (x,x) \in X_x \vdash 
(\MV (X_x - \{(x,x)\}) \cap V_o = \emptyset) \]
这是显然的。
\end{proof}
\end{lemma}

\begin{lemma}
\[ \begin{split} \SBP\ &(V_x,X_x)\ (V_o,X_o),\ (x,y) \in X_x,\ 
x = \underline{(\ c\ v_1\ \cdots\ v_n \ )},\ 
y = \underline{(\ c\ w_1\ \cdots\ w_n \ )} \vdash\\
&\SBP\ (V_x,X_x')\ (V_o,X_o)\ \land\ 
((V_x,X_x')\cup(V_o,X_o) \Meq (V_x,X_x)\cup(V_o,X_o)) \end{split}\]
\[ \text{其中\quad} X_x' = X_x - \{(x,y)\} \cup \bigcup_{i=1\cdots n} \{(v_i,w_i)\} \]
\begin{proof}
首先证明 $((V_x,X_x')\cup(V_o,X_o) \Meq (V_x,X_x)\cup(V_o,X_o))$ 部分,

\noindent 这是成立的因为
$x = \underline{(\ c\ v_1\ \cdots
\ v_n\ )},\ y = \underline{(\ c\ w_1\ \cdots\ w_n\ )}$ 下
\[ \{f \mbar \Inst_V f\ x = \Inst_V f\ y \} = 
\{ f \mbar \bigwedge_{i=1\cdots n} \Inst_V f\ v_i = \Inst_V f\ w_i \} \]
然后证明 $\SBP\ (V_x,X_x')\ (V_o,X_o)$ 只要证明如下命题即可
\[ \begin{split} \SBP\ &(V_x,X_x)\ (V_o,X_o),\ (x,y) \in X_x,\ 
x = \underline{(\ c\ v_1\ \cdots\ v_n \ )},\ 
y = \underline{(\ c\ w_1\ \cdots\ w_n \ )} \vdash\\
&\MV(X_x - \{(x,y)\} \cup \bigcup_{i=1\cdots n} \{(v_i,w_i)\}) = 
\MV(X_x)
\end{split} \]
这是成立的因为
\[ x = \underline{(\ c\ v_1\ \cdots\ v_n \ )},\ 
y = \underline{(\ c\ w_1\ \cdots\ w_n \ )} \vdash
\mathrm{v}(x)\cup\mathrm{v}(y) = \bigcup_{i=1\cdots n} \mathrm{v}(v_i)
\cup \mathrm{v}(w_i) \]
\end{proof}
\end{lemma}

\begin{lemma}
\[ \begin{split} \SBP\ &(V_x,X_x)\ (V_o,X_o),\ (x,y) \in X_x,\ 
x = \underline{(\ u\ )},\ u \in V_x \vdash\\
&\SBP\ (V_x',X_x')\ (V_o',X_o')\ \land\ 
((V_x',X_x')\cup(V_o',X_o') \Meq (V_x,X_x)\cup(V_o,X_o)) \end{split}\]
其中
\begin{align*}
&X_x' = \IM\ (\EI \{(x,y)\})\ (X_x - \{(x,y)\})&
&V_x' = V_x - \{u\}&\\
&X_o' = X_o \cup \{(x,y)\}&
&V_o' = V_o \cup \{u\}&
\end{align*}
\begin{proof}
令$V = V_x \cup V_o$
\[ V_x' \cup V_o' = (V_x - \{u\}) \cup (V_o \cup \{u\}) = V \]
首先证明 $(V_x',X_x')\cup(V_o',X_o') \Meq (V_x,X_x)\cup(V_o,X_o)$ 部分,即
\[ \begin{split} \SBP\ &(V_x,X_x)\ (V_o,X_o),\ (x,y) \in X_x,\ 
x = \underline{(\ u\ )},\ u \in V_x \vdash\\
&(\MF(V,X_x' \cup X_o') = \MF(V,X_x \cup X_o)) \end{split}\]
由引理 \ref{L.MF.XUX}
\[ \begin{split} \SBP\ &(V_x,X_x)\ (V_o,X_o),\ (x,y) \in X_x,\ 
x = \underline{(\ u\ )},\ u \in V_x \vdash\\
&(\MF(V,X_x') \cap \MF(V,X_o') = \MF(V,X_x) \cap \MF(V,X_o))
\end{split}\]

\noindent 这是成立的因为
$x = \underline{(\ c\ v_1\ \cdots
\ v_n\ )},\ y = \underline{(\ c\ w_1\ \cdots\ w_n\ )}$ 下
\[ \{f \mbar \Inst_V f\ x = \Inst_V f\ y \} = 
\{ f \mbar \bigwedge_{i=1\cdots n} \Inst_V f\ v_i = \Inst_V f\ w_i \} \]
然后证明 $\SBP\ (V_x,X_x')\ (V_o,X_o)$ 只要证明如下命题即可
\[ \begin{split} \SBP\ &(V_x,X_x)\ (V_o,X_o),\ (x,y) \in X_x,\ 
x = \underline{(\ c\ v_1\ \cdots\ v_n \ )},\ 
y = \underline{(\ c\ w_1\ \cdots\ w_n \ )} \vdash\\
&\MV(X_x - \{(x,y)\} \cup \bigcup_{i=1\cdots n} \{(v_i,w_i)\}) = 
\MV(X_x)
\end{split} \]
这是成立的因为
\[ x = \underline{(\ c\ v_1\ \cdots\ v_n \ )},\ 
y = \underline{(\ c\ w_1\ \cdots\ w_n \ )} \vdash
\mathrm{v}(x)\cup\mathrm{v}(y) = \bigcup_{i=1\cdots n} \mathrm{v}(v_i)
\cup \mathrm{v}(w_i) \]
\end{proof}
\end{lemma}

\begin{algorithm}
\caption{ $\mathrm{SolveM_r}$ } \label{alg:SolveMr}
\begin{algorithmic}[1]
\Require 任意的类型匹配系统 $(V_x,X_x)$,
已解的类型匹配系统 $\Solved(V_o,X_o)$
\Statex 满足 $X_x \neq \emptyset \ \land\ \SBP\ (V_x,X_x)\ (V_o,X_o)$
\Ensure 类型匹配系统 $(V_x',X_x')$,已解的类型匹配系统 $\Solved (V_o',X_o')$
\Statex 满足 $\SBP\ (V_x,X_x)\ (V_o,X_o)
\ \land\ ((V_x,X_x) \cup (V_o,X_o) \Meq (V_x',X_x') \cup 
(V_o',X_o')) $
\State 从 $X_x$ 中随意取出元素 $(x,y) \in X_x$,
$x = \underline{(\ c\ v_1\ \cdots\ v_n \ )},\ 
y = \underline{(\ d\ w_1\ \cdots\ w_m \ )}$
\State $X_x - (x,y) \rightarrow X_x$
\Comment {接下来将削除方程组中的等式 $(x,y)$ 而保持方程组的解不变}
\If {$x = y$} \label{l:mrif1}
\State 输出 $((V_x,X_x),(V_o,X_o))$ \label{l:mro1}
\Comment{恒等式的情况可以直接削除} \Else
\If {$n \neq 0 \ \land\ m \neq 0$} \label{l:mrif2}
\Comment {若等式两边均不是类型变量或单个类型}
\If {$c = d$}
\Comment {$(c = d) \Rightarrow (n = m)$}
\For {$i = 1\cdots n$}
\Comment {等式$(x,y)$就可以分解为$n$个子等式 $(v_i,w_i)$}
\State $\{(v_i,w_i)\} \cup X_x \rightarrow X_x$
\Comment{将这些等式分别加入方程组}
\EndFor
\State 输出 $((V_x,X_x),(V_o,X_o))$ \label{l:mro2}
\Comment{就完成了$(x,y)$ 的削除}
\Else \State 输出 $(\MNS,\MNS)$ \Comment{$c \neq d$ 时,方程组一定无解}
\label{l:mro3} \EndIf
\Else \Comment{等式两边某一方是类型变量或单个类型}
\If {$c \in \Uv\ \lor\ d \in \Uv$} 
\Comment{若等式两边某一方是类型变量}
\State 令 $c,\ d$ 中属于 $\Uv$ 的为 $u$,另一方为 $v$
\State $\{(u,v)\} \cup X_o \rightarrow X_o\quad,\quad
\{u\} \cup V_o \rightarrow V_o\quad,\quad V_x - \{u\} \rightarrow
V_x\quad,\quad \{\} \rightarrow X_x'$
\For {$(s,t)\in X_x$}
\State $\{(\MI((V_o,X_o),s),\MI((V_o,X_o),t))\} \cup X_x' \rightarrow X_x'$
\Comment {$\MI$ 为算法 \ref{alg:MI}}
\EndFor
\State 输出 $((V_x,X_x'),(V_o,X_o))$ \label{l:mro4}
\Else \Comment{若等式两边都是单个类型而不是类型变量}
\State 输出 $(\MNS,\MNS)$ \Comment{因为$x \neq y$ 这时也是不可解的}
\label{l:mro5} \EndIf
 \EndIf \EndIf
\end{algorithmic}
\end{algorithm}

\begin{algorithm}
\caption{求解类型匹配系统 SolveM } \label{alg:SolveM}
\begin{algorithmic}[1]
\Require 任意的类型匹配系统 $(V,X)$
\Ensure $M' \in \EQs$ 满足 $M'$ 是已解的即 $\Solved M'$ 或者等于 $\MNS$ 
,且 $M \Meq M'$
\State $\{\} \rightarrow V' \quad,\quad \{\} \rightarrow X'$
\While {$(V,X) \neq \MNS \ \land\ X \neq \emptyset$}
\State $\mathrm{SolveM_r}((V,X),(V',X')) \rightarrow ((V,X),(V',X'))$
\Comment {$\mathrm{SolveM_r}$ 是算法 \ref{alg:SolveMr}}
\EndWhile
\State 输出 $(V',X')$
\end{algorithmic}
\end{algorithm}


算法 SolveM 用于求解任意给定的类型匹配系统,主要是对子算法 \SolveMr
的迭代。接下来将分析 \SolveMr 的性质以证明 SolveM 的正确性。

\begin{lemma}[\SolveMr 的正确性1]
\[ \Solved\ M_o,\ X_x \neq \emptyset \vdash
(\mathrm{SolveM_r}(M_x,M_o) = 
(M_x',M_o'))\ \land\ (M_x \cup M_o \Meq M_x' \cup M_o') \]
其中 \begin{align*}
&M_x = (V_x,X_x)&&M_o = (V_o,X_o)&
&M_x' = (V_x',X_x')&&M_o' = (V_o',X_o')&
\end{align*}
\begin{proof} 对任意的 $(x,y) \in X_x$,在行 \ref{l:mro1} 的情况下
命题等价于
\[ \Solved(V_o,X_o),(x,x) \in X_x \vdash
(V_o \cup V_x, X_o \cup X_x) \Meq (V_o \cup V_x, X_o \cup X_x
- \{(x,x)\})  \]
由引理 \ref{L.Meq.refl} 直接证明。

\hfill

\noindent 行 \ref{l:mro2} 的输出,命题等价于
\[ \begin{split}
\Solved(V_o,X_o)&,\ (x,y) \in X_x,\ x = \underline{(\ c\ v_1\ \cdots
\ v_n\ )},\ y = \underline{(\ c\ w_1\ \cdots\ w_n\ )} \vdash\\
&(V_o \cup V_x, X_o \cup X_x) \Meq (V_o \cup V_x, X_o \cup X_x
- \{(x,y)\} \cup \bigcup_{i=1\cdots n} (v_i,w_i)) \end{split} \]
这是成立的,因为 $x = \underline{(\ c\ v_1\ \cdots
\ v_n\ )},\ y = \underline{(\ c\ w_1\ \cdots\ w_n\ )}$ 下
\[ \{f \mbar \Inst_V f\ x = \Inst_V f\ y \} = 
\{ f \mbar \bigwedge_{i=1\cdots n} \Inst_V f\ v_i = \Inst_V f\ w_i \} \]

\hfill

\noindent 行 \ref{l:mro3} 的情况下,$M_x \cup M_o$ 是无解的因为
$(x,y) \in X_x, x = \underline{(\ c\ v_1\ \cdots
\ v_n\ )},\ y = \underline{(\ d\ w_1\ \cdots\ w_p\ )}$ 时
\[ c \neq d \Rightarrow \nexists f.\ \Inst_V f\ x = \Inst_V f\ y \]
行 \ref{l:mro5} 的无解情况也是一样的。

\hfill

\noindent 行 \ref{l:mro4} 的输出是主要的难点。

\end{proof}
\end{lemma}


\begin{theo}[类型匹配系统的求解]
\[ \forall M.\ \exists M'.\ (M \Meq M') \ \land\ (\Solved M' \lor 
\NoSolution M') \]
另一个等价的表达是
\[ \forall M.\ \Solvable M \Rightarrow \exists M'.\ (M \Meq M')\ \land\ 
\Solved M'\]
\begin{proof} 使用下述算法对任意的 $M$ 找出与其M等价的 $M'$ 
且要么是 $\Solved M'$ 要么是 $\NoSolution M'$
\end{proof}
\end{theo}

\begin{theo} 
\[ \forall q_1\ q_2.\ q_1,\ q_2 \in \Pi_\mathrm{E}^\forall 
\ \land\ (\mathrm{QI}\ q_1 \cap \mathrm{QI}\ q_2 \neq \emptyset)
\Rightarrow
\exists q.\ q \in \Pi_\mathrm{E}^\forall\ \land\ ((\mathrm{QI}\ q_1)
\cap (\mathrm{QI}\ q_2) = \mathrm{QI}\ q) \]
\end{theo}

\begin{proof} 证明等价命题
\[ \begin{split}
\forall n\ q_1\ q_2\ u_1\ u_2.\ q_1,\ &q_2 \in \Pi_\mathrm{E}^\forall 
\ \land\ (\mathrm{QI}\ q_1 \cap \mathrm{QI}\ q_2 \neq \emptyset)
\ \land\ (u_1 = \mathrm{QB}\ q_1) \ \land\  (u_2 = \mathrm{QB}\ q_2)\\
\land\ (n = &\max(\mathrm{h}\ u_1,\ \mathrm{h}\ u_2)) \Rightarrow
\exists q.\ q \in \Pi_\mathrm{E}^\forall\ \land\ ((\mathrm{QI}\ q_1)
\cap (\mathrm{QI}\ q_2) = \mathrm{QI}\ q)
\end{split} \]
对 $n$ 使用归纳法

\noindent 在 $n = 0$ 时,由定理 \ref{T.cah} 性质1
不存在 $q_1,\ q_2$ 使得 $n = \max(\mathrm{h}\ q_1,\ \mathrm{h}\ q_2)$

\noindent 在 $n = 1$ 时,由定理 \ref{T.cah} 性质1
$\quad \mathrm{h}\ q_1 = \mathrm{h}\ q_2 = 1\quad$
由定理 \ref{T.cah} 性质3
\[ u_1 = \underline{(\ c_1\ )} \quad\quad u_2 = \underline{(\ c_2\ )}
\quad\quad c_1,\ c_2 \in U \]
\end{proof}


类型变量主要用于全称量化中,
集合$U_0$ 表示预定义的类型如整数 int 以及类型构造器(Type Constructor)
如列表类型构造器 α list,


$\ES$ 形式系统是一种改造过的 λ2 演算,且非常接近SML语言


但抛开与 λ2 的联系而
直接论述$\ES$ 形式系统本身更易与表述。先暂且不管与 λ2 的联系而开始直接
论述。

\subsection{类型系统}

$\ES$

类似λ2演算,令字符串集合 $U_v$ 表示$\ES$的类型系统上所有的类型变量,
字符串集合$U_0$表示所有预定义的类型,集合$ \UE = U_v \cup U_0 $
\[ \PiE = \bnf{\UE \mbar \UE \rightarrow \UE} \]
表示所有的非全称量化类型或简单地叫做普通类型,
\[ \PiAE = \bnf{\PiE \mbar \forall U_v\ \PiAE } \]
表示全称量化类型或简单地叫做量化类型。


