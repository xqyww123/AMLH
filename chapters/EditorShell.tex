\chapter{$\ES$编辑壳层}

$\amlh$ 已经在上一章论述清楚。但$\amlh$只是一个抽象机理论,尽管被
定义在 HOL 交互式证明工具的 HOL 逻辑上,借助定理证明工具似乎具象了一点,
但显然不能让用户直接操作数学命题与定理。
需要有一个壳层包裹起$\amlh$理论并向外提供给用户编辑程序的功能,
这是本章将论述的编辑壳层$\ES$的意义。

\section{$\ES$概述}

$\ES$ 是一个形式语言,且具有同名为$\ES$的
基于 System F 的一阶多态类型系统。围绕此构建起同名为$\ES$的状态机,
最后是名为$\ES$编辑壳层的软件工具。

接下来将分别论述$\ES$形式语言、$\ES$类型系统、$\ES$状态机。

\section{$\ES$ 形式语言}

首先定义类型。

\begin{defin}[类型集$\PiE$]
字符串集合 $U$ 为用于表示类型的字母表,
函数$\xa : U \mapsto \mathbb{N}$表示类型的元数(Arity),
从$U$元数为0的子集$U_\mathrm{v}$:
$U_\mathrm{v} \subseteq U\ \land\ \forall u.\ u \in U_\mathrm{v}
\Rightarrow \xa(u) = 0$ 表示类型变量,主要用于全称量化。

所有由字母表 $U$ 组成的字符串记为集合$\mathrm{String}(U)$,
$\ES$ 的类型集 $\Pi_\mathrm{E}$ 是满足以下条件的$\mathrm{String}(U)$最小子集
\begin{equation} \label{Def.PiE}
\forall u.\ u \in U \Rightarrow \forall \bm{v}.\ \bm{v} 
\in \Pi_\mathrm{E}^{\xa(u)} \Rightarrow \underline{(u\ \bm{v}_1\ 
\cdots\ \bm{v}_{\xa(u)})} \in \Pi_\mathrm{E}
\end{equation}
其中 $\Pi_\mathrm{E}^{\xa(u)}$ 表示 $\Pi_\mathrm{E}$ 的 $\xa(u)$ 维向量
空间,特别的 $\Pi_\mathrm{E}^0 = \{\mathbf{0}\}$
\end{defin}

\begin{example}[$\Pi_\mathrm{E}$] 以下命题成立:
\[ \forall u.\ u \in U\ \land\ (\xa(u) = 0)\ \Rightarrow\ 
\underline{(u)} \in \Pi_\mathrm{E} \]
\[ \forall u\ v.\ u \in U\ \land\ v \in U_\mathrm{v}\ \land\ (\xa(u) = 1)
\ \Rightarrow\ \underline{(u\ v)} \in \Pi_\mathrm{E} \]
\[ \forall u\ v.\ u \in U\ \land\ v \in \Pi_\mathrm{E}\ \land\ (\xa(u) = 1)
\ \Rightarrow\ \underline{(u\ v)} \in \Pi_\mathrm{E} \]
\end{example}

\begin{theo}(类型的结构) \label{TS}
\[ \forall u'.\ u' \in \PiE \Rightarrow \exists! u\ \bm{v}.\ 
u \in U_\mathrm{\bm{v}}\ \land\ \bm{v} \in \PiE^{\xa(u)}\ \land\ 
(u' = \underline{(u\ \bm{v}_1\ \cdots\ \bm{v}_{\xa(u')})}) \]
\end{theo}
\begin{proof} 首先唯一性是显然的,由字符串理论就可以得到。
对存在性的证明使用反证法,假设
\begin{equation} \label{Hypo.TS}
\exists u'.\ u' \in \PiE \Rightarrow \forall u\ \bm{v}.\ 
u \in U_\mathrm{\bm{v}}\ \land\ \bm{v} \in \PiE^{\xa(u)}\ \land\ 
(u' \neq \underline{(u\ \bm{v}_1\ \cdots\ \bm{v}_{\xa(u')})})
\end{equation}
现证明 $\PiE - \{u'\}$ 满足条件 \ref{Def.PiE} 而
$\PiE - \{u'\} \subset \PiE$ 这样就构造了悖论,因为 $\PiE$ 不再是满足
条件 \ref{Def.PiE} 的最小子集,$\PiE - \{u'\}$ 比 $\PiE$更小。
即证明
\[ \forall u.\ u \in U \Rightarrow \forall \bm{v}.\ 
\bm{v} \in \PiE^{\xa(u)} - \{u'\} \Rightarrow
\underline{(u\ \bm{v}_1\ \cdots\ \bm{v}_{\xa(u)})} \in \PiE - \{u'\} \]
因为有
\[ \forall u.\ u \in U \Rightarrow \forall \bm{v}.\ 
\bm{v} \in \Pi_\mathrm{E}^{\xa(u)} \Rightarrow
\underline{(u\ \bm{v}_1\ \cdots\ \bm{v}_{\xa(u)})} \in \Pi_\mathrm{E} \]
所以只要证明
\[ \underline{(u\ \bm{v}_1\ \cdots\ \bm{v}_{\xa(u)})} \neq u' \]
而由反证假设 \ref{Hypo.TS} 这是成立的,故而悖论被构造进而命题得证。
\end{proof}

定理 \ref{TS} 意味着一切类型$u$都具有且唯一地具有如下格式
\[ \underline{u_0\ \bm{v}_1\ \cdots\ \bm{v}_{\xa(u_0)}} \]
其中 $u_0 \in \mathrm{U}$,$\bm{v}_1,\ \cdots,\ \bm{v}_{\xa(u_0)} \in \PiE$

即每一个类型都构成一颗树,类型变量与0元类型构造器是叶子。

\begin{defin}[类型的构造器、元数、高度] \label{Def.Th}
函数 $\mathrm{c}:\Pi_\mathrm{E} \mapsto U$ 表示类型的构造器。
\[ \mathrm{c}\ \underline{(u\ v_1\ \cdots\ v_n)} = u\]
函数 $\xa:\Pi_\mathrm{E} \mapsto \mathbb{N}$ 表示类型的元数。
\[ \xa\ \underline{(u\ v_1\ \cdots\ v_n)} = n\]
$\xa:\Pi_\mathrm{E} \mapsto \mathbb{N}$ 不会跟上文定义的
$\xa:U \mapsto \mathbb{N}$冲突,因为定义域不重合,且两者具有相同的意义,
不会造成歧义。

\noindent 函数 $\mathrm{h}:\Pi_\mathrm{E} \mapsto \mathbb{N}$ 表示类型的高度。
\[ \mathrm{h}\ \underline{(u\ v_1\ \cdots\ v_n)} = 1 + \max(\mathrm{h}\ v_1,\ \cdots,\ 
\mathrm{h}\ v_n) \]
\end{defin}

有如下性质

\begin{theo}[类型的元数与高度的性质] \label{T.cah}
\[ \forall u.\ \mathrm{h}\ u  \geq 1 \quad\quad\text{(1)}\quad\quad
\quad\quad\quad\forall u.\ \xa(\mathrm{c}\ u) = \xa\ u \quad\quad\text{(2)} \]
\[ \forall u.\ (\mathrm{h}\ u = 1) \Rightarrow \exists \mathrm{c}.\ \mathrm{c} \in U
\ \land\ u = \underline{(\mathrm{c})} \tag{3} \]
\[ \begin{split}
\forall u.\ (\mathrm{h}\ u > 1) \Rightarrow \exists \mathrm{c}\ &\bm{v}.\ \mathrm{c} \in U
\ \land\ \bm{v} \in \PiE^{\xa(u)}\ \land\ u = \underline{
(\mathrm{c}\ \bm{v}_1\ \cdots\ \bm{v}_{\xa(u)})} \ \land\ \\
&(\forall i.\ 1 \leq i \leq \xa(u) \Rightarrow \mathrm{h}\ \bm{v}_i < \mathrm{h}\ u )
\end{split} \tag{4} \]
\end{theo}
\begin{proof} 由定义 \ref{Def.Th} 与定理 \ref{TS} 直接得到。
\end{proof}

这样就可以关于类型的高度进行归纳法。

\begin{defin}[全称量化类型 $\PiAE$]
集合 $\Pi_\mathrm{E}^\forall$ 表示全称量化类型,由所有满足如下语法的字符串构成。
\[ \Pi_\mathrm{E} \mbar \forall U_\mathrm{v}\ \Pi_\mathrm{E} \]
同样有记号 $\forall v_1\ \cdots\ v_n.\ b$ 表示 $\underline{\forall v_1
\ \cdots\  \forall v_n\ b}$
\end{defin}

\begin{defin}[全称量化类型的相关属性]
函数 $\mathrm{QV} : \Pi_\mathrm{E}^\forall \rightarrow \powerset(
U_\mathrm{v})$ 表示全称量化类型的绑定变量集。
\[ \mathrm{QV}(\forall v_1\ \cdots\ \v_n.\ b) = \{v_1,\ \cdots,\ v_n\} \]
函数 $\mathrm{QB} : \Pi_\mathrm{E}^\forall \rightarrow \Pi_\mathrm{E}$
表示全称量化的类型体。
\[ \mathrm{QB}(\forall v_1\ \cdots\ \v_n.\ b) = b \]
\end{defin}

\begin{defin}[实例化] \label{Def.Inst}
函数 $\mathrm{Inst} : (U_\mathrm{v} \rightarrow \Pi_\mathrm{E})
\rightarrow \Pi_\mathrm{E} \rightarrow \Pi_\mathrm{E}$ 对类型
进行变量实例化。
\begin{align*}
\mathrm{Inst}\ f\ \underline{(v)} &= \xif v \in U_\mathrm{v} \xthen
 f\ v \xelse \underline{(v)} \\
\mathrm{Inst}\ f\ \underline{(u\ v_1\ \cdots\ v_{\xa(u)})} &= 
\underline{(u\ f(v_1)\ \cdots\ f(v_{\xa(u)}))}
\end{align*}

函数 $\mathrm{Inst}_\forall : (U_\mathrm{v} \rightarrow \Pi_\mathrm{E})
\rightarrow \Pi_\mathrm{E}^\forall \rightarrow \Pi_\mathrm{E}$ 
实例化全称量化类型。
\begin{gather*}
\mathrm{Inst}_\forall\ f\ q = \mathrm{Inst}\ \ (\lambda v.\ \xif v \in
\mathrm{QV}(q) \xthen f\ v \xelse v)\ \ \mathrm{QB}(q)
\end{gather*}
即$\mathrm{Inst}_\forall$只会实例化全称量化的类型变量。
\end{defin}

\begin{defin}[α等价] \label{Def.aE}
二元关系$\sim_\alpha$ 定义为
\[ (q_1 \sim_\alpha q_2) = (\forall f.\ \mathrm{Inst}_\forall\ f\ q_1
= \mathrm{Inst}_\forall\ f\ q_2) \]
显然是一种等价关系,被叫做α等价。
\end{defin}

α等价类$\Pi_\mathrm{E}^\forall/[\sim_\alpha]$即是本质不同的全称量化类型。

\begin{defin}[实例化类型集] \label{Def.QI}
全称量化类型$q$的实例化类型集$\mathrm{QI}\ q$为
\[ \mathrm{QI}\ q = \{\mathrm{Inst}_\forall\ f\ q\mbar f \in 
(U_\mathrm{v} \rightarrow \Pi_\mathrm{E})\} \]
\end{defin}
\begin{theo}[实例化类型集的相等即是$\sim_\alpha$等价]
\[(q_1 \sim_\alpha q_2) = (\mathrm{QI}\ q_1 = \mathrm{QI}\ q_2)\]
\end{theo}
\begin{proof} 由定义 \ref{Def.aE} 与定义 \ref{Def.QI} 直接得到。
\end{proof}
\begin{theo}[实例化类型集非空]
\[\forall q.\ \mathrm{QI}\ q \neq \emptyset\]
\end{theo}
\begin{proof}
因为 $\quad\forall q.\ \mathrm{Inst}_\forall\ \I\ q = \mathrm{QB}\ q 
\quad$ 所以有 $\quad\forall q.\ \mathrm{QB}\ q \in \mathrm{QI}\ q$
\end{proof}

接下来尝试证明一个重要命题 
\[ \forall q_1\ q_2.\ q_1,\ q_2 \in \Pi_\mathrm{E}^\forall 
\ \land\ (\mathrm{QI}\ q_1 \cap \mathrm{QI}\ q_2 \neq \emptyset)
\Rightarrow
\exists q.\ q \in \Pi_\mathrm{E}^\forall\ \land\ ((\mathrm{QI}\ q_1)
\cap (\mathrm{QI}\ q_2) = \mathrm{QI}\ q) \]
并找到一个算法用于求解上述的 $q$

首先要引入诸多工具的定义。

\begin{defin}[类型方程组] 类型方程组集 $\EQs$ 是集合
\[ \EQs = \powerset(\Uv) \times \powerset(\PiE \times \PiE) \]
类型方程组是所有集合 $\EQs$ 中的元素。
\end{defin}
\begin{notation}[类型方程组的记号] 
记号
\[ \begin{Bmatrix}
x_1 &=& y_1 \\
x_2 &=& y_2 \\
x_3 &=& y_3 \\
&\cdots& 
\end{Bmatrix}_V \]
表示类型方程组
\[ (V,\ \{ (x_1, y_1),\ (x_2, y_2),\ (x_3, y_3),\ \cdots \}) \]
其中 $x_1 = y_1,\ \cdots$ 被叫做方程组中的方程,
$V$ 是类型方程组的变量集。
\end{notation}
\begin{defin}[类型方程的解与解集] \label{Def.MF}
一个类型方程组$(V,X)$的解$f$是一个$(\Uv \rightarrow \PiE)$函数,
满足
\[ \forall u_1\ u_2.\ (u_1,u_2) \in X \Rightarrow 
\Inst f\ u_1 = \Inst f\ u_2 \]
所有这样的解构成的集合叫解集。

函数 $\MF : \EQs \rightarrow \powerset(\Uv \rightarrow \PiE)$
将一个类型方程映射到其解集
\[ \MF X = \{ f \mbar \forall u_1\ u_2.\ (u_1,u_2) \in X
\Rightarrow \Inst_V f\ u_1 = \Inst_V f\ u_2 \} \]
\end{defin}

\begin{defin}[类型方程组的M等价] \label{Def.Meq}
二元等价关系 $\Meq$
\[ \forall A\ B.\ (A \Meq B) = (\MF A = \MF B) \]
\end{defin}
\begin{theo}
\[ \forall A\ v.\ A \Meq (A \cup \{(v,v)\}) \]
\end{theo}
\begin{proof} 由定义 \ref{Def.Meq} 定义 \ref{Def.MF} 直接得到
\end{proof}

\begin{defin}[有意义类型方程组] \label{Def.SF}
有意义(Senseful Form)类型方程组函数
$\SF : \powerset(\PiE \times \PiE) \rightarrow \powerset(\PiE \times \PiE)$
\[ \SF X = \{(x,y) \mbar (x,y) \in X \ \land\ x \neq y\} \]
即是削除了恒等式 $(x,x)$ 后的方程组。
\end{defin}

\begin{defin}[已解的类型方程组] \label{Def.Solved}
类型方程组在的{\it 已解}形式下的未定变量(Undefined Variable)函数
$\UDV : \powerset(\PiE \times \PiE) \rightarrow \powerset(\Uv)$
\[ \UDV X = \{x \mbar (x,y) \in \SF X \} \]
一个类型方程组 $X$ 若满足 $\Solved X$ 则被叫做{\it 已解}的。
\[ \begin{split}
\Solved X = (\forall x\ &y.\ (x,y) \in \SF X \Rightarrow
  x \in \Uv \ \land\ \BV y \cap \UDV X = \emptyset)\ \land \\
  & (\forall x\ y_1\ y_2.\ (x,y_1) \in \SF X \ \land\ (x,y_2) \in \SF X
  \Rightarrow (y_1 = y_2))
\end{split} \]
\end{defin}

已解方程组 $\Solved X$ 中
任何一个方程 $(x, y) \in X$,首先 $x \in \Uv$ 是一个变量,
这意味着
\[ \Solved X \Rightarrow \UDV X \subseteq \Uv \]
其次对于任意的 $x$ 至多只有一个 $y$ 使 $(x,y) \in X$,
且 $y$ 中不包含任何未定变量,这保证解 $f$ 在 $x$ 点处被唯一确定
 $f\ x = y$,已解方程组中
 所有的方程都如此就允许直接地写出已解方程组的解

\begin{theo}[已解方程组的解]
\[ \Solved X \vdash \MF X = \{ f \mbar (x,y) \in \SF X \ \land\ 
(f\ x = y) \} \]
更一般的,若将函数视作特殊的二元关系即 
$(\Uv \rightarrow \PiE) \subset (\Uv \times \PiE)$
那么
\[ \Solved X \vdash \MF X = \{ f \mbar f \in (\Uv \times \PiE) \ \land\ 
\SF X \subset f \} \]
\end{theo}
\begin{proof} 由定义 \ref{Def.MF} 命题等价于
\[ \Solved X \vdash (\forall u_1\ u_2.\ (u_1,u_2) \in X
\Rightarrow \Inst f\ u_1 = \Inst f\ u_2) = (\forall x\ y.\ 
(x,y) \in \SF X \ \land\ (f\ x = y))\]
由定义 \ref{Def.SF} 命题等价与
\[ \Solved X,\ (x,y) \in \SF X \vdash 
(\Inst f\ x = \Inst f\ y) = (f\ x = y) \tag{1} \]
由定义 \ref{Def.Solved}
\[ \Solved X,\ (x,y) \in \SF X \vdash x \in \Uv \tag{2} \]
且由定义 \ref{Def.Inst}
\[ x \in \Uv \vdash \Inst f\ x = f\ x \tag{3} \]
\end{proof}



\begin{theo} 
\[ \forall q_1\ q_2.\ q_1,\ q_2 \in \Pi_\mathrm{E}^\forall 
\ \land\ (\mathrm{QI}\ q_1 \cap \mathrm{QI}\ q_2 \neq \emptyset)
\Rightarrow
\exists q.\ q \in \Pi_\mathrm{E}^\forall\ \land\ ((\mathrm{QI}\ q_1)
\cap (\mathrm{QI}\ q_2) = \mathrm{QI}\ q) \]
\end{theo}

\begin{proof} 证明等价命题
\[ \begin{split}
\forall n\ q_1\ q_2\ u_1\ u_2.\ q_1,\ &q_2 \in \Pi_\mathrm{E}^\forall 
\ \land\ (\mathrm{QI}\ q_1 \cap \mathrm{QI}\ q_2 \neq \emptyset)
\ \land\ (u_1 = \mathrm{QB}\ q_1) \ \land\  (u_2 = \mathrm{QB}\ q_2)\\
\land\ (n = &\max(\mathrm{h}\ u_1,\ \mathrm{h}\ u_2)) \Rightarrow
\exists q.\ q \in \Pi_\mathrm{E}^\forall\ \land\ ((\mathrm{QI}\ q_1)
\cap (\mathrm{QI}\ q_2) = \mathrm{QI}\ q)
\end{split} \]
对 $n$ 使用归纳法

\noindent 在 $n = 0$ 时,由定理 \ref{T.cah} 性质1
不存在 $q_1,\ q_2$ 使得 $n = \max(\mathrm{h}\ q_1,\ \mathrm{h}\ q_2)$

\noindent 在 $n = 1$ 时,由定理 \ref{T.cah} 性质1
$\quad \mathrm{h}\ q_1 = \mathrm{h}\ q_2 = 1\quad$
由定理 \ref{T.cah} 性质3
\[ u_1 = \underline{(c_1)} \quad\quad u_2 = \underline{(c_2)}
\quad\quad c_1,\ c_2 \in U \]
\end{proof}


类型变量主要用于全称量化中,
集合$U_0$ 表示预定义的类型如整数 int 以及类型构造器(Type Constructor)
如列表类型构造器 α list,


$\ES$ 形式系统是一种改造过的 λ2 演算,且非常接近SML语言


但抛开与 λ2 的联系而
直接论述$\ES$ 形式系统本身更易与表述。先暂且不管与 λ2 的联系而开始直接
论述。

\subsection{类型系统}

$\ES$

类似λ2演算,令字符串集合 $U_v$ 表示$\ES$的类型系统上所有的类型变量,
字符串集合$U_0$表示所有预定义的类型,集合$ \UE = U_v \cup U_0 $
\[ \PiE = \bnf{\UE \mbar \UE \rightarrow \UE} \]
表示所有的非全称量化类型或简单地叫做普通类型,
\[ \PiAE = \bnf{\PiE \mbar \forall U_v\ \PiAE } \]
表示全称量化类型或简单地叫做量化类型。


