\chapter{用于智能合约的 \Eamlh 的实现}\label{Ch.Implementation}

\section{用于智能合约的抽象机器 \amlhS 在 HOL 上的实现}

首先论述 HOL 交互式定理证明器上用于智能合约的抽象机器 \amlhS 的实现,
这一实现通过 HOL 定理证明器上4867行的 SML 语言完成。
状态方面采用计算与状态分离的方案,将所有状态转移以指令的方式记录而在
智能合约所有计算完成后的最后写入区块链,这样更安全。

\begin{defin}[现象的定义] 类型 phenomenon 表示现象
\[ \begin{split}
\mathrm{phenomenon} \Coloneqq & \PhV\ \mathrm{num}\ 
\mathrm{num} \mbar \PhS\ \mathrm{phenomenon}\ 
\mathrm{phenomenon} \\
\mbar & \PhP \mathrm{phenomenon}
\end{split} \]
$\PhV x\ b$ 表示 $b$ 个二进制表示的数字 $x$。$\PhS p_1\ p_2$ 表示两个值
在内存上的直接拼接构成的现象。$\PhP p$ 表示对现象 $p$ 的指针。
\[ \begin{array}{lcllcl}
\PB\ (\PhV x\ b) &\coloneqq& b&
\PX\ (\PhV x\ b) &\coloneqq& x\\
\PB\ (\PhS p_1\ p_2) &\coloneqq& \PB p_1 + \PB p_2\quad\quad&
\PX\ (\PhS p_1\ p_2) &\coloneqq& \PX(p_1) + \PX(p_2) \cdot 256^{\PB(p_1)}\\
\PB\ (\PhP p) &\coloneqq& \mathrm{PointerSize} &&&
\end{array} \]
$\PX\ (\PhP p)$ 故意的不去定义。PointerSize 为常量固定为智能合约执行环境
的位数,一般为32。
现象的分割工具,函数 $\PL,\ \PU$
    \begin{align*} \PL n\ p &= \PhV\ (\PX p \Mod 256^n)\ n&
        \PU n\ p &= \PhV\ (\PX p \Div 256^n)\ (\PB p \dotminus n)&
    \end{align*}
合法的现象
\begin{gather*} \begin{align*}
\VP\ (\PhV\ x\ b) &\coloneqq (x < 256^b)&
\VP\ (\PhP\ p) &\coloneqq \VP p&
\end{align*}\\
\VP\ (\PhS\ p_1\ p_2) \coloneqq \VP p_1 \ \land\ \VP p_2
\end{gather*}
现象集的 0 元素是 $\mathbf{PhV}\ 0\ 0$,
占用0个比特位的值,且其是合法的。
\end{defin}

\begin{defin}[\amlhS 上的理解]
\amlhS 上理解的定义基本基于定义 \ref{Def.itp} ,只是固定了理解的现象集
的位数目。
    \[ \begin{split}
        \itp{\alpha} \Coloneqq \mathbf{Noesis}\ &(\alpha \rightarrow \phenomenon)
        \ (\phenomenon \rightarrow \alpha)\\
    &(\alpha\ \mathrm{set})\ (\phenomenon\ \mathrm{set})\ \mathrm{num}
    \end{split} \]
\begin{gather*}
\begin{align*}
\mathbf{NOE\_LIGHT}\ (\mathbf{Noesis}\ l\ tr\ s_e\ s_p\ s) & = l&
\mathbf{NOE\_TRANSCEND}\ (\mathbf{Noesis}\ l\ tr\ s_e\ s_p\ s) & = tr&\\
\mathbf{NOE\_SET}\ (\mathbf{Noesis}\ l\ tr\ s_e\ s_p\ s) & = s_e&
\mathbf{NOE\_SIZE}\ (\mathbf{Noesis}\ l\ tr\ s_e\ s_p\ s) &= s&
\end{align*}\\
\mathbf{NOE\_PSET}\ (\mathbf{Noesis}\ l\ tr\ s_e\ s_p\ s)
 = s_p \cap \{ p \mbar \PB p = s \}
 \end{gather*}
同样有记号
\begin{gather*}
\begin{align*}
    \mathbf{Li}_i\ & \coloneqq \mathbf{NOE\_LIGHT}\ i&
    \mathbf{Tr}_i\ & \coloneqq \mathbf{NOE\_TRANSCEND}\ i &
\end{align*}\\ \begin{align*}
    \mathbf{Sp}_i\ & \coloneqq \mathbf{NOE\_PSET}\ i&
    \mathbf{Si}_i\ & \coloneqq \mathbf{NOE\_SIZE}\ i&
    \mathbf{Se}_i\ & \coloneqq \mathbf{NOE\_SET}\ i&
\end{align*} \end{gather*}
$\mathbf{Si}_i$ 表示理解 $i$ 的现象集的比特数。
\end{defin}

\begin{defin}[\amlhS 的 Noesis 对应] 在定义 \ref{Def.TR} 的基础上
加入现象合法性。
\[ p \widesim{i} e \coloneqq \mathbf{V}_i\ \land\ \VP\ p\ \land\ 
p \in \mathbf{Sp}_i\ \land\ (\mathbf{Tr}_i\ p = e) \]
\end{defin}

其余理论均与第 \ref{Ch.AmLH} 章相同。

\begin{defin}[链的抽象表达]
\amlhS 将链上数据抽象为各个由标识区分的键值表,使用有限映射表示,
类型 chain 为此别名 
\[ \mathrm{chain} \Coloneqq (\mathrm{string},\ \mathrm{phenomenon})
\mapsto \mathrm{phenomenon} \]
链数据是作为字符串的表标识与作为现象的键的元组到作为现象的值的有限映射。
$\Dom_c\ \phi,\ \Ima_c\ \phi$ 表示一个链数据 $\phi$ 的名为 $c$ 的表
的所有键与值
  \[ \begin{array}{lcl}
    \Dom_c\ \phi &=& \{x\mbar (c,x) \in \Dom \phi\} \\
    \Ima_c\ \phi &=& \{\phi\ (c,x) \mbar (c,x) \in \Dom \phi \}
\end{array} \]
类型 chain\_table 综合了表标识与键值理解
  \begin{gather*}
  \combtyp{chain\_table}{(\alpha,\ \beta)} \Coloneqq \mathbf{TC}\ 
  \mathrm{string}\ (\itp{\alpha})\ (\itp\beta)\\
  \begin{align*}
    (\mathbf{TC}\ c\ i\ j)_\mathrm{c} &\coloneqq c&
    (\mathbf{TC}\ c\ i\ j)_\mathrm{i} &\coloneqq i&
    (\mathbf{TC}\ c\ i\ j)_\mathrm{j} &\coloneqq j&
  \end{align*} \end{gather*}
记号 $\phi^{(\mathbf{TC}\ c\ i\ j)}$ 表示链数据 $\phi$ 的标识为 $c$ 的表
在键理解 $i$ 与值理解 $j$ 的作用下的本体对应
\[ \begin{array}{lrcl}
  &\phi^{(\mathbf{TC}\ c\ i\ j)} &=& \Tr_j \ccirc\ \phi \ccirc \Li_i\\
    \Dom&\phi^{(\mathbf{TC}\ c\ i\ j)} &=& \{\epsilon\mbar (c,\Li_i 
    \epsilon) \in \Dom \phi\} \\
    \Ima&\phi^{(\mathbf{TC}\ c\ i\ j)} &=& \{\Tr_j(\phi\ (c,\Li_i 
    \epsilon))\mbar (c,\Li_i \epsilon) \in \Dom \phi \}
\end{array} \]
$t_0$ 为表 $t$ 的零元素
  \[ (\mathbf{TC}\ c\ i\ j)_0 = \Tr_j\ (\PhV\ 0\ \Si_j) \]
类型 write\_chain 表示对链的写入操作命令,也是一个别名
\[ \mathrm{write\_chain} \Coloneqq ((\mathrm{string},\ 
\mathrm{phenomenon}),\ \mathrm{phenomenon}) \]
元组 $((name,\ key),\ value)$ 表示以值 $value$ 写入到表 $name$ 的
键 $key$ 的写入操作。
有限映射的更新操作 $\fupdate$ 就表示对链数据的写入。
一系列辅助函数 
\begin{gather*} \begin{array}{lcl}
\Has &:& \combtyp{table}{(\alpha,\ \beta)} \rightarrow \mathrm{chain}
  \rightarrow \alpha \rightarrow \mathrm{bool}\\
  \Read &:& \combtyp{table}{(\alpha,\ \beta)} \rightarrow
  \mathrm{chain} \rightarrow \alpha \rightarrow \beta \\
  \ReadZ &:& \combtyp{table}{(\alpha,\ \beta)} \rightarrow
  \mathrm{chain} \rightarrow \alpha \rightarrow \beta \\
  \Write &:& \combtyp{table}{(\alpha,\ \beta)} \rightarrow
  \alpha \rightarrow \beta \rightarrow \mathrm{write\_chain}
\end{array}\\
\begin{array}{lcl}
\Has\ (\TC c\ i\ j)\ \phi\ k &\coloneqq& (c,\ \Li_i k) \in \Dom \phi\\
\Read\ (\TC c\ i\ j)\ \phi\ k &\coloneqq& \Tr_j(\phi\ (c,\ \Li_i\ k))\\
  \ReadZ\ (\TC c\ i\ j)\ \phi\ k &\coloneqq& \left\{ \begin{split}
&\xif \Has\ (\TC c\ i\ j)\ \phi\ k \xthen \Tr_j(\phi\ (c,\ \Li_i\ k))\\
  &\quad\quad\quad\xelse \Tr_j(\PhV\ 0\ \Si_j) \end{split} \right. \\
\Write\ (\TC c\ i\ j)\ k\ v &\coloneqq& ((c,\ \Li_i k),\ \Li_j v)
\end{array} \end{gather*}
有一些显然的定理
  \[ \begin{array}{rclcl}
    &\vdash&\Has\ t\ \phi\ k &=& k \in \phi^t\\
    &\vdash&\Read\ t\ \phi\ k &=& \phi^t\ k\\
    &\vdash&\ReadZ\ t\ \phi\ k &=& \xif k \in \phi^t \xthen
    \phi^t\ k \xelse t_0\\
    &\vdash&\Read\ t\ (\phi \fupdate \Write\ t\ k\ v)\ k' &=&
    (\xif k = k' \xthen v \xelse \Read\ t\ \phi\ k') \\
  t_1 \neq t_2 &\vdash& \Read\ t_1\ (\phi \fupdate \Write\ 
    t_2\ k\ v)\ k' &=& \Read\ t_1\ \phi\ k' \\
    &\vdash&\ReadZ\ t\ (\phi \fupdate \Write\ t\ k\ v)\ k' &=&
    (\xif k = k' \xthen v \xelse \ReadZ\ t\ \phi\ k') \\
    t_1 \neq t_2 &\vdash& \ReadZ\ t_1\ (\phi \fupdate \Write\ 
    t_2\ k\ v)\ k' &=& \ReadZ\ t_1\ \phi\ k' \\
    &\vdash& \ReadZ\ t\ \emptyset\ k &=& t_0 \\
    k \in \phi^t &\vdash& \Ima_t\ (\phi \fupdate \Write\ t\ k\ v) 
    &=& \Ima \phi^t - \{ \phi^t\ k \} \cup \{ v \}\\
    k \notin \phi^t &\vdash& \Ima_t\ (\phi \fupdate \Write\ t\ k\ v) 
    &=& \Ima \phi^t \cup \{ v \}
  \end{array} \]
\end{defin}

\begin{defin}[智能合约调用响应]
类型 response 表示所期望的智能合约调用的返回类型。
  \[ \combtyp{response}{\alpha} \Coloneqq ((\combtyp{list}{
    \mathrm{write\_chain}}),\ \alpha) \]
$(l,\ x)$ 表示以 $x$ 为返回值,$l$ 为链数据写入命令序列的智能合约调用
响应。$l$ 列表中的每一项元素都是 write\_chain 类型描述的链数据的写入命令
。一个智能合约的作为外部接口的函数必须返回 response 类型,编译时会在
每个外部接口函数的实现的最后逐一遍历写入命令序列 $l$,逐一将命令执行并
写入进链中。智能合约调用响应在实现上的内存结构是
\begin{center}
\begin{tabular}{|c|c|} \hline
\text{指向写入命令序列 $l$ 的指针}&\text{计算结果 $x$}\\
\text{PointerSize bytes}&\text{$\mathbf{Si}_i$ bytes}\\ \hline
\end{tabular}
\end{center}
\end{defin}

\amlhS 所有定义的理解列于表 \ref{tab.IC.noesis} 与表 
 \ref{tab.IC.noesis2} 中。所有的基元指令及其定义与 Noesis 同构列于
表 \ref{tab.IC.primop} 与表 \ref{tab.IC.primop2} 中。

\begin{table}[hp]
\begin{threeparttable}
\centering \caption{\amlhS 中实现的理解} \label{tab.IC.noesis}
\begin{tabularx}{\linewidth}{ |c|c|c|c|X|p{2cm}| } \hline
\textbf{理解} & \textbf{本体集} & \textbf{现象集} & 
$\mathbf{Si}$ \textbf{值} & $\mathbf{Li}$ \textbf{映射} & 
$\mathbf{Tr}$ \textbf{映射} \\ \hline
$\NatSegI\ n$ & $\{x\mbar x < n\}$ & $\mathbb{U}_\mathrm{pv}$ &
$\lceil \log_{256}\ n \rceil$ & 
$\lambda e.\ \PhV e\ \lceil \log_{256}\ n \rceil$ & $\PX$ 
\\ \hline
$\BoolI$ & $\univ{bool}$ & $\mathbb{U}_\mathrm{pv}$ & $1$ &
$\lambda e.\ \PhV\ ($\newline$\xif\ e\ \xthen\ 1\ \xelse\ 0)\ 1$&
$\lambda v.$ \newline $\PX v > 0$  \\ \hline 
$\AddressI$ & $\{ n \mbar n < \mathrm{AdrSize} \}$ &
$\mathbb{U}_\mathrm{pv}$ & AdrSize & $\lambda e.\ \PhV
e\ \mathrm{AdrSize}$ & $\PX$ \\ \hline
$\OneI\ x$ & $\{ x \}$ & $\mathbb{U}_\mathrm{pv}$ & 0 & 
$\K(\PhV 0\ 0)$ & $\K x$ \\ \hline
$\ListI\ i$ & $\mathrm{EVERY}\ \mathbf{Se}_i$ & 虚构 & Ps & 虚构 
& 虚构 \\ \hline
$\CWI$ & $(\mathbb{U}_\mathrm{str} \times \mathbb{U}_\mathrm{ph}) \times
\mathbb{U}_\mathrm{ph}$ & 虚构 & $1 + 2 \mathrm{Ps}$ & 虚构 
& 虚构 \\ \hline
$i \times j$ & $\mathbf{Se}_i \times \mathbf{Se}_j$ & 
见注1 & $\Si_i + \Si_j$ & $\lambda(x,y).\ \PhS\ \Li_i(x)\ 
\Li_j(x)$ & 见注2 \\ \hline
\end{tabularx}
\begin{tablenotes} \small 
\item $\mathbb{U}_\mathrm{pv}$ 是 $\{ \PhV x\ b \}$ 的简写,
表示现象类型的所有元素中由 $\PhV$ 构造的。
$\mathbb{U}_\mathrm{ph}$ 是 $\univ{phenomenon}$ 的简写,
表示现象类型中所有的元素。
$\mathbb{U}_\mathrm{str}$ 是 $\univ{string}$ 的简写。
$\mathrm{Ps}$ 是 PointerSize 的简写。
 EVERY 是 HOL 系统库中的函数,$\mathrm{EVERY}\ s$ 
表示所有元素都属于 $s$ 集的所有列表构成的集合。
\item[注1] 理解 $i \times j$ 的现象集是 $\{\PhS p_1\ p_2
\mbar p_1\in \Sp_i\land p_2 \in \Sp_j\}$,表格空间有限故列于此。
\item[注2] 理解 $i \times j$ 的\textbf{Tr}映射是 
    $\lambda p.\ (\Tr_i(\PL\ \Si_i\ p),\ \Tr_j(\PU\ \Si_i\ p))$,
    表格空间有限故列于此。
\end{tablenotes}
\end{threeparttable}
\newline \newline
\begin{threeparttable}
\centering \caption{\amlhS 中实现的理解(绪)} \label{tab.IC.noesis2}
\begin{tabular}{ |c|p{14.8cm}| } \hline
\textbf{理解} & \textbf{描述} \\ \hline
$\NatSegI\ n$ & 不超过 $n$ 的自然数 \\ \hline
$\BoolI$ & bool 值 \\ \hline $\AddressI$ & 
智能合约场景下的账户标识,在以太坊中是 \texttt{address} 类型,
EOS.IO 中是 \texttt{name} 类型,常量 AdrSize 表示标识的大小,
    以字节为单位\\ \hline
  $\OneI\ x$ & 单元素集合 $\{x\}$ 的理解,用于表示状态输入 \\ \hline
$\ListI\ i$ & 对元素使用 $i$ 理解的列表理解,虚构定义而来,
在内存中的表示是一个指针,故有 PointerSize 的大小 \\ \hline
$\CWI$ & 是链数据的写入命令的理解,虚构定义而来 \\ \hline
$i\times j$ & 表示理解 $i,\ j$ 的笛卡儿积 \\ \hline
\end{tabular}\end{threeparttable}\end{table}%
\begin{table}[hp]%
\begin{threeparttable}
\centering \caption{\amlhS 中的常量} \label{tab.IC.const}
\begin{tabularx}{\linewidth}{ |c|c|X| }\hline
  \textbf{常量} & \textbf{Noesis 对应} & \textbf{定义与描述} \\ \hline
$\Li_{\NatSegI n} x$ & $\Li_{\NatSegI n} x \widesim[2]{\NatSegI\ n}
  x$ & $\Li_{\NatSegI n} x = \PhV\ x\ \B_n \quad\quad$ 自然数常量
\\\hline\emptylist & $\emptylist \widesim{\ListI\ i} \mathrm{Nil}$ &
  虚构定义而来,Nil 是 HOL 中定义的空序列。\\ \hline
\texttt{T} & $\texttt{T} \widesim{\BoolI} \T$ & $\texttt{T} = \PhV 1\ 1$
\\ \hline 
\texttt{F} & $\texttt{F} \widesim{\BoolI} \F$ & $\texttt{F} = \PhV 0\ 1$
\\ \hline
\end{tabularx}
\end{threeparttable}%
  \newline \newline
\begin{threeparttable}%
\centering \caption{\amlhS 中基元函数的定义} \label{tab.IC.primop2}
\begin{tabularx}{\linewidth}{ |c|X| }\hline
\textbf{基元函数} & \textbf{定义} \\ \hline
$\mathbf{IAdd}\ n$ & $\PB p_1 = \PB p_2 = n \vdash
\mathbf{IAdd}\ n\ p_1\ p_2 \coloneqq
\PhV\ (\PX p_1 + \PX p_2 \mod 256^n)\ n$ \\ \hline
$\mathbf{ISub}\ n$ & $\PB p_1 = \PB p_2 = n \vdash
\mathbf{ISub}\ n\ p_1\ p_2 \coloneqq
\PhV\ (\PX p_1 - \PX p_2 \mod 256^n)\ n$ \\ \hline
$\mathbf{IMul}\ n$ & $\PB p_1 = \PB p_2 = n \vdash
\mathbf{IMul}\ n\ p_1\ p_2 \coloneqq
\PhV\ (\PX p_1 * \PX p_2 \mod 256^n)\ n$ \\ \hline
$\mathbf{ILt}\ n$ & $\PB p_1 = \PB p_2 = n \vdash
\mathbf{ILt}\ n\ p_1\ p_2 \coloneqq
\PhV\ (\xif \PX p_1 < \PX p_2 \xthen 1 \xelse 0)\ n$ \\ \hline
$\mathbf{ILe}\ n$ & $\PB p_1 = \PB p_2 = n \vdash
\mathbf{ILe}\ n\ p_1\ p_2 \coloneqq
\PhV\ (\xif \PX p_1 \leq \PX p_2 \xthen 1 \xelse 0)\ n$ \\ \hline
$\mathbf{IEq}\ n$ & $\PB p_1 = \PB p_2 = n \vdash
\mathbf{IEq}\ n\ p_1\ p_2 \coloneqq
\PhV\ (\xif \PX p_1 = \PX p_2 \xthen 1 \xelse 0)\ n$ \\ \hline
    $\mathbf{IConv}_{m,n}$ & $\PB p = m \vdash
\mathbf{IConv}_{m,n}\ p \coloneqq \PhV\ (\PX p \Mod 256^n)\ n$ \\\hline
$\mathbf{INot}$ & $ \PB p = 1 \vdash
\mathbf{INot}\ p = \PhV\ (\xif \PX p = 0 \xthen 1 \xelse 0)\ 1$\\ \hline
$\mathbf{Append}\ i$ & 虚构定义自同构
$\mathbf{Append}\ i \proctr{i|\ListI\ i|\ListI\ 
i}{\lambda x\ l.\ \T} (::)$ \\ \hline
  $\mathbf{Write}\ t$ & 虚构定义自同构 $\mathbf{Write}\ t
  \proctr{i|j|\CWI}{\lambda k\ v.\ \T} \Write t$ \\ \hline
$\mathbf{Read}\ t$ & 虚构定义自同构 $\mathbf{Read}\ t
  \proctr{\OneI\ x|i|j}{\Has\ t} \Read t$ \\ \hline
$\mathbf{Has}\ t$ & 虚构定义自同构 $\mathbf{Has}\ t
  \proctr{\OneI\ x|i|\BoolI}{\lambda x\ k.\ \T} \Has t$ \\\hline
$\mathbf{Read0}\ t$ & 虚构定义自同构 
$\mathbf{Read0}\ t \proctr{\OneI\ x|i|j}{\lambda x\ k.\ \T} \ReadZ\ t$
  \\\hline
$\mathbf{Cart}\ n_1\ n_2$ & $\PB p_1 = n_1,\ \PB p_2 = n_2 \vdash
  \mathbf{Cart}\ p_1\ p_2 \coloneqq \PhS p_1\ p_2$ \\ \hline
    $\mathbf{Seg}_{l,m,n}$ & $\PB p = l \vdash
    \mathbf{Seg}_{l,m,n}\ p \coloneqq 
    \PhV\ (\PX p \Div 256^m \Mod 256^n)\ n$ \\ \hline
$\mathbf{If}$ & $\PB p_c = 1\vdash \mathbf{If}\ p_c\ p_a\ p_b 
  \coloneqq \xif \PX p_c = 1 \xthen p_a \xelse p_b$ \\ \hline
\end{tabularx}\end{threeparttable}\end{table}
\begin{table}[hp]\begin{threeparttable}
\centering \caption{\amlhS 中基元函数的 Noesis 同构}
\label{tab.IC.primop} \begin{tabular}{ |c|p{5.4cm}|p{7.7cm}| } \hline
\textbf{基元函数} & \textbf{ Noesis 同构 } & \textbf{描述} \\ \hline
$\mathbf{IAdd}\ \B_n$ &
$\mathbf{IAdd}\ \B_n \proctr{\NatSegI\ n|\NatSegI\ n|
\NatSegI\ n}{\lambda x\ y.\ x + y < 256^n} (+) $ &
$\B_n$ 字节整数加法的自然数加法对应 \\ \hline
$\mathbf{ISub}\ \B_n$ & $\mathbf{ISub}\ \B_n \proctr{\NatSegI\ n|\NatSegI\ n|
\NatSegI\ n}{\lambda x\ y.\ x \geq y} (-)$ & $\B_n$ 字节整数减法的自然数减法
对应 \\ \hline
  $\mathbf{IMul}\ \B_n$ & $\mathbf{IMul}\ \B_n \proctr{\NatSegI\ n|\NatSegI\ n|
\NatSegI\ n}{\lambda x\ y.\ x * y < 256^n} (\times)$ & $\B_n$ 字节整数乘法
的自然数乘法对应 \\ \hline
  $\mathbf{ILt}\ \B_n$ & $\mathbf{ILt}\ \B_n \proctr{\NatSegI\ n|\NatSegI\ n|
\BoolI}{\lambda x\ y.\ \T} (<)$ & $\B_n$ 字节整数小于判断的自然数小于对应 
\\ \hline
$\mathbf{ILe}\ \B_n$ & $\mathbf{ILe}\ \B_n \proctr{\NatSegI\ n|\NatSegI\ n|
\BoolI}{\lambda x\ y.\ \T} (\leq)$ & $\B_n$ 字节整数小于等于的自然数小于等于对应 \\ \hline
$\mathbf{IEq}\ \B_n$ & $\mathbf{IEq}\ \B_n \proctr{\NatSegI\ n|\NatSegI\ n|
\BoolI}{\lambda x\ y.\ \T} (=)$ & $\B_n$ 字节整数等于判断的自然数等于对应 \\ \hline
$\mathbf{IEq}\ \mathrm{AdrSize}$ & $\mathbf{IEq}\ \mathrm{AdrSize}
\proctr{\AddressI|\AddressI|\BoolI}{\lambda x\ y.\ \T} (=)$ & 
$\B_n$ 字节整数等于判断的账户标识对应 \\ \hline
$\mathbf{IConv}\ \Si_i\ \Si_j$ & $\mathbf{IConv}\ \Si_i\ \Si_j \proctr
    {\NatSegI\ \Si_i|\NatSegI\ \Si_j}{\lambda x.\ x < 256^{\Si_j}}\I$
    & 表示自然数的现象的等值范围变换 \\ \hline
$\mathbf{INot}$ & $\mathbf{INot} \proctr{\BoolI|\BoolI}
{\lambda x.\ \T} (\lnot)$ & bool 值取反 \\ \hline
$\mathbf{Append}\ i$ & $\mathbf{Append}\ i \proctr{i|\ListI\ i|\ListI\ 
i}{\lambda l\ x.\ \T} (::)$ & 增加元素到列表的末尾,$(::)$
是 HOL 系统库中列表的增加函数,$1::[2]$ 表示 $[1,2]$ \\ \hline
$\mathbf{Has}\ t$ & $\mathbf{Has}\ t \proctr{\OneI\ x|i|\BoolI}
{\K (\K \T)} \Has\ t$ &
查询链数据$x$的标识为$c$的表是否拥有键 $k$
  \\ \hline
$\mathbf{Read}\ t$ & $\mathbf{Read}\ t\proctr{\OneI\ x|i|j}{\Has t}
  \Read\ t$ &
读取链数据$x$的标识为$c$的表的$k$键 \\ \hline
$\mathbf{Read0}\ t$ & $\mathbf{Read0}\ t \proctr{\OneI\ x|i|j}
  {\K (\K \T)} \ReadZ\ t$ &
若链数据$x$的表$c$中存在$k$键则读取,否则返回 $j$ 理解的字节大小的0值
\\ \hline $\mathbf{Write}\ t$ & $\mathbf{Write}\ t \proctr{i|j|\CWI}
  {\lambda k\ v.\ \T} \Write\ t$ &
写入链数据的表$c$的键$k$为$v$的操作,产生写入命令 \\ \hline
$\mathbf{Cart}\ \Si_i\ \Si_j$ & $\mathbf{Cart}\ \Si_i\ \Si_j
  \proctr{i|j|i\times j}{\lambda x\ y.\ \T}$\newline
  $(\lambda x\ y.\ (x,\ y))$ & 合并两个值而构造元组 \\ \hline
$\mathbf{Cart}\ \Si_i\ \Si_j$ & $\mathbf{Cart}\ \Si_i\ \Si_j
  \proctr{\ListI\ \CWI|i|\RsI\ i}
  {\lambda l\ x.\ \T}$\newline$(\lambda l\ x.\ (l,\ x))$ &
  当第一个元素为链写入指令序列时,由定义 
  \ref{D.contract.interface} 此元组就构成了智能合约的响应 \\ \hline
$\mathbf{Seg}_{\Si_i+\Si_j,0,\Si_i}$ & $\mathbf{Seg}_
  {\Si_i+\Si_j,0,\Si_i} \proctr{i\times j|i}{\K \T} \mathrm{fst}$ &
取出元组的第一个元素 \\ \hline
    $\mathbf{Seg}_{\Si_i+\Si_j,\Si_i,\Si_j}$ & $
    \mathbf{Seg}_{\Si_i+\Si_j,\Si_i,\Si_j} 
    \proctr{i\times j|j}{\K \T} \mathrm{snd}$ &
取出元组的第二个元素 \\ \hline
  \textbf{If} & 不具有 Noesis 同构,但具有类似性质,见定理
  \ref{T.If.prop}& 分支 \\ \hline
\end{tabular}
  \begin{tablenotes} \small
  \item 因空间不够,记号 $\B_n \coloneqq \lceil \log_{256} n \rceil$。
    fst 与 snd 在 HOL 中预定义。
\end{tablenotes}
\end{threeparttable} 
\end{table}

\begin{theo}[分支 $\mathbf{If}$ 的性质] \label{T.If.prop}
值得一提的是分支 $\mathbf{If}$,其定义已在表 \ref{tab.IC.primop2} 中列出
,其不具有 Noesis 同构形式的性质,而是
\[ \begin{split}
\forall p_c\ P\ p_a\ i\ \epsilon_a\ p_b\ &\epsilon_b.\ 
  p_c \widesim{\BoolI} P \Rightarrow (P \Rightarrow p_a \widesim{i}
  \epsilon_a) \Rightarrow (\lnot P \Rightarrow p_b \widesim{i}
  \epsilon_b) \Rightarrow \\
  & \mathbf{If}\ p_c\ p_a\ p_b \widesim{i} \xif P \xthen \epsilon_a
  \xelse \epsilon_b
\end{split} \]
  \begin{proof} 若 $\PX p_c = 1$ 则由 $\mathbf{If}$ 定义,有
   $ \mathbf{If}\ p_c\ p_a\ p_b = p_a $,
  再由理解 $\BoolI$ 前提 $p_c \widesim {\BoolI} P$ 下 $P = \T$ 于是
    有 $p_a \widesim{i} \epsilon_a$ 于是命题得证。
若 $\PX p_c \neq 1$ 则由 $\mathbf{If}$ 定义,有
   $ \mathbf{If}\ p_c\ p_a\ p_b = p_b $,
  再由理解 $\BoolI$ 前提 $p_c \widesim {\BoolI} P$ 下 $P = \F$ 于是
    有 $p_b \widesim{i} \epsilon_b$ 于是命题得证。
  \end{proof}
\end{theo}

\begin{defin}[智能合约的外部接口] \label{D.contract.interface}
定义理解 $\RsI\ i$ 表示以 $i$ 理解为返回内容的智能合约调用响应
  \[ \RsI\ i \coloneqq i \times (\ListI\ \CWI) \]
所有 \amlhS 的智能合约所暴露的可供外部调用的接口应是一种现象函数,
且其具有至少一种 Noesis 同构,此同构满足第一个参数的理解必为 
$\OneI\ (x:\mathrm{chain})$,且返回值的理解必为 $\RsI\ k$。即
函数 $f$ 应满足如下形式的 Noesis 同构
\[ f \proctr{\OneI\ x|\cdots|\RsI\ k}{cond} \phi \]
\end{defin}

表 \ref{tab.IC.primop2} 定义的基元指令可以直接应用但过于基层,
诸如$\mathbf{IAdd}\ n$ 操作要求给定运算的位宽,显然并不适宜要求用户
每次都显示地给出位宽。\amlhS 提供基于这些基元函数构建的高层函数库。

$\mathbf{AddN_S},\ \mathbf{SubN_S},\ \mathbf{MulN_S},\ 
\mathbf{LeN_S},\ \mathbf{LtN_S},\ \mathbf{EqN_S}$ 
提供自动范围扩张的自然数运算。
\begin{align*}
  \mathbf{AddN_S}\ n\ m\ p\ q \coloneqq\ & \mathbf{IAdd}\ \max(\B_n,\B_m)\ 
  (\xif \B_n \geq \B_m \xthen p \xelse \mathbf{IConv}\ \B_n\ \B_m\ p)\\
  &(\xif \B_m \geq \B_n \xthen p \xelse \mathbf{IConv}\ \B_m\ \B_n\ q)\\
\mathbf{SubN_S}\ n\ m\ p\ q \coloneqq\ & \mathbf{ISub}\ n\ p\ 
  (\xif \B_m = \B_n \xthen p \xelse \mathbf{IConv}\ \B_m\ \B_n\ q)\\
\mathbf{MulN_S}\ n\ m\ p\ q \coloneqq\ & \mathbf{IMul}\ \max(\B_n,\B_m)\ 
  (\xif \B_n \geq \B_m \xthen p \xelse \mathbf{IConv}\ \B_n\ \B_m\ p)\\
  &(\xif \B_m \geq \B_n \xthen p \xelse \mathbf{IConv}\ \B_m\ \B_n\ q)\\
\mathbf{LeN_S}\ n\ m\ p\ q \coloneqq\ &\mathbf{ILe}\ \max(\B_n,\B_m)\ 
  (\xif \B_n \geq \B_m \xthen p \xelse \mathbf{IConv}\ \B_n\ \B_m\ p)\\
  &(\xif \B_m \geq \B_n \xthen p \xelse \mathbf{IConv}\ \B_m\ \B_n\ q)\\
\mathbf{LtN_S}\ n\ m\ p\ q \coloneqq\ &\mathbf{ILt}\ \max(\B_n,\B_m)\ 
  (\xif \B_n \geq \B_m \xthen p \xelse \mathbf{IConv}\ \B_n\ \B_m\ p)\\
  &(\xif \B_m \geq \B_n \xthen p \xelse \mathbf{IConv}\ \B_m\ \B_n\ q)\\
\mathbf{EqN_S}\ n\ m\ p\ q \coloneqq\ &\mathbf{IEq}\ \max(\B_n,\B_m)\ 
  (\xif \B_n \geq \B_m \xthen p \xelse \mathbf{IConv}\ \B_n\ \B_m\ p)\\
  &(\xif \B_m \geq \B_n \xthen p \xelse \mathbf{IConv}\ \B_m\ \B_n\ q)\\
\end{align*}
\[ \begin{array}{lcr} \forall n\ m.\ 
  \mathbf{AddN_S}\ n\ m &\proctr{\NatSegI\ n|
  \NatSegI\ m|\NatSegI\ \max(n,m)}{\lambda x\ y.\ x + y < \max
  (n,m)}& (+) \\\forall n\ m.\ 
  \mathbf{SubN_S}\ n\ m &\proctr{\NatSegI\ n|
\NatSegI\ m|\NatSegI\ n}{\lambda x\ y.\ x \geq y}& (-)\\ \forall n\ m.\ 
  \mathbf{MulN_S}\ n\ m &\proctr{\NatSegI\ n|
  \NatSegI\ m|\NatSegI\ \max(n,m)}{\lambda x\ y.\ x \times y < \max
  (n,m)}& (\times)\\ \forall n\ m.\ 
  \mathbf{LeN_S}\ n\ m &\proctr{\NatSegI\ n|
\NatSegI\ m|\NatSegI\ \max(n,m)}{\K\ (\K \T)}& (\leq)\\ \forall n\ m.\ 
  \mathbf{LtN_S}\ n\ m &\proctr{\NatSegI\ n|
  \NatSegI\ m|\NatSegI\ \max(n,m)}{\K\ (\K \T)}& (<)\\ \forall n\ m.\ 
  \mathbf{EqN_S}\ n\ m &\proctr{\NatSegI\ n|
  \NatSegI\ m|\NatSegI\ \max(n,m)}{\K\ (\K \T)}& (=)
\end{array} \] \newline
容易证明 $\mathbf{AddN_S},\ \mathbf{SubN_S},\ \mathbf{MulN_S},\ 
\mathbf{LeN_S},\ \mathbf{LtN_S},\ \mathbf{EqN_S}$ 具有
上述 Noesis 同构,允许进行如下的定理推导以程序构造。
\begin{center}
\begin{prooftree}
  \AxiomC{$\forall n\ m.\ \mathbf{AddN_S}\ n\ m \proctr{\NatSegI\ n|
  \NatSegI\ m|\NatSegI\ \max(n,m)}{\lambda x\ y.\ x + y < \max
  (n,m)} (+)$} 
  \AxiomC{$x \widesim{\NatSegI\ 42} \omega$}
  \RightLabel{(Noesis 调用)} \BinaryInfC{$\forall m.\ 
  \mathbf{AddN_S}\ 42\ m\ x \proctr{\NatSegI\ m|\NatSegI\ \max(42,m)}
  {\lambda y.\ \omega + y < \max(42,m)} (+\ \omega)$}
\end{prooftree}
  \begin{prooftree}
    \AxiomC{$\omega + \upsilon < 666$}
\AxiomC{$\forall m.\ 
\mathbf{AddN_S}\ 42\ m\ x \proctr{\NatSegI\ m|\NatSegI\ \max(42,m)}
  {\lambda y.\ \omega + y < \max(42,m)} (+\ \omega)$}
  \AxiomC{$y \widesim{\NatSegI\ 666} \upsilon$}
  \RightLabel{(Noesis 调用)}
  \BinaryInfC{$\mathbf{AddN_S}\ 42\ 666\ x\ y \proctr{\NatSegI\ 666}
  {\omega + \upsilon < 666} (\omega + \upsilon)$}
    \RightLabel{(Noesis 条件削除)}
    \BinaryInfC{$\mathbf{AddN_S}\ 42\ 666\ x\ y \widesim{\NatSegI\ 666} 
    \omega + \upsilon$}
  \end{prooftree}
\end{center}
而在编译时,将 $\mathbf{AddN_S}\ 42\ 666\ x\ y$ 根据定义展开
\[ \begin{split}
\mathbf{AddN_S}\ 42\ 666\ x\ y =\ & \mathbf{IAdd}\ \max(\B_{42},\B_{666})\ 
  (\xif \B_{42} \geq \B_{666} \xthen x \xelse \mathbf{IConv}\ \B_{42}\ \B_{666}\ x)\\
  &(\xif \B_{666} \geq \B_{42} \xthen x \xelse \mathbf{IConv}\ \B_{666}\ \B_{42}\ y)\\
  =\ & \mathbf{IAdd}\ 2\ (\mathbf{IConv}\ 1\ 2\ x)\ y
\end{split} \]
就得到了编译结果 $\mathbf{IAdd}\ 2\ (\mathbf{IConv}\ 1\ 2\ x)\ y$。

附录 \ref{Ch.example.transfer} 给出了一个完整的 \amlhS 上转账合约的示例。

\section{抽象机器 \amlhS 的编译} \label{Sec.compile}

抽象机器 \amlh 的编译分为两步,首先程序被其定义展开成基元函数的组合,
成为一种中间表达(IR)。IR 由 HOL 作为文本输出而交给外部程序,外部程序
由基元函数到目标平台指令集的映射,将基元函数的组合翻译至目标平台。
用户程序到 IR 的编译是在 HOL 内部进行而被保障的,IR 到目标平台的编译
是在外部进行,是可能有错误的。

本文的 \amlhS 的基元函数存在到 WASM 的映射,其上的程序可以被编译到 WASM
,进而在任何 WASM 执行平台上运行。EOS.IO 等智能合约平台支持以 WASM 程序
作为智能合约。

WASM 的特点,只有 \Wi{32},\Wi{64},\Wf{32},\Wf{64} 
四种类型,分别表示 32位整数、64位整数、32位浮点数、64位浮点数。本工作的
\amlhS 尚不支持浮点数,故只考虑 \Wi{32} 与 \Wi{64}。

\amlhS 中每一个理解都具有确定的字节数,理解的现象以此字节数排布在内存中
。计算时,所有小于等于4字节的现象以 WASM 的 \Wi{32} 表示,大于4字节但
小于等于8字节的以 \Wi{64} 表示,大于8字节的以指针表示。
本工作的实现中指针以32位整数 \Wi{32} 表示,但这是可变的。
 $\mathcal{T}_b$ 表示 $b$ 个字节的值的 WASM 类型
 \[ \mathcal{T}_b \coloneqq \left\{ \begin{array}{lcl} 
\Wi{32} &\when& b \leq 4\\\Wi{64} &\when& 4 < b \leq 8\\
   \Wpointer &\when& 8 < b \end{array} \right. \]

WASM 上的指令也为各种类型设计了不同版本。例如整数加法有 \texttt{i32.add}
,\texttt{i64.add} 两种版本。\amlhS 中的基元运算也根据字节长度有不同
版本,例如 $\mbar{IAdd}\ n$ 表示 $n$ 个字节的整数加法基元函数,
具体参见表 \ref{tab.IC.primop2}。$n$ 个字节的基元函数就对应 \Wtyp{n} 
类型的版本的指令。基元函数 $\mathbf{IAdd}\ n$ 对应的 WASM 指令是
\Winst{\Wtyp{n}.add},这一记号表示类型 \Wtyp{n} 版本的 \texttt{add} 
指令。

以 WASM 为智能合约架构的智能合约平台一般提供外部函数以实现高精度运算。
目前的 \amlhS 尚不支持高精度整数运算,仅支持最高64位的运算,故 $n > 8$
,$\Wtyp{n} = \Wpointer$ 情况的整数运算不需要考虑。

WASM 采用一种树形结构表达, \amlhS 的 IR 的也同以树形结构表达,与 WASM
的结构类似。

\amlh 中的 Let 结构表示本地变量,
\[ \mathrm{Let}\ p\ (\lambda x.\ p') \]
将被编译为,赋值 $p$ 给名为 $x$ 的本地变量,并在编译 $p'$ 中,所有对
$x$ 的引用都编译为读取本地变量 $x$。

\amlhS 的 IR 中分支表达为
\[ \mathbf{If}\quad condition\quad branch_A\quad branch_B \]
对应的 WAST 表达为
\[ (\texttt{if}\quad (condition)\quad (\texttt{then}\quad branch_A)
\quad (\texttt{else}\quad branch_B)) \]
整个 $branch_A$ 与 $branch_B$ 都在分支块中,只有那些重复的计算被重用
地置于之前的临时变量赋值中。最后循环结构因 \amlhS 目前尚不支持而
不需要讨论。表 \ref{tab.primop.WASM} 列出了基元函数到 WASM 指令的映射。

\begin{table}[hp]
\begin{threeparttable}
\centering \caption{\amlhS 的基元函数到 WASM 指令的映射}
\label{tab.primop.WASM}
\begin{tabularx}{\linewidth}{ |c|c|c|X| } \hline
{\bfseries 基元函数} & {\bfseries 条件} & {\bfseries WASM 指令} &
{\bfseries 说明}\\ \hline
$\mathbf{IAdd}\ b$ & $b \leq 8$ & \Winst{\Wtyp{b}.add} & 
不支持 $b > 8$ 时的高精度运算 \\ \hline
$\mathbf{ISub}\ b$ & $b \leq 8$ & \Winst{\Wtyp{b}.sub} & 
不支持 $b > 8$ 时的高精度运算 \\ \hline
$\mathbf{IMul}\ b$ & $b \leq 8$ & \Winst{\Wtyp{b}.mul} &
不支持 $b > 8$ 时的高精度运算 \\ \hline
$\mathbf{ILt}\ b$ & $b \leq 8$ & \Winst{\Wtyp{b}.lt} &
不支持 $b > 8$ 时的高精度运算 \\ \hline
$\mathbf{ILe}\ b$ & $b \leq 8$ & \Winst{\Wtyp{b}.le} &
不支持 $b > 8$ 时的高精度运算 \\ \hline
$\mathbf{IEq}\ b$ & $b \leq 8$ & \Winst{\Wtyp{b}.eq} &
不支持 $b > 8$ 时的高精度运算 \\ \hline
$\mathbf{IConv}\ a\ b$ & $a \leq 4\ \land\ b \leq 4$ 
& 不生成指令 & 不需要任何计算 \\ \hline
$\mathbf{IConv}\ a\ b$ & $a \leq 4\ \land\ 4 < b \leq 8$ 
& \texttt{i64.extend\_u/i32} & \\ \hline
$\mathbf{IConv}\ a\ b$ & $4 < a \leq 8\ \land\ b < 4$ 
& \texttt{i32.wrap/i64} &  \\ \hline
$\mathbf{IConv}\ a\ b$ & $8 < a\ \land\ 8 < b$
& 不生成指令 & 不需要任何计算 \\ \hline
$\mathbf{IConv}\ a\ b$ & 其余情况
& 不支持 & 不支持任何高精度运算 \\ \hline
$\mathbf{INot}$ & $\T$ & \Winst{i32.eqz} & \\ \hline
$\mathbf{Append}\ i$ & $\T$ & \Winst{call $\texttt{append}\_{\Si_i}$} 
& 根据给定的理解 $i$ 的字节大小 $\Si_i$,生成 $\Si_i$ 字节版本的
列表增添函数并调用。\\ \hline
$\mathbf{Has}\ t$ & $\T$ & \Winst{call $\texttt{has}_t$}
& 生成表 $t$ 版本的区块链数据查找函数并调用。\\ \hline
$\mathbf{Read}\ t$ & $\T$ & \Winst{call $\texttt{read}_t$}
& 生成表 $t$ 版本的区块链数据读取函数并调用。\\ \hline
$\mathbf{Read0}\ t$ & $\T$ & \Winst{call $\texttt{read0}_t$}
& 生成表 $t$ 版本的区块链数据读取函数并调用。\\ \hline
$\mathbf{Write}\ t$ & $\T$ & \Winst{call $\texttt{write}_t$}
& 生成表 $t$ 版本的区块链数据写入命令构造函数并调用。\\ \hline
$\mathbf{Cart}\ n\ m$ & $\T$ & \Winst{call $\texttt{cart}_{n,m}$}
& 生成表 $n,m$ 版本的内存拼接函数并调用。\\ \hline
$\mathbf{Seg}\ n\ m$ & $\T$ & \Winst{call $\texttt{seg}_{n,m}$}
& 生成表 $n,m$ 版本的内存部分读取函数并调用。\\ \hline
$\mathbf{If}$ & $\T$ & \Winst{if} & 见正文的论述 \\ \hline
\end{tabularx}\end{threeparttable}\end{table}

一些基元函数不能直接对应到一个 WASM 上的指令,而是要通过预先构建的
WASM 上的函数,这些函数可以理解为 \amlhS 运行时的系统库,
当然这些函数也可以内联地展开在基元函数调用的地方。
现在将论述表 \ref{tab.primop.WASM} 中提及的系统库函数 $\texttt{has}_t$,
$\texttt{read}_t$,$\texttt{read0}_t$,$\texttt{write}_t$,
$\texttt{cart}_{n,m}$,$\texttt{seg}_{n,m}$。

\begin{description}
\item[全局值 \texttt{heap\_pointer : \Wpointer}] 指向堆的末尾。
\amlhS 在 WASM 的编译实现的堆正向增长。
\texttt{heap\_pointer} 的初始值是一常量,此值之前的为常量段内存,
之后为堆。
\item[函数 \texttt{malloc(size\ :\ \Wpointer)}\ :\ \Wpointer] 
在堆中分配大小为 \texttt
{size} 字节的内存。将 \texttt{heap\_pointer} 的值增加 \texttt{size}
而返回 \texttt{heap\_pointer} 的原始值。
\item[函数 $\texttt{has}_{(\mathbf{TC}\ c\ i\ j)}\texttt{(t\ :\ 
\Wtyp{\Si_i})}\ :\ \Wi{32}$] 根据目标智能合约平台,
调用智能合约平台提供的接口,查询链数据的表$c$
中是否有键 $k$,并返回 \Wi{32} 类型的 Bool 值。
\item[函数 $\texttt{read}_{(\mathbf{TC}\ c\ i\ j)}\texttt{(k\ :\ 
\Wtyp{\Si_i})\ :\ \Wtyp{\Si_j}}$]
根据目标智能合约平台,调用智能合约平台提供的接口,读取链数据的表$c$
的键 $k$。并根据 $\Si_j$ 决定是以指针方式还是值方式返回数据,并返回类型
\Wtyp{\Si_j}。
\item[函数 $\texttt{write}_{(\mathbf{TC}\ c\ i\ j)}\texttt{(k\ :\ 
\Wtyp{\Si_i},\ v\ :\ \Wtyp{\Si_j})\ :\ \Wpointer}$] 
构造链数据写入命令。
首先调用 \texttt{malloc} 函数分配适当的空间将键值存于内存,即,当 
$\Si_i \leq 8$ 时分配 $\Si_i$ 字节的内存并存储 $k$ 的值,当 $\Si_i
> 8$ 时 $k$ 就是指向键数据的指针。$v$ 的处理相同。
链数据写入命令的内存结构为
\begin{center}
\begin{tabular}{|c|c|c|c|c|} \hline
\text{表标识}&\text{键指针}&
\text{键大小}&\text{值指针}&\text{值大小}\\
\text{1byte}&\text{PointerSize} bytes&
\text{4bytes}&\text{PointerSize} bytes&\text{4byte} \\ \hline
\end{tabular}
\end{center}
表标识占用1字节,理论上表标识可以是任意字符串,但在编译实现中,
一段合约所有使用的表标识被唯一地分配$0\sim255$的编号,使用此编号表示
表标识,故一个合约支持访问的表数量不超过256个。\texttt{write} 函数
如上地分配并写入此内存结构,最后返回此内存结构的指针。
\item[函数 $\texttt{seg}_{l,m,n}\texttt{(v\ :\ \Wtyp{l})\ :\ 
\Wtyp{n}}$] 返回 $v$ 的第 $m$ 字节到第 $m+n-1$ 字节(从0开始)的共
$n$ 个字节的数据。若 $l > 8$ 则 $v$ 为指针,将指针位移 $m$ 并根据 $n$
的值,若 $n > 8$ 则直接返回指针 $v + m$ 否则读取 $v + m$ 处的数据并
根据 \Wtyp{n} 返回适当的 \Wi{32} 或 \Wi{64}。若 $l \leq 8$ 则 $v$
为一值,$(v\ \texttt{>>}\ 8m)\ \texttt{\&}\ ((1\ \texttt{<<}
\ 8n) - 1)$ 即是返回值,注意其中 $8m$,$((1\ \texttt{<<}\ 8n) - 1)$是
常量。
\end{description}

\begin{algorithm}[b]
\caption{用户程序的包装算法} \label{alg:package}
\begin{algorithmic}[1]
\Statex 对编译好的用户程序 user\_func
\State 读取来自智能合约平台的输入
\State 以输入调用 user\_func 并记返回值为 $(l, x)$,其中 $l$ 为
链数据写入命令序列,$x$ 为计算结果
\State 逐一遍历命令序列 $l$ 并执行链数据写入
\State 将计算结果 $x$ 返回给智能合约平台
\end{algorithmic}
\end{algorithm}


以上,由表 \ref{tab.primop.WASM} 以及上述的运行时函数,可以将 \amlhS 
的 IR 翻译成 WASM,进而完成编译。
而这一编译是非常简洁的,外部编译器所肩负的职责仅仅是对 \amlh 的 IR 进行
几乎指令对指令的翻译,不需要任何复杂的编排或逻辑。外部编译程序可以
非常简短,以有效降低缺陷的出现可能,并使对其的形式化验证更加容易。

编译好的用户程序会进行最后的包装,在函数执行之前插入处理来自智能合约
平台的输入的代码,并在函数之后增加对链数据写入指令的执行程序。
这两部分都是固定的。算法 \ref{alg:package} 描述了这一过程。
最后包装好的函数就是可以被智能合约执行平台调用并能
有效地计算输出并修改区块链数据的智能合约。

附录 \ref{Ch.example.transfer} 给出了对一段转账示例程序的编译结果,
对照本节内容可以清晰地看到 \amlhS 上的 IR 到 WASM 的编译是如何
指令对指令的简洁的。

