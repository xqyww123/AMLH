\chapter{用于智能合约的 \Eamlh 的实现}

\section{用于智能合约的抽象机器 \amlhS 在 HOL 上的实现}

首先论述 HOL 交互式定理证明器上用于智能合约的抽象机器 \amlhS 的实现,
这一实现通过 HOL 定理证明器上4867行的 SML 语言完成。
状态方面采用计算与状态分离的方案,将所有状态转移以指令的方式记录而在
智能合约所有计算完成后的最后写入区块链,这样更安全。

\begin{defin}[现象的定义] 类型 phenomenon 表示现象
\[ \begin{split}
\mathrm{phenomenon} \Coloneqq & \PhV\ \mathrm{num}\ 
\mathrm{num} \mbar \PhS\ \mathrm{phenomenon}\ 
\mathrm{phenomenon} \\
\mbar & \PhP \mathrm{phenomenon}
\end{split} \]
$\PhV x\ b$ 表示 $b$ 个二进制表示的数字 $x$。$\PhS p_1\ p_2$ 表示两个值
在内存上的直接拼接构成的现象。$\PhP p$ 表示对现象 $p$ 的指针。
\[ \begin{array}{lcllcl}
\PB\ (\PhV x\ b) &\coloneqq& b&
\PX\ (\PhV x\ b) &\coloneqq& x\\
\PB\ (\PhS p_1\ p_2) &\coloneqq& \PB p_1 + \PB p_2\quad\quad&
\PX\ (\PhS p_1\ p_2) &\coloneqq& \PX(p_1) + \PX(p_2) \cdot 256^{\PB(p_1)}\\
\PB\ (\PhP p) &\coloneqq& \mathrm{PointerSize} &&&
\end{array} \]
$\PX\ (\PhP p)$ 故意的不去定义。PointerSize 为常量固定为智能合约执行环境
的位数,一般为32。
现象的分割工具,函数 $\PL,\ \PU$
    \begin{align*} \PL n\ p &= \PhV\ (\PX p \Mod 256^n)\ n&
        \PU n\ p &= \PhV\ (\PX p \Div 256^n)\ (\PB p \dotminus n)&
    \end{align*}
合法的现象
\begin{gather*} \begin{align*}
\VP\ (\PhV\ x\ b) &\coloneqq (x < 256^b)&
\VP\ (\PhP\ p) &\coloneqq \VP p&
\end{align*}\\
\VP\ (\PhS\ p_1\ p_2) \coloneqq \VP p_1 \ \land\ \VP p_2
\end{gather*}
现象集的 0 元素是 $\mathbf{PhV}\ 0\ 0$,
占用0个比特位的值,且其是合法的。
\end{defin}

\begin{defin}[\amlhS 上的理解]
\amlhS 上理解的定义基本基于定义 \ref{Def.itp} ,只是固定了理解的现象集
的位数目。
    \[ \begin{split}
        \itp{\alpha} \Coloneqq \mathbf{Noesis}\ &(\alpha \rightarrow \phenomenon)
        \ (\phenomenon \rightarrow \alpha)\\
    &(\alpha\ \mathrm{set})\ (\phenomenon\ \mathrm{set})\ \mathrm{num}
    \end{split} \]
\begin{gather*}
\begin{align*}
\mathbf{NOE\_LIGHT}\ (\mathbf{Noesis}\ l\ tr\ s_e\ s_p\ s) & = l&
\mathbf{NOE\_TRANSCEND}\ (\mathbf{Noesis}\ l\ tr\ s_e\ s_p\ s) & = tr&\\
\mathbf{NOE\_SET}\ (\mathbf{Noesis}\ l\ tr\ s_e\ s_p\ s) & = s_e&
\mathbf{NOE\_SIZE}\ (\mathbf{Noesis}\ l\ tr\ s_e\ s_p\ s) &= s&
\end{align*}\\
\mathbf{NOE\_PSET}\ (\mathbf{Noesis}\ l\ tr\ s_e\ s_p\ s)
 = s_p \cap \{ p \mbar \PB p = s \}
 \end{gather*}
同样有记号
\begin{gather*}
\begin{align*}
    \mathbf{Li}_i\ & \coloneqq \mathbf{NOE\_LIGHT}\ i&
    \mathbf{Tr}_i\ & \coloneqq \mathbf{NOE\_TRANSCEND}\ i &
\end{align*}\\ \begin{align*}
    \mathbf{Sp}_i\ & \coloneqq \mathbf{NOE\_PSET}\ i&
    \mathbf{Si}_i\ & \coloneqq \mathbf{NOE\_SIZE}\ i&
    \mathbf{Se}_i\ & \coloneqq \mathbf{NOE\_SET}\ i&
\end{align*} \end{gather*}
$\mathbf{Si}_i$ 表示理解 $i$ 的现象集的比特数。
\end{defin}

\begin{defin}[\amlhS 的 Noesis 对应] 在定义 \ref{Def.TR} 的基础上
加入现象合法性。
\[ p \widesim{i} e \coloneqq \mathbf{V}_i\ \land\ \VP\ p\ \land\ 
p \in \mathbf{Sp}_i\ \land\ (\mathbf{Tr}_i\ p = e) \]
\end{defin}

其余理论均与第 \ref{Ch.AmLH} 章相同。

\begin{defin}[链的抽象表达]
\amlhS 将链上数据抽象为各个由标识区分的键值表,使用有限映射表示,
类型 chain 为此别名 
\[ \mathrm{chain} \Coloneqq (\mathrm{string},\ \mathrm{phenomenon})
\mapsto \mathrm{phenomenon} \]
类型 write\_chain 表示对链的写入操作命令,也是一个别名
\[ \mathrm{write\_chain} \Coloneqq ((\mathrm{string},\ 
\mathrm{phenomenon}),\ \mathrm{phenomenon}) \]
元组 $((name,\ key),\ value)$ 表示以值 $value$ 写入到表 $name$ 的
键 $key$ 的写入操作。
有限映射的更新操作 $\fupdate$ 就表示对链数据的写入。
\end{defin}

链数据写入命令的内存结构为
\begin{center}
\begin{tabular}{|c|c|c|c|c|} \hline
\text{表标识}&\text{键指针}&
\text{键大小}&\text{值指针}&\text{值大小}\\
\text{1byte}&\text{PointerSize} bytes&
\text{4bytes}&\text{PointerSize} bytes&\text{4byte} \\ \hline
\end{tabular}
\end{center}
表标识占用1字节,理论上表标识可以是任意字符串,但在最终的编译实现中,
一段合约所有使用的表标识会被唯一地分配$0\sim255$的编号,使用此编号表示
表标识,故一个合约支持访问的表数量不超过256个。

\begin{defin}[智能合约调用响应]
类型 response 表示所期望的智能合约调用的返回类型。
\[ \combtyp{response}{\alpha} \Coloneqq \Rsp \alpha\ (\combtyp{list}{
    \mathrm{write\_chain}}) \]
$\Rsp\ x\ l$ 表示以 $x$ 为返回值,$l$ 为链数据写入命令序列的智能合约调用
响应。$l$ 列表中的每一项元素都是 write\_chain 类型描述的链数据的写入命令
。一个智能合约的作为外部接口的函数必须返回 response 类型,编译时会在
每个外部接口函数的实现的最后逐一遍历写入命令序列 $l$,逐一将命令执行并
写入进链中。智能合约调用响应在实现上的内存结构是
\begin{center}
\begin{tabular}{|c|c|} \hline
\text{指向写入命令序列 $l$ 的指针}&\text{计算结果 $x$}\\
\text{PointerSize bytes}&\text{$\mathbf{Si}_i$ bytes}\\ \hline
\end{tabular}
\end{center}
\end{defin}

\amlhS 所有定义的理解列于表 \ref{tab.IC.noesis} 与表 
 \ref{tab.IC.noesis2} 中。所有的基元指令及其定义与 Noesis 同构列于
表 \ref{tab.IC.primop} 与表 \ref{tab.IC.primop2} 中。

\begin{table}[hp]
\begin{threeparttable}
\centering \caption{\amlhS 中实现的理解} \label{tab.IC.noesis}
\begin{tabular}{ |c|c|c|c|p{4.5cm}|p{2cm}| } \hline
\textbf{理解} & \textbf{本体集} & \textbf{现象集} & 
$\mathbf{Si}$ \textbf{值} & $\mathbf{Li}$ \textbf{映射} & 
$\mathbf{Tr}$ \textbf{映射} \\ \hline
$\NatSegI\ n$ & $\{x\mbar x < n\}$ & $\mathbb{U}_\mathrm{pv}$ &
$\lceil \log_{256}\ n \rceil$ & 
$\lambda e.\ \PhV e\ \lceil \log_{256}\ n \rceil$ & $\PX$ 
\\ \hline
$\BoolI$ & $\univ{bool}$ & $\mathbb{U}_\mathrm{pv}$ & $1$ &
$\lambda e.\ \PhV\ ($\newline$\xif\ e\ \xthen\ 1\ \xelse\ 0)\ 1$&
$\lambda v.$ \newline $\PX v > 0$  \\ \hline 
$\AddressI$ & $\{ n \mbar n < \mathrm{AdrSize} \}$ &
$\mathbb{U}_\mathrm{pv}$ & AdrSize & $\lambda e.\ \PhV
e\ \mathrm{AdrSize}$ & $\PX$ \\ \hline
$\OneI\ s$ & $\{ s \}$ & $\mathbb{U}_\mathrm{pv}$ & 0 & 
$\K(\PhV 0\ 0)$ & $\K s$ \\ \hline
$\ListI\ i$ & $\mathrm{EVERY}\ \mathbf{Se}_i$ & 虚构 & Ps & 虚构 
& 虚构 \\ \hline
$\CWI$ & $(\mathbb{U}_\mathrm{str} \times \mathbb{U}_\mathrm{ph}) \times
\mathbb{U}_\mathrm{ph}$ & 虚构 & $1 + 2 \mathrm{Ps}$ & 虚构 
& 虚构 \\ \hline
$\RsI\ i$ & $\{\Rsp\ x\ l \mbar x \in \mathbf{Se}_i\}$ & 虚构 & 
$\mathrm{Ps} + \mathbf{Si}_i$ & 虚构 
& 虚构 \\ \hline
$i \times j$ & $\mathbf{Se}_i \times \mathbf{Se}_j$ & 
见注1 & $\Si_i + \Si_j$ & $\lambda(x,y).\ \PhS\ \Li_i(x)\ 
\Li_j(x)$ & 见注2 \\ \hline
\end{tabular}
\begin{tablenotes} \small 
\item $\mathbb{U}_\mathrm{pv}$ 是 $\{ \PhV x\ b \}$ 的简写,
表示现象类型的所有元素中由 $\PhV$ 构造的。
$\mathbb{U}_\mathrm{ph}$ 是 $\univ{phenomenon}$ 的简写,
表示现象类型中所有的元素。
$\mathbb{U}_\mathrm{str}$ 是 $\univ{string}$ 的简写。
$\mathrm{Ps}$ 是 PointerSize 的简写。
 EVERY 是 HOL 系统库中的函数,$\mathrm{EVERY}\ s$ 
表示所有元素都属于 $s$ 集的所有列表构成的集合。
\item[注1] 理解 $i \times j$ 的现象集是 $\{\PhS p_1\ p_2
\mbar p_1\in \Sp_i\land p_2 \in \Sp_j\}$,表格空间有限故列于此。
\item[注2] 理解 $i \times j$ 的\textbf{Tr}映射是 
    $\lambda p.\ (\Tr_i(\PL\ \Si_i\ p),\ \Tr_j(\PU\ \Si_i\ p))$,
    表格空间有限故列于此。
\end{tablenotes}
\end{threeparttable}
\newline \newline
\begin{threeparttable}
\centering \caption{\amlhS 中实现的理解(绪)} \label{tab.IC.noesis2}
\begin{tabular}{ |c|p{14.8cm}| } \hline
\textbf{理解} & \textbf{描述} \\ \hline
$\NatSegI\ n$ & 不超过 $n$ 的自然数 \\ \hline
$\BoolI$ & bool 值 \\ \hline $\AddressI$ & 
智能合约场景下的账户标识,在以太坊中是 \texttt{address} 类型,
EOS.IO 中是 \texttt{name} 类型,常量 AdrSize 表示标识的大小,
    以字节为单位\\ \hline
$\OneI\ s$ & 单元素集合 $\{s\}$ 的理解,用于表示
状态输入$\{ s \}$ \\ \hline
$\ListI\ i$ & 对元素使用 $i$ 理解的列表理解,虚构定义而来,
在内存中的表示是一个指针,故有 PointerSize 的大小 \\ \hline
$\CWI$ & 是链数据的写入命令的理解,虚构定义而来 \\ \hline
$\RsI\ i$ & 表示以 $i$ 理解为返回内容的智能合约调用响应 \\ \hline
$i\times j$ & 表示理解 $i,\ j$ 的笛卡儿积 \\ \hline
\end{tabular}\end{threeparttable}\end{table}%
\begin{table}[b]
\begin{threeparttable}%
\centering \caption{\amlhS 中基元函数的定义} \label{tab.IC.primop2}
\begin{tabularx}{\linewidth}{ |c|X| }\hline
\textbf{基元函数} & \textbf{定义} \\ \hline
$\mathbf{IAdd}\ n$ & $\PB p_1 = \PB p_2 = n \Rightarrow
\mathbf{IAdd}\ n\ p_1\ p_2 \coloneqq
\PhV\ (\PX p_1 + \PX p_2 \mod 256^n)\ n$ \\ \hline
$\mathbf{ISub}\ n$ & $\PB p_1 = \PB p_2 = n \Rightarrow
\mathbf{ISub}\ n\ p_1\ p_2 \coloneqq
\PhV\ (\PX p_1 - \PX p_2 \mod 256^n)\ n$ \\ \hline
$\mathbf{IMul}\ n$ & $\PB p_1 = \PB p_2 = n \Rightarrow
\mathbf{IMul}\ n\ p_1\ p_2 \coloneqq
\PhV\ (\PX p_1 * \PX p_2 \mod 256^n)\ n$ \\ \hline
$\mathbf{ILt}\ n$ & $\PB p_1 = \PB p_2 = n \Rightarrow
\mathbf{ILt}\ n\ p_1\ p_2 \coloneqq
\PhV\ (\xif \PX p_1 < \PX p_2 \xthen 1 \xelse 0)\ n$ \\ \hline
$\mathbf{ILe}\ n$ & $\PB p_1 = \PB p_2 = n \Rightarrow
\mathbf{ILe}\ n\ p_1\ p_2 \coloneqq
\PhV\ (\xif \PX p_1 \leq \PX p_2 \xthen 1 \xelse 0)\ n$ \\ \hline
$\mathbf{IEq}\ n$ & $\PB p_1 = \PB p_2 = n \Rightarrow
\mathbf{IEq}\ n\ p_1\ p_2 \coloneqq
\PhV\ (\xif \PX p_1 = \PX p_2 \xthen 1 \xelse 0)\ n$ \\ \hline
    $\mathbf{IConv}_{m,n}$ & $\PB p = m \Rightarrow
\mathbf{IConv}_{m,n}\ p \coloneqq \PhV\ (\PX p \Mod 256^n)\ n$ \\\hline
$\mathbf{INot}$ & $ \PB p = 1 \Rightarrow
\mathbf{INot}\ p = \PhV\ (\xif \PX p = 0 \xthen 1 \xelse 0)\ 1$\\ \hline
$\mathbf{Append}\ i$ & 虚构定义自同构
$\mathbf{Append}\ i \proctr{i|\ListI\ i|\ListI\ 
i}{\lambda l\ x.\ \T} (::)$ \\ \hline
$\mathbf{Write}\ c\ i\ j$ & 虚构定义自同构 $\mathbf{Write}\ c\ i\ j 
\proctr{i|j|\CWI}{\lambda k\ v.\ \T}
(\lambda k\ v.\ ((c,\ \mathbf{Li}_i\ k),\ \mathbf{Li}_j\ v))$ \\ \hline
$\mathbf{Read}\ c\ x\ i\ j$ & 虚构定义自同构 $\mathbf{Read}\ c\ x\ i\ j 
\proctr{\OneI\ x|i|j}{\lambda x\ k.\ (c,k) \in \Dom x}
(\lambda x\ k.\ x\ (c,k))$ \\ \hline
$\mathbf{Has}\ c\ x\ i$ & 虚构定义自同构 $\mathbf{Has}\ c\ x\ i 
  \proctr{\OneI\ x|i|\BoolI}{\K (\K \T)}
  (\lambda x\ k.\ (c,k) \in \Dom x)$ \\\hline
$\mathbf{Cart}$ & $\mathbf{Cart}\ p_1\ p_2 \coloneqq \PhS p_1\ p_2$ \\ \hline
    $\mathbf{Seg}_{m,n}$ & $\mathbf{Seg}_{m,n}\ p \coloneqq \PhV\ 
    (\PX p \Div 256^m \Mod 256^n)\ n$ \\ \hline
$\mathbf{If}$ & $\PB p_c = 1\Rightarrow \mathbf{If}\ p_c\ p_a\ p_b 
  \coloneqq \xif \PX p_c = 1 \xthen p_a \xelse p_b$ \\ \hline
\end{tabularx}\end{threeparttable}\end{table}
\begin{table}[hp]\begin{threeparttable}
\centering \caption{\amlhS 中基元函数的 Noesis 同构}
\label{tab.IC.primop} \begin{tabular}{ |c|p{5.4cm}|p{7.7cm}| } \hline
\textbf{基元函数} & \textbf{ Noesis 同构 } & \textbf{描述} \\ \hline
$\mathbf{IAdd}\ \B_n$ &
$\mathbf{IAdd}\ \B_n \proctr{\NatSegI\ n|\NatSegI\ n|
\NatSegI\ n}{\lambda x\ y.\ x + y < 256^n} (+) $ &
$\B_n$ 字节整数加法的自然数加法对应 \\ \hline
$\mathbf{ISub}\ \B_n$ & $\mathbf{ISub}\ \B_n \proctr{\NatSegI\ n|\NatSegI\ n|
\NatSegI\ n}{\lambda x\ y.\ x \geq y} (-)$ & $\B_n$ 字节整数减法的自然数减法
对应 \\ \hline
  $\mathbf{IMul}\ \B_n$ & $\mathbf{IMul}\ \B_n \proctr{\NatSegI\ n|\NatSegI\ n|
\NatSegI\ n}{\lambda x\ y.\ x * y < 256^n} (\times)$ & $\B_n$ 字节整数乘法
的自然数乘法对应 \\ \hline
  $\mathbf{ILt}\ \B_n$ & $\mathbf{ILt}\ \B_n \proctr{\NatSegI\ n|\NatSegI\ n|
\BoolI}{\lambda x\ y.\ \T} (<)$ & $\B_n$ 字节整数小于判断的自然数小于对应 
\\ \hline
$\mathbf{ILe}\ \B_n$ & $\mathbf{ILe}\ \B_n \proctr{\NatSegI\ n|\NatSegI\ n|
\BoolI}{\lambda x\ y.\ \T} (\leq)$ & $\B_n$ 字节整数小于等于的自然数小于等于对应 \\ \hline
$\mathbf{IEq}\ \B_n$ & $\mathbf{IEq}\ \B_n \proctr{\NatSegI\ n|\NatSegI\ n|
\BoolI}{\lambda x\ y.\ \T} (=)$ & $\B_n$ 字节整数等于判断的自然数等于对应 \\ \hline
$\mathbf{IEq}\ \mathrm{AdrSize}$ & $\mathbf{IEq}\ \mathrm{AdrSize}
\proctr{\AddressI|\AddressI|\BoolI}{\lambda x\ y.\ \T} (=)$ & 
$\B_n$ 字节整数等于判断的账户标识对应 \\ \hline
$\mathbf{IConv}\ \Si_i\ \Si_j$ & $\mathbf{IConv}\ \Si_i\ \Si_j \proctr
    {\NatSegI\ \Si_i|\NatSegI\ \Si_j}{\lambda x.\ x < 256^{\Si_j}}\I$
    & 表示自然数的现象的等值范围变换 \\ \hline
$\mathbf{INot}$ & $\mathbf{INot} \proctr{\BoolI|\BoolI}
{\lambda x.\ \T} (\lnot)$ & bool 值取反 \\ \hline
$\mathbf{Append}\ i$ & $\mathbf{Append}\ i \proctr{i|\ListI\ i|\ListI\ 
i}{\lambda l\ x.\ \T} (::)$ & 增加元素到列表的末尾,$(::)$
是 HOL 系统库中列表的增加函数,$1::[2]$ 表示 $[1,2]$ \\ \hline
$\mathbf{Write}\ c\ i\ j$ & $\mathbf{Write}\ c\ i\ j 
\proctr{i|j|\CWI}{\lambda k\ v.\ \T}$\newline$
(\lambda k\ v.\ ((c,\ \mathbf{Li}_i\ k),\ \mathbf{Li}_j\ v))$ &
写入链数据的表$c$的键$k$为$v$的操作,产生写入命令 \\ \hline
$\mathbf{Read}\ c\ x\ i\ j$ & $\mathbf{Read}\ c\ x\ i\ j 
\proctr{\OneI\ x|i|j}{\lambda x\ k.\ (c,k) \in \Dom x}$\newline$
(\lambda k.\ x\ (c,k))$ &
对给定的链数据 $x$,表标识 $c$,读取链数据$x$的表$c$的$k$键 \\ \hline
$\mathbf{Has}\ c\ x\ i$ & $\mathbf{Has}\ c\ x\ i 
  \proctr{\OneI\ x|i|\BoolI}{\K (\K \T)}$\newline$
(\lambda x\ k.\ (c,k) \in \Dom x)$ &
对给定的链数据 $x$,表标识 $c$,查询链数据$x$的表$c$是否拥有键 $k$ 
  \\ \hline
$\mathbf{Cart}$ & $\mathbf{Cart} \proctr{i|j|i\times j}{\lambda k.\ (c,k) \in \Dom x}$\newline$
(\lambda k.\ x\ (c,k))$ &
对给定的链数据 $x$,表标识 $c$,读取链数据$x$的表$c$的$k$键 \\ \hline
    $\mathbf{Seg}_{0,\Si_i}$ & $\mathbf{Seg}_{0,\Si_i} 
    \proctr{i\times j|i}{\K \T} \mathrm{fst}$ &
取出元组的第一个元素,fst 为 HOL 系统库的函数。 \\ \hline
    $\mathbf{Seg}_{\Si_i,\Si_j}$ & $\mathbf{Seg}_{\Si_i,\Si_j} 
    \proctr{i\times j|j}{\K \T} \mathrm{snd}$ &
取出元组的第二个元素,snd 为 HOL 系统库的函数。 \\ \hline
  \textbf{If} & 不具有 Noesis 同构,但具有类似性质,见定理
  \ref{T.If.prop}& 分支 \\ \hline
\end{tabular}
  \begin{tablenotes} \small
  \item 因空间不够,记号 $\B_n \coloneqq \lceil \log_{256} n \rceil$
\end{tablenotes}
\end{threeparttable} 
\end{table}

\begin{theo}[分支 $\mathbf{If}$ 的性质] \label{T.If.prop}
值得一提的是分支 $\mathbf{If}$,其定义已在表 \ref{tab.IC.primop2} 中列出
,其不具有 Noesis 同构形式的性质,而是
\[ \begin{split}
\forall p_c\ P\ p_a\ i\ \epsilon_a\ p_b\ &\epsilon_b.\ 
  p_c \widesim{\BoolI} P \Rightarrow (P \Rightarrow p_a \widesim{i}
  \epsilon_a) \Rightarrow (\lnot P \Rightarrow p_b \widesim{i}
  \epsilon_b) \Rightarrow \\
  & \mathbf{If}\ p_c\ p_a\ p_b \widesim{i} \xif P \xthen \epsilon_a
  \xelse \epsilon_b
\end{split} \]
  \begin{proof} 若 $\PX p_c = 1$ 则由 $\mathbf{If}$ 定义,有
   $ \mathbf{If}\ p_c\ p_a\ p_b = p_a $,
  再由理解 $\BoolI$ 前提 $p_c \widesim {\BoolI} P$ 下 $P = \T$ 于是
    有 $p_a \widesim{i} \epsilon_a$ 于是命题得证。
若 $\PX p_c \neq 1$ 则由 $\mathbf{If}$ 定义,有
   $ \mathbf{If}\ p_c\ p_a\ p_b = p_b $,
  再由理解 $\BoolI$ 前提 $p_c \widesim {\BoolI} P$ 下 $P = \F$ 于是
    有 $p_b \widesim{i} \epsilon_b$ 于是命题得证。
  \end{proof}
\end{theo}

\begin{defin}[智能合约的外部接口]
所有 \amlhS 的智能合约所暴露的可供外部调用的接口应是一种现象函数,
且其具有至少一种 Noesis 同构,此同构满足第一个参数的理解必为 
$\OneI\ (x:\mathrm{chain})$,且返回值的理解必为 $\RsI\ k$。即
函数 $f$ 应满足如下形式的 Noesis 同构
\[ f \proctr{\OneI\ x|\cdots|\RsI\ k}{cond} \phi \]
\end{defin}

表 \ref{tab.IC.primop2} 定义的基元指令可以直接应用但过于基层,
诸如$\mathbf{IAdd}\ n$ 操作要求给定运算的位宽,显然并不适宜要求用户
每次都显示地给出位宽。\amlhS 提供基于这些基元函数构建的高层函数库。

$\mathbf{AddN_S},\ \mathbf{SubN_S},\ \mathbf{MulN_S},\ 
\mathbf{LeN_S},\ \mathbf{LtN_S},\ \mathbf{EqN_S}$ 
提供自动范围扩张的自然数运算。
\begin{align*}
  \mathbf{AddN_S}\ n\ m\ p\ q \coloneqq\ & \mathbf{IAdd}\ \max(\B_n,\B_m)\ 
  (\xif \B_n \geq \B_m \xthen p \xelse \mathbf{IConv}\ \B_n\ \B_m\ p)\\
  &(\xif \B_m \geq \B_n \xthen p \xelse \mathbf{IConv}\ \B_m\ \B_n\ q)\\
\mathbf{SubN_S}\ n\ m\ p\ q \coloneqq\ & \mathbf{ISub}\ n\ p\ 
  (\xif \B_m = \B_n \xthen p \xelse \mathbf{IConv}\ \B_m\ \B_n\ q)\\
\mathbf{MulN_S}\ n\ m\ p\ q \coloneqq\ & \mathbf{IMul}\ \max(\B_n,\B_m)\ 
  (\xif \B_n \geq \B_m \xthen p \xelse \mathbf{IConv}\ \B_n\ \B_m\ p)\\
  &(\xif \B_m \geq \B_n \xthen p \xelse \mathbf{IConv}\ \B_m\ \B_n\ q)\\
\mathbf{LeN_S}\ n\ m\ p\ q \coloneqq\ &\mathbf{ILe}\ \max(\B_n,\B_m)\ 
  (\xif \B_n \geq \B_m \xthen p \xelse \mathbf{IConv}\ \B_n\ \B_m\ p)\\
  &(\xif \B_m \geq \B_n \xthen p \xelse \mathbf{IConv}\ \B_m\ \B_n\ q)\\
\mathbf{LtN_S}\ n\ m\ p\ q \coloneqq\ &\mathbf{ILt}\ \max(\B_n,\B_m)\ 
  (\xif \B_n \geq \B_m \xthen p \xelse \mathbf{IConv}\ \B_n\ \B_m\ p)\\
  &(\xif \B_m \geq \B_n \xthen p \xelse \mathbf{IConv}\ \B_m\ \B_n\ q)\\
\mathbf{EqN_S}\ n\ m\ p\ q \coloneqq\ &\mathbf{IEq}\ \max(\B_n,\B_m)\ 
  (\xif \B_n \geq \B_m \xthen p \xelse \mathbf{IConv}\ \B_n\ \B_m\ p)\\
  &(\xif \B_m \geq \B_n \xthen p \xelse \mathbf{IConv}\ \B_m\ \B_n\ q)\\
\end{align*}
\[ \begin{array}{lcr} \forall n\ m.\ 
  \mathbf{AddN_S}\ n\ m &\proctr{\NatSegI\ n|
  \NatSegI\ m|\NatSegI\ \max(n,m)}{\lambda x\ y.\ x + y < \max
  (n,m)}& (+) \\\forall n\ m.\ 
  \mathbf{SubN_S}\ n\ m &\proctr{\NatSegI\ n|
  \NatSegI\ m|\NatSegI\ n}{\lambda x\ y.\ x > y}& (-) \\ \forall n\ m.\ 
  \mathbf{MulN_S}\ n\ m &\proctr{\NatSegI\ n|
  \NatSegI\ m|\NatSegI\ \max(n,m)}{\lambda x\ y.\ x \times y < \max
  (n,m)}& (\times)\\ \forall n\ m.\ 
  \mathbf{LeN_S}\ n\ m &\proctr{\NatSegI\ n|
\NatSegI\ m|\NatSegI\ \max(n,m)}{\K\ (\K \T)}& (\leq)\\ \forall n\ m.\ 
  \mathbf{LtN_S}\ n\ m &\proctr{\NatSegI\ n|
  \NatSegI\ m|\NatSegI\ \max(n,m)}{\K\ (\K \T)}& (<)\\ \forall n\ m.\ 
  \mathbf{EqN_S}\ n\ m &\proctr{\NatSegI\ n|
  \NatSegI\ m|\NatSegI\ \max(n,m)}{\K\ (\K \T)}& (=)
\end{array} \] \newline
容易证明 $\mathbf{AddN_S},\ \mathbf{SubN_S},\ \mathbf{MulN_S},\ 
\mathbf{LeN_S},\ \mathbf{LtN_S},\ \mathbf{EqN_S}$ 具有
上述 Noesis 同构,允许进行如下的定理推导以程序构造。
\begin{center}
\begin{prooftree}
  \AxiomC{$\forall n\ m.\ \mathbf{AddN_S}\ n\ m \proctr{\NatSegI\ n|
  \NatSegI\ m|\NatSegI\ \max(n,m)}{\lambda x\ y.\ x + y < \max
  (n,m)} (+)$} 
  \AxiomC{$x \widesim{\NatSegI\ 42} \omega$}
  \RightLabel{(Noesis 调用)} \BinaryInfC{$\forall m.\ 
  \mathbf{AddN_S}\ 42\ m\ x \proctr{\NatSegI\ m|\NatSegI\ \max(42,m)}
  {\lambda y.\ \omega + y < \max(42,m)} (+\ \omega)$}
\end{prooftree}
  \begin{prooftree}
    \AxiomC{$\omega + \upsilon < 666$}
\AxiomC{$\forall m.\ 
\mathbf{AddN_S}\ 42\ m\ x \proctr{\NatSegI\ m|\NatSegI\ \max(42,m)}
  {\lambda y.\ \omega + y < \max(42,m)} (+\ \omega)$}
  \AxiomC{$y \widesim{\NatSegI\ 666} \upsilon$}
  \RightLabel{(Noesis 调用)}
  \BinaryInfC{$\mathbf{AddN_S}\ 42\ 666\ x\ y \proctr{\NatSegI\ 666}
  {\omega + \upsilon < 666} (\omega + \upsilon)$}
    \RightLabel{(Noesis 条件削除)}
    \BinaryInfC{$\mathbf{AddN_S}\ 42\ 666\ x\ y \widesim{\NatSegI\ 666} 
    \omega + \upsilon$}
  \end{prooftree}
\end{center}
而在编译时,将 $\mathbf{AddN_S}\ 42\ 666\ x\ y$ 根据定义展开
\[ \begin{split}
\mathbf{AddN_S}\ 42\ 666\ x\ y =\ & \mathbf{IAdd}\ \max(\B_{42},\B_{666})\ 
  (\xif \B_{42} \geq \B_{666} \xthen x \xelse \mathbf{IConv}\ \B_{42}\ \B_{666}\ x)\\
  &(\xif \B_{666} \geq \B_{42} \xthen x \xelse \mathbf{IConv}\ \B_{666}\ \B_{42}\ y)\\
  =\ & \mathbf{IAdd}\ 2\ (\mathbf{IConv}\ 1\ 2\ x)\ y
\end{split} \]
就得到了编译结果 $\mathbf{IAdd}\ 2\ (\mathbf{IConv}\ 1\ 2\ x)\ y$。

\section{示例:\amlhS 上的转账合约}

定义理解,使用符号 \currency 表示本示例中的货币理解
\begin{align*} 
  \currency &\coloneqq \NatSegI\ \mathrm{TS}&
  \B_\currency &\coloneqq \lceil \log_{256} \mathrm{TS}
  \rceil&
\end{align*}
TS 表示该货币的总供应量,于是理解 \currency 保证了其
本体一定在总供应量内。

先由公理律得到参数定理
\begin{flalign}
&\text{由公理得}& \dot{c} \widesim[2]{\OneI\ \phi} \phi &\vdash 
\dot{c} \widesim[2]{\OneI\ \phi} \phi & \label{f.a.c} \\
&\text{由公理得}& A \widesim{\AddressI} a &\vdash 
A \widesim{\AddressI} a & \label{f.a.a} \\
&\text{由公理得}& B \widesim{\AddressI} b &\vdash 
B \widesim{\AddressI} b & \label{f.a.b} \\
&\text{由公理得}& M \widesim{\currency} m &\vdash M \widesim{\currency} m&
\label{f.a.m} \end{flalign}
$\dot{c}$ 是状态参数,$\phi : \mathrm{chain}$ 是链状态,
$a : \mathrm{address}$ 是转账方,$b : \mathrm{address}$ 是被转账方,
$m:\mathrm{num}$ 是转账金额,链数据中的账户表标识为 
$\mathrm{c_\currency}$,表为
\[\tcurrency \coloneqq \mathbf{TC}\ \ccurrency\ \AddressI\ \currency \]
调用条件为 $P_0$
\[ P_0 \coloneqq (\Sigma\ \Ima_\tcurrency \phi = \mathrm{TS}) \]
即表 $\tcurrency$ 的所有值的和始终等于总供应量,即货币不会丢失也不会增多
。之后将证明转账合约执行后的链数据仍将保持此属性,以证明转账的合理性。
在下文中,简写记号
\begin{align*}
\tilde{c} &\coloneqq \dot{c} \widesim[2]{\OneI\ \phi} \phi&
\tilde{a} &\coloneqq A \widesim{\AddressI} a&
\tilde{b} &\coloneqq B \widesim{\AddressI} b&
\tilde{m} &\coloneqq M \widesim{\currency} m&
\end{align*}
帮助缩短公式长度。此程序会用到如下基元函数或者预定义函数或者常量
\begin{flalign}
&\text{基元}& &\vdash \mathbf{Has}\ t
\proctr{\OneI\ x|i|\BoolI}{\lambda x\ k.\ \T} \Has t & \label{f.Has} \\
&\text{基元}&&\vdash 
\mathbf{Read0}\ t \proctr{\OneI\ x|i|j}{\lambda x\ k.\ \T} \ReadZ\ t
&\label{f.Read} \\
&\text{基元}&& \vdash \mathbf{Write}\ t \proctr{i|j|\CWI}
  {\lambda k\ v.\ \T} \Write\ t & \label{f.Write}\\
&\text{基元}&&\vdash \mathbf{ILt}\ \B_n \proctr{\NatSegI\ n|\NatSegI\ n|
  \BoolI}{\lambda x\ y.\ \T} (<) &\label{f.lt}\\
&\text{基元}&&\vdash\mathbf{IAdd}\ \B_n \proctr{\NatSegI\ n|
\NatSegI\ n|\NatSegI\ n}{\lambda x\ y.\ x + y < 256^n} (+)
  &\label{f.A} \\
&\text{基元}&&\vdash\mathbf{ISub}\ \B_n \proctr{\NatSegI\ n|
  \NatSegI\ n|\NatSegI\ n}{\lambda x\ y.\ x \geq y} (-) & \label{f.S} \\
&\text{常量}& &\vdash \emptylist \widesim[2]{\ListI\ i} \mathrm{Nil}&
  \label{f.Nil}\\
&\text{基元}&&\vdash\mathbf{Append}\ i \proctr{i|\ListI\ i|\ListI\ 
  i}{\lambda x\ l.\ \T} (::)& \label{f.Append}\\
  &\text{基元}&\begin{split}&\vdash
  p_c \widesim{\BoolI} P \Rightarrow (P \Rightarrow p_a \widesim{i}
  \epsilon_a) \Rightarrow (\lnot P \Rightarrow p_b \widesim{i}
  \epsilon_b) \Rightarrow \\
  &\quad\quad \mathbf{If}\ p_c\ p_a\ p_b \widesim{i} \xif P \xthen 
  \epsilon_a \xelse \epsilon_b \end{split}& \label{f.If} \\
&\text{基元}& &\vdash \mathbf{Cart} \proctr{\ListI\ \CWI|i|\RsI\ i}
  {\lambda l\ x.\ \T}(\lambda l\ x.\ (l,\ x)) & \label{f.Rsp} \\
  &\text{常量}& &\vdash \texttt{T} \widesim{\BoolI} \T& \label{f.T} \\
  &\text{常量}& &\vdash \texttt{F} \widesim{\BoolI} \F& \label{f.F} \\
\end{flalign}
现在可以演绎定理以构建程序
\begin{flalign}
&\text{\ref{f.Read}\ SPEC $\tcurrency$}&
&\vdash \mathbf{Read0}\ \tcurrency \proctr{\OneI\ x|\AddressI|\currency}
  {\lambda x\ k.\ \T} \ReadZ\ \tcurrency& \label{f.Rc1} \\
&\text{\ref{f.Rc1} 调用 \ref{f.a.c}}& \tilde{c} & 
\vdash \mathbf{Read0}\ \tcurrency\ \dot{c} \proctr{\AddressI|\currency}
  {\lambda k.\ \T} \ReadZ\ \tcurrency\ \phi&\label{f.Rcc}  \\
&\text{\ref{f.Rcc} 调用 \ref{f.a.a}}& \tilde{c},\ \tilde{a}
& \vdash \mathbf{Read0}\ \tcurrency\ \dot{c}\ A \widesim{\currency}
\ReadZ\ \tcurrency\ \phi\ a& \label{f.Rcca} \\
&\text{\ref{f.lt} 调用 \ref{f.Rcca}}& \tilde{c},\ \tilde{a}
  & \vdash \mathbf{ILt}\ \B_\currency\ (
  \mathbf{Read0}\ \tcurrency\ \dot{c}\ A)
\proctr{\currency|\BoolI}{\K \T} (\lambda x.\ \ReadZ\ \tcurrency\ \phi
  \ a< x) & \label{f.LRcca} \\
  &\text{\ref{f.LRcca} 调用 \ref{f.a.m}}& \tilde{c},\ \tilde{a},\ &
  \tilde{m} \vdash \mathbf{ILt}\ \B_\currency\ (
  \mathbf{Read0}\ \tcurrency\ \dot{c}\ A)\ M
\widesim{\BoolI} (\ReadZ\ \tcurrency\ \phi\ a < M) & \label{f.LRccam} \\
&\text{简写记号}& P_1 &\coloneqq \ReadZ\ \tcurrency\ \phi\ a
  < M & \label{f.P2} \\
&\text{\ref{f.S} 调用 \ref{f.Rcca}}& \begin{split}\tilde{c},\ \tilde{a}
  &\vdash \mathbf{ISub}\ \B_\currency\ (\mathbf{Read0}\ 
    \tcurrency\ \dot{c}\ A) 
  \proctr{\currency|\currency}{\lambda y.\ \ReadZ\ \tcurrency\ \phi\ a
\geq y}\\ &\quad\quad (\lambda y.\ (\ReadZ\ \tcurrency\ \phi\ a) - y)
  \end{split} & \label{f.SRcca}\\
&\text{\ref{f.SRcca} 调用 \ref{f.a.m}}&  \tilde{c},\ \tilde{a},\ &
  \tilde{m},\ \lnot P_1 \vdash \mathbf{ISub}\ \B_\currency\ 
  (\mathbf{Read0}\ \tcurrency\ \dot{c}\ A)\ M \widesim{\currency}(\ReadZ
  \ \tcurrency\ \phi\ a) - m& \label{f.SRccam}\\
&\text{\ref{f.Write} SPEC $\tcurrency$}& \tilde{c} &\vdash 
  \mathbf{Write}\ \tcurrency \proctr{\AddressI|\currency|\CWI}
  {\lambda k\ v.\ \T} \Write\ \tcurrency& \label{f.Wc}\\
&\text{\ref{f.Wc} 调用 \ref{f.a.a}}& \tilde{c},\ \tilde{a} &\vdash
  \mathbf{Write}\ \tcurrency\ A \proctr{\currency|\CWI}
  {\lambda v.\ \T} \Write\ \tcurrency\ a& \label{f.Wca}\\
&\text{\ref{f.Wca} 调用 \ref{f.SRccam}}& \begin{split}
  \tilde{c},\ \tilde{a},\ &\tilde{m},\
  \lnot P_1 \vdash \mathbf{Write}\ \tcurrency\ 
  A\ (\mathbf{ISub}\ \B_\currency\ 
  (\mathbf{Read0}\ \tcurrency\ \dot{c}\ A)\ M) \\&
  \quad\quad\widesim[2]{\CWI}\Write\ \tcurrency\ a\ ((\ReadZ
\ \tcurrency\ \phi\ a) - m) \end{split}& \label{f.WcaSRcam} \\
  &\text{\ref{f.Append} 调用 \ref{f.WcaSRcam}}& \begin{split}
\tilde{c},\ \tilde{a},\ &\tilde{m},\ \lnot P_1 \vdash\mathbf{Append}\ (
  \mathbf{Write}\ \tcurrency\ A\ (\mathbf{ISub}\ \B_\currency\ 
  (\mathbf{Read0}\ \tcurrency\ \dot{c}\ A)\ M))\\
    & \proctr{\ListI\ \CWI|\ListI\ \CWI}{\lambda l.\ \T} 
    (\lambda l.\ (\Write\ \tcurrency\ a\ ((\ReadZ
\ \tcurrency\ \phi\ a) - m)) :: l)
  \end{split} & \label{f.LWcaSRcam1} \\
  &\text{\ref{f.LWcaSRcam1} 调用 \ref{f.Nil}}& \begin{split}
\tilde{c},\ \tilde{a},\ &\tilde{m},\ \lnot P_1 \vdash\mathbf{Append}\ (
  \mathbf{Write}\ \tcurrency\ A\ (\mathbf{ISub}\ \B_\currency\ 
  (\mathbf{Read0}\ \tcurrency\ \dot{c}\ A)\ M))\ \emptylist\\
    & \widesim[3]{\ListI\ \CWI} [\Write\ \tcurrency\ a\ ((\ReadZ
    \ \tcurrency\ \phi\ a) - m)]
  \end{split} & \label{f.LWcaSRcam} \\
&\parbox{1.7cm}{\linespread{1}\selectfont 抽象范畴内的证明}&
P_0,\ &\lnot P_1 \vdash \ReadZ\ \tcurrency\ \phi\ b + m < \mathrm{TS}
\leq 256^{\B_\currency} & \label{f.P.addOK} \\
  &\text{\ref{f.Rcc} 调用 \ref{f.a.b}}& \tilde{c},\ \tilde{b} &\vdash 
\mathbf{Read0}\ \tcurrency\ \dot{c}\ B \widesim{\currency}
  \ReadZ\ \tcurrency\ \phi\ k & \label{f.Rccb}\\
&\text{\ref{f.Rccb} 调用 \ref{f.A}}& \begin{split}
  \tilde{c},\ \tilde{b} &\vdash \mathbf{IAdd}\ \B_\currency\ 
  (\mathbf{Read0}\ \tcurrency\ \dot{c}\ B) \proctr{\currency|\currency}
{\lambda y.\ (\ReadZ\ \tcurrency\ \phi\ b) + y < 256^{\B_\currency}}\\
&(\lambda y.\ (\ReadZ\ \tcurrency\ \phi\ b) + y) \end{split} 
  &\label{f.ARccb} \\
  &\parbox{2.3cm}{\ref{f.ARccb} 调用 \ref{f.a.m} 且由
\ref{f.P.addOK}}& \tilde{c},\ \tilde{b},\ &\tilde{m},\ P_0,\ \lnot P_1
\vdash
\mathbf{IAdd}\ \B_\currency\ (\mathbf{Read0}\ \tcurrency\ \dot{c}\ B)\ 
  M \widesim{\currency} (\ReadZ\ \tcurrency\ \phi\ b) + m & 
  \label{f.ARccbm} \\
&\text{\ref{f.Wc} 调用 \ref{f.a.b}}& \tilde{c},\ \tilde{b}&\vdash 
  \mathbf{Write}\ \tcurrency\ B \proctr{\currency|\CWI}
  {\lambda v.\ \T} \Write\ \tcurrency\ b &\label{f.Wcb}\\
&\text{\ref{f.Wcb} 调用 \ref{f.ARccbm}}& \begin{split}
  \tilde{c},\ \tilde{b},\ &\tilde{m},\ P_0,\ \lnot P_1 
  \vdash \mathbf{Write}\ \tcurrency\ B\ (
\mathbf{IAdd}\ \B_\currency\ (\mathbf{Read0}\ \tcurrency\ \dot{c}\ B)\ 
  M) \\ & \widesim[2]{\CWI} \Write\ \tcurrency\ b\ (
(\ReadZ\ \tcurrency\ \phi\ b) + m) \end{split} &\label{f.WcbARccbm}\\
&\text{\ref{f.Append} 调用 \ref{f.WcbARccbm}}& \begin{split}
  \tilde{c},\ \tilde{b},\ &\tilde{m},\ P_0,\ \lnot P_1 
  \vdash \mathbf{Append}\ (\mathbf{Write}\ \tcurrency\ B\ (
\mathbf{IAdd}\ \B_\currency\ (\mathbf{Read0}\ \tcurrency\ \dot{c}\ B)\ 
M))\\ & \proctr{\ListI\ \CWI|\ListI\ \CWI}{\lambda l.\ \T} (\lambda l.\ 
(\Write\ \tcurrency\ b\ ((\ReadZ\ \tcurrency\ \phi\ b) + m))::l)
\end{split} & \label{f.LWcbARcbm}\\
&\text{\ref{f.LWcbARcbm} 调用 \ref{f.LWcaSRcam}}& \begin{split}
  \tilde{c},\ \tilde{b},\ &\tilde{m},\ P_0,\ \lnot P_1 
  \vdash \mathbf{Append}\ (\mathbf{Write}\ \tcurrency\ B\ (
\mathbf{IAdd}\ \B_\currency\ (\mathbf{Read0}\ \tcurrency\ \dot{c}\ B)\ 
  M))\\&\quad\quad(\mathbf{Append}\ (
  \mathbf{Write}\ \tcurrency\ A\ (\mathbf{ISub}\ \B_\currency\ 
  (\mathbf{Read0}\ \tcurrency\ \dot{c}\ A)\ M))\ \emptylist) \\ 
  &\widesim[3]{\ListI\ \CWI} [(\Write\ \tcurrency\ b\ 
  ((\ReadZ\ \tcurrency\ \phi\ b) + m)),\\&\quad\quad(
  \Write\ \tcurrency\ a\ ((\ReadZ\ \tcurrency\ \phi\ a) - m))]
\end{split} & \label{f.LLtrans1} \\
&\text{\ref{f.Rsp} 调用 \ref{f.LLtrans1}}& \begin{split}
  \tilde{c},\ \tilde{b},\ &\tilde{m},\ P_0,\ \lnot P_1 
  \vdash \mathbf{Cart}\ (\mathbf{Append}\ (
  \mathbf{Write}\ \tcurrency\ B\ (
\mathbf{IAdd}\ \B_\currency\ (\mathbf{Read0}\ \tcurrency\ \dot{c}\ B)\ 
  M))\\&\quad\quad(\mathbf{Append}\ (
  \mathbf{Write}\ \tcurrency\ A\ (\mathbf{ISub}\ \B_\currency\ 
  (\mathbf{Read0}\ \tcurrency\ \dot{c}\ A)\ M))\ \emptylist)) \\ 
  &\proctr{i|\RsI\ i}{\lambda x.\ \T} (\lambda x.\ 
  ([(\Write\ \tcurrency\ b\ 
  ((\ReadZ\ \tcurrency\ \phi\ b) + m)),\\&\quad\quad(
  \Write\ \tcurrency\ a\ ((\ReadZ\ \tcurrency\ \phi\ a) - m))],\ x))
\end{split} & \label{f.LLtrans2} \\
&\text{\ref{f.LLtrans2} 调用 \ref{f.T}}& \begin{split}
  \tilde{c},\ \tilde{b},\ &\tilde{m},\ P_0,\ \lnot P_1 
  \vdash \mathbf{Cart}\ (\mathbf{Append}\ (
  \mathbf{Write}\ \tcurrency\ B\ (
\mathbf{IAdd}\ \B_\currency\ (\mathbf{Read0}\ \tcurrency\ \dot{c}\ B)\ 
  M))\\&\quad\quad(\mathbf{Append}\ (
  \mathbf{Write}\ \tcurrency\ A\ (\mathbf{ISub}\ \B_\currency\ 
  (\mathbf{Read0}\ \tcurrency\ \dot{c}\ A)\ M))\ \emptylist))\ 
  \texttt{T}\\ 
  &\widesim{\RsI\ \BoolI} ([(\Write\ \tcurrency\ b\ 
  ((\ReadZ\ \tcurrency\ \phi\ b) + m)),\\&\quad\quad(
  \Write\ \tcurrency\ a\ ((\ReadZ\ \tcurrency\ \phi\ a) - m))],\ \T)
\end{split} & \label{f.LLtrans} \\
&\text{\ref{f.If} 调用 \ref{f.LRccam}}& \begin{split}
  \tilde{c},\ \tilde{a},\ &\tilde{m} \vdash
  (P_1 \Rightarrow p_a \widesim{i} \epsilon_a) \Rightarrow 
  (\lnot P_1 \Rightarrow p_b \widesim{i} \epsilon_b) \Rightarrow \\
  \mathbf{If}\ &(\mathbf{ILt}\ \B_\currency\ (
  \mathbf{Read0}\ \tcurrency\ \dot{c}\ A)\ M)\ p_a\ p_b \widesim{i} 
\xif P_1 \xthen \epsilon_a \xelse \epsilon_b \end{split}& \label{f.bc}
\\ &\text{\ref{f.Rsp} 调用 \ref{f.Nil}}&&\vdash \mathbf{Cart}\ 
\emptylist \proctr{i|\RsI\ i}{\lambda x.\ \T} (\lambda x.\ (\Nil,\ x))
& \label{f.b11}\\
&\text{\ref{f.b11} 调用 \ref{f.F}}&&\vdash \mathbf{Cart}\ 
\emptylist\ \texttt{F} \widesim{\RsI\ \BoolI} (\Nil,\ \F) 
& \label{f.b1} \\
&\text{\ref{f.bc} 调用 \ref{f.b1}}&\begin{split}
 \tilde{c},\ \tilde{a},\ &\tilde{m} \vdash
(\lnot P_1 \Rightarrow p_b \widesim{\RsI\ \BoolI} \epsilon_b)
  \Rightarrow \\&\mathbf{If}\ (\mathbf{ILt}\ \B_\currency\ ( 
  \mathbf{Read0}\ \tcurrency\ \dot{c}\ A)\ M)\ (
  \mathbf{Cart}\ \emptylist\ \texttt{F})\ p_b\\
  &\widesim{\RsI\ \BoolI} 
\xif P_1 \xthen (\Nil,\ \F) \xelse \epsilon_b \end{split}& \label{f.ook}
\\ &\text{\ref{f.ook} 调用 \ref{f.LLtrans}}& \begin{split}
  \tilde{c},\ \tilde{a},\ &\tilde{b},\ \tilde{m},\ P_0 \vdash 
  \mathbf{If}\ (\mathbf{ILt}\ \B_\currency\ ( 
  \mathbf{Read0}\ \tcurrency\ \dot{c}\ A)\ M)\ (
  \mathbf{Cart}\ \emptylist\ \texttt{F})\\ &\quad(
  \mathbf{Cart}\ (\mathbf{Append}\ (\mathbf{Write}\ \tcurrency\ B\ (
\mathbf{IAdd}\ \B_\currency\ (\mathbf{Read0}\ \tcurrency\ \dot{c}\ B)\ 
  M)))\\&\quad\quad(\mathbf{Append}\ (
  \mathbf{Write}\ \tcurrency\ A\ (\mathbf{ISub}\ \B_\currency\ 
  (\mathbf{Read0}\ \tcurrency\ \dot{c}\ A)\ M))\ \emptylist)\ 
  \texttt{T})\\ 
  &\widesim{\RsI\ \BoolI} \xif P_1 \xthen (\Nil,\ \F) \xelse
  ([(\Write\ \tcurrency\ b\ 
  ((\ReadZ\ \tcurrency\ \phi\ b) + m)),\\&\quad\quad(
  \Write\ \tcurrency\ a\ ((\ReadZ\ \tcurrency\ \phi\ a) - m))],\ \T)
\end{split} & \label{f.OK1} \\
&\text{\ref{f.OK1} 条件 $P_0$}& \begin{split}
  \tilde{c},\ \tilde{a},\ &\tilde{b},\ \tilde{m} \vdash 
  \mathbf{If}\ (\mathbf{ILt}\ \B_\currency\ ( 
  \mathbf{Read0}\ \tcurrency\ \dot{c}\ A)\ M)\ (
  \mathbf{Cart}\ \emptylist\ \texttt{F})\\ &\quad(
  \mathbf{Cart}\ (\mathbf{Append}\ (\mathbf{Write}\ \tcurrency\ B\ (
\mathbf{IAdd}\ \B_\currency\ (\mathbf{Read0}\ \tcurrency\ \dot{c}\ B)\ 
  M)))\\&\quad\quad(\mathbf{Append}\ (
  \mathbf{Write}\ \tcurrency\ A\ (\mathbf{ISub}\ \B_\currency\ 
  (\mathbf{Read0}\ \tcurrency\ \dot{c}\ A)\ M))\ \emptylist)\ 
  \texttt{T})\\ 
  &\proctr{\RsI\ \BoolI}{P_0} \xif P_1 \xthen (\Nil,\ \F) \xelse
  ([(\Write\ \tcurrency\ b\ 
  ((\ReadZ\ \tcurrency\ \phi\ b) + m)),\\&\quad\quad(
  \Write\ \tcurrency\ a\ ((\ReadZ\ \tcurrency\ \phi\ a) - m))],\ \T)
\end{split} & \label{f.OK2} \\
&\text{\ref{f.OK2} 抽象 $\tilde{m}$}& \begin{split}
  \tilde{c},\ \tilde{a},\ &\tilde{b} \vdash (\lambda M.\ 
  \mathbf{If}\ (\mathbf{ILt}\ \B_\currency\ ( 
  \mathbf{Read0}\ \tcurrency\ \dot{c}\ A)\ M)\ (
  \mathbf{Cart}\ \emptylist\ \texttt{F})\\ &\quad(
  \mathbf{Cart}\ (\mathbf{Append}\ (\mathbf{Write}\ \tcurrency\ B\ (
\mathbf{IAdd}\ \B_\currency\ (\mathbf{Read0}\ \tcurrency\ \dot{c}\ B)\ 
  M)))\\&\quad\quad(\mathbf{Append}\ (
  \mathbf{Write}\ \tcurrency\ A\ (\mathbf{ISub}\ \B_\currency\ 
  (\mathbf{Read0}\ \tcurrency\ \dot{c}\ A)\ M))\ \emptylist)\ 
  \texttt{T}))\\ 
  &\proctr{\currency|\RsI\ \BoolI}{\lambda m.\ P_0} (\lambda m.\ 
\xif P_1 \xthen (\Nil,\ \F) \xelse ([(\Write\ \tcurrency\ b\\&
  ((\ReadZ\ \tcurrency\ \phi\ b) + m)),(
  \Write\ \tcurrency\ a\ ((\ReadZ\ \tcurrency\ \phi\ a) - m))],\ \T))
\end{split} & \label{f.OK3} \\
&\text{\ref{f.OK3} 抽象 $\tilde{b}$}& \begin{split}
  \tilde{c},\ \tilde{a},\ &\vdash (\lambda B\ M.\ 
  \mathbf{If}\ (\mathbf{ILt}\ \B_\currency\ ( 
  \mathbf{Read0}\ \tcurrency\ \dot{c}\ A)\ M)\ (
  \mathbf{Cart}\ \emptylist\ \texttt{F})\\ &\quad(
  \mathbf{Cart}\ (\mathbf{Append}\ (\mathbf{Write}\ \tcurrency\ B\ (
\mathbf{IAdd}\ \B_\currency\ (\mathbf{Read0}\ \tcurrency\ \dot{c}\ B)\ 
  M)))\\&\quad\quad(\mathbf{Append}\ (
  \mathbf{Write}\ \tcurrency\ A\ (\mathbf{ISub}\ \B_\currency\ 
  (\mathbf{Read0}\ \tcurrency\ \dot{c}\ A)\ M))\ \emptylist)\ 
  \texttt{T}))\\ 
&\proctr{\AddressI|\currency|\RsI\ \BoolI}{\lambda b\ m.\ P_0} 
  (\lambda b\ m.\ 
\xif P_1 \xthen (\Nil,\ \F) \xelse ([(\Write\ \tcurrency\ b\\&
  ((\ReadZ\ \tcurrency\ \phi\ b) + m)),(
  \Write\ \tcurrency\ a\ ((\ReadZ\ \tcurrency\ \phi\ a) - m))],\ \T))
\end{split} & \label{f.OK4} \\\hline
&\text{\ref{f.OK4} 抽象 $\tilde{a}$}& \begin{split}
  \tilde{c}&\vdash (\lambda A\ B\ M.\ 
  \mathbf{If}\ (\mathbf{ILt}\ \B_\currency\ ( 
  \mathbf{Read0}\ \tcurrency\ \dot{c}\ A)\ M)\ (
  \mathbf{Cart}\ \emptylist\ \texttt{F})\\ &\quad(
  \mathbf{Cart}\ (\mathbf{Append}\ (\mathbf{Write}\ \tcurrency\ B\ (
\mathbf{IAdd}\ \B_\currency\ (\mathbf{Read0}\ \tcurrency\ \dot{c}\ B)\ 
  M)))\\&\quad\quad(\mathbf{Append}\ (
  \mathbf{Write}\ \tcurrency\ A\ (\mathbf{ISub}\ \B_\currency\ 
  (\mathbf{Read0}\ \tcurrency\ \dot{c}\ A)\ M))\ \emptylist)\ 
  \texttt{T}))\\ 
&\proctr{\AddressI|\AddressI|\currency|\RsI\ \BoolI}
  {\lambda a\ b\ m.\ P_0} (\lambda a\ b\ m.\ 
\xif P_1 \xthen (\Nil,\ \F) \xelse ([(\Write\ \tcurrency\ b\\&
  ((\ReadZ\ \tcurrency\ \phi\ b) + m)),(
  \Write\ \tcurrency\ a\ ((\ReadZ\ \tcurrency\ \phi\ a) - m))],\ \T))
\end{split} & \label{f.OK5}  \\
&\text{\ref{f.OK5} 抽象 $\tilde{c}$}& \begin{split}
  &\vdash (\lambda \dot{c}\ A\ B\ M.\ 
  \mathbf{If}\ (\mathbf{ILt}\ \B_\currency\ ( 
  \mathbf{Read0}\ \tcurrency\ \dot{c}\ A)\ M)\ (
  \mathbf{Cart}\ \emptylist\ \texttt{F})\\ &\quad(
  \mathbf{Cart}\ (\mathbf{Append}\ (\mathbf{Write}\ \tcurrency\ B\ (
\mathbf{IAdd}\ \B_\currency\ (\mathbf{Read0}\ \tcurrency\ \dot{c}\ B)\ 
  M)))\\&\quad\quad(\mathbf{Append}\ (
  \mathbf{Write}\ \tcurrency\ A\ (\mathbf{ISub}\ \B_\currency\ 
  (\mathbf{Read0}\ \tcurrency\ \dot{c}\ A)\ M))\ \emptylist)\ 
  \texttt{T}))\\ 
&\proctr{\OneI\ \phi|\AddressI|\AddressI|\currency|\RsI\ \BoolI}
  {\lambda \phi\ a\ b\ m.\ P_0} (\lambda \phi\ a\ b\ m.\ 
\xif P_1 \xthen (\Nil,\ \F) \xelse ([(\Write\ \tcurrency\ b\\&
  ((\ReadZ\ \tcurrency\ \phi\ b) + m)),(
  \Write\ \tcurrency\ a\ ((\ReadZ\ \tcurrency\ \phi\ a) - m))],\ \T))
\end{split} & \label{f.OK6}
\end{flalign}

