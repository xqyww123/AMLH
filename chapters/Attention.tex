\section{记号与混淆区分}

本工作在形式系统的记号上使用逻辑学领域通用的记号,来自计算机科学或其他学科的读者
可以参考Greg 在 \textit{An introduction to substructural logics} 
第二章前几节形式定义的表达方式\cite{restall2002introduction}。

首先是证明系统(Proof System)的区分,一些计算机领域可能将其指代证明用的软件系统而
被误解。证明系统是逻辑学下证明理论中清晰而确定的概念。
本文不尝试定义如此基础的理论概念而仅留下参考文献\cite{restall2002introductionP31}。
简单来说,证明系统是一种由一组公理与一组推导规则构成的形式系统。

缩写 HOL 既代表高阶逻辑(High Order Logic)也代表 HOL交互式定理证明器
(Higher Order Logic Interactive Theorem Prover),同时也表示HOL证明器上所使用
的HOL逻辑,一种高阶逻辑的修改版本。
必须区分HOL逻辑与高阶逻辑,HOL逻辑是HOL定理证明器上所使用的形式化的证明系统,
这一证明系统非常类似高阶逻辑但并不完全等同,这就是普通的一个证明系统,仅仅是跟高阶逻辑
相似,而HOL定理证明器上的证明亦仅是此证明系统上的形式证明。
一个例子是,HOL逻辑这一证明系统上无法良好地定义 NaN ,而可以证明 $0 \div 0 = 0$,参见
John Harrison 的文章 \cite{harrison1993constructing}。


