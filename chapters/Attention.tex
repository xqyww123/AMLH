\section{记号与混淆区分}

本工作一贯地使用符号逻辑(Symbolic Logic)领域通用的记号,来自计算机科学或其他学科的读者
可以参考 Restall 在 \textit{An introduction to substructural logics} 
第二章前几节形式定义的表达方式\cite{restall2002introduction}。

首先是证明系统(Proof System)的区分,
一些计算机领域可能将其指代证明用的软件系统而被误解。
证明系统是逻辑学下证明理论中清晰而确定的概念。
本文不尝试定义如此基础的理论概念而仅留下参考文献\cite{restall2002introductionP31}。
简单来说,证明系统是一种证明理论(Proof Theory)领域的形式系统,
由一组公理与一组用于推理定理的演绎规则构成。

缩写 HOL 既代表高阶逻辑(High Order Logic)也代表 HOL交互式定理证明器
(Higher Order Logic Interactive Theorem Prover),
同时也表示HOL证明器上所使用的HOL逻辑(HOL Logic),
一种类似高阶逻辑的公理系统(Axiomatic System)。
而公理系统是形式系统专用于数理逻辑领域时的别称,所以HOL逻辑亦是形式系统。
必须区分HOL逻辑与高阶逻辑,HOL逻辑是HOL定理证明器上所使用的证明系统,
这一证明系统非常类似高阶逻辑但完全不等同,是仅仅相似但完全不同的两个
形式系统。
而HOL定理证明器上的证明亦仅是HOL逻辑上的形式证明,HOL定理证明器
从来无法证明高阶逻辑上的任何命题。
一个例子是,HOL逻辑上无法良好地定义 NaN ,以至可以证明 $0 \div 0 = 0$,
参见 Harrison 的文章 \cite{harrison1993constructing}。

不熟悉证明理论的读者可能因此质疑HOL定理证明器的可靠性,
但这是多余的。近代兴起的数学形式主义(Mathematical Formalism)
已将曾在王座上的经典逻辑拉下神坛,
世界上存在着诸多的形式系统、公理系统、证明系统,没有任何理由偏爱一方
而歧视一方。本文不是证明理论的文献,有兴趣的读者应当更了解David Hilbert
引领的数学形式主义。

