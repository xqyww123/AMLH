\section{示例:\amlhS 上的转账合约}

定义理解,使用符号 \currency 表示本示例中的货币理解
\begin{align*} 
  \currency &\coloneqq \NatSegI\ \mathrm{TS}&
  \B_\currency &\coloneqq \lceil \log_{256} \mathrm{TS}
  \rceil&
\end{align*}
TS 表示该货币的总供应量,于是理解 \currency 保证了其
本体一定在总供应量内。

先由公理律得到参数定理
\begin{flalign}
&\text{由公理得}& \dot{c} \widesim[2]{\OneI\ \phi} \phi &\vdash 
\dot{c} \widesim[2]{\OneI\ \phi} \phi & \label{f.a.c} \\
&\text{由公理得}& A \widesim{\AddressI} a &\vdash 
A \widesim{\AddressI} a & \label{f.a.a} \\
&\text{由公理得}& B \widesim{\AddressI} b &\vdash 
B \widesim{\AddressI} b & \label{f.a.b} \\
&\text{由公理得}& M \widesim{\currency} m &\vdash M \widesim{\currency} m&
\label{f.a.m} \end{flalign}
$\dot{c}$ 是状态参数,$\phi : \mathrm{chain}$ 是链状态,
$a : \mathrm{address}$ 是转账方,$b : \mathrm{address}$ 是被转账方,
$m:\mathrm{num}$ 是转账金额,链数据中的账户表标识为 
$\mathrm{c_\currency}$,表为
\[\tcurrency \coloneqq \mathbf{TC}\ \ccurrency\ \AddressI\ \currency \]
调用条件为 $P_0$
\[ P_0 \coloneqq (\Sigma\ \Ima_\tcurrency \phi = \mathrm{TS}) \]
即表 $\tcurrency$ 的所有值的和始终等于总供应量,即货币不会丢失也不会增多
。之后将证明转账合约执行后的链数据仍将保持此属性,以证明转账的合理性。
在下文中,简写记号
\begin{align*}
\tilde{c} &\coloneqq \dot{c} \widesim[2]{\OneI\ \phi} \phi&
\tilde{a} &\coloneqq A \widesim{\AddressI} a&
\tilde{b} &\coloneqq B \widesim{\AddressI} b&
\tilde{m} &\coloneqq M \widesim{\currency} m&
\end{align*}
帮助缩短公式长度。此程序会用到如下基元函数或者预定义函数
\begin{flalign}
&\text{基元}& &\vdash \mathbf{Has}\ (\mathbf{TC}\ c\ i\ j)
  \proctr{\OneI\ x|i|\BoolI}{\K (\K \T)}
(\lambda x\ k.\ (c,\ \Li_i\ k) \in \Dom x) & \label{f.Has} \\
&\text{基元}& &\vdash \mathbf{Read}\ (\mathbf{TC}\ 
  c\ i\ j) \proctr{\OneI\ x|i|j}
  {\lambda x\ k.\ (c,\ \Li_i k) \in \Dom x}
  (\lambda x\ k.\ \Tr_j(x (c,\ \Li_i k)))& \label{f.Read} \\
&\text{基元}&&\vdash \mathbf{ILt}\ \B_n \proctr{\NatSegI\ n|\NatSegI\ n|
  \BoolI}{\lambda x\ y.\ \T} (<) &\label{f.lt}\\
&\text{基元}&&\vdash\mathbf{IAdd}\ \B_n \proctr{\NatSegI\ n|
\NatSegI\ n|\NatSegI\ n}{\lambda x\ y.\ x + y < 256^n} (+)
  &\label{f.A} \\
&\text{基元}&&\vdash\mathbf{ISub}\ \B_n \proctr{\NatSegI\ n|
  \NatSegI\ n|\NatSegI\ n}{\lambda x\ y.\ x \geq y} (-) & \label{f.S}
\end{flalign}
现在可以演绎定理以构建程序
\begin{flalign}
&\text{\ref{f.Has}\ SPEC\ $\tcurrency$}&
&\vdash \mathbf{Has}\ \tcurrency\ \proctr{\OneI\ x|\currency|\BoolI}
{\K (\K \T)}(\lambda x\ k.\ (\tcurrency,k) \in \Dom x)&\label{f.Hasc} \\
&\text{\ref{f.Hasc} 调用 \ref{f.a.c}}& \tilde{c} &
\vdash \mathbf{Has}\ \tcurrency\ \dot{c} \proctr{\currency|\BoolI}
{\K \T}(\lambda k.\ (\tcurrency,k) \in \Dom \dot{c})& \label{f.Hcx} \\
&\text{\ref{f.Hcx} 调用 \ref{f.a.a}}& \tilde{c},\ \tilde{a} &
\vdash \mathbf{Has}\ \tcurrency\ \dot{c}\ A \widesim{\BoolI}
(\tcurrency,a) \in \Dom \dot{c}& \label{f.Hcxa1} \\
&\text{\ref{f.Read}\ SPEC $\tcurrency$}&
&\vdash \mathbf{Read}\ \tcurrency \proctr{\OneI\ x|\currency|\currency}
 {\lambda x\ k.\ (\tcurrency,k) \in \Dom x}
  (\lambda x\ k.\ x\ (\tcurrency,k))& \label{f.Rc1} \\
&\text{\ref{f.Rc1} 调用 \ref{f.a.c}}& \tilde{c} & \vdash \mathbf{Read}\ 
\tcurrency\ \dot{c} \proctr{\currency|\currency}{\lambda k.\ 
  (\tcurrency,k) \in \Dom \phi}(\lambda k.\ \phi\ (\tcurrency,k))&
  \label{f.Rcc}  \\
&\text{简写记号}& P_1 &\coloneqq (\tcurrency,a) \in \Dom \phi&\\
&\text{\ref{f.Rcc} 调用 \ref{f.a.a}}& \tilde{c},\ \tilde{a},\ P_1
& \vdash \mathbf{Read}\ \tcurrency\ \dot{c}\ A \widesim{\currency}
\phi\ (\tcurrency,a)& \label{f.Rcca} \\
&\text{\ref{f.lt} 调用 \ref{f.Rcca}}& \tilde{c},\ \tilde{a},\ P_1
  & \vdash \mathbf{ILt}\ \B_\currency\ (
  \mathbf{Read}\ \tcurrency\ \dot{c}\ A)
  \proctr{\currency|\BoolI}{\K \T} (\lambda x.\ \phi\ (\tcurrency,a)
  < x) & \label{f.LRcca} \\
&\text{\ref{f.LRcca} 调用 \ref{f.a.m}}& \tilde{c},\ \tilde{a},\ P_1
  & \vdash \mathbf{ILt}\ \B_\currency\ (
  \mathbf{Read}\ \tcurrency\ \dot{c}\ A)\ M
  \widesim{\BoolI} (\phi\ (\tcurrency,a) < M) & \label{f.LRccam} \\
  &\text{简写记号}& P_2 &\coloneqq \phi\ (\tcurrency,a) < M & \label
  {f.P2} \\
  &\text{\ref{f.S} 调用 \ref{f.Rcca}}& \tilde{c},\ \tilde{a},\ P_1
&\vdash
\mathbf{ISub}\ \B_\currency\ (\mathbf{Read}\ \tcurrency\ \dot{c}\ A) 
  \proctr{\currency|\currency}{\lambda y.\ \phi\ (\tcurrency,a)
  \geq y} (\lambda y.\ \phi\ (\tcurrency,a) - y)& \label{f.SRcca}\\
&\text{\ref{f.SRcca} 调用 \ref{f.a.m}}&  \tilde{c},\ \tilde{a},\ P_1
,\ \lnot P_2 &\vdash \mathbf{ISub}\ \B_\currency\ 
  (\mathbf{Read}\ \tcurrency\ \dot{c}\ A)\ M
  \widesim{\currency} \phi\ (\tcurrency,a) - m& \label{SRccam}\\
  &\parbox{1.7cm}{\linespread{1}\selectfont 抽象范畴内的证明}&
  P_0,\ \lnot P_2 &\vdash \phi\ (\tcurrency,\ b) + M < \mathrm{TS}&
  \label{f.P.addOK} \\
&\text{\ref{f.Rcc} 调用 \ref{f.a.b}}& \tilde{b} &\vdash \mathbf{Read}\ 
\tcurrency\ \dot{c}\ b \proctr{\currency|\currency}{\lambda k.\ 
(\tcurrency,k) \in \Dom \phi}(\lambda k.\ \phi\ (\tcurrency,k))& 
  \label{f.Rccb}\\
  %&\text{\ref{f.A} 调用 \ref{f}}&
  &\text{\ref{f.S} 调用 \ref{f.Rcca}}& \begin{split}
    \tilde{c},\ \tilde{a},\ P_1
    &\vdash\mathbf{SubN_S}\ \mathrm{TS}\ m\ 
    (\mathbf{Read}\ \tcurrency\ \dot{c}\ A) \proctr{\NatSegI\ m|
  \currency}{\lambda y.\ \phi\ (\tcurrency,a) > y}\\ & \quad\quad\quad
  (\lambda y.\ \phi\ (\tcurrency,a) - y) \end{split} &
\end{flalign}
