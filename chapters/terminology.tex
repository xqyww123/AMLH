\Nchapter{术语、符号与中英文对照}

本章列出本文用到的所有术语与中英文的对照。部分术语没有通用的中译名,
读者可以根据其英文名称查找相关文献。

\begin{multicols}{2}
\begin{labeling}{alligator}
\item [元数]        arity
\item [编纂系统]    Collation System
\item [演绎规则]    Deductive Rule
\item [形式语言]    Formal Language
\item [形式系统]    Formal System
\item [泛型]        Generic
\item [高阶逻辑]    High Order Logic
\item [中间表达]    Intermediate Representation
\item [直觉逻辑]    Intuitionistic Logic
\item [数理对象]    mathematical object
\item [证明系统]    Proof System
\item [类型系统]    Type System
\item [类型构造器]  Type Constructor
\item [全称量化]    universal quantification
\item [唯一解读性]  unique readability
\item [肯定前件]    Modus Ponens
\end{labeling}
\end{multicols}

此外本文使用如下的记号,它们部分是通用的,部分是尚未有通用的记号而本文
选取了尽可能通用的。

\begin{align*}
    &\begin{array}{ll}
        \sequence{X}& \text{由集合}\ X\ \text{中的元算构成的序列}\\
        x\mathbin\Vert y& \text{序列}\ x,\ y\ \text{的连接}\\
        \abs{X} & \text{集合或序列}\ X \text{的大小}
    \end{array}&
    &\begin{array}{ll}
        x\concat y & \text{字符串}\ x,\ y\ \text{的拼接}\\
        \underline{\mathrm{abc}} & \text{字面量为 abc 的字符串}\\
        \abs{X} & \text{集合或序列}\ X \text{的大小}
    \end{array}&
\end{align*}
