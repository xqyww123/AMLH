\chapter*{术语、符号与中英文对照}

本章列出本文用到的所有术语与中英文的对照。部分术语没有通用的中译名,
读者可以根据其英文名称查找相关文献。
\begin{center}
    \begin{tabular}{ p{2.5cm} p{5cm} p{2.5cm} p{5cm} } 
    %\begin{tabular}{ l l l l } 
\terminology{WASM}{WebAssembly}&
\terminology{SML}{Standard Meta Language}\\
\terminology{安全严苛}{safety critical}&
\terminology{本体}{noumenon, νoούμενον}\\
\terminology{编纂系统}{Collation System}&
\terminology{定理证明器}{Theorem Prover}\\
\terminology{定理证明器}{Proof Assistance\footnote{Theorem Prover 与 Proof Assistance
      在机器证明界是同义词。}}&
\terminology{对应}{correspondence}\\
\terminology{泛型}{Generic}&
\terminology{符号逻辑}{Symbolic Logic}\\
\terminology{高阶逻辑}{High Order Logic}&
\terminology{公理系统}{Axiomatic System}\\
\terminology{构造演算}{Calculus of Constructions}&
\terminology{归纳}{induct}\\
\terminology{结构化归纳}{structual induction}&
\terminology{肯定前件}{Modus Ponens}\\
\terminology{类型构造器}{Type Constructor}&
\terminology{类型关系}{typing}\\
\terminology{类型系统}{Type System}&
\terminology{理解}{understanding, nóēsis, νόησῐς}\\
\terminology{排中律}{law of excluded middle}&
\terminology{皮亚诺数}{Peano number}\\
\terminology{全称量化}{Universal Quantification}&
\terminology{数理对象}{Mathematical Object}\\
\terminology{同构}{homomorphism}&
\terminology{唯一解读性}{Unique Readability}\\
\terminology{现象}{phenomenon, φαινόμενον}&
      \terminology{项}{term}\\
      \terminology{形式化方法}{Formal Method}&
\terminology{形式化描述}{Formal Specification}\\
      \terminology{形式系统}{Formal System}&
      \terminology{形式语言}{Formal Language}\\
      \terminology{演绎}{deduct}&
\terminology{演绎律}{Deductive Rule}\\
      \terminology{依赖类型}{Dependent Type}&
\terminology{元数}{arity}\\
      \terminology{证明系统}{Proof System}&
\terminology{指称语义}{Denotation Semantic}\\
      \terminology{直觉逻辑}{Intuitionistic Logic}&
      \terminology{中间表达}{Intermediate Representation}
\end{tabular}
\end{center}

此外本文使用如下的记号,它们部分是通用的,部分是尚未有通用的记号而本文
选取了尽可能通用的。

\begin{align*}
    &\begin{array}{ll}
        \sequence{X}& \text{由集合}\ X\ \text{中的元素构成的序列}\\
        x\mathbin\Vert y& \text{序列}\ x,\ y\ \text{的连接}\\
        \abs{X} & \text{集合或序列}\ X \text{的大小}\\
        \Dom f & \text{映射 $f$ 的原像}\\
        A \rightarrow B & \text{从 $A$ 到 $B$ 的映射}\\
        \bnf{B} & \text{满足BNF语法 $B$ 的字符串集合}\\
        type \coloneqq \cdots & \text{HOL 逻辑上项的定义} \\
    \end{array}&
    &\begin{array}{ll}
        x\concat y & \text{字符串}\ x,\ y\ \text{的拼接}\\
        \underline{\mathrm{abc}} & \text{字面量为 abc 的字符串}\\
        \Gamma \vdash A & \text{由前提 $\Gamma$ 可得结论 $A$}\\
        \Ima f & \text{映射 $f$ 的像}\\
        A \mapsto B & \text{从 $A$ 到 $B$ 的有限映射}\\
        type \Coloneqq \cdots & \text{HOL 逻辑上类型的定义} \\
        &
    \end{array}&
\end{align*}

\noindent 一些过于基础的数学记号并未在上标列出。另外注意本文遵循 HOL 逻辑
使用符号 $\Rightarrow$ 表示逻辑 $if$,替代常用记号 $\rightarrow$,
而 $\rightarrow$ 表示 HOL 逻辑中的函数类型。
\begin{center}
  \AxiomC{$\Gamma_1 \vdash P \Rightarrow Q$}
  \AxiomC{$\Gamma_2 \vdash P$} \RightLabel{(Modus Ponens)}
  \BinaryInfC{$\Gamma_1 \cup \Gamma_2 \vdash Q$} \DisplayProof
\end{center}
