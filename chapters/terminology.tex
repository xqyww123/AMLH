\Nchapter{术语、符号与中英文对照}

本章列出本文用到的所有术语与中英文的对照。部分术语没有通用的中译名,
读者可以根据其英文名称查找相关文献。
\begin{center}
    \begin{tabular}{ p{2.5cm} p{5cm} p{2.5cm} p{5cm} } 
    %\begin{tabular}{ l l l l } 
    \terminology{元数}{arity}&
\terminology{编纂系统}{Collation System}\\
    \terminology{演绎律}{Deductive Rule}&
\terminology{形式语言}{Formal Language}\\
    \terminology{形式系统}{Formal System}&
\terminology{泛型}{Generic}\\
    \terminology{高阶逻辑}{High Order Logic}&
\terminology{中间表达}{Intermediate Representation}\\
    \terminology{直觉逻辑}{Intuitionistic Logic}&
\terminology{数理对象}{mathematical object}\\
    \terminology{证明系统}{Proof System}&
\terminology{类型系统}{Type System}\\
    \terminology{类型构造器}{Type Constructor}&
\terminology{全称量化}{universal quantification}\\
    \terminology{唯一解读性}{unique readability}&
\terminology{肯定前件}{Modus Ponens}\\
        \terminology{现象}{phenomenon, φαινόμενον}&
        \terminology{理解}{understanding, nóēsis, νόησῐς}\\
        \terminology{本体}{noumenon, νoούμενον}&
        \terminology{同构}{homomorphism}\\
        \terminology{对应}{correspondence}&
        \terminology{演绎}{deduct}\\
        \terminology{归纳}{induct}&
        \terminology{符号逻辑}{Symbolic Logic}
\end{tabular}
\end{center}

此外本文使用如下的记号,它们部分是通用的,部分是尚未有通用的记号而本文
选取了尽可能通用的。

\begin{align*}
    &\begin{array}{ll}
        \sequence{X}& \text{由集合}\ X\ \text{中的元算构成的序列}\\
        x\mathbin\Vert y& \text{序列}\ x,\ y\ \text{的连接}\\
        \abs{X} & \text{集合或序列}\ X \text{的大小}
    \end{array}&
    &\begin{array}{ll}
        x\concat y & \text{字符串}\ x,\ y\ \text{的拼接}\\
        \underline{\mathrm{abc}} & \text{字面量为 abc 的字符串}\\
        \abs{X} & \text{集合或序列}\ X \text{的大小}
    \end{array}&
\end{align*}
