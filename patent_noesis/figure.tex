\begin{figure}[h] 
  \centering \begin{tikzpicture}\useasboundingbox (-5,-5) rectangle (5,5);
\baselineskip=0.5cm
    \draw (0,0) node (O) {编辑壳层} circle [radius=1.2];
  \path[name path=vCbarD] (60:3.3) -- ([turn]-90:1.5);
  \path[name path=vCbarU] (60:3.3) -- ([turn]90:1.5);
  \path[name path=hCbarD] (15:1.5) -- ([turn]45:4);
  \path[name path=hCbarU] (105:1.5) -- ([turn]-45:4);
  \draw [name intersections={of=vCbarD and hCbarD, by=x}] 
  (60:1.5) arc [start angle=60, end angle=15, radius=1.5] -- (x) -- ([turn]90:1.5);
  \draw [name intersections={of=vCbarU and hCbarU, by=x}]
  (60:1.5) arc [start angle=60, end angle=105, radius=1.5] -- (x) -- ([turn]-90:1.5);
  \draw [align=center] (60:2.4) node {编译\\后端};
  \draw (60:4) node[dart,draw,minimum size=1cm, shape border rotate=60,
    shape border uses incircle,fill=white] (B) {编译};
  \draw [->] (65:0.9) -- ([turn]-5:0.8);
  \draw [<-] (55:0.9) -- ([turn]5:0.8);

  \path[name path=vbarD] (-60:3.7) -- ([turn]-90:1.5);
  \path[name path=vbarU] (-60:3.7) -- ([turn]90:1.5);
  \path[name path=hbarD] (-105:1.5) -- ([turn]45:4);
  \path[name path=hbarU] (-15:1.5) -- ([turn]-45:4);
  \draw [name intersections={of=vbarD and hbarD, by=x}] 
  (-60:1.5) arc [start angle=-60, end angle=-105, radius=1.5] -- (x) -- ([turn]90:1.5);
  \draw [name intersections={of=vbarU and hbarU, by=x}]
  (-60:1.5) arc [start angle=-60, end angle=-15, radius=1.5] -- (x) -- ([turn]-90:1.5);
  \draw [align=center] (-60:2.5) node {HOL\\证明器};
  \draw [fill=white] (-60:4.6) +(-60:1.3) -- +(-150:1.3) -- +(120:1.3) -- + (30:1.3) -- +(-60:1.3);
  \draw (-60:4.6) node [align=center] (A) {\amlh};
  \draw [->] (-55:0.9) -- ([turn]-5:0.8);
  \draw [<-] (-65:0.9) -- ([turn]5:0.8);
  \path[name path=Vbar] (-4,1.2) -- (-4,-1.2);
  \path[name path=Dhbar] (-135:1.5) -- +(-3,0);
  \path[name path=Uhbar] (135:1.5) -- +(-3,0);
  \draw [name intersections={of=Vbar and Uhbar, by=x}] 
  (-1.5,0) arc [start angle=180, end angle=135, radius=1.5] -- (x) -- (-4,0);
  \draw [name intersections={of=Vbar and Dhbar, by=x}]
  (-1.5,0) arc [start angle=180, end angle=225, radius=1.5] -- (x) -- (-4,0);
  \draw [align=center] (-2.8,0) node {编辑器\\前端};
  \draw [<-] (175:1) -- +(-0.8,0);
  \draw [->] (-175:1) -- +(-0.8,0);
  \draw (-4.4,0) node[left] {用户};
  \draw [<-] (175:1) ++ (-2.7cm,0) -- +(-0.6,0);
  \draw [->] (-175:1) ++ (-2.7cm,0) -- +(-0.6,0);
  %\path (B) ([turn]-90:2) node (BX) {};
  %\draw [->,thick,rounded corners=1cm] (A) ([turn]90:1.8) -- (0:1.8) -- (BX);
\end{tikzpicture}
\caption{软件 \Eamlh 的主体框架} \label{fig:total-frame}
\end{figure}

\begin{figure}[h] 
  \centering \begin{tikzpicture} \baselineskip=0.5cm
    \draw (0,0) node (O) {编辑壳层} circle [radius=1.2];
  \path[name path=vCbarXD] (60:3.3) -- ([turn]-90:1.5);
  \path[name path=vCbarXU] (60:3.3) -- ([turn]90:1.5);
  \path[name path=hCbarXD] (15:1.5) -- ([turn]45:4);
  \path[name path=hCbarXU] (105:1.5) -- ([turn]-45:4);
  \draw [name intersections={of=vCbarXD and hCbarXD, by=x}] 
  (60:1.5) arc [start angle=60, end angle=15, radius=1.5] -- (x) -- ([turn]90:1.5);
  \draw [name intersections={of=vCbarXU and hCbarXU, by=x}]
  (60:1.5) arc [start angle=60, end angle=105, radius=1.5] -- (x) -- ([turn]-90:1.5);
  \draw [align=center] (60:2.4) node {编译\\后端};
  \draw (60:4) node[dart,draw,minimum size=1cm, shape border rotate=60,
    shape border uses incircle,fill=white] (B) {编译};
  \draw [->] (65:0.9) -- ([turn]-5:0.8);
  \draw [<-] (55:0.9) -- ([turn]5:0.8);

  \path[name path=vbarXD] (-60:3.7) -- ([turn]-90:1.5);
  \path[name path=vbarXU] (-60:3.7) -- ([turn]90:1.5);
  \path[name path=hbarXD] (-105:1.5) -- ([turn]45:4);
  \path[name path=hbarXU] (-15:1.5) -- ([turn]-45:4);
  \draw [name intersections={of=vbarXD and hbarXD, by=x}] 
  (-60:1.5) arc [start angle=-60, end angle=-105, radius=1.5] -- (x) -- ([turn]90:1.5);
  \draw [name intersections={of=vbarXU and hbarXU, by=x}]
  (-60:1.5) arc [start angle=-60, end angle=-15, radius=1.5] -- (x) -- ([turn]-90:1.5);
  \draw [align=center] (-60:2.5) node {HOL\\证明器};
  \draw [fill=white] (-60:4.6) +(-60:1.3) -- +(-150:1.3) -- +(120:1.3) -- + (30:1.3) -- +(-60:1.3);
  \draw (-60:4.6) node [align=center] (A) {\amlh};
  \draw [->] (-55:0.9) -- ([turn]-5:0.8);
  \draw [<-] (-65:0.9) -- ([turn]5:0.8);
  \path[name path=VbarX] (-4,1.2) -- (-4,-1.2);
  \path[name path=DhbarX] (-135:1.5) -- +(-3,0);
  \path[name path=UhbarX] (135:1.5) -- +(-3,0);
  \draw [name intersections={of=VbarX and UhbarX, by=x}] 
  (-1.5,0) arc [start angle=180, end angle=135, radius=1.5] -- (x) -- (-4,0);
  \draw [name intersections={of=VbarX and DhbarX, by=x}]
  (-1.5,0) arc [start angle=180, end angle=225, radius=1.5] -- (x) -- (-4,0);
  \draw [align=center] (-2.8,0) node {编辑器\\前端};
  \draw [<-] (175:1) -- +(-0.8,0);
  \draw [->] (-175:1) -- +(-0.8,0);
  \draw (-4.4,0) node[left] {用户};
  \draw [<-] (175:1) ++ (-2.7cm,0) -- +(-0.6,0);
  \draw [->] (-175:1) ++ (-2.7cm,0) -- +(-0.6,0);
  \path (B) ([turn]-90:2) node (BX) {};
  \draw [->,thick,rounded corners=1cm] (A) ([turn]90:1.8) -- (0:1.8) -- (BX);
\draw [align=center] (A) ([turn]90:2) node[right]{等价变换\\展开成基元指令\\与常量的表达};
    \draw [align=center] (2.7,0) node[right] {得到中间表达};
    \draw [align=center] (BX) node[right] {编译中间表达\\至目标执行环境};
\end{tikzpicture}
\caption{编译流程示意} \label{fig:compilation}
\end{figure}

\begin{figure}
\begin{center}
    \AxiomC{$\Gamma_1 \vdash f_p \proctr{i|j|\cdots|l}{cond} f_e$}
    \AxiomC{$\Gamma_2 \vdash a_p \widesim{i} a_e$}
    \RightLabel{(调用)}
    \BinaryInfC{$\Gamma_1 \cup \Gamma_2 \vdash f_p\ a_p \proctr{j|\cdots|l}{cond\ a_e}
      f_e\ a_e$}
\DisplayProof

  \vspace{\baselineskip}
    \AxiomC{$\Gamma_1,\ a_p \widesim{i} a_e \vdash b_p \proctr{j|\cdots|l}{cond} b_e$}
    \RightLabel{(抽象)}
    \UnaryInfC{$\Gamma_1 \vdash (\lambda a_p.\ b_p)
      \proctr{i|j|\cdots|l}{\lambda a_e.\ cond\ a_e} (\lambda a_e.\ b_e)$}
\DisplayProof

\vspace{\baselineskip}
\begin{tabular}{c c}
    \AxiomC{$\Gamma, cond \vdash p \widesim{i} e$}
    \RightLabel{(一阶同构引入)}
    \UnaryInfC{$\Gamma \vdash p \proctr{i}{cond} e$} \DisplayProof &
    \AxiomC{$\Gamma_1 \vdash p \proctr{i}{cond} e$}
    \AxiomC{$\Gamma_2 \vdash cond$}
    \RightLabel{(一阶同构削除)}
    \BinaryInfC{$\Gamma_1 \cup \Gamma_2 \vdash p \widesim{i} e$} \DisplayProof
\end{tabular}

\vspace{\baselineskip}
    \AxiomC{$\Gamma \vdash p \widesim{i} e$}
    \RightLabel{(对应记号 $x$)}
    \UnaryInfC{$\Gamma,\ ((x_\mathrm{p} = p)\ \land\ (
    x_\mathrm{e} = e)) \vdash
     x_\mathrm{p} \widesim{i}  x_\mathrm{e}$} \DisplayProof

\vspace{\baselineskip}
  \AxiomC{$\Gamma \vdash p \proctr{i|\cdots|k}{cond} e$}
    \RightLabel{(同构记号 $x$)}
    \UnaryInfC{$\Gamma,\ ((x_\mathrm{p} = p)\ \land\ (
    x_\mathrm{e} = e)) \vdash
     x_\mathrm{p} \proctr{i|\cdots|k}{cond}  x_\mathrm{e}$} \DisplayProof

\vspace{\baselineskip}
  \AxiomC{$\Gamma,\ ((x_\mathrm{p} = p)\ \land\ (x_\mathrm{e} = e 
  ))\vdash p' \widesim{i}  e'$}
    \RightLabel{(对应记号 $x$ 削除)}
  \UnaryInfC{$\Gamma \vdash \mathrm{Let}\ p\ (\lambda x_\mathrm{p}.\ p')
    \widesim{i} \mathrm{Let}\ e\ (\lambda x_\mathrm{e}.\ e')$}\DisplayProof

\vspace{\baselineskip}

\AxiomC{$\Gamma,\ ((x_\mathrm{p} = p)\ \land\ (x_\mathrm{e} = e))
  \vdash p' \proctr{i|\cdots|k}{cond}  e'$}
  \RightLabel{(同构记号 $x$ 削除)}
\UnaryInfC{$\Gamma \vdash \mathrm{Let}\ p\ (\lambda x_\mathrm{p}.\ p')
\proctr{i|\cdots|k}{cond} \mathrm{Let}\ e\ (\lambda x_\mathrm{e}.\ e')$} \DisplayProof

\vspace{\baselineskip}

\begin{tabular}{c c}
\AxiomC{$\Gamma \vdash p \widesim{i_1 \cdot i_2} (e_1,\ e_2)$}
    \RightLabel{(合并分解1)}
\UnaryInfC{$\Gamma \vdash p \widesim{i_1} e_1$} \DisplayProof&
\AxiomC{$\Gamma \vdash p \widesim{i_1 \cdot i_2} (e_1,\ e_2)$}
    \RightLabel{(合并分解2)}
\UnaryInfC{$\Gamma \vdash p \widesim{i_2} e_2$} \DisplayProof
\end{tabular}

\vspace{\baselineskip}

\AxiomC{$\Gamma_1 \vdash p \widesim{i_1} e_1$}
\AxiomC{$\Gamma_2 \vdash p \widesim{i_2} e_2$}
    \RightLabel{(合并)}
\BinaryInfC{$\Gamma_1 \cup \Gamma_2 \vdash p \widesim{i_1 \cdot i_2} (e_1,\ e_2)$}
\DisplayProof
\end{center}

\caption{\noesishol 上的演绎律} \label{fig:noesis-hol-rule}
\end{figure}

\begin{figure} \centering
  \begin{tikzpicture}
    \draw (-3.5,0) node [shade,draw,cylinder,shape border rotate=90,minimum width=3cm,
    bottom color=white!50!gray,top color=white,minimum height=1cm,
    aspect=.15] {Noesis 系统};
    \draw (-3.5,-1) node {朴素 Noesis 系统};
    \draw (0,0) node [shade,draw,cylinder,shape border rotate=90,
    bottom color=white!50!gray,top color=white,
    minimum width=3cm, minimum height=1cm, aspect=.2] {HOL 逻辑};
    \draw (0,-1) node {\noesishol};
    \draw (0,0.9) node [shade,draw,cylinder,shape border rotate=90,
    bottom color=white!50!gray,top color=white,
    minimum width=2cm, minimum height=1cm, aspect=.1] {Noesis 系统};
    \draw (4,0) node [shade,draw,cylinder,shape border rotate=90,
    bottom color=white!50!gray,top color=white,
    minimum width=4cm, minimum height=1cm, aspect=.3] {HOL 逻辑};
    \draw (4,-1) node {\amlh};
    \draw (4,1) node [shade,draw,cylinder,shape border rotate=90,
    bottom color=white!50!gray,top color=white,
    minimum width=3cm, minimum height=1cm, aspect=.2] {Noesis 系统};
    \foreach \x in {3,4.2}
    \draw [align=center] (\x,2.2) node [shade,draw,cylinder,shape border rotate=90,
    bottom color=white!50!gray,top color=white,
    minimum height=1.5cm, aspect=.17, font={\linespread{1.1}\tiny}] {基\\元\\指\\令};
    \foreach \x in {3.6,4.8}
    \draw [align=center] (\x,2.2) node [shade,draw,cylinder,shape border rotate=90,
    bottom color=white!50!gray,top color=white,
    minimum height=1.5cm, aspect=.17, font={\linespread{1.1}\tiny}] {常\\量};
    \draw (8.5,0) node [shade,draw,cylinder,shape border rotate=90,
    bottom color=white!50!gray,top color=white,
    minimum width=4cm, minimum height=1cm, aspect=.3] {HOL 逻辑};
    \draw (8.5,-1) node {\amlh 上的程序};
    \draw (8.5,1) node [shade,draw,cylinder,shape border rotate=90,
    bottom color=white!50!gray,top color=white,
    minimum width=3cm, minimum height=1cm, aspect=.2] {Noesis 系统};
    \foreach \x in {7.5,8.7}
    \draw [align=center] (\x,2.2) node [shade,draw,cylinder,shape border rotate=90,
    bottom color=white!50!gray,top color=white,
    minimum height=1.5cm, aspect=.17, font={\linespread{1.1}\tiny}] {基\\元\\指\\令};
    \foreach \x in {8.1,9.3}
    \draw [align=center] (\x,2.2) node [shade,draw,cylinder,shape border rotate=90,
    bottom color=white!50!gray,top color=white,
    minimum height=1.5cm, aspect=.17, font={\linespread{1.1}\tiny}] {常\\量};
    \draw (8.5,3.2) node [shade,draw,cylinder,shape border rotate=90,
    bottom color=white!50!gray,top color=white,
    minimum width=3cm, minimum height=1cm, aspect=.2] {\amlh 上的程序};
  \end{tikzpicture}
\caption{朴素 Noesis 系统、\noesishol、\amlh、\amlh 上的程序之间的关系} 
  \label{fig:noesis-rela}
\end{figure}
