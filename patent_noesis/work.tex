\chapter{本发明的技术方案}
\section{本发明采用的技术方案}
图 \ref{fig:naive_noesis_lang} 表示朴素 Noesis 形式系统的形式语言,
其中 $\hat{\mathcal{P}}, \hat{\mathcal{E}}, \mathcal{I}$ 为给定的单词集,
记号 $\bnf{BNF}$ 表示 BNF 语法,最后的 $L_\mathrm{N}$ 即为朴素 Noesis 形式系统的形式语言。
图 \ref{fig:naive_noesis_rule} 为朴素 Noesis 形式系统的演绎规则。
图 \ref{fig:naive_noesis_lang} 与图 \ref{fig:naive_noesis_rule} 共同表示了
朴素 Noesis 形式系统。
其中参数律用于构建参数,一阶同构引入律与一阶同构削除律为技术性操作无实际意义,应用律
用于函数调用,抽象律用于函数构建,对应记号律、同构记号律用于临时变量的引入,
对应记号削除律、同构记号削除律用于函数构建前临时变量的封装。

\section{本发明的关键点}
\subsection{类型系统的扩广 Noesis 形式系统}

\chapter{附图}
\begin{figure}
\caption{朴素 Noesis 形式系统的形式语言} \label{fig:naive_noesis_lang}
\begin{gather*}
  \begin{align*} \hat{\mathcal{P}} &\subseteq \Words&
\hat{\mathcal{E}} &\subseteq \Words& \mathcal{I} &\subseteq \Words&
  \end{align*}\\
  \begin{array}{lcrcccccccl}
  \mathcal{P} &=& \bnf{&\hat{\mathcal{P}}&\mbar&
    (\mathcal{P}\ \mathcal{P})&
    \mbar&(\lambda\ \hat{\mathcal{P}}\ \mathcal{P})&\mbar&
    (\mathrm{Let}\ \hat{\mathcal{P}\ \mathcal{P}\ \mathcal{P}})&}\\
    \mathcal{E} &=& \bnf{&\hat{\mathcal{E}}&\mbar&
    (\mathcal{E}\ \mathcal{E})&
    \mbar&(\lambda\ \hat{\mathcal{E}}\ \mathcal{E})&\mbar&
    (\mathrm{Let}\ \hat{\mathcal{E}}\ \mathcal{E}\ \mathcal{E})&}
  \end{array} \\
\begin{align*}
 N_1 &= \bnf{\mathcal{P} \widesim{\mathcal{I}} \mathcal{E}}&
 N^* &= \bnf{\mathcal{P} \proctr{\mathcal{I}|\cdots|\mathcal{I}}{}
\mathcal{E}}& 
\Gamma_\mathrm{N} &= \powerset(N_1) & \end{align*} \\
L_\mathrm{N} = \bnf{\Gamma_\mathrm{N} \vdash N}
\end{gather*}
\end{figure}

\begin{figure}
\caption{朴素 Noesis 形式系统的演绎规则} \label{fig:naive_noesis_rule}
\vspace{6mm}

\begin{minipage}{0.25\linewidth} \begin{prooftree} \centering
    \AxiomC{\hfill}
    \RightLabel{(参数)}
    \UnaryInfC{$p \widesim{i} e \vdash p \widesim{i} e$}
\end{prooftree}\end{minipage}%
\begin{minipage}{0.4\linewidth} \begin{prooftree}
    \AxiomC{$\Gamma \vdash p \widesim{i} e$}
    \RightLabel{(一阶同构引入)}
    \UnaryInfC{$\Gamma \vdash p \proctr{i}{} e$}
\end{prooftree}\end{minipage}%
\begin{minipage}{0.3\linewidth} \begin{prooftree}
    \AxiomC{$\Gamma \vdash p \proctr{i}{} e$}
    \RightLabel{(一阶同构削除)}
    \UnaryInfC{$\Gamma \vdash p \widesim{i} e$}
\end{prooftree}\end{minipage}%}

\vspace{3mm}

\begin{minipage}{0.45\linewidth} \begin{prooftree}
    \AxiomC{$\Gamma_1 \vdash p \widesim{i} e$}
    \AxiomC{$\Gamma_2 \vdash f \proctr{i|j|\cdots|k}{} \phi$}
    \RightLabel{(应用)}
    \BinaryInfC{$\Gamma_1 \cup \Gamma_2 \vdash
    f\ p \proctr{j|\cdots|k}{} \phi\ e$}
\end{prooftree}\end{minipage}
\begin{minipage}{0.5\linewidth} \begin{prooftree}
    \AxiomC{$\Gamma,\ p \widesim{i} e \vdash 
    f \proctr{j|\cdots|k}{} \phi$}
    \RightLabel{(抽象)}
    \UnaryInfC{$\Gamma \vdash \lambda p.\ f \proctr{i|j|\cdots|k}{}
    \lambda e.\ \phi$}
\end{prooftree}\end{minipage}

\vspace{3mm}

\begin{minipage}{\linewidth}\begin{prooftree}
    \AxiomC{$\Gamma \vdash p \widesim{i} e$}
    \RightLabel{(对应记号 $x$)}
  \UnaryInfC{$\Gamma,\ ((x_\mathrm{p} = p)\ \land\ (
  x_\mathrm{e} = e)) \vdash
     x_\mathrm{p} \widesim{i}  x_\mathrm{e}$}
\end{prooftree}\end{minipage}

\vspace{3mm}

\begin{minipage}{\linewidth} \begin{prooftree}
  \AxiomC{$\Gamma \vdash p \proctr{i|\cdots|k}{cond} e$}
    \RightLabel{(同构记号 $x$)}
  \UnaryInfC{$\Gamma,\ ((x_\mathrm{p} = p)\ \land\ (
  x_\mathrm{e} = e)) \vdash
     x_\mathrm{p} \proctr{i|\cdots|k}{cond}  x_\mathrm{e}$}
\end{prooftree}\end{minipage}

\vspace{3mm}

\begin{minipage}{\linewidth}\begin{prooftree}
    \AxiomC{$\Gamma,\ x_\mathrm{p} = p,\ x_\mathrm{e} = e \vdash
     p' \widesim{i}  e'$}
    \RightLabel{(对应记号 $x$ 削除)}
    \UnaryInfC{$\Gamma \vdash \mathrm{Let}\ x_\mathrm{p}\ p\ p'
    \widesim{i} \mathrm{Let}\ x_\mathrm{e}\ e\ e'$}
\end{prooftree}\end{minipage}

\vspace{3mm}

\begin{minipage}{\linewidth}\begin{prooftree}
  \AxiomC{$\Gamma,\ x_\mathrm{p} = p,\ x_\mathrm{e} = e \vdash
   p' \proctr{i|\cdots|k}{cond}  e'$}
  \RightLabel{(同构记号 $x$ 削除)}
  \UnaryInfC{$\Gamma \vdash \mathrm{Let}\ x_\mathrm{p}\ p\ p'
  \proctr{i|\cdots|k}{cond} \mathrm{Let}\ x_\mathrm{e}\ e\ e'$}
\end{prooftree}\end{minipage}

\vspace{6mm}
\end{figure}



\begin{figure} \centering
\begin{tikzpicture}
\draw (0,0) node {编辑壳层} circle [radius=2];
%\path[name=bar1] (-3,)
\draw (0,2.5) arc [start angle=90, end angle=135, radius=2.5] -- ++ (0,2);
\draw (0,2.5) arc [start angle=90, end angle=45, radius=2.5];
\end{tikzpicture}
\end{figure}
