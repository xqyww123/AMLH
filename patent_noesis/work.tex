\chapter{本发明的技术方案}
\section{本发明采用的技术方案}

本发明首先开创性地提出 Noesis 对应关系,并以 Noesis 对应关系 $\widesim{\cdot}$ 替代
传统编程语言中的类型关系,以 Noesis 对应关系构成的形式系统 Noesis 形式系统
替代传统的类型系统,以此构建程序,这是本发明的核心技术。

具体而言,Noesis 对应关系 $\widesim{\cdot}$ 是一个三元关系
具有形式 $x \widesim{i} \epsilon$,其中 $x$ 表示程序部分,名叫 {\it 现象}
({\it phenomenon, φαινόμενον}),$i$ 表示理解方式叫做 {\it 理解}
({\it understanding, nóēsis, νόησῐς}),$\epsilon$ 表示程序 $x$ 在理解 $i$ 下的抽象对应,
表示抽象意义,名叫 {\it 本体}({\it noumenon, νoούμενον})。

朴素 Noesis 形式系统围绕 Noesis 对应关系,提供演绎 Noesis 对应关系的定理以构建程序的能力。
图 \ref{fig:naive_noesis_lang} 表示朴素 Noesis 形式系统的形式语言,
其中 $\hat{\mathcal{P}}, \hat{\mathcal{E}}, \mathcal{I}$ 为给定的单词集,
记号 $\bnf{BNF}$ 表示 BNF 语法,最后的 $L_\mathrm{N}$ 即为朴素 Noesis 形式系统的形式语言。
图 \ref{fig:naive_noesis_rule} 为朴素 Noesis 形式系统的演绎规则。
图 \ref{fig:naive_noesis_lang} 与图 \ref{fig:naive_noesis_rule} 共同表示了
朴素 Noesis 形式系统。
其中参数律用于构建参数,一阶同构引入律与一阶同构削除律为技术性操作无实际意义,应用律
用于函数调用,抽象律用于函数构建,对应记号律、同构记号律用于临时变量的引入,
对应记号削除律、同构记号削除律用于函数构建前临时变量的封装。

\noesishol 是 Noesis 形式系统在交互式高阶逻辑证明器(HOL interactive theorem prover)
的逻辑(HOL 逻辑,HOL Logic)上的实现,由图 \ref{fig:noesis-hol-rule} 表示。
它允许使用 HOL 定理证明器应用 Noesis 形式系统,进而演绎 Noesis 对应关系的定理以构建程序。

本发明提出的新技术是取代传统类型系统而以 Noesis 形式系统通过演绎 Noesis 对应关系的
定理以构建程序,这种技术通过一个名为 \Eamlh 的软件实现,并最终提供给用户使用。
图 \ref{fig:total-frame} 为软件 \Eamlh 的整体架构。


Noesis 形式系统实现在 HOL 交互式定理证明器上,
编辑壳层操作 HOL 交互式定理证明器演绎 Noesis 定理以构造程序。

程序的编译流程为图 \ref{fig:compilation} 。


\section{本发明的关键点}
\subsection{使用类型系统的扩广 Noesis 形式系统构造程序并形式化验证}

类型关系被替换成 Noesis 对应,类型系统被替换成围绕 Noesis 对应
构造的名为 {\it Noesis 系统}({\it Noesis system }) 的形式系统。
本发明不再关注类型关系而是围绕 Noesis 对应,
通过构造具有形式 $x \widesim{i} \epsilon$ 的定理以构建程序,
无论是通过对已有定理的演绎还是从头证明。

\chapter{附图}
\begin{figure}
\caption{朴素 Noesis 形式系统的形式语言} \label{fig:naive_noesis_lang}
\begin{gather*}
  \begin{align*} \hat{\mathcal{P}} &\subseteq \Words&
\hat{\mathcal{E}} &\subseteq \Words& \mathcal{I} &\subseteq \Words&
  \end{align*}\\
  \begin{array}{lcrcccccccl}
  \mathcal{P} &=& \bnf{&\hat{\mathcal{P}}&\mbar&
    (\mathcal{P}\ \mathcal{P})&
    \mbar&(\lambda\ \hat{\mathcal{P}}\ \mathcal{P})&\mbar&
    (\mathrm{Let}\ \hat{\mathcal{P}\ \mathcal{P}\ \mathcal{P}})&}\\
    \mathcal{E} &=& \bnf{&\hat{\mathcal{E}}&\mbar&
    (\mathcal{E}\ \mathcal{E})&
    \mbar&(\lambda\ \hat{\mathcal{E}}\ \mathcal{E})&\mbar&
    (\mathrm{Let}\ \hat{\mathcal{E}}\ \mathcal{E}\ \mathcal{E})&}
  \end{array} \\
\begin{align*}
 N_1 &= \bnf{\mathcal{P} \widesim{\mathcal{I}} \mathcal{E}}&
 N^* &= \bnf{\mathcal{P} \proctr{\mathcal{I}|\cdots|\mathcal{I}}{}
\mathcal{E}}& 
\Gamma_\mathrm{N} &= \powerset(N_1) & \end{align*} \\
L_\mathrm{N} = \bnf{\Gamma_\mathrm{N} \vdash N}
\end{gather*}
\end{figure}

\begin{figure}
\caption{朴素 Noesis 形式系统的演绎规则} \label{fig:naive_noesis_rule}
\vspace{6mm}

\begin{minipage}{0.25\linewidth} \begin{prooftree} \centering
    \AxiomC{\hfill}
    \RightLabel{(参数)}
    \UnaryInfC{$p \widesim{i} e \vdash p \widesim{i} e$}
\end{prooftree}\end{minipage}%
\begin{minipage}{0.4\linewidth} \begin{prooftree}
    \AxiomC{$\Gamma \vdash p \widesim{i} e$}
    \RightLabel{(一阶同构引入)}
    \UnaryInfC{$\Gamma \vdash p \proctr{i}{} e$}
\end{prooftree}\end{minipage}%
\begin{minipage}{0.3\linewidth} \begin{prooftree}
    \AxiomC{$\Gamma \vdash p \proctr{i}{} e$}
    \RightLabel{(一阶同构削除)}
    \UnaryInfC{$\Gamma \vdash p \widesim{i} e$}
\end{prooftree}\end{minipage}%}

\vspace{3mm}

\begin{minipage}{0.45\linewidth} \begin{prooftree}
    \AxiomC{$\Gamma_1 \vdash p \widesim{i} e$}
    \AxiomC{$\Gamma_2 \vdash f \proctr{i|j|\cdots|k}{} \phi$}
    \RightLabel{(应用)}
    \BinaryInfC{$\Gamma_1 \cup \Gamma_2 \vdash
    f\ p \proctr{j|\cdots|k}{} \phi\ e$}
\end{prooftree}\end{minipage}
\begin{minipage}{0.5\linewidth} \begin{prooftree}
    \AxiomC{$\Gamma,\ p \widesim{i} e \vdash 
    f \proctr{j|\cdots|k}{} \phi$}
    \RightLabel{(抽象)}
    \UnaryInfC{$\Gamma \vdash \lambda p.\ f \proctr{i|j|\cdots|k}{}
    \lambda e.\ \phi$}
\end{prooftree}\end{minipage}

\vspace{3mm}

\begin{minipage}{\linewidth}\begin{prooftree}
    \AxiomC{$\Gamma \vdash p \widesim{i} e$}
    \RightLabel{(对应记号 $x$)}
  \UnaryInfC{$\Gamma,\ ((x_\mathrm{p} = p)\ \land\ (
  x_\mathrm{e} = e)) \vdash
     x_\mathrm{p} \widesim{i}  x_\mathrm{e}$}
\end{prooftree}\end{minipage}

\vspace{3mm}

\begin{minipage}{\linewidth} \begin{prooftree}
  \AxiomC{$\Gamma \vdash p \proctr{i|\cdots|k}{cond} e$}
    \RightLabel{(同构记号 $x$)}
  \UnaryInfC{$\Gamma,\ ((x_\mathrm{p} = p)\ \land\ (
  x_\mathrm{e} = e)) \vdash
     x_\mathrm{p} \proctr{i|\cdots|k}{cond}  x_\mathrm{e}$}
\end{prooftree}\end{minipage}

\vspace{3mm}

\begin{minipage}{\linewidth}\begin{prooftree}
    \AxiomC{$\Gamma,\ x_\mathrm{p} = p,\ x_\mathrm{e} = e \vdash
     p' \widesim{i}  e'$}
    \RightLabel{(对应记号 $x$ 削除)}
    \UnaryInfC{$\Gamma \vdash \mathrm{Let}\ x_\mathrm{p}\ p\ p'
    \widesim{i} \mathrm{Let}\ x_\mathrm{e}\ e\ e'$}
\end{prooftree}\end{minipage}

\vspace{3mm}

\begin{minipage}{\linewidth}\begin{prooftree}
  \AxiomC{$\Gamma,\ x_\mathrm{p} = p,\ x_\mathrm{e} = e \vdash
   p' \proctr{i|\cdots|k}{cond}  e'$}
  \RightLabel{(同构记号 $x$ 削除)}
  \UnaryInfC{$\Gamma \vdash \mathrm{Let}\ x_\mathrm{p}\ p\ p'
  \proctr{i|\cdots|k}{cond} \mathrm{Let}\ x_\mathrm{e}\ e\ e'$}
\end{prooftree}\end{minipage}

\vspace{6mm}
\end{figure}


\begin{figure}[h] 
  \centering \begin{tikzpicture}\useasboundingbox (-5,-5) rectangle (5,5);
\baselineskip=0.5cm
    \draw (0,0) node (O) {编辑壳层} circle [radius=1.2];
  \path[name path=vCbarD] (60:3.3) -- ([turn]-90:1.5);
  \path[name path=vCbarU] (60:3.3) -- ([turn]90:1.5);
  \path[name path=hCbarD] (15:1.5) -- ([turn]45:4);
  \path[name path=hCbarU] (105:1.5) -- ([turn]-45:4);
  \draw [name intersections={of=vCbarD and hCbarD, by=x}] 
  (60:1.5) arc [start angle=60, end angle=15, radius=1.5] -- (x) -- ([turn]90:1.5);
  \draw [name intersections={of=vCbarU and hCbarU, by=x}]
  (60:1.5) arc [start angle=60, end angle=105, radius=1.5] -- (x) -- ([turn]-90:1.5);
  \draw [align=center] (60:2.4) node {编译\\后端};
  \draw (60:4) node[dart,draw,minimum size=1cm, shape border rotate=60,
    shape border uses incircle,fill=white] (B) {编译};
  \draw [->] (65:0.9) -- ([turn]-5:0.8);
  \draw [<-] (55:0.9) -- ([turn]5:0.8);

  \path[name path=vbarD] (-60:3.7) -- ([turn]-90:1.5);
  \path[name path=vbarU] (-60:3.7) -- ([turn]90:1.5);
  \path[name path=hbarD] (-105:1.5) -- ([turn]45:4);
  \path[name path=hbarU] (-15:1.5) -- ([turn]-45:4);
  \draw [name intersections={of=vbarD and hbarD, by=x}] 
  (-60:1.5) arc [start angle=-60, end angle=-105, radius=1.5] -- (x) -- ([turn]90:1.5);
  \draw [name intersections={of=vbarU and hbarU, by=x}]
  (-60:1.5) arc [start angle=-60, end angle=-15, radius=1.5] -- (x) -- ([turn]-90:1.5);
  \draw [align=center] (-60:2.5) node {HOL\\证明器};
  \draw [fill=white] (-60:4.6) +(-60:1.3) -- +(-150:1.3) -- +(120:1.3) -- + (30:1.3) -- +(-60:1.3);
  \draw (-60:4.6) node [align=center] (A) {\amlh};
  \draw [->] (-55:0.9) -- ([turn]-5:0.8);
  \draw [<-] (-65:0.9) -- ([turn]5:0.8);
  \path[name path=Vbar] (-4,1.2) -- (-4,-1.2);
  \path[name path=Dhbar] (-135:1.5) -- +(-3,0);
  \path[name path=Uhbar] (135:1.5) -- +(-3,0);
  \draw [name intersections={of=Vbar and Uhbar, by=x}] 
  (-1.5,0) arc [start angle=180, end angle=135, radius=1.5] -- (x) -- (-4,0);
  \draw [name intersections={of=Vbar and Dhbar, by=x}]
  (-1.5,0) arc [start angle=180, end angle=225, radius=1.5] -- (x) -- (-4,0);
  \draw [align=center] (-2.8,0) node {编辑器\\前端};
  \draw [<-] (175:1) -- +(-0.8,0);
  \draw [->] (-175:1) -- +(-0.8,0);
  \draw (-4.4,0) node[left] {用户};
  \draw [<-] (175:1) ++ (-2.7cm,0) -- +(-0.6,0);
  \draw [->] (-175:1) ++ (-2.7cm,0) -- +(-0.6,0);
  %\path (B) ([turn]-90:2) node (BX) {};
  %\draw [->,thick,rounded corners=1cm] (A) ([turn]90:1.8) -- (0:1.8) -- (BX);
\end{tikzpicture}
\caption{软件 \Eamlh 的主体框架} \label{fig:total-frame}
\end{figure}

\begin{figure}[h] 
  \centering \begin{tikzpicture} \baselineskip=0.5cm
    \draw (0,0) node (O) {编辑壳层} circle [radius=1.2];
  \path[name path=vCbarXD] (60:3.3) -- ([turn]-90:1.5);
  \path[name path=vCbarXU] (60:3.3) -- ([turn]90:1.5);
  \path[name path=hCbarXD] (15:1.5) -- ([turn]45:4);
  \path[name path=hCbarXU] (105:1.5) -- ([turn]-45:4);
  \draw [name intersections={of=vCbarXD and hCbarXD, by=x}] 
  (60:1.5) arc [start angle=60, end angle=15, radius=1.5] -- (x) -- ([turn]90:1.5);
  \draw [name intersections={of=vCbarXU and hCbarXU, by=x}]
  (60:1.5) arc [start angle=60, end angle=105, radius=1.5] -- (x) -- ([turn]-90:1.5);
  \draw [align=center] (60:2.4) node {编译\\后端};
  \draw (60:4) node[dart,draw,minimum size=1cm, shape border rotate=60,
    shape border uses incircle,fill=white] (B) {编译};
  \draw [->] (65:0.9) -- ([turn]-5:0.8);
  \draw [<-] (55:0.9) -- ([turn]5:0.8);

  \path[name path=vbarXD] (-60:3.7) -- ([turn]-90:1.5);
  \path[name path=vbarXU] (-60:3.7) -- ([turn]90:1.5);
  \path[name path=hbarXD] (-105:1.5) -- ([turn]45:4);
  \path[name path=hbarXU] (-15:1.5) -- ([turn]-45:4);
  \draw [name intersections={of=vbarXD and hbarXD, by=x}] 
  (-60:1.5) arc [start angle=-60, end angle=-105, radius=1.5] -- (x) -- ([turn]90:1.5);
  \draw [name intersections={of=vbarXU and hbarXU, by=x}]
  (-60:1.5) arc [start angle=-60, end angle=-15, radius=1.5] -- (x) -- ([turn]-90:1.5);
  \draw [align=center] (-60:2.5) node {HOL\\证明器};
  \draw [fill=white] (-60:4.6) +(-60:1.3) -- +(-150:1.3) -- +(120:1.3) -- + (30:1.3) -- +(-60:1.3);
  \draw (-60:4.6) node [align=center] (A) {\amlh};
  \draw [->] (-55:0.9) -- ([turn]-5:0.8);
  \draw [<-] (-65:0.9) -- ([turn]5:0.8);
  \path[name path=VbarX] (-4,1.2) -- (-4,-1.2);
  \path[name path=DhbarX] (-135:1.5) -- +(-3,0);
  \path[name path=UhbarX] (135:1.5) -- +(-3,0);
  \draw [name intersections={of=VbarX and UhbarX, by=x}] 
  (-1.5,0) arc [start angle=180, end angle=135, radius=1.5] -- (x) -- (-4,0);
  \draw [name intersections={of=VbarX and DhbarX, by=x}]
  (-1.5,0) arc [start angle=180, end angle=225, radius=1.5] -- (x) -- (-4,0);
  \draw [align=center] (-2.8,0) node {编辑器\\前端};
  \draw [<-] (175:1) -- +(-0.8,0);
  \draw [->] (-175:1) -- +(-0.8,0);
  \draw (-4.4,0) node[left] {用户};
  \draw [<-] (175:1) ++ (-2.7cm,0) -- +(-0.6,0);
  \draw [->] (-175:1) ++ (-2.7cm,0) -- +(-0.6,0);
  \path (B) ([turn]-90:2) node (BX) {};
  \draw [->,thick,rounded corners=1cm] (A) ([turn]90:1.8) -- (0:1.8) -- (BX);
\draw [align=center] (A) ([turn]90:2) node[right]{等价变换\\展开成基元指令\\与常量的表达};
    \draw [align=center] (2.7,0) node[right] {得到中间表达};
    \draw [align=center] (BX) node[right] {编译中间表达\\至目标执行环境};
\end{tikzpicture}
\caption{编译流程示意} \label{fig:compilation}
\end{figure}

\begin{figure}
\begin{center}
    \AxiomC{$\Gamma_1 \vdash f_p \proctr{i|j|\cdots|l}{cond} f_e$}
    \AxiomC{$\Gamma_2 \vdash a_p \widesim{i} a_e$}
    \RightLabel{(调用)}
    \BinaryInfC{$\Gamma_1 \cup \Gamma_2 \vdash f_p\ a_p \proctr{j|\cdots|l}{cond\ a_e}
      f_e\ a_e$}
\DisplayProof

  \vspace{\baselineskip}
    \AxiomC{$\Gamma_1,\ a_p \widesim{i} a_e \vdash b_p \proctr{j|\cdots|l}{cond} b_e$}
    \RightLabel{(抽象)}
    \UnaryInfC{$\Gamma_1 \vdash (\lambda a_p.\ b_p)
      \proctr{i|j|\cdots|l}{\lambda a_e.\ cond\ a_e} (\lambda a_e.\ b_e)$}
\DisplayProof

\vspace{\baselineskip}
\begin{tabular}{c c}
    \AxiomC{$\Gamma, cond \vdash p \widesim{i} e$}
    \RightLabel{(一阶同构引入)}
    \UnaryInfC{$\Gamma \vdash p \proctr{i}{cond} e$} \DisplayProof &
    \AxiomC{$\Gamma_1 \vdash p \proctr{i}{cond} e$}
    \AxiomC{$\Gamma_2 \vdash cond$}
    \RightLabel{(一阶同构削除)}
    \BinaryInfC{$\Gamma_1 \cup \Gamma_2 \vdash p \widesim{i} e$} \DisplayProof
\end{tabular}

\vspace{\baselineskip}
    \AxiomC{$\Gamma \vdash p \widesim{i} e$}
    \RightLabel{(对应记号 $x$)}
    \UnaryInfC{$\Gamma,\ ((x_\mathrm{p} = p)\ \land\ (
    x_\mathrm{e} = e)) \vdash
     x_\mathrm{p} \widesim{i}  x_\mathrm{e}$} \DisplayProof

\vspace{\baselineskip}
  \AxiomC{$\Gamma \vdash p \proctr{i|\cdots|k}{cond} e$}
    \RightLabel{(同构记号 $x$)}
    \UnaryInfC{$\Gamma,\ ((x_\mathrm{p} = p)\ \land\ (
    x_\mathrm{e} = e)) \vdash
     x_\mathrm{p} \proctr{i|\cdots|k}{cond}  x_\mathrm{e}$} \DisplayProof

\vspace{\baselineskip}
  \AxiomC{$\Gamma,\ ((x_\mathrm{p} = p)\ \land\ (x_\mathrm{e} = e 
  ))\vdash p' \widesim{i}  e'$}
    \RightLabel{(对应记号 $x$ 削除)}
  \UnaryInfC{$\Gamma \vdash \mathrm{Let}\ p\ (\lambda x_\mathrm{p}.\ p')
    \widesim{i} \mathrm{Let}\ e\ (\lambda x_\mathrm{e}.\ e')$}\DisplayProof

\vspace{\baselineskip}

\AxiomC{$\Gamma,\ ((x_\mathrm{p} = p)\ \land\ (x_\mathrm{e} = e))
  \vdash p' \proctr{i|\cdots|k}{cond}  e'$}
  \RightLabel{(同构记号 $x$ 削除)}
\UnaryInfC{$\Gamma \vdash \mathrm{Let}\ p\ (\lambda x_\mathrm{p}.\ p')
\proctr{i|\cdots|k}{cond} \mathrm{Let}\ e\ (\lambda x_\mathrm{e}.\ e')$} \DisplayProof

\vspace{\baselineskip}

\begin{tabular}{c c}
\AxiomC{$\Gamma \vdash p \widesim{i_1 \cdot i_2} (e_1,\ e_2)$}
    \RightLabel{(合并分解1)}
\UnaryInfC{$\Gamma \vdash p \widesim{i_1} e_1$} \DisplayProof&
\AxiomC{$\Gamma \vdash p \widesim{i_1 \cdot i_2} (e_1,\ e_2)$}
    \RightLabel{(合并分解2)}
\UnaryInfC{$\Gamma \vdash p \widesim{i_2} e_2$} \DisplayProof
\end{tabular}

\vspace{\baselineskip}

\AxiomC{$\Gamma_1 \vdash p \widesim{i_1} e_1$}
\AxiomC{$\Gamma_2 \vdash p \widesim{i_2} e_2$}
    \RightLabel{(合并)}
\BinaryInfC{$\Gamma_1 \cup \Gamma_2 \vdash p \widesim{i_1 \cdot i_2} (e_1,\ e_2)$}
\DisplayProof
\end{center}

\caption{\noesishol 上的演绎律} \label{fig:noesis-hol-rule}
\end{figure}

\begin{figure} \centering
  \begin{tikzpicture}
    \draw (-3.5,0) node [shade,draw,cylinder,shape border rotate=90,minimum width=3cm,
    bottom color=white!50!gray,top color=white,minimum height=1cm,
    aspect=.15] {Noesis 系统};
    \draw (-3.5,-1) node {朴素 Noesis 系统};
    \draw (0,0) node [shade,draw,cylinder,shape border rotate=90,
    bottom color=white!50!gray,top color=white,
    minimum width=3cm, minimum height=1cm, aspect=.2] {HOL 逻辑};
    \draw (0,-1) node {\noesishol};
    \draw (0,0.9) node [shade,draw,cylinder,shape border rotate=90,
    bottom color=white!50!gray,top color=white,
    minimum width=2cm, minimum height=1cm, aspect=.1] {Noesis 系统};
    \draw (4,0) node [shade,draw,cylinder,shape border rotate=90,
    bottom color=white!50!gray,top color=white,
    minimum width=4cm, minimum height=1cm, aspect=.3] {HOL 逻辑};
    \draw (4,-1) node {\amlh};
    \draw (4,1) node [shade,draw,cylinder,shape border rotate=90,
    bottom color=white!50!gray,top color=white,
    minimum width=3cm, minimum height=1cm, aspect=.2] {Noesis 系统};
    \foreach \x in {3,4.2}
    \draw [align=center] (\x,2.2) node [shade,draw,cylinder,shape border rotate=90,
    bottom color=white!50!gray,top color=white,
    minimum height=1.5cm, aspect=.17, font={\linespread{1.1}\tiny}] {基\\元\\指\\令};
    \foreach \x in {3.6,4.8}
    \draw [align=center] (\x,2.2) node [shade,draw,cylinder,shape border rotate=90,
    bottom color=white!50!gray,top color=white,
    minimum height=1.5cm, aspect=.17, font={\linespread{1.1}\tiny}] {常\\量};
    \draw (8.5,0) node [shade,draw,cylinder,shape border rotate=90,
    bottom color=white!50!gray,top color=white,
    minimum width=4cm, minimum height=1cm, aspect=.3] {HOL 逻辑};
    \draw (8.5,-1) node {\amlh 上的程序};
    \draw (8.5,1) node [shade,draw,cylinder,shape border rotate=90,
    bottom color=white!50!gray,top color=white,
    minimum width=3cm, minimum height=1cm, aspect=.2] {Noesis 系统};
    \foreach \x in {7.5,8.7}
    \draw [align=center] (\x,2.2) node [shade,draw,cylinder,shape border rotate=90,
    bottom color=white!50!gray,top color=white,
    minimum height=1.5cm, aspect=.17, font={\linespread{1.1}\tiny}] {基\\元\\指\\令};
    \foreach \x in {8.1,9.3}
    \draw [align=center] (\x,2.2) node [shade,draw,cylinder,shape border rotate=90,
    bottom color=white!50!gray,top color=white,
    minimum height=1.5cm, aspect=.17, font={\linespread{1.1}\tiny}] {常\\量};
    \draw (8.5,3.2) node [shade,draw,cylinder,shape border rotate=90,
    bottom color=white!50!gray,top color=white,
    minimum width=3cm, minimum height=1cm, aspect=.2] {\amlh 上的程序};
  \end{tikzpicture}
\caption{朴素 Noesis 系统、\noesishol、\amlh、\amlh 上的程序之间的关系} 
  \label{fig:noesis-rela}
\end{figure}


